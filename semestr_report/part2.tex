\chapter{Анализ физических ограничений для решения задачи в ординарных и двойных каскадах}\label{ch:ch2}

Как уже было отмечено в главе \ref{ch1}, лишь некоторые из предложенных к настоящему моменту способов обогащения регенерированного урана потенциально способны решить задачу обогащения регенерата произвольного исходного состава в условиях одновременного выполнения ограничений на концентрации сразу нескольких изотопов и при заданном отношении между массой получаемого НОУ и исходной смеси регенерированного урана, поступающего для обогащения. В первую очередь, это касается разбавляющих схем на основе ординарного каскада. 
Однако, проведенный в главе \ref{ch1} теоретический анализ не позволяет априори определить при каких условиях может быть применена та или иная схема. В рамках настоящей главы кратко представлены результаты вычислительных экспериментов и сопутствующего им теоретического анализа, направленных на оценку возможности применения одиночных каскадных схем с разбавлением и двойных каскадов для получения обогащенного регенерированного урана в условиях многократного рецикла. 

\section{Модификации каскадных схем на основе ординарного каскада: методическая часть}\label{ch2_stat}

Вернемся к рассмотренным в Главе \ref{ch1} модификациям каскадных схем, основанных на ординарном каскаде (рисунок \ref{fig:diagram1ch3}), для обогащения регенерированного урана с одновременным разбавлением четных изотопов. Каждая из схем, представленных на рисунке \ref{fig:diagram1ch3} реализует один из возможных способов разбавления регенерированного урана. Отличия в способах состоят в том, в каком именно узле осуществляют разбавление регенерата, либо в  используемой в качестве разбавителя смеси. Например, в качестве разбавителя может выступать природный уран, НОУ из природного сырья.


\begin{figure}[ht]
  \centerfloat{\includegraphics[scale=0.7]{cascades/ord_all}}
  \caption{Схемы на основе ординарного каскада. Обозначения: $E$ -- поток питающего схему регенерата, $F_n$ -- поток разбавителя (природного урана или низкообогащенного урана); $W$ -- поток отвального ОГФУ тяжелого конца каскада; $P$ -- товарный низкообогащенный уран}\label{fig:diagram1ch3}
\end{figure}

Достоинства и недостатки подобных схем на теоретическом уровне подробно проанализированы в Главе \ref{ch1}. Однако, как следует из литературного обзора, для данных схем возможность их применения продемонстрирована, в первую очередь, на примере составов, характеризующихся относительно низким содержанием четных изотопов. Например, концентрация изотопа $^{232}$U в поступающем в обогащение регенерате составляет величину $1,5\cdot10^{-7}$\%, что в разы ниже значений, которые могут достигаться в условиях многократного рецикла (вплоть до $1\cdot10^{-6}$\%) \cite{palkinDesignanalyticalResearchRefinement2010}. Кроме того, при рассмотрении подобных каскадных схем, как правило, исходят из того, что уровень обогащения $^{235}$U регенерата или разбавителя в них, не превышает 10\%. В связи с этим представляется целесообразным проведение дополнительного исследования, которое бы позволило изучить взаимосвязи параметров модификаций каскадных схем обогащения регенерата на основе ординарного каскада при различных условиях. В свою очередь, это позволит оценить потенциал таких каскадных схем для обогащения регенерированного урана при различных внешних условиях. 

Таким образом, основная цель описанных ниже вычислительных экспериментов -- оценить возможности рассматриваемых модификаций ординарного каскада для обогащения регенерированного урана в условиях заметных (до одного порядка) колебаний концентраций четных изотопов, что характерно для его многократного рецикла. Рассмотрены случаи обогащения регенерированного урана двух составов с различным исходным содержанием чётных изотопов (см. таблицу \ref{is_compositions_2_5}). Выбранные составы отвечают регенерированному урану, выделенному из ОЯТ реакторов ВВЭР-1000 и -1200 при различных внешних условиях  и заимствованы из работ, посвященных анализу закономерностей изменения изотопного состава регенерата при его многократном рецикле \cite{palkinDesignanalyticalResearchRefinement2010,nevinicaToplivnyyCiklLegkovodnogo2019}. Оба состава характеризуются содержанием $^{232}$U, которое превышает предельно допустимый уровень концентарции $^{232}$U в конечном НОУ-продукте ($5\cdot10^{-7}$\%). Выбор подобных загрязненных четными изотопами составов регенерата имитирует сложности, которые могут возникать при обогащении регенерированного урана в условиях его многократного рецикла.  

\begin{table}[h]
  \centering
  \normalsize\begin{tabulary}{1.0\textwidth}{|c|c|c|c|c|c|c|}
  \hline Состав № & Массовое число & 232 & 233 & 234 & 235 & 236 \\
  \hline 1 & C, \% & $6,62\cdot10^{-7}$ & $1,19\cdot10^{-6}$ & $3,28\cdot10^{-2}$ & 1,43 & 0,9932 \\
  2 & C, \% &  $1,03\cdot10^{-6}$ & $1,3\cdot10^{-6}$ & $3,91\cdot10^{-2}$ & 1,07 & 1,45 \\\hline
  \end{tabulary}
  \caption{{Изотопные составы регенерата различных циклов.{\label{is_compositions_2_5}}}}
\end{table}

При реализации вычислительных экспериментов общая постановка задачи соответствовала формулировке, приведенной в Главе \ref{ch1}. С учётом конкретных выбранных ограничений ее можно представить в следующем виде.

Из заданной массы исходного регенерированного урана необходимо получить заданную массу товарного НОУ, отвечающего следующим требованиям:

\begin{enumerate}
  \item Концентрация $^{235}$U в конечном продукте составляет 4,95\%, значение характерно для современных легководных реакторов \cite{solovevaCennostiOYaTKak2019};
  \item Расход регенерированного урана на единицу конечного продукта в виде низкообогащенного урана: 0,93 кг на 1 кг НОУ \cite{smirnovApplyingEnrichmentCapacities2018};
  \item Концентрация $^{235}$U в потоке отвала задана равной 0,1\% \cite{smirnovEvolutionIsotopicComposition2012};
  \item Соотношение $^{234}$U к $^{235}$U не должно превышать значения 0,02;
  \item Влияние изотопа $^{236}$U на нейтронно-физические характеристики топлива должно быть скомпенсировано дополнительным обогащением по $^{235}$U, для расчёта которого использована линейная функция $\Delta C_{235,P}=K_{c}\times C_{236,P}$, где $K_{c}$ -- коэффициент компенсации реактивности, принятый равным 0,29 \cite{smirnovApplyingEnrichmentCapacities2018};
  \item Концентрация $^{232}$U ограничена величиной $5\cdot10^{-7}$\% \cite{smirnovApplyingEnrichmentCapacities2018}.
\end{enumerate}

Решение подобной задачи означает поиск такого набора параметров каждой из представленных на рисунке \ref{fig:diagram1ch3} схем, который обеспечит одновременное удовлетворение перечисленных выше условий. 

Как следует из анализа рисунка, каждая из схем подразумевает использование ординарного каскада, который обогащает либо регенерированный уран, либо природный в зависимости от выбранного способа. Для моделирования процесса обогащения урана в каскаде в рамках работы была выбрана модель R-каскада \cite{sulaberidzeTeoriyaKaskadovDlya2011}. При этом для всех рассматриваемых схем при расчёте параметров каскада задавали концентрации $^{235}$U в его внешних выходящих потоках. Таким образом, расчёт параметров такого каскада был сведен к одной из возможных постановок задач при моделировании процессов разделения в каскадах -- задаче проектировочного расчета (см. Главу \ref{ch2_theory}) и подразумевал следующую математическую постановку задачи. 

Задано: 

\begin{enumerate}
  \item концентрации компонентов исходной смеси регенерированного урана -- ${C}_{i,E}$;
  \item коэффициент разделения для единочной разности массовых чисел одиночного разделительного элемента -- ${q}_{0}$;
  \item величина одного из внешних потоков каскада, например, $P$, $E$ или $F_n$ (см. рис. \ref{fig:diagram1ch3});
  \item концентрации $^{235}$U в потоках отбора и отвала каскада -- ${C_{235, P}}$, ${C_{235, W}}$;
  \item номера компонентов, для которых выполнено условие несмешивания по относительным концентрациям -- $n$ и $k$.
\end{enumerate}

В процессе расчёта необходимо определить следующие параметры: 

\begin{enumerate}
  \item величины $N$ и $f$;
  \item концентрации ${C}_{i,P}$ и ${C}_{i,W}$ ($i \neq n$, где индекс n соответствует $^{235}$U) в потоках $P$ и $W$; 
  \item отношения внешних потоков каскада -- $P/F$, $W/F$;
  \item распределения потока и концентраций компонентов по ступеням каскада -- $L_{s}, C_{i,s} (i = 1,.., m)$;
  \item значения срезов потоков на ступенях -- $\theta_{s}$;
  \item остальные внутренние параметры каскада. 
\end{enumerate}

Описанная постановка задачи расчета параметров ординарного R-каскада требует численного решения системы нелинейных уравнений, из которых возможно определить величины $N$ и $f$, отвечающие заданным значениям концентраций целевого компонента во внешних потоках (см. раздел \ref{R_cas}). Указанная система может быть записана в следующем виде: 

\begin{equation}\label{dpdw}
  \begin{cases}
  \Delta_{P} = {(C_{235, P})}_{calc}-{(C_{235, P})}_{given}\\
  \Delta_{W} = {(C_{235, W})}_{calc}-{(C_{235, W})}_{given}
  \end{cases}\,
\end{equation}

, где ${(C_{235, P})}_{calc}$, ${(C_{235, W})}_{calc}$ -- рассчитанные концентрации $^{235}$U в потоках отбора и отвала каскада, соответственно; ${(C_{235, P})}_{given}$, ${(C_{235, W})}_{given}$ -- заданные концентрации $^{235}$U в потоках отбора и отвала каскада, соответственно; а $\Delta_{P}$, $\Delta_{W}$ -- невязки по $C_{235, P}$ и $C_{235, W}$, соответственно. 

Представленная система (\ref{dpdw}) в случае R-каскада формируется на основе соотношений для расчета концентраций компонентов в потоках отбора и отвала каскада и представляет собой систему нелинейных алгебраических уравнений. В этом случае суть процедуры расчета параметров каскада состоит сначала в итерационном поиске величин $N$ и $f$, после чего возможно аналитически рассчитать остальные параметры каскада. Итерирование величин $N$ и $f$ осуществляют на основе одного из известных методов решения систем нелинейных уравнений, например, метода Ньютона, применяя их к системе (\ref{dpdw}). 

При использовании модели R-каскада в указанных выше уравнениях удобно использовать соотношения (\ref{GrindEQ__1_72_}), (\ref{GrindEQ__1_73_}). В этом случае из решения системы определяют величины $R_{n k}^{W}$ и $R_{n k}^{P}$, после чего аналитически рассчитать остальные внешние параметры R-каскада по соотношениям (\ref{GrindEQ__1_70_})-(\ref{GrindEQ__1_77_}), а также рассчитать значений для $N$ и $f$. Окончив расчет параметров каскада можно легко аналитически рассчитать состав получаемого конечного продукта после смешивания отбора каскада с разбавителем.

В последующих разделах настоящей главы представлены результаты моделирования обогащения регенерата в каждой из трёх схем, представленных на рисунке \ref{fig:diagram1ch3}. 
Основная цель проведенного моделирования -- оценить возможность использования схем на основе ординарного каскада для обогащения регенерированного урана с повышенным содержанием чётных изотопов и, в первую очередь, $^{232}$U. Во всех случаях первоначально в качестве обогащаемого состава был рассмотрен состав №1 (табл. \ref{is_compositions_2_5}), имеющий более низкое содержание чётных изотопов, по отношению к составу 2. Учитывая то, что основная цель вычислительных экспериментов состояла в оценке применимости рассматриваемых схем для решения поставленной задачи, рассмотрение сначала менее загрязненного состава, в случае невозможности получить решение, позволит сделать вывод о невозможности использования той или иной схемы и для более загрязненных четными изотопами составов.

Для моделирования представленных на рисунке \ref{fig:diagram1ch3} схем в рамках работы разработан оригинальный программный код на языке Julia, реализующий процедуры численного расчета параметров ординарных каскадов для различных постановок задач.

\subsection{Схема с разбавлением природным ураном предварительно обогащенного регенерата}

Рассмотрим каскадную схему, в которой регенерат сначала обогащают до уровня, превышающего необходимую для товарного НОУ концентрацию $^{235}$U, а затем разбавляют, например, природным ураном (рис. \ref{o1}). Необходимо проверить существование такого набора параметров данной схемы, который одновременно обеспечит получение заданной массы товарного НОУ заданного обогащения по $^{235}$U, соблюдение ограничений на концентрации четных изотопов в НОУ, а также условие полного использования регенерата. Заметим, что данная схема допускает возможность использования в качестве разбавителя любой другой урановой смеси, не содержащей изотопов $^{232}$U и $^{236}$U. В качестве одного из вариантов разбавителей может быть использован и низкообогащенный уран, полученный обогащением природного урана.

\begin{figure}[ht]
  \centerfloat{\includegraphics[scale=1.2]{cascades/ord1}}
  \caption{Схема разбавления предварительно обогащенного регенерата природным ураном или низкообогащенным ураном. Обозначения: $E$ -- поток питающего схему регенерата; $P_0$ -- поток отбора легкой фракции каскада; $F_n$ -- поток разбавителя (природного урана или низкообогащенного урана); $W$ -- поток отвального ОГФУ тяжелого конца каскада; $P$ -- поток товарного низкообогащенного урана}\label{o1}
\end{figure}


Для получения обогащенного урана, удовлетворяющего всем требованиям, необходимо определить величину $C_{235, P_0}$ и отношение потоков $P_0$ и разбавителя. Указанные параметры определяли итерационно по следующей схеме. Сначала задавали начальное приближение для  $C_{235, P_0}$, после чего рассчитывали параметры каскада по описанной в предыдущем разделе процедуре. Далее, зная состав смеси урана в потоке $P_0$, определяли соотношение между потоками природного урана или обогащенного регенерата для получения финального продукта. При этом полученный в результате смешивания поток товарного НОУ должен отвечать ограничениям по концентрациям чётных изотопов, а отношение массы полученного НОУ к массе исходного регенерата должно соответствовать заданной величине. Это означает, что для успешного решения задачи одновременно должны быть выполнены условия на концентрации изотопов $^{232,234,235,236}$U и обеспечено заданное отношение между расходом регенерата и конечным продуктом. 

Однако при известной и заданной концентрации $^{235}$U в потоке разбавителя управляющих параметров в такой схеме только два: $C_{235, P_0}$ и отношение ${P_0}{/}{F_n}$. При этом существует четыре условия, которые должны быть выполнены одновременно: ограничение на концентрации изотопов $^{232,234}$U, достижение заданной концентрации $^{235}$U с учетом компенсации $^{236}$U и заданное соотношение между потоками $E$ и $P$. Очевидно, что в этом случае не для любых исходных данных возможно подобрать требуемые параметры каскадной схемы.

Для иллюстрации сложности решения подобной задачи рассмотрим некоторые вспомогательные функции $\chi_1$ и $\chi_2$ (\ref{d1}).

\begin{equation}\label{d1} 
  \begin{cases}
    \chi_1=\left[C_{235,P\textit{экв.}}-\left(C_{235,P\textit{NU}}+K_{c} \times C_{236,P}\right)\right]\\
    \chi_2=\left[{(C_{232,P})}_{lim}-C_{232,P}\right]
  \end{cases}\
\end{equation}
,
где $K_{c}$ -- коэффициент компенсации реактивности, $(C_{232,P})_{lim}$ -- величина предельно допустимой концентрации $^{232}$U в товарном НОУ (в рассматриваемом примере $5\cdot10^{-7}$\%).

По своему физическому смыслу величина $\chi_1$ представляет собой отклонение  $C_{235, P}$ -- концентрации изотопа $^{235}$U (выраженное в долях) в конечном продукте (после смешивания) -- от заданной величины, с учетом компенсации $^{236}$U. Величина $\chi_2$ представляет собой разность фактической концентрации $^{232}$U в окончательном продукте и требуемой величины в соответствии с принятым ограничением. Из таких определений очевидно, что в случае получения продукта, отвечающего необходимым условиям величина $\chi_1$ должна быть равна 0 в пределах заданной точности решения задачи, а величина $\chi_2$ должна быть $\leq0$ при одних и тех же параметрах схемы. 

Оговоримся, что, строго говоря, к этой условиям (\ref{d1}) следует добавить аналогичное условие $\chi_3=\left[C_{234,P}/C_{235,P}-D\right]$ (D -- заданное предельное значение относительной концентрации $^{234}$U к $^{235}$U в конечном НОУ-продукте $P$). Это означает, что ещё одним условием успешного решения задачи обогащения регенерата будет являться условие $\chi_3=\leq0]$. Как показали серии расчетов, в большинстве ситуаций это условие оказывается выполненным. Далее, будет показано что основную роль для успешного решения задачи играют условия, накладываемые на величины $\chi_1$ и $\chi_2$.

В работе проведены вычислительные эксперименты, в которых варьировали концентрацию $C_{235, P_0}$ и отношение потока $P_0$ к разбавителю из природного урана. Диапазон варьирования величины $C_{235, P_0}$ выбран исходя из соображений достижения величины 90\%, а диапазон варьирования отношения потоков $P_0$ к разбавителю выбрали равным 1--30, исходя из предварительных расчетов. Для каждого случая пытались решить задачу обогащения регенерированного урана при описанных в разделе \ref{ch2_stat} внешних условиях. Концентрацию $^{235}$U в потоке $W$ задавали равной 0,1\%. При расчёте параметров ординарного каскада во всех случаях предполагали, что в каскаде было реализовано условие несмешивания по относительной концентрации компонентов $^{235}UF_6$ и $^{238}UF_6$, так как эти два компонента имеются во всех используемых исходных смесях. Величину коэффициента разделения для компонентов  $^{235}UF_6$ к $^{238}UF_6$ приняли равной 1,2 \cite{smirnovEvolutionIsotopicComposition2012}. Расчёты выполнены на примере регенерированного урана состава 1 (табл. \ref{is_compositions_2_5}).

На рис. \ref{delta1}--\ref{delta4} представлены зависимости величин $\chi_1$ и $\chi_2$ от соотношения смешиваемых потоков (в диапазоне от 1 до 20, в который попадают все значимые для иллюстрации значения величин $\chi_1$ и $\chi_2$), взятых для различных значений $C_{235, P_0}$ (рис. \ref{delta1}--\ref{delta4}). Значения $C_{235, P_0}$ на рис. \ref{delta1}--\ref{delta4} представлены в диапазоне от 7 до 50\% в иллюстративных целях,  также как и соотношения смешиваемых потоков, так как эти диапазоны достаточны, чтобы продемонстрировать невозможность одновременного удовлетворения условия компенсации $^{236}$U и выполнения заданного ограничения по $^{232}$U в получаемом товарном НОУ. Чтобы сопоставить указанные величины на одном рисунке, величина $\chi_2$ была умножена на специально подобранный числовой коэффициент, равный $10^{6}$.

\begin{figure}[ht]
  \begin{minipage}{.5\textwidth}
    \centering
    \includegraphics[width=.8\linewidth]{images/plots/7}  
    \caption{Концентрация $^{235}$U в предварительно обогащенном регенерата равна 7\%}
    \label{delta1}
  \end{minipage}
  \begin{minipage}{.5\textwidth}
    \centering
    \includegraphics[width=.8\linewidth]{images/plots/15}  
    \caption{Концентрация $^{235}$U в предварительно обогащенном регенерата равна 15\%}
    \label{delta2}
  \end{minipage}
  \begin{minipage}{.5\textwidth}
    \centering
    \includegraphics[width=.8\linewidth]{images/plots/30}  
    \caption{Концентрация $^{235}$U в предварительно обогащенном регенерата равна 30\%}
    \label{delta3}
  \end{minipage}
  \begin{minipage}{.5\textwidth}
    \centering
    \includegraphics[width=.8\linewidth]{images/plots/50}  
    \caption{Концентрация $^{235}$U в предварительно обогащенном регенерата равна 50\%}
    \label{delta4}
  \end{minipage}
 \end{figure}

Как видно из рисунков \ref{delta1}-\ref{delta4} функции $\chi_1$ во всех случаях оказывается равной 0 в случаях, когда $\chi_2$ оказывается отрицательной, что означает превышение предельно допустимого значения концентрации $^{232}$U в продукте. Таким образом, полученные результаты показывают невозможность одновременного удовлетворения условия компенсации $^{236}$U и выполнения заданного ограничения по $^{232}$U в получаемом товарном НОУ. Очевидно, что в случае обогащения регенерата состава №2 ситуация окажется ещё более худшей, ввиду более высокого исходного содержания четных изотопов. 

Полученные результаты свидетельствуют о том, что данную схему нельзя рассматривать в качестве способа обогащения регенерированного урана в условиях его многократного рецикла, по крайней мере для выбранных в данном примере или более строгих ограничениях на концентрацию $^{232}$U. Основная причина таких результатов состоит в том, что рассматриваемая схема имеет количество свободных параметров, меньшее, чем количество условий требующих одновременного выполнения. Поэтому только в частных случаях возможно решить задачу, например, когда в обогащение поступает регенерированный уран с относительно низким исходным содержанием четных изотопов, что может соответствовать, например, обогащению регенерата первого рецикла.


\subsection{Схема с разбавлением предварительно обогащенного регенерата низкообогащенным ураном}\label{ch2_1_1}

Если в схеме, рассмотренной выше (рис. \ref{o1}), заменить разбавитель $(F_n)$ с природного урана на низкообогащенный уран, не содержащий четных изотопов (например, изготовленный из природного урана), то у подобной схемы, тем самым, появится дополнительный управляющий параметр -- концентрация $^{235}$U в потоке $F_n$, которую можно также варьировать.  Несмотря на это, в подобном варианте каскадной схемы все еще может быть не выполнено условие максимального использования регенерированного урана, состоящее в равенстве отношения потоков $E$ и $P$ заданной величине, так как количество свободных параметров все равно будет меньшим, чем количество условий. В качестве еще одного параметра схемы можно рассматривать концентрацию $^{235}$U в потоке отвала каскада -- $C_{235, W}$. Необходимо учесть, что данную величину возможно варьировать только в относительно узком диапазоне от 0,1\% до 0,6\% или ниже. 

Для оценки возможности решения задачи с использованием каскадной схемы рисунка \ref{o1} проведены вычислительные эксперименты, в рамках которых варьировались следующие параметры схемы:
\begin{enumerate}
  \item концентрация $^{235}$U в потоке $P_0$ ($C_{235, P_0}$);
  \item концентрация $^{235}$U в потоке $F_n$ ($C_{235, F_n}$);
  \item концентрация $^{235}$U в потоке $W$ ($C_{235, W}$).
\end{enumerate}

Варьирование этих параметров позволяет регулировать относительные концентрации компонентов в смеси, в частности, пары изотопов $^{232}$U и $^{235}$U. Это позволяет обеспечивать получение различных вариантов изотопного состава конечного продукта, возможно один из которых сможет удовлетворить все заданные ограничения одновременно. 

В части основных параметров каскада расчеты проводили при тех же значениях, что и в примерах, описанных в разделе \ref{ch2_1_1}. Как показали результаты расчетов, рассматриваемая модификация каскадной схемы обогащения регенерата может одновременно удовлетворить все ограничения на концентрации четных изотопов в продукте. Однако ни в одном из рассчитанных вариантов не удалось добиться получения заданного отношения между исходным регенератом и продуктом. Данное утверждение иллюстрирует рисунок \ref{Figure_10}. Анализ кривых на указанном рисунке показывает невозможность выполнить условия возврата заданной доли регенерата на единицу продукта в такой каскадной схеме в доступном диапазоне варьирования её свободных параметров. Причем отношение используемого регенерата к НОУ-продукту меняется лишь незначительно с изменением $C_{235, W}$, и остается постоянным при разных концентрациях $C_{235, P_0}$. Последнее означает, что изменение данного параметра может оказывать влияние на величину затрат работы разделения или расхода природного урана, но не на долю возвращаемого в воспроизводство топлива регенерата.

% , для которого удельный расход природного урана составляет $\approx$7,93 (см. Приложение).

% \begin{figure}[ht]
%   \centerfloat{\includegraphics[scale=0.5]{images/plots/sc2_LEU_D}}
%   \caption{Концентрация $^{235}$U в разбавителе, необходимая для получения свежего НОУ для различных концентраций $^{235}$U в потоках продукта и отвала каскада, обогащающего регенерат}\label{fig:sc2_LEU_D}
% \end{figure}

% Проиллюстрировать затраты на работу разделения от составных  частей каскадной схемы, можно с помощью рис.\ref{myplot}, который показывает, что доля центрифуг для приготовления разбавителя из природного урана выше, чем доля центрифуг, задействованных для предварительного обогащения регенерата, и эта пропорция уменьшается с понижением содержания $^{235}$U в $W_0$.

% \begin{figure}[ht]
%   \centerfloat{\includegraphics[scale=0.5]{images/plots/myplot}}
%   \caption{Отношение количества центрифуг в каскаде, производящем разбавитель из природного урана, к количеству центрифуг, задействованных для предварительного обогащения регенерата, для различных концентраций $^{235}$U в потоках продукта и отвала каскада, обогащающего регенерат}\label{myplot}
% \end{figure}

Подытоживая анализ результатов вычислительных экспериментов для данной модификации ординарного каскада для обогащения регенерированного урана, можно заключить, что она непригодна для решения задачи обогащения в условиях многократного рецикла, так как с помощью нее нет возможности использовать весь регенерированный уран на производство НОУ-продукта, как показано на рис. \ref{Figure_10}. Тем не менее, рассмотренная каскадная схема может быть использована для решения задачи повторного использования урана для возврата (дообогащения) регенерата с относительно низким исходным содержанием чётных изотопов, что соответствует первому рециклу или низкой глубине выгорания топлива.


\begin{figure}[ht]
  \centerfloat{\includegraphics[scale=0.5]{images/plots/3_6}}
  \caption{Расход регенерата на единицу конечного НОУ-продукта для различных  $C_{235, P}$ и $C_{235, W}$ каскада, обогащающего регенерат (кривые совпадают)}\label{Figure_10}
\end{figure}

Также можно заметить, что концентрация $^{235}$U в НОУ-разбавителе (рис. \ref{fig3__7}) близка к значениям, требуемым в конечном продукте.

\begin{figure}[ht]
  \centerfloat{\includegraphics[scale=0.5]{images/plots/3__7}}
  \caption{Концентрация $^{235}$U в НОУ-разбавителе $F_n$ для различных $C_{235, P}$ и $C_{235, W}$ каскада, обогащающего регенерат}\label{fig3__7}
\end{figure}

\subsection{Анализ схемы с разбавлением предварительно обогащенного природного урана регенератом}

Проанализируем возможность решения задачи обогащения регенерированного урана со всеми ограничениями в каскадной схеме с разбавлением предварительно обогащенного природного урана регенератом (рис. \ref{o2}). Принцип работы такой схемы состоит в том, что предварительно обогащенный природный уран смешивается с возвращаемым в топливный цикл регенерированным ураном. Уровень предварительного обогащения (перед смешением) природного урана и отношение потоков обогащенного природного урана к регенерату определяются исходя из условий задачи. Таким образом, данная схема в принципе аналогична схеме рисунка \ref{o1}. Это означает, что ей присущ тот же принципиальный недостаток, что и упомянутой схеме, а именно: число ее управляющих параметров меньше, чем число условий, требующих одновременного выполнения. 

\begin{figure}[ht]
  \centerfloat{\includegraphics[scale=1.1]{cascades/ord2}}
  \caption{Схема каскада с разбавлением предварительно обогащенного природного урана регенератом. Обозначения: $E$ -- регенерат, $F_n$ -- природный уран; $W$ -- поток отвального ОГФУ тяжелого конца каскада; $P$ -- поток товарного низкообогащенного урана}\label{o2}
\end{figure}

Для ответа на вопрос о возможности использования данной схемы для обогащения регенерата в условиях многократного рецикла были проведены вычислительные эксперименты, в рамках которых варьировали величину концентрации $^{235}$U в обогащенном природном уране и отношение, в котором смешиваются разбавитель с регенератом с целью найти такой набор параметров схемы, при которых сформулированная в разделе \ref{ch2_stat} задача будет решена. Как и в рассмотренных выше примерах моделирование процесса обогащения урана в каскаде осуществляли с использованием R-каскада. Концентрацию $^{235}$U в отвале каскада задавали равной 0,1\%.

Из результатов вычислительных экспериментов следует, что для рассматриваемой схемы возможно получение решения, удовлетворяющего заданным ограничениям на концентрации изотопов $^{232,234,236}$U. Однако, как и в случае применения схемы рис. \ref{o1}, одновременно с этими условиями не удаётся удовлетворить условие возврата заданной массы регенерата. В результате вместо заданной величины отношения массы исходного регенерата к продукту -- 0,93, фактические значения не превысили величины 0,755. Данные результаты свидетельствуют о том, что такая схема обогащения регенерата не решает поставленную задачу для произвольного изотопного состава регенерата и, следовательно, не может быть применена в условиях многократного рецикла урана в топливе легководных реакторов.

\subsection{Анализ схемы с разбавлением регенерата природным ураном перед подачей в ординарный трехпоточный каскад}

Еще одним вариантом каскадной схемы для обогащения регенерированного урана, основанной на использовании ординарного каскада является схема, в которой смешивание и разбавление регенерата происходит непосредственно перед подачей в каскад для последующего обогащения (рис. \ref{o3}). В качестве разбавителя здесь, как правило, рассматривают природный уран. Соотношение, в котором следует смешать природный и регенерированный уран определяют, исходя из ограничений на концентрации четных изотопов в конечном НОУ-продукте.

\begin{figure}[ht]
  \centerfloat{\includegraphics[scale=1.1]{cascades/ord3}}
  \caption{Схема каскада со смешением регенерата и природного урана перед подачей на питание ординарного каскада. Обозначения: $E$ -- поток питающего схему регенерата, $F_n$ -- поток разбавителя; $W$ -- поток отвального ОГФУ тяжелого конца каскада; $P$ -- поток товарного низкообогащенного урана}\label{o3}
\end{figure}

В случае, если разбавитель известен, то для такой схемы существует единственный управляющий параметр -- это отношение, в котором смешиваются потоки регенерата и природного урана. Очевидно, что с ростом доли регенерата в совокупном питании каскада (смесь потоков $E$ и $F_n$) будут возрастать концентрации чётных изотопов в потоке отбора каскада. Это обуславливает тот факт, что существует некоторое критическое значение этого отношения, начиная с которого уже невозможно будет соблюсти, как минимум, ограничение на концентрацию изотопа $^{232}$U. Для рассматриваемой схемы проведены вычислительные эксперименты, в которых варьировали отношение разбавления между регенератом и разбавителем, в качестве которого рассматривали уран природного состава. Кроме того, в диапазоне 0,05--0,3\% варьировали концентрацию изотопа $^{235}$U в потоке отвала каскада -- $C_{235, W}$. Как и во всех рассмотренных в рамках данной главы примерах исходные условия соответствовали задаче, описанной в разделе \ref{ch2_stat}, а в качестве расчётной модели использован R-каскад. Расчёты проведены на примере состава 1 таблицы \ref{is_compositions_2_5}. 

На рис. \ref{sc3_1.second} отражена взаимозависимость отношения потоков $E/P$ и концентрации $^{232}$U в конечном продукте. Кривые построены при различных $C_{235, W}$. Как следует из анализа представленных зависимостей, во всех случаях величина концентрации  $^{232}$U достигает предельного значения ($5\cdot10^{-7}$\%) ранее, чем отношение между исходным регенератом и продуктом достигнет требуемого значения -- 0,93. Это означает, что и эта схема также не позволяет решить полностью задачу обогащения регенерата с относительно высоким содержанием чётных изотопов, не позволяя расходовать заданное количество регенерата на единицу конечного НОУ-продукта. Иными словами схема также оказывается непригодной к обогащению регенерированного урана в условиях многократного рецикла в допустимом диапазоне изменения ее параметров.

\begin{figure}[ht]
  \centerfloat{\includegraphics[scale=0.5]{images/plots/3_11}}
  \caption{Расход регенерированного урана на единицу НОУ-продукта  при различной концентрации $^{232}$U в питающем потоке каскада для различных концентраций $^{232}$U в потоке НОУ-продукта}\label{sc3_1.second}
\end{figure}


% \begin{figure}[ht]
%   \centerfloat{\includegraphics[scale=0.5]{images/plots/3.11new}}
%   \caption{Расход регенерированного урана на единицу НОУ-продукта  при различной концентрации $^{232}$U в питающем потоке каскада для различных концентраций $^{235}$U в потоке отвала. Обозначения: см. - смесь природного урана и регенерата, подаваемая на питание каскада}\label{sc3_1.second}
% \end{figure}

\subsection{Аналитический подход к оценке возможности использования модификаций ординарного каскада для обогащения регенерированного урана в условиях многократного рецикла}

Описанные выше результаты вычислительных экспериментов, проведенных для анализа применимости схем на основе простейших модификаций ординарного каскада для решения сформулированной в главе \ref{ch1} задачи обогащения регенерированного урана, показали что такие схемы не могут решить подобную задачу в условиях  многократного рецикла. Это обусловлено ухудшением изотопного состава урана по мере прохождения им серии топливных циклов, что выражается в накоплении $^{232}$U и других чётных изотопов. При этом исходная концентрация $^{232}$U питающей смеси, начиная со второго рецикла, превышает уровень допустимый в конечном продукте, поэтому схемы, основанные на ординарном каскаде, которые только разбавляют этот изотоп, не эффективны для решения поставленной задачи. Тем не менее, если рассматривать подобные схемы для обогащения регенерированного урана, прошедшего только однократное облучение или допустить <<послабление>> ограничений на концентрации чётных изотопов, то подобные схемы, безусловно, могут быть применены для решения задачи обогащения регенерированного урана.

При этом закономерно возникает следующий вопрос: возможно ли априорно оценить возможность решить поставленную выше (\ref{ch2_stat}) задачу в таких модификациях ординарного каскада? 

На этот вопрос можно ответить, обратившись к уравнениям баланса компонентов в каскаде \ref{GrindEQ__1_21_} по крайней мере в случае вариантов каскадных схем, где регенерированный уран поступает в каскад для обогащения. Если записать уравнение \ref{GrindEQ__1_21_} для изотопа $^{232}$U и, учитывая, его малую концентрации в исходной смеси сделать предположение о том, что его концентрация в отвале каскада будет стремиться к нулю. Данное предположение может быть вполне оправдано, если отвальная часть каскада имеет достаточное число ступеней. В этом случае $^{232}$U, являясь самым лёгким в смеси регенерированного урана, будет активнее остальных компонентов концентрироваться в отборе каскада. Это означает, что для изотопа $^{232}$U уравнение \ref{GrindEQ__1_21_} можно переписать в следующем виде, пренебрегая слагаемым с потоком отвала каскада:

\begin{equation} \label{GrindEQ__1_21__} 
  \begin{array}{l} {\quad \quad \quad \quad \quad  E+F=P+W,} \\ {FC_{i,F} + EC_{i,E} =PC_{i,P} +WC_{i,W} ,\;  i=1,2,...,m.} \end{array} 
\end{equation} 

% \begin{equation}
% \label{eq_232_balance}
%   C_{232,P} \approx \frac{RepU}{P} C_{232,RepU}
% \end{equation}

\begin{equation}
  \label{eq_232_balance_}
    C_{232,P} \approx \frac{E}{P} C_{232,E}
  \end{equation}

Величина $\frac{E}{P}$ в приведенном выше уравнении и является отношением (исходный регенерат)/продукт. Если учесть, что типичные значения этого отношения составляют величину $\approx$0,9-0,95, то станет очевидно, что это условие будет выполнено только, если концентрация $^{232}$U в исходном регенерате ниже, чем ограничение на $^{232}$U в конечном продукте. 
С помощью уравнения \ref{eq_232_balance_} можно вычислить максимально возможную долю питающего потока, содержащего $^{232}$U, как неизвестную переменную уравнения \ref{eq_232_balance_}. Например, для состава 1 таблицы \ref{is_compositions_2_5}, который был использован в рассмотренных выше примерах получаем:

% \begin{equation}
%   \label{eq_232_balance_X}
%     5 \times 10^{-7} \% \approx X \times 6.622 \times 10^{-7} \% \Rightarrow X \approx 0.755
% \end{equation}

% \begin{equation}
%   \label{eq_232_balance_X}
%     \frac{RepU}{P} \leq 0,755
% \end{equation}

\begin{equation}
  \label{eq_232_balance_X_}
    \frac{E}{P} \leq 0,755
\end{equation}

% \begin{equation}
%   \label{eq_232_balance_X_}
%     \frac{E}{P} \leq 0,5
% \end{equation}

Снова анализируя представленные в предыдущих разделах данные, легко увидеть, что полученные в результате прямого численного расчёта предельные величины отношений (исходный регенерат)/продукт приблизительно и составляют такую величину.

Интересно, что используя уравнение \ref{eq_232_balance_}, легко оценить максимальное содержание $^{232}$U в исходном регенерате, при котором еще будет возможно решить задачу обогащения регенерата. Это можно сделать, подставив в \ref{eq_232_balance_} величины $C_{232,P}$ и $\frac{E}{P}$ и вычислив $C_{232,E}$. Например, для $\frac{E}{P}=0,93$ и $C_{232,P}=5\cdot10^{-7}\%$ получаем величину $\approx 5,37\cdot10^{-7}\%$, оказывается меньше, чем в составах №1 и №2 таблицы \ref{is_compositions_2_5}.

Описанный выше подход позволяет аналитически оценить возможность применения схем на основе простейших модификаций ординарного каскада, исходя из изотопного состава регенерата. Следует отметить также, что подобные оценки можно также применять и для каскадных схем, в которых регенерат разбавляют уже внутри каскада, путём его подачи в качестве дополнительного питания, поскольку такие схемы по сути являются также только разбавляющими.

\subsection{Общий вывод для схем возврата регенерата в ЯТЦ на основе ординарного каскада}\label{sec:ch2/sec2}

Обобщая описанные выше результаты вычислительных экспериментов, проведенных для различных модификаций ординарного каскада для обогащения и разбавления регенерированного урана, можно сделать следующие основные выводы:
\begin{enumerate}
  \item Представленные на рисунке \ref{fig:diagram1ch3} варианты каскадных схем принципиально не решают задачу обогащения регенерированного урана при одновременном выполнении условий на концентрации четных изотопов в товарном НОУ и обеспечения расходования заданной массы регенерата на получение этого НОУ для составов регенерата с исходным высоким содержанием четных изотопов. Например, для концентраций $^{232}$U, исходно превышающих предельные значения для товарного НОУ. 
  \item Основная причина невозможности решения задачи состоит в том, что в рассматриваемых схемах число свободных параметров оказывается меньшим, чем число условий, которые необходимо одновременно удовлетворить. В результате такие схемы могут обеспечить решение задачи только в некоторых частных случаях, например, когда в обогащение поступает регенерированный уран с относительно низкими исходными концентрациями четных изотопов, что может соответствовать обогащению регенерата первого рецикла.
  \item Продемонстрирован аналитический подход к оценке возможности применения схем на основе ординарного каскада для решения задачи, основанный на анализе исходного изотопного состава поступающего в обогащение регенерированного урана.
  \item Полученные результаты однозначно свидетельствует о том, что для обогащения регенерированного урана в условиях многократного рецикла необходимо использование более сложных вариантов каскадных схем, которые смогут позволить полностью решить поставленную задачу обогащения регенерата безотносительно к его исходному составу.
\end{enumerate}

\section{Обоснование необходимости составных схем}\label{sec:ch2/sec2}

Как следует из описанных выше результатов, на текущий момент в принципе имеются способы, позволяющие обеспечить выполнение требований по четным изотопам урана при обогащении регенерата. Однако основной проблемой, решаемой в рамках настоящей диссертационной работы, является поиск варианта каскадной схемы, позволяющей одновременно выполнить ограничения по концентрациям четных изотопов и задействовать в обогащении весь имеющийся регенерат в условиях неопределенности его изотопного состава при многократном рецикле.

Если анализировать причины невозможности возврата массы регенерата в производство топлива в различных модификациях ординарного каскада для обогащения регенерата в условиях многократного рецикла, то становится очевидным, что это, во многом, связано с нарастанием относительных концентраций <<легких>> изотопов (в первую очередь $^{232}$U) и $^{235}$U. Поскольку данные изотопы концентрируются вместе на легком <<конце>> каскада, то единственным способом понизить отношение их концентраций -- это разбавить материалом, не содержащим $^{232}$U, $^{236}$U. Как показали результаты, описанных в этой главе вычислительных экспериментов, для составов с относительно высоким исходным содержанием $^{232}$U невозможно подобрать такой разбавитель, чтобы удовлетворить одновременно и условие полного возврата массы регенерата в цикл и ограничения на содержание четных изотопов.

Из приведенного выше анализа следует, что эффективная каскадная схема для обогащения регенерата урана при многократном рецикле должна обеспечивать не только разбавление регенерата, но и хотя бы частичную его очистку от чётных изотопов. Поэтому возможные варианты решения задачи, по-видимому, должны быть основаны на использовании схем двойных каскадов, в том числе, описанных в Главе \ref{ch1}. В связи с этим представляет интерес оценка возможности прямого обогащения регенерата с повышенным содержанием чётных изотопов в двойном каскаде с целью решения задачи обогащения регенерата в наиболее общей постановке. Этот вопрос и рассмотрен в следующих разделах настоящей главы.



\section{Двойной каскад}\label{sec:ch2/dvoynoy}

Возможностью осуществить решение является двойной каскад, представленный в обзоре Гл. \ref{ch1/dvoynoy}, представляющий собой последовательное соединение двух каскадов (рис. \ref{fig:double_ru}). 

\begin{figure}[ht]
  \centerfloat{\includegraphics[scale=1.0]{cascades/Double_core}}
  \caption{Двойной каскад. Обозначения: $E$ -- поток питающего схему регенерата, $W_1$ -- поток отвального ОГФУ тяжелого конца каскада; $P$ -- конечный НОУ продукт на основе регенерата; $P_2$ -- отход двойного каскада в виде высокообогащенного урана; $F_0$ -- природный уран; $P_0$ -- дополнительно производимый НОУ продукт для возможности загрузить активную зону реактора}\label{fig:double_ru}
\end{figure}

Анализируя уравнения, описывающие систему из двух <<квазиидеальных>> каскадов, а также учитывая сформулированную выше постановку задачи, получаем 3 неизвестных переменные: 1) $C_{235, P_1}$; 2) $C_{235, P_2}$; 3) $C_{235, P}$; и два независимых уравнения, характеризующих целевую систему:

\begin{enumerate}
    \item $\delta_{1}=C_{235,P\textit{ экв.}}-(C_{235,P\textit{ NU}}+\Delta C_{235})$ -- невязка по концентрации $^{235}$U в конечном НОУ-продукте, с учетом поправки на присутствие изотопа $^{236}$U. Величина $\delta_{1}$ фактически определяет точность достижения условия компенсации $^{236}$U;
    \item $\delta_{2}=C_{232,P\textit{ расч.}}-C_{232,P\textit{ треб.}}$ -- разница между рассчитанным значением концентрации $^{232}$U в конечном НОУ-продукте и заданным ограничением для концентрации этого изотопа.
\end{enumerate}

Таким образом, для нахождения 3-х переменных, имея 2 уравнения, получаем 1 свободную переменную, которую следует рассматривать в в качестве оптимизационной. В качестве такой переменной была выбрана $C_{235, P_2}$, которая затем была подобрана в ходе оптимизационной процедуры на критерий затрат работы разделения. Отметим также, что дополнительно для нахождения оптимального распределения потока по ступеням (для подбора формы каскада, соответствующей наименьшему суммарному потоку) было осуществлено варьирование величин $g_{i}$, которое было организовано перебором возможных опорных компонент $M_{k1}$ и $M_{k2}$ для ординарных каскадов 1 и 2, входящих в схему.


В качестве ключевых оцениваемых характеристик будем опираться на те же интегральные характеристики, отражающие экономику разделительного процесса, что использовались для анализа схем на основе ординарного каскада:

\begin{enumerate}
  \item $\delta(\frac{\Delta A}{P})=1-\frac{\Delta A}{P}/\frac{\Delta A}{P}_{(ord.)}$ -- экономия работы разделения относительно референтной схемы трехпоточного каскада для обогащения природного урана (см. Приложение). Наибольшая экономия соответствует минимуму суммарного потока схемы \ref{GrindEQ__1_73_}. Если величина отрицательная, абсолютное значение соответствует потерям работы разделения, по сравнению с референтной схемы трехпоточного каскада для обогащения природного урана;
  \item $\delta(\frac{F_{NU}}{P})=1-\frac{F_{NU}}{P}/\frac{F_{NU}}{P}_{(ord.)}$ -- экономия природного урана относительно референтной схемы трехпоточного каскада для обогащения природного урана (см. Приложение).  Наибольшая экономия соответствует минимуму удельного расхода природного урана схемы. Если величина отрицательна, абсолютное значение соответствует перерасходу природного урана, по сравнению с референтной схемы трехпоточного каскада для обогащения природного урана.
\end{enumerate}

, а также на такой показатель как максимум степени извлечения из регенерата (\ref{RecR2}):

\begin{equation} \label{RecR2} 
    Y_{E} = \frac{P\cdot C_{235,P}}{E \cdot C_{235,E}}        
\end{equation} 

Ниже приведены результаты анализа применимости схемы двойного каскада для возврата регенерата в ЯТЦ (табл. \ref{pure_double2and5}).\\


\begin{table}[ht]
  \centering
  \begin{tabular}{|c|c|c|}
  \hline \diagbox{Параметр}{Состав р-та №} & 1 & 2\\ \hline
  $\delta(\frac{\Delta A}{P}), \%$ & $4.19$ & $-8.35$\\ \hline
  $\delta(\frac{F_{NU}}{P}), \%$ & $19.76$ & $0.083$\\ \hline
  \hline $Y_{E}, \%$ & $89.19$ & $90.84$\\ \hline
  $\frac{E}{P}$       & $4.71$ & $11.2$\\ \hline
  $EPP$  & $9.315$ & $23.71$\\ \hline
  \hline $C_{232,P}\cdot10^{7}, \%$ & $5.0$ & $5.0$\\ \hline
  $C_{234,\text{P}}, \%$  & $0.117$ & $0.190$\\ \hline
  $C_{235,\text{P}}, \%$  & $5.996$ & $7.715$\\ \hline
  $C_{236,\text{P}}, \%$  & $3.609$ & $9.541$\\ \hline
  $C_{235,P_{1}, \%}$       & $6.32$ & $10.18$\\ \hline
  $C_{235,P_{2}, \%}$       & $77.92$ & $43.12$\\ \hline
\end{tabular}
\caption{Параметры схемы двойного каскада для возврата регенерата в рецикл.{\label{pure_double2and5}}}
\end{table}

Исходя из анализа результатов, приведенных в табл. \ref{pure_double2and5}, схема двойного каскада позволяет возможность решить поставленную выше (\ref{ch2_stat}) задачу, однако чтобы достичь возврата регенерата в требуемом отношении, требуется производить дополнительное количество НОУ-продукта из отдельных сырьевых материалов -- природного урана или его производных. Иными словами, схема двойного каскада позволяет израсходовать весь регенерированный уран на производство конечного НОУ-продукта, однако изотопа $^{235}$U недостаточно для воспроизводства топлива для повторной загрузки активной зоны реактора (АЗ), поэтому для того чтобы произвести необходимое количество топлива для загрузки АЗ, необходимо дополнять НОУ, произведенный на основе регенерата, низкообогащенным ураном, произведенным из дополнительных источников.  А так как на 1 кг конечного НОУ-продукта требуется расходовать 4,71 кг и 11,2 кг регенерата для составов 1 и 2, соответственно, получаемый продукт необходимо дополнять $4.71/0.93 - 1=4.06$кг и $11.2/0.93 - 1=11.04$кг НОУ-продукта, произведенными из прочих источников. Таким образом, для схемы двойного каскада, приведенные в табл. \ref{pure_double2and5} интегральные показатели экономии природного урана и работы разделения расчитываются для всей массы производимого НОУ-продукта, производимого для загрузки в активную зону реактора.

Несмотря на формальное удовлетворение заданных условий задачи возврата регенерата (постановки, приведенной в \ref{ch2_stat}), такая схема производит НОУ с высоким содержанием $^{236}$U, превышающим 3,5\% для состава 1, и с превышающей концентрацию $^{235}$U для состава 2. Такое обстоятельство негативно характеризует эффективность схемы двойного каскада по двум причинам:

\begin{itemize}
  \item во-первых, более высокое содержание $^{236}$U в НОУ-продукте обуславливает необходимость обеспечивать более существенную добавку $^{235}$U для компенсации вносимого 236-м в ядерный реактор паразитного нейтронного поглощения;
  \item во-вторых, присутствие в материале $^{236}$U ускоряет накопление изотопа $^{232}$U, так как $^{236}$U являясь предшественником $^{232}$U в цепочке ядерных превращений \cite{smirnovEvolutionIsotopicComposition2012}). А так как ограничение на содержание $^{232}$U в свежем ядерном топливе -- является основным препятствием для использовании регенерата в производстве низкообогащенного урана, такое обстоятельство негативно сказывается на пригодности такого матерала для многократного рецикла.
\end{itemize}

Отсюда вытекает необходимость модификации двойного каскада, которая позволит нивелировать недостаток схемы, состоящий в высокой концентрации $^{236}$U в конечном НОУ-продукте, развивая принципиальную возможность получить с помощью такой схемы решение поставленной (в \ref{ch2_stat}) задачи. 

Заметим также, что с использованием схемы двойного каскада для обоих составов получение продукта заданных качеств сопряжено с превышением допустимого ограничения на производство ВОУ (20\% содержания $^{235}$U в потоке $P_{2}$).





% Примечание (черновик)

% Для состава 1:

% 4.71/0.93=5.06

% $\delta(\frac{\Delta A}{P}), \%$ = 21.2/5.06=4.19

% $\delta(\frac{F_{NU}}{P})$=1/5.06=0.1976


% Для состава 2:

% 11.2/0.93=12.04

% $\delta(\frac{\Delta A}{P}), \%$ = -100.58/12.04=-8.35

% $\delta(\frac{F_{NU}}{P})$=1/12.04=0.083


\clearpage
