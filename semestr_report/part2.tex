\chapter{Анализ физических ограничений для решения задачи в ординарных каскадах}\label{ch:ch2}

Как уже было отмечено в главе 1, лишь некоторые из предложенных к настоящему моменту способов обогащения регенерированного урана потенциально способны решить задачу обогащения регенерата произвольного исходного состава в условиях одновременного выполнения ограничений на концентрации сразу нескольких изотопов и при заданной пропорции между продуктом и исходной смесью. В первую очередь, это касается разбавляющих схем на основе ординарного каскада. 
Однако проведенный ранее теоретический анализ не позволяет однозначно утверждать при каких условиях может быть применена та или иная схема. В рамках настоящей главы кратко представлены результаты серии вычислительных экспериментов и сопутствующего им теоретического анализа, направленных на то, чтобы выявить область возможного применения одиночных каскадных схем с разбавлением для получения обогащенного регенерированного урана в условиях многократного рецикла. 

\section{Постановка задачи и методическая часть}

Для изучения возможностей разбавляющих схем на основе ординарного каскада были рассмотрены случаи обогащения регенерированного урана с различных исходным содержанием чётных изотопов (см. \ref{is_compositions_2_5}). Выбранные составы соответствуют составам регенерированного урана, выделенных из ОЯТ реакторов ВВЭР-1000 и -1200 при различных внешних условиях \cite{palkinDesignanalyticalResearchRefinement2010,nevinicaToplivnyyCiklLegkovodnogo2019}. Отметим, что оба состава характеризуются довольно высоким содержанием $^{232}$U. Выбор подобных загрязненных четными изотопами составов регенерата иммитирует сложности, которые могут возникать при обогащении регенерированного урана в условиях многократного рецикла.  
При реализации вычислительных экспериментов общая постановка задачи соответствовала формулировке, приведенной в Главе 1. С учётом конкретных выбранных ограничений задачу можно сформулировать следующим образом.

Из заданной массы исходного регенерированного урана необходимо получить заданную массу товарного НОУ, отвечающего следующим требованиям:

\begin{enumerate}
  \item Концентрация в конечном продукте составляет 4.95\%, значение характерно для современных легководных реакторов \cite{solovevaCennostiOYaTKak2019}.
  \item Расход регенерированного урана на единицу конечного продукта в виде низкообогащенного урана: 0,93 кг на 1 кг НОУ \cite{smirnovApplyingEnrichmentCapacities2018}.
  \item Концентрация $^{235}$U в потоке отвала задана равной 0.1\% \cite{smirnovEvolutionIsotopicComposition2012};
  \item Соотношение $^{234}$U к $^{235}$U не должно превышать 0.02.
  \item Влияние изотопа $^{236}$U на нейтронно-физические характеристики топлива должно быть скомпенсировано дополнительным обогащение по $^{235}$U, для расчёта которого коэффициент компенсации реактивности принят равным 0.29 \cite{smirnovApplyingEnrichmentCapacities2018}.
  \item Концентрация $^{232}$U ограничена величиной $5\cdot10^{-7}$\% \cite{smirnovApplyingEnrichmentCapacities2018}.
\end{enumerate}

В качестве схем обогащения регенерированного урана рассмотрены схемы, представленные на рисунке \ref{fig:diagram1}.
Каждая из них подразумевает использование ординарного каскада. Для моделирования процессо обогащения урана в каскаде использовали модель R-каскада \cite{sulaberidzeTeoriyaKaskadovDlya2011}. При этом во всех случаях при расчёте параметров каскада задавали концентрации $^{235}$U в его внешних выходящих потоках. Под расчётом параметров такого каскада подразумевали следующую задачу. 
Задано: состав обогащаемой смеси; параметры одиночного разделительного аппарата; величина одного из внешних потоков каскада, например, потока отбора или питания; концентрации $^{235}$U в потоках отбора и отвала каскада; в случае расчёта на основе модели R-каскада задают также пару компонентов, по относительным концентрациям которых выполняется условие несмешивания.
В процессе расчёта необходимо определить следующие параметры: число ступеней в каскаде и номер ступени подачи внешнего питания, концентрации всех компонентов (кроме $^{235}$U) в выходящих из каскада потоках, величины неизвестных внешних потоков каскада, распределения потока и концентраций компонентов по ступеням каскада и все остальные внутренние параметры. 
Такая постановка задачи требует численного решения системы нелинейных уравнений, возникающих для невязок концентраций $^{235}$U в выходящих потоках. Указанная система может быть записана в следующем виде: 

$\Delta_{P} = {(C_{235, P})}_{calc}-{(C_{235, P})}_{given}$

$\Delta_{W} = {(C_{235, W})}_{calc}-{(C_{235, W})}_{given}$

где ${(C_{235, P})}_{calc}$, ${(C_{235, W})}_{calc}$ - рассчитанные концентрации $^{235}$U в потоках отбора и отвала каскада, соответственно; ${(C_{235, P})}_{given}$, ${(C_{235, W})}_{given}$ - заданные концентрации $^{235}$U в потоках отбора и отвала каскада, соответственно; $\Delta_{P}$, $\Delta_{W}$ -- невязки по концентрациям $^{235}$U в потоках отбора и отвала каскада, соответственно. 

Для получения уравнений на невязки использованы соотношения (\ref{GrindEQ__1_72_}), (\ref{GrindEQ__1_73_}). Из решения подобной системы можно определить величины $R_{n k}^{W}$ и $R_{n k}^{P}$, после чего аналитически рассчитать остальные внешние параметры R-каскада по соотношениям (\ref{GrindEQ__1_70_})-(\ref{GrindEQ__1_77_}).   



\begin{table}[h]
  \centering
  \normalsize\begin{tabulary}{1.0\textwidth}{CCCCCCC}
  Цикл № & Массовое число & 232 & 233 & 234 & 235 & 236 \\
  2 & C, \% & 6.62e-7 & 1.19e-6 &    3.28e-2 & 1.43 & 0.9932 \\
   &  &  &  &  &  &  \\
  5 & C, \% &  1.03e-6 &   1.3e-6 &  3.91e-2 & 1.07 & 1.45 \\
   &  &  &  &  &  &  \\
  \end{tabulary}
  \caption{{Изотопные составы регенерата различных циклов.{\label{is_compositions_2_5}}}}
\end{table}


Ниже представлены результаты моделирования обогащения регенерата в каждой из трёх схем, представленных на рисунке \ref{fig:diagram1}. 
Основная цель проведенного моделирования - оценить возможность использования схем на основе ординарного каскада для обогащения регенерированного урана с повышенным содержанием чётных изотопов и, в первую очередь, $^{232}$U. Во всех случаях в качестве обогащаемого состава был рассмотрен состав, соответствующий второму рециклу и, как следствие, имеющий более низкое содержание чётных изотопов, по отношению к составу пятого рецикла.

\subsection{Схема с разбавлением предварительно обогащенного регенерата}

Рассмотрим каскадную схему, в которой регенерат сначала обогащают до уровня, превышающего требуемую концентрацию $^{235}$U, а затем разбавляют, например, природным ураном (рис. \ref{o1}). Такая схема позволяет обогатить регенерат до требуемого условием задачи содержания изотопа $^{235}$U в конечном продукте (НОУ), а также выполнить ограничения на $^{232}$U. При этом необходимо проверить соблюдение и остальных условий решаемой задачи обогащения регенерата. Заметим, что данная схема также подразумевает возможность использования в качестве разбавителя любой другой урановой смеси, не содержащей изотопов $^{232}$U и $^{236}$U. В качестве одного из вариантов разбавителей может быть использован и низкообогащенный уран, полученный обогащением природного урана.

\begin{figure}[ht]
  \centerfloat{\includegraphics[scale=0.2]{cascades/ordinary/1}}
  \caption{Схема разбавления предварительно обогащенного регенерата природным ураном или низкообогащенным ураном. Обозначения: $P_0$ -- поток отбора легкой фракции каскада; $W_0$ -- поток отвального ОГФУ тяжелого конца каскада; $CM.$ -- узел смешения, на выходе из которого получается конечный продукт $НОУ$ -- низкообогащенный уран}\label{o1}
\end{figure}

Для получения обогащенного урана, удовлетворяющего всем требованиям, необходимо определить величину концентрации $^{235}$U в потоке $P_0$ и пропорцию смешивания потоков $P_0$ и разбавителя. Указанные параметры определяли итерационно по следующей схеме. Сначала задавали начальное приближение для концентрации $^{235}$U в потоке $P_0$, после чего рассчитывали параметры каскада по описанной в предыдущем разделе процедуре. Далее, зная состав смеси урана в потоке $P_0$ на основе простейшей пропорции определяли соотношение между потоками природного урана или обогащенного регенерата для получения финального продукта. При этом полученный в результате смешивания поток товарного НОУ должен отвечать ограничениям по концентрациям чётных изотопов, а отношение массы полученного НОУ к массе исходного регенерата должно соответствовать заданной величине. Это означает, что для успешного решения задачи одновременно должны быть выполнены условия на концентрации изотопов $^{232,234,235,236}$U и обеспечена заданная пропорция между расходом регенерата и конечным продуктом. Однако при известной и заданной концентрации $^{235}$U в потоке разбавителя управляющих параметров в такой схеме только два: концентрация $^{235}$U в потоке $P_0$ и пропорция смешивания обогащенного регенерата и разбавителя. Очевидно, что в этом случае не для любых исходных данных возможно подобрать требуемые параметры каскадной схемы. 
Для иллюстрации сложности решения подобной задачи рассмотрим некоторые вспомогательные функции $\delta_1$ и $\delta_2$, которые определены следующим образом:

\begin{equation} \label{d1} 
  \delta_1=\left[C_{235}^P-\left(C_n^P+KKP\times C_{236}^P\right)\right]
\end{equation} 

\begin{equation} \label{d2} 
    \delta_2=\left[C_{232}^P-5\times10^{-7}\right],             
\end{equation}

где ККР -- коэффициент компенсации реактивности.

По своему физическому смыслу величина $\delta_1$ представляет собой отклонение концентрации изотопа $^{235}$U (выраженное в долях) в конечном продукте (после смешивания) от заданной величины, с учетом компенсации $^{236}$U. Величина $\delta_2$ представляет собой разность фактической концентрации $^{232}$U в окончательном продукте и требуемой величины в соответствии с принятым ограничением. Из таких определений очевидно, что в случае получения продукта, отвечающего необходимым условиям обе функции должны быть равны 0, причём при одних и тех же параметрах схемы. Фактически, функции $\delta_1$ и $\delta_2$ задают для рассматриваемой каскадной схемы систему уравнений, из которой можно найти её параметры при заданной отношение потоков регенерата и продукта. Оговоримся, что строго говоря к этой системе надо добавить аналогичное условие на концентрацию $^{234}$U, однако как будет показано ниже, даже в таком более простом варианте каскадная схема не может обеспечить одновременное выполнение всех требований, предъявляемых к составу конечного продукта.   
В рамках работы были проведены вычислительные эксперименты, в которых варьировали концентрацию $^{235}$U в потоке $P_0$, а пропорцию между $P_0$ и разбавителем из природного урана варьировали в диапазоне от 1 до 20. При этом для большинства полученных вариантов отношение (исходный регенерат)/конечный продукт не было равным 0,93. Концентрацию $^{235}$U в потоке $W_0$ задавали равной 0,1\%. При расчёте параметров ординарного каскада во всех случаях предполагали, что в каскаде было реализовано несмешивание по относительной концентрации компонентов $^{235}UF_6$ и $^{236}UF_6$. Такое условие было выбрано на основе серии предварительных расчётов. Величину коэффициента разделения для компонентов  $^{235}UF_6$ к $^{238}UF_6$ приняли равной 1.2  \cite{smirnovEvolutionIsotopicComposition2012}. 
На рис. \ref{delta1}--\ref{delta4} представлены зависимости величин $\delta_1$ и $\delta_2$ от соотношения смешиваемых потоков, взятых для различных значений концентрации $^{235}$U (рис. \ref{delta1}--\ref{delta4}) в потоке $P_0$. Чтобы сопоставить указанные величины на одном рисунке, величина $\delta_2$ была взята с поправкой (умножена на специально подобранный числовой коэффициент, который был равен $10^{-5}$).
Как видно из рисунков \ref{delta1}-\ref{delta4} функции $\delta_1$ и $\delta_2$ имеют "нули" при различных значениях аргумента. Таким образом, полученные результаты показывают невозможность одновременного удовлетворения условия компенсации $^{236}$U и выполнения заданного ограничения по $^{232}$U в получаемом товарном НОУ. По крайней мере это справедливо для рассмотренного изотопного состава. Дополнительные расчёты для состава пятого рецикла подтвердили те же выводы. Следовательно, данную схему нельзя рассматривать в качестве способа обогащения регенерированного урана в условиях его многократного рецикла. .


\begin{figure}[ht]
  \begin{minipage}{.5\textwidth}
    \centering
    % include first image
    \includegraphics[width=.8\linewidth]{images/plots/15}  
    \caption{Концентрация $^{235}$U в предварительно обогащенном регенерата равна 15\%}
    \label{delta1}
  \end{minipage}
  \begin{minipage}{.5\textwidth}
    \centering
    % include second image
    \includegraphics[width=.8\linewidth]{images/plots/30}  
    \caption{Концентрация $^{235}$U в предварительно обогащенном регенерата равна 30\%}
    \label{delta2}
  \end{minipage}
  \begin{minipage}{.5\textwidth}
    \centering
    % include second image
    \includegraphics[width=.8\linewidth]{images/plots/50}  
    \caption{Концентрация $^{235}$U в предварительно обогащенном регенерата равна 50\%}
    \label{delta3}
  \end{minipage}
  \begin{minipage}{.5\textwidth}
    \centering
    % include second image
    \includegraphics[width=.8\linewidth]{images/plots/65}  
    \caption{Концентрация $^{235}$U в предварительно обогащенном регенерата равна 65\%}
    \label{delta4}
  \end{minipage}
  % \caption{Невязки для $^{235}$U и $^{232}$U. Обозначения: $\frac{NatU}{P_{0}}$ -- отношение доли природного урана к потоку обогащенного регенерата.}
  % \label{fig:deltas_ordinar}
 \end{figure}


\subsection{Схема с разбавлением предварительно обогащенного регенерата низкообогащенным ураном}

Если в схеме, рассмотренной выше (рис. \ref{o1}), заменить разбавитель предварительно обогащенного регенерата (рис. \ref{o1}) с природного урана на низкообогащенный уран, не содержащий четных изотопов (например, изготовленный из природного урана), для данного состава можно найти решение, когда одновременно выполнены условия равенства нулю обеих невязок ($\delta_1$ и $\delta_2$). Данная схема может быть рассмотрена в качестве модификации варианта, описанного в предыдущем разделе. Нахождение решения для такой схемы обусловлено появлением дополнительного управляющего параметра -- концентрации $^{235}$U в потоке $НОУ$-разбавителя.  Несмотря на это, в подобном варианте каскадной схемы может быть не выполнено условие максимального использования регенерированного урана, состояющее в равенстве отношения потоков исходного регенерата и товарного НОУ заданной величине.

Чтобы оценить возможность решения задачи в рассматриваемой каскадной схеме проведены вычислительные эксперименты, в рамках которых варьировали величины область допустимых значений параметров схемы, а также определить при каких параметрах схема наиболее эффективна, исследованы диапазоны изменения следующих параметров схемы: $^{235}$U в концах каскада: $P_0$ и $W_0$ (рис. \ref{o1}). Это поможет проиллюстрировать как меняются доли природного урана и регенерата в конечном продукте и как меняются затраты работы разделения, за счет изменения длины каскада. Это позволит исследовать возможность достижения заданной пропорции возврата регенерата. 

\begin{figure}[ht]
  \centerfloat{\includegraphics[scale=0.5]{images/plots/sc2_2}}
  \caption{Расход природного урана на единицу НОУ-продукта  для различных концентраций $^{235}$U в потоках продукта и отвала каскада, обогащающего регенерат}\label{fig:sc2_2}
\end{figure}

Рис. \ref{fig:sc2_2} иллюстрирует зависимость расхода природного урана (NatU) на единицу НОУ-продукта (LEU Product) от концентрации $^{235}$U в обогащенном регенерате ($P_0$ на рис. \ref{o1}) для разных значений концентрации $^{235}$U в отвале каскада, получаемом из регенерата. Значения уровня расхода природного урана на единицу НОУ-продукта на уровне $\approx$6,5-6,7 для всего исследуемого диапазона параметров, меньше, чем расход природного урана в схеме ординарного каскада ($\approx$7,93 на ед.продукта, смотри Приложение), когда он используется без добавки из регенерата для производства свежего НОУ топлива, аналогичного по исходным требованиям к продукту. Для значений концентрации $^{235}$U в $W_0$ на уровнях 0,05\% и 0,15\%, значения расхода природного урана в схеме ординарного каскада для обогащения природного урана, будут составлять $\approx$7,41 и $\approx$8,55, соответственно. Таким образом, при заданном интервале параметров, использование схемы с разбавлением обогащенного регенерата позволяет обеспечить экономию природного урана на уровне 15--28\%, что говорит о целесообразности использования регенерата в производстве НОУ.

Заметим, что по $^{235}$U регенерат обогащается до значений (рис.  \ref{fig:sc2_2}), превышающих пороговое значение для НОУ в 20\%, что может быть неприемлемым в виду требований соблюдения условий нераспространения ядерных материалов \cite{brownOriginsSignificanceLimit2016}. Именно эти значения соответствуют наилучшим показателям экономии природного урана, хотя переход 20\%-й границы для $^{235}$U, дает прирост экономии природного урана менее 1\%. Более существенное улучшение экономии (>3\%), как видно на графике \ref{fig:sc2_2}, происходит с ростом уровня извлечения $^{235}$U из регенерата при понижении концентрации $^{235}$U в обедняемом потоке регенерата $W_0$. 
% Эффект экономии природного урана от понижения $^{235}$U в $W_0$, а также от повышения $^{235}$U в $P_0$, обсуловлен возможностью использовать НОУ-разбавитель с более низким уровнем $^{235}$U, что проиллюстрировано на рис.\ref{fig:sc2_LEU_D}.

Уменьшение концентрации $^{235}$U в обедняемом потоке регенерата $W_0$ связано с еще одним преимуществом: более высокий уровень обеднения исходной смеси ($^{235}$U в $W_0$ при концентрации 0,05\%), дает существенный вклад в экономию работы разделения, что показано на рис.\ref{Figure_13}, за счет более эффективного извлечения $^{235}$U из регенерата. Область отрицательных значений потерь работы разделения (SW loss) на рис. \ref{Figure_13} соответствует ее экономии. Слияние кривых, соответствующих различным концентрациям $^{235}$U в обогащенном регенерате, обусловлено эквивалентностью масс $^{235}$U в каждом из этих потоков, что отражает постоянство вклада обогащенного регенерата в формирование конечного продукта, и соответствует постоянному уровню извлечения $^{235}$U.

% \begin{figure}[ht]
%   \centerfloat{\includegraphics[scale=0.5]{images/plots/sc2_LEU_D}}
%   \caption{Концентрация $^{235}$U в разбавителе, необходимая для получения свежего НОУ для различных концентраций $^{235}$U в потоках продукта и отвала каскада, обогащающего регенерат}\label{fig:sc2_LEU_D}
% \end{figure}


\begin{figure}[ht]
  \centerfloat{\includegraphics[scale=0.5]{images/plots/Figure_13}}
  \caption{Потери работы разделения по отношению к ординарному каскаду для обогащения природного урана для различных концентраций $^{235}$U в потоках продукта и отвала каскада, обогащающего регенерат}\label{Figure_13}
\end{figure}

Анализ графика \ref{Figure_10} зависимости расхода регенерата на единицу НОУ-продукта ($\frac{NatU}{P_{0}}$) от концентрации $^{235}$U в $W_0$, показывает невозможность выполнить условия возврата заданной доли регенерата на единицу продукта при любых параметрах каскадной схемы ($^{235}$U в $W_0$ и в $P_0$). Причем пропорция используемого регенерата к НОУ-продукту меняется лишь незначительно концентрацией $^{235}$U в $W_0$, а удлинение обогащающей части каскада (рост $^{235}$U в $P_0$) не приводит росту вовлечения регенерата.

\begin{figure}[ht]
  \centerfloat{\includegraphics[scale=0.5]{images/plots/Figure_10}}
  \caption{Расход регенерата на единицу конечного НОУ-продукта для различных концентраций $^{235}$U в потоках продукта и отвала каскада, обогащающего регенерат}\label{Figure_10}
\end{figure}

% Проиллюстрировать затраты на работу разделения от составных  частей каскадной схемы, можно с помощью рис.\ref{myplot}, который показывает, что доля центрифуг для приготовления разбавителя из природного урана выше, чем доля центрифуг, задействованных для предварительного обогащения регенерата, и эта пропорция уменьшается с понижением содержания $^{235}$U в $W_0$.

% \begin{figure}[ht]
%   \centerfloat{\includegraphics[scale=0.5]{images/plots/myplot}}
%   \caption{Отношение количества центрифуг в каскаде, производящем разбавитель из природного урана, к количеству центрифуг, задействованных для предварительного обогащения регенерата, для различных концентраций $^{235}$U в потоках продукта и отвала каскада, обогащающего регенерат}\label{myplot}
% \end{figure}

На основе анализа рис.\ref{fig:sc2_2}-\ref{Figure_10}, можно заключить, что схема с разбавлением обогащенного регенерата смесью, не содержащей минорных изотопов, непригодна для решения задачи обогащения в условиях многократного рецикла, так как с помощью нее нет возможности использовать весь регенерированный уран на производство НОУ-продукта, как показано на рис. \ref{Figure_13}.

Однако, такую схему можно использовать для решения задачи повторного использования урана для возврата (дообогащения) регенерата на первом рецикле. Поэтому важно исследовать закономерности, характерные для схемы с разбавлением обогащенного регенерата.
Анализ рабочих диапазонов для схемы демонстрирует, что уменьшение концентрации $^{235}$U в $W_0$ позволяет достичь экономии природного урана на единицу продукта (рис. \ref{fig:sc2_2}), за счет понижения концентрации $^{235}$U в НОУ-разбавителе (рис. \ref{fig:sc2_LEU_D}), а также сэкономить работу разделения (рис. \ref{Figure_13}), при том что расход регенерата на единицу продукта будет меньше на пренебрежимо малую величину (рис. \ref{Figure_10}). Анализ графиков (рис.\ref{fig:sc2_2} и \ref{fig:sc2_LEU_D}) демонстрирует возможность с увеличением уровня обогащения регенерата получить рост экономии природного урана за счет меньшей необходимой концентрации $^{235}$U в НОУ-разбавителе. Однако, следует отметить, что превышение концентрации $^{235}$U уровня НОУ в 20\% может быть недопустимо на некоторых разделительных производствах, так как такой материал попадает в категорию высокообогащенного урана (ВОУ), на производство которого наложены ограничения \cite{gusevProliferationResistanceAnalysis2019}.


\subsection{Анализ схемы с разбавлением предварительно обогащенного природного урана регенератом}

Проанализируем возможность решения задачи обогащения регенерированного урана со всеми ограничениями в каскадной схеме с разбавлением предварительно обогащенного природного урана регенератом (рис. \ref{o2}). Принцип работы такой схемы состоит в том, что предварительно обогащенный природный уран смешивается с возвращаемым в топливный цикл регенерированным ураном. Уровень предварительного обогащения (перед смешением) природного урана и отношение потоков обогащенного природного урана к регенерату определяются исходя из условий задачи. Таким образом, данная схема в принципе аналогична схеме с разбавлением предварительно обогащенного регенерата природным ураном (рис. \ref{o1}). Это означает, что ей присущи те же проблемы, что упомянутой схеме. А именно, в ней число управляющих параметров меньше, чем число условий, предъявляемых к конечному продукту. 

\begin{figure}[ht]
  \centerfloat{\includegraphics[scale=0.2]{cascades/ordinary/2}}
  \caption{Схема каскада с разбавлением предварительно обогащенного природного урана регенератом. Обозначения: $P_0$ -- поток отбора легкой фракции каскада; $W_0$ -- поток отвального ОГФУ тяжелого <<конца>> каскада; $CM.$ -- узел смешения, на выходе из которого получается конечный НОУ-продукт $НОУ$  -- низкообогащенный уран}\label{o2}
\end{figure}

Для окончательного ответа на вопрос о возможности использования данной схемы для обогащения составов регенерата второго и пятого рециклов были проведены вычислительные эксперименты, в рамках которых варьировали величину концентрации $^{235}$U в обогащенном природном уране и пропорцию смешивания разбавителя и регенерата. Расчёты также проведены с использованием R-каскада, концентрацию $^{235}$U в отвале каскада также задавали равной 0,1\%.
Из результатов вычислительных экспериментов следует, что для рассматриваемой схемы возможно получение решения, удовлетворяющего заданным ограничениям на концентрации изотопов $^{232,234,236}$U. Однако, как и в случае применения схемы рис. \ref{o1}, одновременно с этими условиями не удаётся удовлетворить условие возврата заданной массы регенерата. В результате вместо заданной величины отношения массы исходного регенерата к продукту - 0,93, фактические значения не превысили величины 0,75. Данные результаты свидетельствуют о том, что такая схема обогащения регенерата не решает поставленную задачу для произвольного изотопного состава регенерата и, следовательно, не может быть применена в условиях многократного рецикла урана в топливе легководных реакторов.

\subsection{Анализ схемы с разбавлением регенерата природным ураном перед подачей в ординарный трехпоточный каскад}

Еще одним вариантом каскадной схемы для обогащения регенерированного урана, основанной на использовании ординарного каскада является схема, в которой смешивание и разбавление регенерата происходит непосредственно перед подачей в каскада для последующего обогащения. В качестве разбавителя здесь рассматривают природный уран. Данная схема изображена на рис. \ref{o3}. Пропорция смешения природного и регенерированного урана определяется предварительно, исходя из ограничений на четные изотопы в конечном НОУ-продукте.

\begin{figure}[ht]
  \centerfloat{\includegraphics[scale=0.25]{cascades/ordinary/3}}
  \caption{Схема каскада со смешением регенерата и природного урана перед подачей на питание ординарного каскада. Обозначения: $P_0$ -- поток отбора легкой фракции каскада; $W_0$ -- поток отвального ОГФУ тяжелого <<конца>> каскада; $CM.$ -- узел смешения входящих сырьевых потоков; $НОУ$ -- конечный НОУ-продукт схемы}\label{o3}
\end{figure}

Для такой схемы существует единственный управляющий параметр - это пропорция смешивания регенерата и природного урана. Очевидно, что с ростом доли регенерата в совокупном питании каскада будут возрастать концентрации чётных изотопов в потоке отбора каскада. Это обуславливает тот факт, что существует некоторое критическое значение пропорции, начиная с которого уже невозможно будет соблюсти, как минимум, ограничение на концентрацию изотопа $^{232}$U. Данное утверждение иллюстрирует рисунок \ref{sc3_1.second}, на котором показано взаимосвязь доли регенерата в питании каскада, концентрации $^{232}$U в конечном продукте и отношения потока исходного регенерата к потоку продукта. Кривые построены при различных концентрациях изотопа $^{235}$U в отвале каскада. Остальные параметры были такими же, как и в предыдущих примерах. В качестве обогащаемого состава регенерата рассмотрен регенерат рецикла 2. Как следует из анализа представленных зависимостей, во всех случаях величина концентрации  $^{232}$U достигает предельного значения ($5\cdot10^{-7}$\%) до того, как пропорция между исходным регенератом и продуктом достигнет требуемого значения 0,93. Это означает, что и эта схема, ни при каком наборе её параметров не позволяет решить в общем случае задачу обогащения регенерата. 

\begin{figure}[ht]
  \centerfloat{\includegraphics[scale=0.5]{images/plots/sc3_1.second}}
  \caption{Расход регенерированного урана на единицу НОУ-продукта  при различной концентрации $^{232}$U в питающем потоке каскада для различных концентраций $^{235}$U в потоке отвала}\label{sc3_1.second}
\end{figure}


\subsection{Общий вывод для схем возврата регенерата в ЯТЦ на основе ординарного каскада}

Описанные выше результаты вычислительных экспериментов, проведенных для анализа применимости схем на основе простейших модификаций ординарного каскада для решения сформулированной в главе 1 задачи обогащения регенерированного урана, показали что такие схемы не могут решить подобную задачу в условиях  многократного рецикла. Это обусловлено ухудшением изотопного состава урана по мере прохождения им серии топливных циклов, что выражается в накоплении $^{232}$U и других чётных изотопов. При этом исходная концентрация $^{232}$U питающей смеси, начиная со второго рецикла, превышает уровень допустимый в конечном продукте, поэтому схемы, основанные на ординарном каскаде, которые только разбавляют этот изотоп, не эффективны для решения поставленной задачи. Тем не менее, если рассмотривать подобные схемы для обогащения регенерированного урана, прошедшего только однократное облучение или допустить "послабление" ограничений на концентрации чётных изотопов, то подобные схемы, безусловно, могут быть применены для решения задачи обогащения регенерированного урана.

При этом закономерно возникает следующий вопрос: возможно ли априорно оценить способность рассматриваемой схемы решить поставленную задачу? 

На этот вопрос можно ответить, обратившись к уравнениям баланса компонентов в каскаде \ref{GrindEQ__1_21_} по крайней мере в случае вариантов каскадных схем, где регенерированный уран поступает в каскад для обогащения. Если записать уравнение \ref{GrindEQ__1_21_} для изотопа $^{232}$U и, учитывая, его малую концентрации в исходной смеси сделать предположение о том, что его концентрация в отвале каскада будет стремиться к нулю. Данное предположение может быть вполне оправдано, если отвальная часть каскада имеет достаточное число ступеней. В этом случае $^{232}$U, являясь самым лёгким в смеси регенерированного урана, будет активнее остальных компонентов концентрироваться в отборе каскада. Это означает, что для изотопа $^{232}$U уравнение \ref{GrindEQ__1_21_} можно переписать в следующем виде:

\begin{equation}
\label{eq_232_balance}
  C_{232_{U}}^{P} \approx \frac{F}{P} C_{232_{U}}^{F}
\end{equation}

Величина $\frac{F}{P}$ в привденном выше уравнении и является отношением (исходный регенерат)/продукт. Если учесть, что типичные значения этого отношения составляют величину $\approx$0,9-0,95, то станет очевидно, что это условие будет выполнено только, если концентрация $^{232}$U в исходном регенерате ниже, чем ограничение на $^{232}$U в конечном продукте. 
С помощью уравнения \ref{eq_232_balance} можно вычислить максимально возможную долю питающего потока, содержащего $^{232}$U, как неизвестную переменную уравнения \ref{eq_232_balance}. На основе состава регенерата второго рецикла получаем \ref{eq_232_balance_X} получаем:

% \begin{equation}
%   \label{eq_232_balance_X}
%     5 \times 10^{-7} \% \approx X \times 6.622 \times 10^{-7} \% \Rightarrow X \approx 0.755
% \end{equation}

\begin{equation}
  \label{eq_232_balance_X}
    \frac{RepU}{P} \approx 0.755
\end{equation}

Легко увидеть, что полученные в результате прямого численного расчёта предельные величины отношений (исходный регенерат)/продукт приблизительно и составлют такую величину. 
Подобный подход позволяет аналитически оценить возможность применения схем на основе простейших модификаций ординарного каскада, исходя из изотопного состава регенерата. Следует отметить также, что подобные оценки можно также применять и для каскадных схем, в которых регенерат разбавляют уже внутри каскада, путём его подачи в качестве дополнительного питания, поскольку такие схемы по сути являются также только разбавляющими.

\section{Обоснование необходимости составных схем}\label{sec:ch2/sec2}

Как следует из вышеприведенных расчетов, на текущий момент имеются способы, позволяющие принципиально решить проблему выполнения требования по четным изотопам урана при обогащении регенерата, и основной проблемой, решаемой в рамках настоящей диссертационной работы, является поиск варианта каскадной схемы, позволяющей одновременно выполнить ограничения по концентрациям четных изотопов и задействовать в обогащении весь имеющийся регенерат в условиях разброса по составу регенерата при его многократном рецикле.

Если анализировать причины невозможности возврата регенерата в производство топлива в многочисленных модификациях каскада для обогащения многократно облученного регенерата, то становится очевидным, что это, во многом, связано с нарастанием относительных концентраций “легких” изотопов (в первую очередь $^{232}$U) и
$^{235}$U, а поскольку данные изотопы концентрируются вместе на легком <<конце>> каскада, то единственным способом понизить отношение их концентраций -- это разбавить материалом, не содержащим $^{232}$U, например на входе в каскад. Как показали результаты, описанных в этой главе вычислительных экспериментов, для составов с достаточно высоким исходным содержанием $^{232}$U невозможно подобрать такой разбавитель, чтобы удовлетворить одновременно и условие полного возврата регенерата в цикл и условия на содержание четных изотопов.

Из приведенного выше анализа следует, что эффективная каскадная схема для обогащения регенерата урана при многократном рецикле должна обеспечивать не только разбавление регенерата, но и его хотя бы частичную очистку от чётных изотопов. Поэтому возможные варианты решения задачи, по-видимому, должны основываться на использование двойных каскадных схем или "гибридных" вариантов, в том числе, описанных в Главе 1. 
\clearpage
