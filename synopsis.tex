%&preformat-synopsis
\RequirePackage[l2tabu,orthodox]{nag} % Раскомментировав, можно в логе получать рекомендации относительно правильного использования пакетов и предупреждения об устаревших и нерекомендуемых пакетах
\PassOptionsToPackage{bookmarks=false}{hyperref}
\documentclass[a5paper,10pt,twoside,openany,article]{memoir}

%%%%%%%%%%%%%%%%%%%%%%%%%%%%%%%%%%%%%%%%%%%%%%%%%%%%%%%
%%%% Файл упрощённых настроек шаблона автореферата %%%%
%%%%%%%%%%%%%%%%%%%%%%%%%%%%%%%%%%%%%%%%%%%%%%%%%%%%%%%

%%% Инициализирование переменных, не трогать!  %%%
\newcounter{tabcap}
\newcounter{tablaba}
\newcounter{tabtita}
\newcounter{showperssign}
\newcounter{showsecrsign}
\newcounter{showopplead}
%%%%%%%%%%%%%%%%%%%%%%%%%%%%%%%%%%%%%%%%%%%%%%%%%%%%%%%

%%% Список публикаций %%%
\makeatletter
\@ifundefined{c@usefootcite}{
  \newcounter{usefootcite}
  \setcounter{usefootcite}{0} % 0 --- два списка литературы;
                              % 1 --- список публикаций автора + цитирование
                              %       других работ в сносках
}{}
\makeatother

\makeatletter
\@ifundefined{c@bibgrouped}{
  \newcounter{bibgrouped}
  \setcounter{bibgrouped}{0}  % 0 --- единый список работ автора;
                              % 1 --- сгруппированные работы автора
}{}
\makeatother

%%% Область упрощённого управления оформлением %%%

%% Управление зазором между подрисуночной подписью и основным текстом %%
\setlength{\belowcaptionskip}{10pt plus 20pt minus 2pt}


%% Подпись таблиц %%
\setcounter{tabcap}{0}  % 0 --- по ГОСТ, номер таблицы и название разделены
                        %       тире, выровнены по левому краю, при
                        %       необходимостина нескольких строках;
                        % 1 --- подпись таблицы не по ГОСТ, на двух и более
                        %       строках, дальнейшие настройки:
%Выравнивание первой строки, с подписью и номером
\setcounter{tablaba}{2} % 0 --- по левому краю;
                        % 1 --- по центру;
                        % 2 --- по правому краю
%Выравнивание строк с самим названием таблицы
\setcounter{tabtita}{1} % 0 --- по левому краю;
                        % 1 --- по центру;
                        % 2 --- по правому краю
%Разделитель записи «Таблица #» и названия таблицы
\newcommand{\tablabelsep}{ }

%% Подпись рисунков %%
%Разделитель записи «Рисунок #» и названия рисунка
\newcommand{\figlabelsep}{~\cyrdash\ }  % (ГОСТ 2.105, 4.3.1)
                                        % "--- здесь не работает

%Демонстрация подписи диссертанта на автореферате
\setcounter{showperssign}{1}  % 0 --- не показывать;
                              % 1 --- показывать
%Демонстрация подписи учёного секретаря на автореферате
\setcounter{showsecrsign}{1}  % 0 --- не показывать;
                              % 1 --- показывать
%Демонстрация информации об оппонентах и ведущей организации на автореферате
\setcounter{showopplead}{1}   % 0 --- не показывать;
                              % 1 --- показывать

%%% Цвета гиперссылок %%%
% Latex color definitions: http://latexcolor.com/
% \definecolor{linkcolor}{rgb}{0.9,0,0}
% \definecolor{citecolor}{rgb}{0,0.6,0}
% \definecolor{urlcolor}{rgb}{0,0,1}
\definecolor{linkcolor}{rgb}{0,0,0} %black
\definecolor{citecolor}{rgb}{0,0,0} %black
\definecolor{urlcolor}{rgb}{0,0,0} %black
          % общие настройки шаблона
\input{common/packages}       % Пакеты общие для диссертации и автореферата
\synopsistrue                 % Этот документ --- автореферат
\input{Synopsis/synpackages}  % Пакеты для автореферата
\usepackage{tabu, tabulary}  %таблицы с автоматически подбирающейся шириной столбцов
\usepackage{fr-longtable}    %ради \endlasthead

% Листинги с исходным кодом программ
\usepackage{fancyvrb}
\usepackage{listings}
\lccode`\~=0\relax %Без этого хака из-за особенностей пакета listings перестают работать конструкции с \MakeLowercase и т. п. в (xe|lua)latex

% Русская традиция начертания греческих букв
\usepackage{upgreek} % прямые греческие ради русской традиции

%%% Микротипографика
%\ifnumequal{\value{draft}}{0}{% Только если у нас режим чистовика
%    \usepackage[final, babel, shrink=45]{microtype}[2016/05/14] % улучшает представление букв и слов в строках, может помочь при наличии отдельно висящих слов
%}{}

% Отметка о версии черновика на каждой странице
% Чтобы работало надо в своей локальной копии по инструкции
% https://www.ctan.org/pkg/gitinfo2 создать небходимые файлы в папке
% ./git/hooks
% If you’re familiar with tweaking git, you can probably work it out for
% yourself. If not, I suggest you follow these steps:
% 1. First, you need a git repository and working tree. For this example,
% let’s suppose that the root of the working tree is in ~/compsci
% 2. Copy the file post-xxx-sample.txt (which is in the same folder of
% your TEX distribution as this pdf) into the git hooks directory in your
% working copy. In our example case, you should end up with a file called
% ~/compsci/.git/hooks/post-checkout
% 3. If you’re using a unix-like system, don’t forget to make the file executable.
% Just how you do this is outside the scope of this manual, but one
% possible way is with commands such as this:
% chmod g+x post-checkout.
% 4. Test your setup with “git checkout master” (or another suitable branch
% name). This should generate copies of gitHeadInfo.gin in the directories
% you intended.
% 5. Now make two more copies of this file in the same directory (hooks),
% calling them post-commit and post-merge, and you’re done. As before,
% users of unix-like systems should ensure these files are marked as
% executable.
\ifnumequal{\value{draft}}{1}{% Черновик
   \IfFileExists{.git/gitHeadInfo.gin}{
      \usepackage[mark,pcount]{gitinfo2}
      \renewcommand{\gitMark}{rev.\gitAbbrevHash\quad\gitCommitterEmail\quad\gitAuthorIsoDate}
      \renewcommand{\gitMarkFormat}{\rmfamily\color{Gray}\small\bfseries}
   }{}
}{} % Пакеты для специфических пользовательских задач

% Новые переменные, которые могут использоваться во всём проекте
% ГОСТ 7.0.11-2011
% 9.2 Оформление текста автореферата диссертации
% 9.2.1 Общая характеристика работы включает в себя следующие основные структурные
% элементы:
% актуальность темы исследования;
\newcommand{\actualityTXT}{Актуальность темы.}
% \newcommand{\actualityTXTAutoref}{Актуальность.}

% степень ее разработанности;
\newcommand{\progressTXT}{Степень разработанности темы.}
% цели и задачи;
\newcommand{\aimTXT}{Целью}
\newcommand{\tasksTXT}{задачи}
% научную новизну;
\newcommand{\noveltyTXT}{Научная новизна:}
% теоретическую и практическую значимость работы;
%\newcommand{\influenceTXT}{Теоретическая и практическая значимость}
% или чаще используют просто
\newcommand{\influenceTXT}{Практическая значимость}
% методологию и методы исследования;
\newcommand{\methodsTXT}{Методология и методы исследования.}
% положения, выносимые на защиту;
\newcommand{\defpositionsTXT}{Основные положения, выносимые на~защиту:}
% степень достоверности и апробацию результатов.
\newcommand{\reliabilityTXT}{Достоверность}
\newcommand{\probationTXT}{Апробация работы.}

\newcommand{\contributionTXT}{Личный вклад.}
\newcommand{\publicationsTXT}{Публикации.}


%%% Заголовки библиографии:

% для автореферата:
\newcommand{\bibtitleauthor}{Перечень основных публикаций по теме диссертации}

% для стиля библиографии `\insertbiblioauthorgrouped`
\newcommand{\bibtitleauthorvak}{В изданиях из списка ВАК РФ}
\newcommand{\bibtitleauthorscopus}{В изданиях, входящих в международную базу цитирования Scopus}
\newcommand{\bibtitleauthorwos}{В изданиях, входящих в международную базу цитирования Web of Science}
\newcommand{\bibtitleauthorother}{В прочих изданиях}
\newcommand{\bibtitleauthorconf}{В сборниках трудов конференций}

% для стиля библиографии `\insertbiblioauthorimportant`:
\newcommand{\bibtitleauthorimportant}{Наиболее значимые \protect\MakeLowercase\bibtitleauthor}

% для списка литературы в диссертации и списка чужих работ в автореферате:
\newcommand{\bibtitlefull}{Список литературы} % (ГОСТ Р 7.0.11-2011, 4)
       % Новые переменные, которые могут использоваться во всём проекте
%%%%%%%%%%%%%%%%%%%%%%%%%%%%%%%%%%%%%%%%%%%%%%%%%%%%%%%
%%%% Файл упрощённых настроек шаблона автореферата %%%%
%%%%%%%%%%%%%%%%%%%%%%%%%%%%%%%%%%%%%%%%%%%%%%%%%%%%%%%

%%% Инициализирование переменных, не трогать!  %%%
\newcounter{tabcap}
\newcounter{tablaba}
\newcounter{tabtita}
\newcounter{showperssign}
\newcounter{showsecrsign}
\newcounter{showopplead}
%%%%%%%%%%%%%%%%%%%%%%%%%%%%%%%%%%%%%%%%%%%%%%%%%%%%%%%

%%% Список публикаций %%%
\makeatletter
\@ifundefined{c@usefootcite}{
  \newcounter{usefootcite}
  \setcounter{usefootcite}{0} % 0 --- два списка литературы;
                              % 1 --- список публикаций автора + цитирование
                              %       других работ в сносках
}{}
\makeatother

\makeatletter
\@ifundefined{c@bibgrouped}{
  \newcounter{bibgrouped}
  \setcounter{bibgrouped}{0}  % 0 --- единый список работ автора;
                              % 1 --- сгруппированные работы автора
}{}
\makeatother

%%% Область упрощённого управления оформлением %%%

%% Управление зазором между подрисуночной подписью и основным текстом %%
\setlength{\belowcaptionskip}{10pt plus 20pt minus 2pt}


%% Подпись таблиц %%
\setcounter{tabcap}{0}  % 0 --- по ГОСТ, номер таблицы и название разделены
                        %       тире, выровнены по левому краю, при
                        %       необходимостина нескольких строках;
                        % 1 --- подпись таблицы не по ГОСТ, на двух и более
                        %       строках, дальнейшие настройки:
%Выравнивание первой строки, с подписью и номером
\setcounter{tablaba}{2} % 0 --- по левому краю;
                        % 1 --- по центру;
                        % 2 --- по правому краю
%Выравнивание строк с самим названием таблицы
\setcounter{tabtita}{1} % 0 --- по левому краю;
                        % 1 --- по центру;
                        % 2 --- по правому краю
%Разделитель записи «Таблица #» и названия таблицы
\newcommand{\tablabelsep}{ }

%% Подпись рисунков %%
%Разделитель записи «Рисунок #» и названия рисунка
\newcommand{\figlabelsep}{~\cyrdash\ }  % (ГОСТ 2.105, 4.3.1)
                                        % "--- здесь не работает

%Демонстрация подписи диссертанта на автореферате
\setcounter{showperssign}{1}  % 0 --- не показывать;
                              % 1 --- показывать
%Демонстрация подписи учёного секретаря на автореферате
\setcounter{showsecrsign}{1}  % 0 --- не показывать;
                              % 1 --- показывать
%Демонстрация информации об оппонентах и ведущей организации на автореферате
\setcounter{showopplead}{1}   % 0 --- не показывать;
                              % 1 --- показывать

%%% Цвета гиперссылок %%%
% Latex color definitions: http://latexcolor.com/
% \definecolor{linkcolor}{rgb}{0.9,0,0}
% \definecolor{citecolor}{rgb}{0,0.6,0}
% \definecolor{urlcolor}{rgb}{0,0,1}
\definecolor{linkcolor}{rgb}{0,0,0} %black
\definecolor{citecolor}{rgb}{0,0,0} %black
\definecolor{urlcolor}{rgb}{0,0,0} %black
        % Упрощённые настройки шаблона
%%% Основные сведения %%%
\newcommand{\thesisAuthorLastName}{Гусев}
\newcommand{\thesisAuthorOtherNames}{Владислав Евгеньевич}
\newcommand{\thesisAuthorInitials}{В.\,Е.}
\newcommand{\thesisAuthor}             % Диссертация, ФИО автора
{%
    \texorpdfstring{% \texorpdfstring takes two arguments and uses the first for (La)TeX and the second for pdf
        \thesisAuthorLastName~\thesisAuthorOtherNames% так будет отображаться на титульном листе или в тексте, где будет использоваться переменная
    }{%
        \thesisAuthorLastName, \thesisAuthorOtherNames% эта запись для свойств pdf-файла. В таком виде, если pdf будет обработан программами для сбора библиографических сведений, будет правильно представлена фамилия.
    }
}
\newcommand{\thesisAuthorShort}        % Диссертация, ФИО автора инициалами
{\thesisAuthorInitials~\thesisAuthorLastName}
\newcommand{\thesisUdk}                % Диссертация, УДК
{\todo{000.000}}
\newcommand{\thesisTitle}              % Диссертация, название
{Каскадные схемы для обогащения регенерированного урана при его многократном рецикле в топливных циклах перспективных энергетических реакторов}
\newcommand{\thesisSpecialtyNumber}    % Диссертация, специальность, номер
{1.3.14}
\newcommand{\thesisSpecialtyTitle}     % Диссертация, специальность, название (название взято с сайта ВАК для примера)
{Теплофизика и теоретическая теплотехника}
%% \newcommand{\thesisSpecialtyTwoNumber} % Диссертация, вторая специальность, номер
%% {\todo{XX.XX.XX}}
%% \newcommand{\thesisSpecialtyTwoTitle}  % Диссертация, вторая специальность, название
%% {\todo{Теория и~методика физического воспитания, спортивной тренировки,
%% оздоровительной и~адаптивной физической культуры}}
\newcommand{\thesisDegree}             % Диссертация, ученая степень
{кандидата технических наук}
\newcommand{\thesisDegreeShort}        % Диссертация, ученая степень, краткая запись
{канд. техн. наук}
\newcommand{\thesisCity}               % Диссертация, город написания диссертации
{Москва}
\newcommand{\thesisYear}               % Диссертация, год написания диссертации
{2024}
\newcommand{\thesisOrganization}       % Диссертация, организация МИНИСТЕРСТВО НАУКИ И ВЫСШЕГО ОБРАЗОВАНИЯ РОССИЙСКОЙ ФЕДЕРАЦИИ
{ФЕДЕРАЛЬНОЕ ГОСУДАРСТВЕННОЕ АВТОНОМНОЕ ОБРАЗОВАТЕЛЬНОЕ УЧРЕЖДЕНИЕ ВЫСШЕГО ОБРАЗОВАНИЯ\\
\textbf {<<Национальный исследовательский ядерный университет <<МИФИ>>\\(НИЯУ МИФИ)}}
\newcommand{\thesisOrganizationShort}  % Диссертация, краткое название организации для доклада
{\todo{НазУчДисРаб}}

\newcommand{\thesisInOrganization}     % Диссертация, организация в предложном падеже: Работа выполнена в ...
{Национальном исследовательском ядерном университете «МИФИ» (НИЯУ МИФИ)}

%% \newcommand{\supervisorDead}{}           % Рисовать рамку вокруг фамилии
\newcommand{\supervisorFio}              % Научный руководитель, ФИО
{Сулаберидзе Георгий Анатольевич}
\newcommand{\supervisorRegalia}          % Научный руководитель, регалии
{кандидат физико-математических наук, доцент}
\newcommand{\supervisorFioShort}         % Научный руководитель, ФИО
{Г.\,А.~Сулаберидзе}
\newcommand{\supervisorRegaliaShort}     % Научный руководитель, регалии
{к.ф.-м.н, доцент}


\newcommand{\consultOneFio}           % Второй научный руководитель, ФИО
{Невиница Владимир Анатольевич}
\newcommand{\consultOneRegalia}       % Второй научный руководитель, регалии
{кандидат технических наук}
\newcommand{\consultOneFioShort}      % Второй научный руководитель, ФИО
{В.\,А.~Невиница}
\newcommand{\consultOneRegaliaShort}  % Второй научный руководитель, регалии
{к.т.н}

% \newcommand{\consultTwoFio}           % Второй научный руководитель, ФИО
% {Невиница Владимир Анатольевич}
% \newcommand{\consultTwoRegalia}       % Второй научный руководитель, регалии
% {кандидат технических наук}
% \newcommand{\consultTwoFioShort}      % Второй научный руководитель, ФИО
% {В.\,А.~Невиница}
% \newcommand{\consultTwoRegaliaShort}  % Второй научный руководитель, регалии
% {к.т.н}

\newcommand{\opponentOneFio}           % Оппонент 1, ФИО
{Палкин Валерий Анатольевич}
\newcommand{\opponentOneRegalia}       % Оппонент 1, регалии
{доктор технических наук}
\newcommand{\opponentOneJobPlace}      % Оппонент 1, место работы
{УРФУ}
\newcommand{\opponentOneJobPost}       % Оппонент 1, должность
{профессор}

\newcommand{\opponentTwoFio}           % Оппонент 2, ФИО
{Алексей Алексеевич Орлов}
\newcommand{\opponentTwoRegalia}       % Оппонент 2, регалии
{доктор технических наук}
\newcommand{\opponentTwoJobPlace}      % Оппонент 2, место работы
{Отделение ядерно-топливного цикла ТПУ}
\newcommand{\opponentTwoJobPost}       % Оппонент 2, должность
{профессор}

\newcommand{\opponentThreeFio}         % Оппонент 3, ФИО
{Букин Алексей Николаевич}
\newcommand{\opponentThreeRegalia}     % Оппонент 3, регалии
{кандидат технических  наук}
\newcommand{\opponentThreeJobPlace}    % Оппонент 3, место работы
{Кафедра технологии изотопов и водородной энергетики РХТУ им. Д.И. Менделеева}
\newcommand{\opponentThreeJobPost}     % Оппонент 3, должность
{старший научный сотрудник}

\newcommand{\leadingOrganizationTitle} % Ведущая организация, дополнительные строки. Удалить, чтобы не отображать в автореферате
{\todo{Федеральное государственное бюджетное образовательное учреждение высшего
 образования -.-.-.-.-.-.-.-.-.-.-.--.-.-.-.-.-.-.-.-.-.-.-.--..--..-}}

\newcommand{\defenseDate}              % Защита, дата
{\todo{--- --- 2024~г.~в~--- часов}}
\newcommand{\defenseCouncilNumber}     % Защита, номер диссертационного совета
{{МИФИ.2.02}}
\newcommand{\defenseCouncilTitle}      % Защита, учреждение диссертационного совета
{{НИЯУ МИФИ}}
\newcommand{\defenseCouncilAddress}    % Защита, адрес учреждение диссертационного совета
{{Москва, Каширское шоссе, 31}}
\newcommand{\defenseCouncilPhone}      % Телефон для справок
{\todo{+7~(000)~00-00-00}}

\newcommand{\defenseSecretaryFio}      % Секретарь диссертационного совета, ФИО
{{Куликов Е.Г.}}
\newcommand{\defenseSecretaryRegalia}  % Секретарь диссертационного совета, регалии
{{к.т.н.}}            % Для сокращений есть ГОСТы, например: ГОСТ Р 7.0.12-2011 + http://base.garant.ru/179724/#block_30000

\newcommand{\synopsisLibrary}          % Автореферат, название библиотеки
{\todo{Название библиотеки}}
\newcommand{\synopsisDate}             % Автореферат, дата рассылки
{\todo{DD mmmmmmmm 2024 года}}

% To avoid conflict with beamer class use \providecommand
\providecommand{\keywords}%            % Ключевые слова для метаданных PDF диссертации и автореферата
{}
           % Основные сведения
\input{common/fonts}          % Определение шрифтов (частичное)
\input{common/styles}         % Стили общие для диссертации и автореферата
\input{Synopsis/synstyles}    % Стили для автореферата
% для вертикального центрирования ячеек в tabulary
\def\zz{\ifx\[$\else\aftergroup\zzz\fi}
%$ \] % <-- чиним подсветку синтаксиса в некоторых редакторах
\def\zzz{\setbox0\lastbox
\dimen0\dimexpr\extrarowheight + \ht0-\dp0\relax
\setbox0\hbox{\raise-.5\dimen0\box0}%
\ht0=\dimexpr\ht0+\extrarowheight\relax
\dp0=\dimexpr\dp0+\extrarowheight\relax
\box0
}

\lstdefinelanguage{Renhanced}%
{keywords={abbreviate,abline,abs,acos,acosh,action,add1,add,%
        aggregate,alias,Alias,alist,all,anova,any,aov,aperm,append,apply,%
        approx,approxfun,apropos,Arg,args,array,arrows,as,asin,asinh,%
        atan,atan2,atanh,attach,attr,attributes,autoload,autoloader,ave,%
        axis,backsolve,barplot,basename,besselI,besselJ,besselK,besselY,%
        beta,binomial,body,box,boxplot,break,browser,bug,builtins,bxp,by,%
        c,C,call,Call,case,cat,category,cbind,ceiling,character,char,%
        charmatch,check,chol,chol2inv,choose,chull,class,close,cm,codes,%
        coef,coefficients,co,col,colnames,colors,colours,commandArgs,%
        comment,complete,complex,conflicts,Conj,contents,contour,%
        contrasts,contr,control,helmert,contrib,convolve,cooks,coords,%
        distance,coplot,cor,cos,cosh,count,fields,cov,covratio,wt,CRAN,%
        create,crossprod,cummax,cummin,cumprod,cumsum,curve,cut,cycle,D,%
        data,dataentry,date,dbeta,dbinom,dcauchy,dchisq,de,debug,%
        debugger,Defunct,default,delay,delete,deltat,demo,de,density,%
        deparse,dependencies,Deprecated,deriv,description,detach,%
        dev2bitmap,dev,cur,deviance,off,prev,,dexp,df,dfbetas,dffits,%
        dgamma,dgeom,dget,dhyper,diag,diff,digamma,dim,dimnames,dir,%
        dirname,dlnorm,dlogis,dnbinom,dnchisq,dnorm,do,dotplot,double,%
        download,dpois,dput,drop,drop1,dsignrank,dt,dummy,dump,dunif,%
        duplicated,dweibull,dwilcox,dyn,edit,eff,effects,eigen,else,%
        emacs,end,environment,env,erase,eval,equal,evalq,example,exists,%
        exit,exp,expand,expression,External,extract,extractAIC,factor,%
        fail,family,fft,file,filled,find,fitted,fivenum,fix,floor,for,%
        For,formals,format,formatC,formula,Fortran,forwardsolve,frame,%
        frequency,ftable,ftable2table,function,gamma,Gamma,gammaCody,%
        gaussian,gc,gcinfo,gctorture,get,getenv,geterrmessage,getOption,%
        getwd,gl,glm,globalenv,gnome,GNOME,graphics,gray,grep,grey,grid,%
        gsub,hasTsp,hat,heat,help,hist,home,hsv,httpclient,I,identify,if,%
        ifelse,Im,image,\%in\%,index,influence,measures,inherits,install,%
        installed,integer,interaction,interactive,Internal,intersect,%
        inverse,invisible,IQR,is,jitter,kappa,kronecker,labels,lapply,%
        layout,lbeta,lchoose,lcm,legend,length,levels,lgamma,library,%
        licence,license,lines,list,lm,load,local,locator,log,log10,log1p,%
        log2,logical,loglin,lower,lowess,ls,lsfit,lsf,ls,machine,Machine,%
        mad,mahalanobis,make,link,margin,match,Math,matlines,mat,matplot,%
        matpoints,matrix,max,mean,median,memory,menu,merge,methods,min,%
        missing,Mod,mode,model,response,mosaicplot,mtext,mvfft,na,nan,%
        names,omit,nargs,nchar,ncol,NCOL,new,next,NextMethod,nextn,%
        nlevels,nlm,noquote,NotYetImplemented,NotYetUsed,nrow,NROW,null,%
        numeric,\%o\%,objects,offset,old,on,Ops,optim,optimise,optimize,%
        options,or,order,ordered,outer,package,packages,page,pairlist,%
        pairs,palette,panel,par,parent,parse,paste,path,pbeta,pbinom,%
        pcauchy,pchisq,pentagamma,persp,pexp,pf,pgamma,pgeom,phyper,pico,%
        pictex,piechart,Platform,plnorm,plogis,plot,pmatch,pmax,pmin,%
        pnbinom,pnchisq,pnorm,points,poisson,poly,polygon,polyroot,pos,%
        postscript,power,ppoints,ppois,predict,preplot,pretty,Primitive,%
        print,prmatrix,proc,prod,profile,proj,prompt,prop,provide,%
        psignrank,ps,pt,ptukey,punif,pweibull,pwilcox,q,qbeta,qbinom,%
        qcauchy,qchisq,qexp,qf,qgamma,qgeom,qhyper,qlnorm,qlogis,qnbinom,%
        qnchisq,qnorm,qpois,qqline,qqnorm,qqplot,qr,Q,qty,qy,qsignrank,%
        qt,qtukey,quantile,quasi,quit,qunif,quote,qweibull,qwilcox,%
        rainbow,range,rank,rbeta,rbind,rbinom,rcauchy,rchisq,Re,read,csv,%
        csv2,fwf,readline,socket,real,Recall,rect,reformulate,regexpr,%
        relevel,remove,rep,repeat,replace,replications,report,require,%
        resid,residuals,restart,return,rev,rexp,rf,rgamma,rgb,rgeom,R,%
        rhyper,rle,rlnorm,rlogis,rm,rnbinom,RNGkind,rnorm,round,row,%
        rownames,rowsum,rpois,rsignrank,rstandard,rstudent,rt,rug,runif,%
        rweibull,rwilcox,sample,sapply,save,scale,scan,scan,screen,sd,se,%
        search,searchpaths,segments,seq,sequence,setdiff,setequal,set,%
        setwd,show,sign,signif,sin,single,sinh,sink,solve,sort,source,%
        spline,splinefun,split,sqrt,stars,start,stat,stem,step,stop,%
        storage,strstrheight,stripplot,strsplit,structure,strwidth,sub,%
        subset,substitute,substr,substring,sum,summary,sunflowerplot,svd,%
        sweep,switch,symbol,symbols,symnum,sys,status,system,t,table,%
        tabulate,tan,tanh,tapply,tempfile,terms,terrain,tetragamma,text,%
        time,title,topo,trace,traceback,transform,tri,trigamma,trunc,try,%
        ts,tsp,typeof,unclass,undebug,undoc,union,unique,uniroot,unix,%
        unlink,unlist,unname,untrace,update,upper,url,UseMethod,var,%
        variable,vector,Version,vi,warning,warnings,weighted,weights,%
        which,while,window,write,\%x\%,x11,X11,xedit,xemacs,xinch,xor,%
        xpdrows,xy,xyinch,yinch,zapsmall,zip},%
    otherkeywords={!,!=,~,$,*,\%,\&,\%/\%,\%*\%,\%\%,<-,<<-},%$
    alsoother={._$},%$
    sensitive,%
    morecomment=[l]\#,%
    morestring=[d]",%
    morestring=[d]'% 2001 Robert Denham
}%

%решаем проблему с кириллицей в комментариях (в pdflatex) https://tex.stackexchange.com/a/103712
\lstset{extendedchars=true,keepspaces=true,literate={Ö}{{\"O}}1
    {Ä}{{\"A}}1
    {Ü}{{\"U}}1
    {ß}{{\ss}}1
    {ü}{{\"u}}1
    {ä}{{\"a}}1
    {ö}{{\"o}}1
    {~}{{\textasciitilde}}1
    {а}{{\selectfont\char224}}1
    {б}{{\selectfont\char225}}1
    {в}{{\selectfont\char226}}1
    {г}{{\selectfont\char227}}1
    {д}{{\selectfont\char228}}1
    {е}{{\selectfont\char229}}1
    {ё}{{\"e}}1
    {ж}{{\selectfont\char230}}1
    {з}{{\selectfont\char231}}1
    {и}{{\selectfont\char232}}1
    {й}{{\selectfont\char233}}1
    {к}{{\selectfont\char234}}1
    {л}{{\selectfont\char235}}1
    {м}{{\selectfont\char236}}1
    {н}{{\selectfont\char237}}1
    {о}{{\selectfont\char238}}1
    {п}{{\selectfont\char239}}1
    {р}{{\selectfont\char240}}1
    {с}{{\selectfont\char241}}1
    {т}{{\selectfont\char242}}1
    {у}{{\selectfont\char243}}1
    {ф}{{\selectfont\char244}}1
    {х}{{\selectfont\char245}}1
    {ц}{{\selectfont\char246}}1
    {ч}{{\selectfont\char247}}1
    {ш}{{\selectfont\char248}}1
    {щ}{{\selectfont\char249}}1
    {ъ}{{\selectfont\char250}}1
    {ы}{{\selectfont\char251}}1
    {ь}{{\selectfont\char252}}1
    {э}{{\selectfont\char253}}1
    {ю}{{\selectfont\char254}}1
    {я}{{\selectfont\char255}}1
    {А}{{\selectfont\char192}}1
    {Б}{{\selectfont\char193}}1
    {В}{{\selectfont\char194}}1
    {Г}{{\selectfont\char195}}1
    {Д}{{\selectfont\char196}}1
    {Е}{{\selectfont\char197}}1
    {Ё}{{\"E}}1
    {Ж}{{\selectfont\char198}}1
    {З}{{\selectfont\char199}}1
    {И}{{\selectfont\char200}}1
    {Й}{{\selectfont\char201}}1
    {К}{{\selectfont\char202}}1
    {Л}{{\selectfont\char203}}1
    {М}{{\selectfont\char204}}1
    {Н}{{\selectfont\char205}}1
    {О}{{\selectfont\char206}}1
    {П}{{\selectfont\char207}}1
    {Р}{{\selectfont\char208}}1
    {С}{{\selectfont\char209}}1
    {Т}{{\selectfont\char210}}1
    {У}{{\selectfont\char211}}1
    {Ф}{{\selectfont\char212}}1
    {Х}{{\selectfont\char213}}1
    {Ц}{{\selectfont\char214}}1
    {Ч}{{\selectfont\char215}}1
    {Ш}{{\selectfont\char216}}1
    {Щ}{{\selectfont\char217}}1
    {Ъ}{{\selectfont\char218}}1
    {Ы}{{\selectfont\char219}}1
    {Ь}{{\selectfont\char220}}1
    {Э}{{\selectfont\char221}}1
    {Ю}{{\selectfont\char222}}1
    {Я}{{\selectfont\char223}}1
    {і}{{\selectfont\char105}}1
    {ї}{{\selectfont\char168}}1
    {є}{{\selectfont\char185}}1
    {ґ}{{\selectfont\char160}}1
    {І}{{\selectfont\char73}}1
    {Ї}{{\selectfont\char136}}1
    {Є}{{\selectfont\char153}}1
    {Ґ}{{\selectfont\char128}}1
}

% Ширина текста минус ширина надписи 999
\newlength{\twless}
\newlength{\lmarg}
\setlength{\lmarg}{\widthof{999}}   % ширина надписи 999
\setlength{\twless}{\textwidth-\lmarg}

\lstset{ %
%    language=R,                     %  Язык указать здесь, если во всех листингах преимущественно один язык, в результате часть настроек может пойти только для этого языка
    numbers=left,                   % where to put the line-numbers
    numberstyle=\fontsize{12pt}{14pt}\selectfont\color{Gray},  % the style that is used for the line-numbers
    firstnumber=1,                  % в этой и следующей строках задаётся поведение нумерации 5, 10, 15...
    stepnumber=5,                   % the step between two line-numbers. If it's 1, each line will be numbered
    numbersep=5pt,                  % how far the line-numbers are from the code
    backgroundcolor=\color{white},  % choose the background color. You must add \usepackage{color}
    showspaces=false,               % show spaces adding particular underscores
    showstringspaces=false,         % underline spaces within strings
    showtabs=false,                 % show tabs within strings adding particular underscores
    frame=leftline,                 % adds a frame of different types around the code
    rulecolor=\color{black},        % if not set, the frame-color may be changed on line-breaks within not-black text (e.g. commens (green here))
    tabsize=2,                      % sets default tabsize to 2 spaces
    captionpos=t,                   % sets the caption-position to top
    breaklines=true,                % sets automatic line breaking
    breakatwhitespace=false,        % sets if automatic breaks should only happen at whitespace
%    title=\lstname,                 % show the filename of files included with \lstinputlisting;
    % also try caption instead of title
    basicstyle=\fontsize{12pt}{14pt}\selectfont\ttfamily,% the size of the fonts that are used for the code
%    keywordstyle=\color{blue},      % keyword style
    commentstyle=\color{ForestGreen}\emph,% comment style
    stringstyle=\color{Mahogany},   % string literal style
    escapeinside={\%*}{*)},         % if you want to add a comment within your code
    morekeywords={*,...},           % if you want to add more keywords to the set
    inputencoding=utf8,             % кодировка кода
    xleftmargin={\lmarg},           % Чтобы весь код и полоска с номерами строк была смещена влево, так чтобы цифры не вылезали за пределы текста слева
}

%http://tex.stackexchange.com/questions/26872/smaller-frame-with-listings
% Окружение, чтобы листинг был компактнее обведен рамкой, если она задается, а не на всю ширину текста
\makeatletter
\newenvironment{SmallListing}[1][]
{\lstset{#1}\VerbatimEnvironment\begin{VerbatimOut}{VerbEnv.tmp}}
{\end{VerbatimOut}\settowidth\@tempdima{%
        \lstinputlisting{VerbEnv.tmp}}
    \minipage{\@tempdima}\lstinputlisting{VerbEnv.tmp}\endminipage}
\makeatother

\DefineVerbatimEnvironment% с шрифтом 12 пт
{Verb}{Verbatim}
{fontsize=\fontsize{12pt}{14pt}\selectfont}

\newfloat[chapter]{ListingEnv}{lol}{Листинг}

\renewcommand{\lstlistingname}{Листинг}

%Общие счётчики окружений листингов
%http://tex.stackexchange.com/questions/145546/how-to-make-figure-and-listing-share-their-counter
% Если смешивать плавающие и не плавающие окружения, то могут быть проблемы с нумерацией
\makeatletter
\AtBeginDocument{%
    \let\c@ListingEnv\c@lstlisting
    \let\theListingEnv\thelstlisting
    \let\ftype@lstlisting\ftype@ListingEnv % give the floats the same precedence
}
\makeatother

% значок С++ — используйте команду \cpp
\newcommand{\cpp}{%
    C\nolinebreak\hspace{-.05em}%
    \raisebox{.2ex}{+}\nolinebreak\hspace{-.10em}%
    \raisebox{.2ex}{+}%
}

%%%  Чересстрочное форматирование таблиц
%% http://tex.stackexchange.com/questions/278362/apply-italic-formatting-to-every-other-row
\newcounter{rowcnt}
\newcommand\altshape{\ifnumodd{\value{rowcnt}}{\color{red}}{\vspace*{-1ex}\itshape}}
% \AtBeginEnvironment{tabular}{\setcounter{rowcnt}{1}}
% \AtEndEnvironment{tabular}{\setcounter{rowcnt}{0}}

%%% Ради примера во второй главе
\let\originalepsilon\epsilon
\let\originalphi\phi
\let\originalkappa\kappa
\let\originalle\le
\let\originalleq\leq
\let\originalge\ge
\let\originalgeq\geq
\let\originalemptyset\emptyset
\let\originaltan\tan
\let\originalcot\cot
\let\originalcsc\csc

%%% Русская традиция начертания математических знаков
\renewcommand{\le}{\ensuremath{\leqslant}}
\renewcommand{\leq}{\ensuremath{\leqslant}}
\renewcommand{\ge}{\ensuremath{\geqslant}}
\renewcommand{\geq}{\ensuremath{\geqslant}}
\renewcommand{\emptyset}{\varnothing}

%%% Русская традиция начертания математических функций (на случай копирования из зарубежных источников)
\renewcommand{\tan}{\operatorname{tg}}
\renewcommand{\cot}{\operatorname{ctg}}
\renewcommand{\csc}{\operatorname{cosec}}

%%% Русская традиция начертания греческих букв (греческие буквы вертикальные, через пакет upgreek)
\renewcommand{\epsilon}{\ensuremath{\upvarepsilon}}   %  русская традиция записи
\renewcommand{\phi}{\ensuremath{\upvarphi}}
%\renewcommand{\kappa}{\ensuremath{\varkappa}}
\renewcommand{\alpha}{\upalpha}
\renewcommand{\beta}{\upbeta}
\renewcommand{\gamma}{\upgamma}
\renewcommand{\delta}{\updelta}
\renewcommand{\varepsilon}{\upvarepsilon}
\renewcommand{\zeta}{\upzeta}
\renewcommand{\eta}{\upeta}
\renewcommand{\theta}{\uptheta}
\renewcommand{\vartheta}{\upvartheta}
\renewcommand{\iota}{\upiota}
\renewcommand{\kappa}{\upkappa}
\renewcommand{\lambda}{\uplambda}
\renewcommand{\mu}{\upmu}
\renewcommand{\nu}{\upnu}
\renewcommand{\xi}{\upxi}
\renewcommand{\pi}{\uppi}
\renewcommand{\varpi}{\upvarpi}
\renewcommand{\rho}{\uprho}
%\renewcommand{\varrho}{\upvarrho}
\renewcommand{\sigma}{\upsigma}
%\renewcommand{\varsigma}{\upvarsigma}
\renewcommand{\tau}{\uptau}
\renewcommand{\upsilon}{\upupsilon}
\renewcommand{\varphi}{\upvarphi}
\renewcommand{\chi}{\upchi}
\renewcommand{\psi}{\uppsi}
\renewcommand{\omega}{\upomega}
   % Стили для специфических пользовательских задач

%%% Библиография. Выбор движка для реализации %%%
\ifnumequal{\value{bibliosel}}{0}{%
    %%% Реализация библиографии встроенными средствами посредством движка bibtex8 %%%

%%% Пакеты %%%
\usepackage{cite}                                   % Красивые ссылки на литературу


%%% Стили %%%
% \bibliographystyle{BibTeX-Styles/utf8gost71u}    % Оформляем библиографию по ГОСТ 7.1 (ГОСТ Р 7.0.11-2011, 5.6.7)
\bibliographystyle{BibTeX-Styles/ugost2008mod}    % Оформляем библиографию по ГОСТ 7.1 (ГОСТ Р 7.0.11-2011, 5.6.7)

\makeatletter
\renewcommand{\@biblabel}[1]{#1.}   % Заменяем библиографию с квадратных скобок на точку
\makeatother
%% Управление отступами между записями
%% требует etoolbox
%% http://tex.stackexchange.com/a/105642
%\patchcmd\thebibliography
% {\labelsep}
% {\labelsep\itemsep=5pt\parsep=0pt\relax}
% {}
% {\typeout{Couldn't patch the command}}

%%% Список литературы с красной строки (без висячего отступа) %%%
%\patchcmd{\thebibliography} %может потребовать включения пакета etoolbox
%  {\advance\leftmargin\labelsep}
%  {\leftmargin=0pt%
%   \setlength{\labelsep}{\widthof{\ }}% Управляет длиной отступа после точки
%   \itemindent=\parindent%
%   \addtolength{\itemindent}{\labelwidth}% Сдвигаем правее на величину номера с точкой
%   \advance\itemindent\labelsep%
%  }
%  {}{}

%%% Цитирование %%%
\renewcommand\citepunct{;\penalty\citepunctpenalty%
    \hskip.13emplus.1emminus.1em\relax}                % Разделение ; при перечислении ссылок (ГОСТ Р 7.0.5-2008)

\newcommand*{\autocite}[1]{}  % Чтобы примеры цитирования, рассчитанные на biblatex, не вызывали ошибок при компиляции в bibtex


\newcommand*{\insertbibliofull}{
\bibliography{biblio/library,biblio/zotero,biblio/external}         % Подключаем BibTeX-базы % После запятых не должно быть лишних пробелов — он "думает", что это тоже имя пути
}

\newcommand*{\insertbiblioauthor}{
\bibliography{biblio/author}         % Подключаем BibTeX-базы % После запятых не должно быть лишних пробелов — он "думает", что это тоже имя пути
}

% \newcommand*{\insertbiblioexternal}{
% \bibliography{biblio/zotero}         % Подключаем BibTeX-базы
% }


%% Счётчик использованных ссылок на литературу, обрабатывающий с учётом неоднократных ссылок
%% Требуется дважды компилировать, поскольку ему нужно считать актуальный внешний файл со списком литературы
\newtotcounter{citenum}
\def\oldcite{}
\let\oldcite=\bibcite
\def\bibcite{\stepcounter{citenum}\oldcite}
 % Встроенная реализация с загрузкой файла через движок bibtex8
}{
    %%% Реализация библиографии пакетами biblatex и biblatex-gost с использованием движка biber %%%

\usepackage{csquotes} % biblatex рекомендует его подключать. Пакет для оформления сложных блоков цитирования.
%%% Загрузка пакета с основными настройками %%%
\makeatletter
\ifnumequal{\value{draft}}{0}{% Чистовик
\usepackage[%
backend=biber,% движок
bibencoding=utf8,% кодировка bib файла
sorting=none,% настройка сортировки списка литературы
style=gost-numeric,% стиль цитирования и библиографии (по ГОСТ)
language=autobib,% получение языка из babel/polyglossia, default: autobib % если ставить autocite или auto, то цитаты в тексте с указанием страницы, получат указание страницы на языке оригинала
autolang=other,% многоязычная библиография
clearlang=true,% внутренний сброс поля language, если он совпадает с языком из babel/polyglossia
defernumbers=true,% нумерация проставляется после двух компиляций, зато позволяет выцеплять библиографию по ключевым словам и нумеровать не из большего списка
sortcites=true,% сортировать номера затекстовых ссылок при цитировании (если в квадратных скобках несколько ссылок, то отображаться будут отсортированно, а не абы как)
%urldate=false,
doi=false,% Показывать или нет ссылки на DOI
isbn=false,% Показывать или нет ISBN, ISSN, ISRN
url=false,
]{biblatex}[2016/09/17]
\ltx@iffilelater{biblatex-gost.def}{2017/05/03}%
{\toggletrue{bbx:gostbibliography}%
\renewcommand*{\revsdnamepunct}{\addcomma}}{}
}{%Черновик
\usepackage[%
backend=biber,% движок
bibencoding=utf8,% кодировка bib файла
sorting=none,% настройка сортировки списка литературы
% defernumbers=true, % откомментируйте, если требуется правильная нумерация ссылок на литературу в режиме черновика. Замедляет сборку
]{biblatex}[2016/09/17]%
}
\makeatother

\ifxetexorluatex
\else
% Исправление случая неподдержки знака номера в pdflatex
    \DefineBibliographyStrings{russian}{number={\textnumero}}
\fi

\ifsynopsis
\ifnumgreater{\value{usefootcite}}{0}{
    \ExecuteBibliographyOptions{autocite=footnote}
    \newbibmacro*{cite:full}{%
        \printtext[bibhypertarget]{%
            \usedriver{%
                \DeclareNameAlias{sortname}{default}%
            }{%
                \thefield{entrytype}%
            }%
        }%
        \usebibmacro{shorthandintro}%
    }
    \DeclareCiteCommand{\smartcite}[\mkbibfootnote]{%
        \usebibmacro{prenote}%
    }{%
        \usebibmacro{citeindex}%
        \usebibmacro{cite:full}%
    }{%
        \multicitedelim%
    }{%
        \usebibmacro{postnote}%
    }
}{}
\fi

%%% Подключение файлов bib %%%
\addbibresource[label=bl-external]{biblio/external.bib}
\addbibresource[label=bl-author]{biblio/author.bib}
\addbibresource[label=bl-author]{biblio/library.bib}
% \addbibresource[label=bl-author]{biblio/zotero.bib}



%http://tex.stackexchange.com/a/141831/79756
%There is a way to automatically map the language field to the langid field. The following lines in the preamble should be enough to do that.
%This command will copy the language field into the langid field and will then delete the contents of the language field. The language field will only be deleted if it was successfully copied into the langid field.
\DeclareSourcemap{ %модификация bib файла перед тем, как им займётся biblatex
    \maps{
        \map{% перекидываем значения полей language в поля langid, которыми пользуется biblatex
            \step[fieldsource=language, fieldset=langid, origfieldval, final]
            \step[fieldset=language, null]
        }
        \map{% перекидываем значения полей numpages в поля pagetotal, которыми пользуется biblatex
            \step[fieldsource=numpages, fieldset=pagetotal, origfieldval, final]
            \step[fieldset=numpages, null]
        }
        \map{% перекидываем значения полей pagestotal в поля pagetotal, которыми пользуется biblatex
            \step[fieldsource=pagestotal, fieldset=pagetotal, origfieldval, final]
            \step[fieldset=pagestotal, null]
        }
        \map[overwrite]{% перекидываем значения полей shortjournal, если они есть, в поля journal, которыми пользуется biblatex
            \step[fieldsource=shortjournal, final]
            \step[fieldset=journal, origfieldval]
            \step[fieldset=shortjournal, null]
        }
        \map[overwrite]{% перекидываем значения полей shortbooktitle, если они есть, в поля booktitle, которыми пользуется biblatex
            \step[fieldsource=shortbooktitle, final]
            \step[fieldset=booktitle, origfieldval]
            \step[fieldset=shortbooktitle, null]
        }
        \map{% если в поле medium написано "Электронный ресурс", то устанавливаем поле media, которым пользуется biblatex, в значение eresource.
            \step[fieldsource=medium,
            match=\regexp{Электронный\s+ресурс},
            final]
            \step[fieldset=media, fieldvalue=eresource]
            \step[fieldset=medium, null]
        }
        \map{% использование media=text по умолчанию
            \step[fieldset=media, fieldvalue=text]
        }
        \map[overwrite]{% стираем значения всех полей issn
            \step[fieldset=issn, null]
        }
        \map[overwrite]{% стираем значения всех полей abstract, поскольку ими не пользуемся, а там бывают "неприятные" латеху символы
            \step[fieldsource=abstract]
            \step[fieldset=abstract,null]
        }
        \map[overwrite]{ % переделка формата записи даты
            \step[fieldsource=urldate,
            match=\regexp{([0-9]{2})\.([0-9]{2})\.([0-9]{4})},
            replace={$3-$2-$1$4}, % $4 вставлен исключительно ради нормальной работы программ подсветки синтаксиса, которые некорректно обрабатывают $ в таких конструкциях
            final]
        }
        \map[overwrite]{ % стираем ключевые слова
            \step[fieldsource=keywords]
            \step[fieldset=keywords,null]
        }
        % реализация foreach различается для biblatex v3.12 и v3.13.
        % Для версии v3.13 эта конструкция заменяет последующие 5 структур map
        % \map[overwrite,foreach={authorvak,authorscopus,authorwos,authorconf,authorother}]{ % записываем информацию о типе публикации в ключевые слова
        %     \step[fieldsource=$MAPLOOP,final=true]
        %     \step[fieldset=keywords,fieldvalue={,biblio$MAPLOOP},append=true]
        % }
        \map[overwrite]{ % записываем информацию о типе публикации в ключевые слова
            \step[fieldsource=authorvak,final=true]
            \step[fieldset=keywords,fieldvalue={,biblioauthorvak},append=true]
        }
        \map[overwrite]{ % записываем информацию о типе публикации в ключевые слова
            \step[fieldsource=authorscopus,final=true]
            \step[fieldset=keywords,fieldvalue={,biblioauthorscopus},append=true]
        }
        \map[overwrite]{ % записываем информацию о типе публикации в ключевые слова
            \step[fieldsource=authorwos,final=true]
            \step[fieldset=keywords,fieldvalue={,biblioauthorwos},append=true]
        }
        \map[overwrite]{ % записываем информацию о типе публикации в ключевые слова
            \step[fieldsource=authorconf,final=true]
            \step[fieldset=keywords,fieldvalue={,biblioauthorconf},append=true]
        }
        \map[overwrite]{ % записываем информацию о типе публикации в ключевые слова
            \step[fieldsource=authorother,final=true]
            \step[fieldset=keywords,fieldvalue={,biblioauthorother},append=true]
        }
        \map[overwrite]{ % добавляем ключевые слова, чтобы различать источники
            \perdatasource{biblio/external.bib}
            \step[fieldset=keywords, fieldvalue={,biblioexternal},append=true]
        }
        \map[overwrite]{ % добавляем ключевые слова, чтобы различать источники
            \perdatasource{biblio/author.bib}
            \step[fieldset=keywords, fieldvalue={,biblioauthor},append=true]
        }
        \map[overwrite]{ % добавляем ключевые слова, чтобы различать источники
            \step[fieldset=keywords, fieldvalue={,bibliofull},append=true]
        }
%        \map[overwrite]{% стираем значения всех полей series
%            \step[fieldset=series, null]
%        }
        \map[overwrite]{% перекидываем значения полей howpublished в поля organization для типа online
            \step[typesource=online, typetarget=online, final]
            \step[fieldsource=howpublished, fieldset=organization, origfieldval]
            \step[fieldset=howpublished, null]
        }
        % Так отключаем [Электронный ресурс]
%        \map[overwrite]{% стираем значения всех полей media=eresource
%            \step[fieldsource=media,
%            match={eresource},
%            final]
%            \step[fieldset=media, null]
%        }
        % временный костыль для biber версии 2.13
        \map[overwrite]{ % добавляем ключевые слова, чтобы различать источники
            \perdatasource{external.bib}
            \step[fieldset=keywords, fieldvalue={,biblioexternal},append=true]
        }
        \map[overwrite]{ % добавляем ключевые слова, чтобы различать источники
            \perdatasource{author.bib}
            \step[fieldset=keywords, fieldvalue={,biblioauthor},append=true]
        }
    }
}

\ifsynopsis
\else
\DeclareSourcemap{ %модификация bib файла перед тем, как им займётся biblatex
    \maps{
        \map[overwrite]{% стираем значения всех полей addendum
            \perdatasource{biblio/author.bib}
            \step[fieldset=addendum, null] %чтобы избавиться от информации об объёме авторских статей, в отличие от автореферата
        }
         % временный костыль для biber версии 2.13
        \map[overwrite]{% стираем значения всех полей addendum
            \perdatasource{author.bib}
            \step[fieldset=addendum, null] %чтобы избавиться от информации об объёме авторских статей, в отличие от автореферата
        }
    }
}
\fi

\defbibfilter{vakscopuswos}{%
    keyword=biblioauthorvak or keyword=biblioauthorscopus or keyword=biblioauthorwos
}

\defbibfilter{scopuswos}{%
    keyword=biblioauthorscopus or keyword=biblioauthorwos
}

%%% Убираем неразрывные пробелы перед двоеточием и точкой с запятой %%%
%\makeatletter
%\ifnumequal{\value{draft}}{0}{% Чистовик
%    \renewcommand*{\addcolondelim}{%
%      \begingroup%
%      \def\abx@colon{%
%        \ifdim\lastkern>\z@\unkern\fi%
%        \abx@puncthook{:}\space}%
%      \addcolon%
%      \endgroup}
%
%    \renewcommand*{\addsemicolondelim}{%
%      \begingroup%
%      \def\abx@semicolon{%
%        \ifdim\lastkern>\z@\unkern\fi%
%        \abx@puncthook{;}\space}%
%      \addsemicolon%
%      \endgroup}
%}{}
%\makeatother

%%% Правка записей типа thesis, чтобы дважды не писался автор
%\ifnumequal{\value{draft}}{0}{% Чистовик
%\DeclareBibliographyDriver{thesis}{%
%  \usebibmacro{bibindex}%
%  \usebibmacro{begentry}%
%  \usebibmacro{heading}%
%  \newunit
%  \usebibmacro{author}%
%  \setunit*{\labelnamepunct}%
%  \usebibmacro{thesistitle}%
%  \setunit{\respdelim}%
%  %\printnames[last-first:full]{author}%Вот эту строчку нужно убрать, чтобы автор диссертации не дублировался
%  \newunit\newblock
%  \printlist[semicolondelim]{specdata}%
%  \newunit
%  \usebibmacro{institution+location+date}%
%  \newunit\newblock
%  \usebibmacro{chapter+pages}%
%  \newunit
%  \printfield{pagetotal}%
%  \newunit\newblock
%  \usebibmacro{doi+eprint+url+note}%
%  \newunit\newblock
%  \usebibmacro{addendum+pubstate}%
%  \setunit{\bibpagerefpunct}\newblock
%  \usebibmacro{pageref}%
%  \newunit\newblock
%  \usebibmacro{related:init}%
%  \usebibmacro{related}%
%  \usebibmacro{finentry}}
%}{}

%\newbibmacro{string+doi}[1]{% новая макрокоманда на простановку ссылки на doi
%    \iffieldundef{doi}{#1}{\href{http://dx.doi.org/\thefield{doi}}{#1}}}

%\ifnumequal{\value{draft}}{0}{% Чистовик
%\renewcommand*{\mkgostheading}[1]{\usebibmacro{string+doi}{#1}} % ссылка на doi с авторов. стоящих впереди записи
%\renewcommand*{\mkgostheading}[1]{#1} % только лишь убираем курсив с авторов
%}{}
%\DeclareFieldFormat{title}{\usebibmacro{string+doi}{#1}} % ссылка на doi с названия работы
%\DeclareFieldFormat{journaltitle}{\usebibmacro{string+doi}{#1}} % ссылка на doi с названия журнала
%%% Тире как разделитель в библиографии традиционной руской длины:
% \renewcommand*{\newblockpunct}{\addperiod\addnbspace\cyrdash\space\bibsentence}
%%% 2024!Убрать тире из разделителей элементов в библиографии:
\renewcommand*{\newblockpunct}{%
   \addperiod\space\bibsentence}%block punct.,\bibsentence is for vol,etc.

%%% Возвращаем запись «Режим доступа» %%%
%\DefineBibliographyStrings{english}{%
%    urlfrom = {Mode of access}
%}
%\DeclareFieldFormat{url}{\bibstring{urlfrom}\addcolon\space\url{#1}}

%%% В списке литературы обозначение одной буквой диапазона страниц англоязычного источника %%%
\DefineBibliographyStrings{english}{%
    pages = {p\adddot} %заглавность буквы затем по месту определяется работой самого biblatex
}

%%% В ссылке на источник в основном тексте с указанием конкретной страницы обозначение одной большой буквой %%%
%\DefineBibliographyStrings{russian}{%
%    page = {C\adddot}
%}

%%% Исправление длины тире в диапазонах %%%
% \cyrdash --- тире «русской» длины, \textendash --- en-dash
\DefineBibliographyExtras{russian}{%
  \protected\def\bibrangedash{%
    \cyrdash\penalty\value{abbrvpenalty}}% almost unbreakable dash
  \protected\def\bibdaterangesep{\bibrangedash}%тире для дат
}
\DefineBibliographyExtras{english}{%
  \protected\def\bibrangedash{%
    \cyrdash\penalty\value{abbrvpenalty}}% almost unbreakable dash
  \protected\def\bibdaterangesep{\bibrangedash}%тире для дат
}

%Set higher penalty for breaking in number, dates and pages ranges
\setcounter{abbrvpenalty}{10000} % default is \hyphenpenalty which is 12

%Set higher penalty for breaking in names
\setcounter{highnamepenalty}{10000} % If you prefer the traditional BibTeX behavior (no linebreaks at highnamepenalty breakpoints), set it to ‘infinite’ (10 000 or higher).
\setcounter{lownamepenalty}{10000}

%%% Set low penalties for breaks at uppercase letters and lowercase letters
%\setcounter{biburllcpenalty}{500} %управляет разрывами ссылок после маленьких букв RTFM biburllcpenalty
%\setcounter{biburlucpenalty}{3000} %управляет разрывами ссылок после больших букв, RTFM biburlucpenalty

%%% Список литературы с красной строки (без висячего отступа) %%%
%\defbibenvironment{bibliography} % переопределяем окружение библиографии из gost-numeric.bbx пакета biblatex-gost
%  {\list
%     {\printtext[labelnumberwidth]{%
%       \printfield{prefixnumber}%
%       \printfield{labelnumber}}}
%     {%
%      \setlength{\labelwidth}{\labelnumberwidth}%
%      \setlength{\leftmargin}{0pt}% default is \labelwidth
%      \setlength{\labelsep}{\widthof{\ }}% Управляет длиной отступа после точки % default is \biblabelsep
%      \setlength{\itemsep}{\bibitemsep}% Управление дополнительным вертикальным разрывом между записями. \bibitemsep по умолчанию соответствует \itemsep списков в документе.
%      \setlength{\itemindent}{\bibhang}% Пользуемся тем, что \bibhang по умолчанию принимает значение \parindent (абзацного отступа), который переназначен в styles.tex
%      \addtolength{\itemindent}{\labelwidth}% Сдвигаем правее на величину номера с точкой
%      \addtolength{\itemindent}{\labelsep}% Сдвигаем ещё правее на отступ после точки
%      \setlength{\parsep}{\bibparsep}%
%     }%
%      \renewcommand*{\makelabel}[1]{\hss##1}%
%  }
%  {\endlist}
%  {\item}

%%% Макросы автоматического подсчёта количества авторских публикаций.
% Печатают невидимую (пустую) библиографию, считая количество источников.
% http://tex.stackexchange.com/a/66851/79756
%
\makeatletter
        \newtotcounter{citenum}
        \defbibenvironment{counter}
            {\setcounter{citenum}{0}\renewcommand{\blx@driver}[1]{}} % begin code: убирает весь выводимый текст
            {} % end code
            {\stepcounter{citenum}} % item code: cчитает "печатаемые в библиографию" источники

        \newtotcounter{citeauthorvak}
        \defbibenvironment{countauthorvak}
            {\setcounter{citeauthorvak}{0}\renewcommand{\blx@driver}[1]{}}
            {}
            {\stepcounter{citeauthorvak}}

        \newtotcounter{citeauthorscopus}
        \defbibenvironment{countauthorscopus}
                {\setcounter{citeauthorscopus}{0}\renewcommand{\blx@driver}[1]{}}
                {}
                {\stepcounter{citeauthorscopus}}

        \newtotcounter{citeauthorwos}
        \defbibenvironment{countauthorwos}
                {\setcounter{citeauthorwos}{0}\renewcommand{\blx@driver}[1]{}}
                {}
                {\stepcounter{citeauthorwos}}

        \newtotcounter{citeauthorother}
        \defbibenvironment{countauthorother}
                {\setcounter{citeauthorother}{0}\renewcommand{\blx@driver}[1]{}}
                {}
                {\stepcounter{citeauthorother}}

        \newtotcounter{citeauthorconf}
        \defbibenvironment{countauthorconf}
                {\setcounter{citeauthorconf}{0}\renewcommand{\blx@driver}[1]{}}
                {}
                {\stepcounter{citeauthorconf}}

        \newtotcounter{citeauthor}
        \defbibenvironment{countauthor}
                {\setcounter{citeauthor}{0}\renewcommand{\blx@driver}[1]{}}
                {}
                {\stepcounter{citeauthor}}

        \newtotcounter{citeauthorvakscopuswos}
        \defbibenvironment{countauthorvakscopuswos}
                {\setcounter{citeauthorvakscopuswos}{0}\renewcommand{\blx@driver}[1]{}}
                {}
                {\stepcounter{citeauthorvakscopuswos}}

        \newtotcounter{citeauthorscopuswos}
        \defbibenvironment{countauthorscopuswos}
                {\setcounter{citeauthorscopuswos}{0}\renewcommand{\blx@driver}[1]{}}
                {}
                {\stepcounter{citeauthorscopuswos}}

        \newtotcounter{citeexternal}
        \defbibenvironment{countexternal}
                {\setcounter{citeexternal}{0}\renewcommand{\blx@driver}[1]{}}
                {}
                {\stepcounter{citeexternal}}
\makeatother

\defbibheading{nobibheading}{} % пустой заголовок, для подсчёта публикаций с помощью невидимой библиографии
\defbibheading{pubgroup}{\section*{#1}} % обычный стиль, заголовок-секция
\defbibheading{pubsubgroup}{\noindent\textbf{#1}} % для подразделов "по типу источника"

%%%Сортировка списка литературы Русский-Английский (предварительно удалить dissertation.bbl) (начало)
%%%Источник: https://github.com/odomanov/biblatex-gost/wiki/%D0%9A%D0%B0%D0%BA-%D1%81%D0%B4%D0%B5%D0%BB%D0%B0%D1%82%D1%8C,-%D1%87%D1%82%D0%BE%D0%B1%D1%8B-%D1%80%D1%83%D1%81%D1%81%D0%BA%D0%BE%D1%8F%D0%B7%D1%8B%D1%87%D0%BD%D1%8B%D0%B5-%D0%B8%D1%81%D1%82%D0%BE%D1%87%D0%BD%D0%B8%D0%BA%D0%B8-%D0%BF%D1%80%D0%B5%D0%B4%D1%88%D0%B5%D1%81%D1%82%D0%B2%D0%BE%D0%B2%D0%B0%D0%BB%D0%B8-%D0%BE%D1%81%D1%82%D0%B0%D0%BB%D1%8C%D0%BD%D1%8B%D0%BC
%\DeclareSourcemap{
%	\maps[datatype=bibtex]{
%		\map{
%			\step[fieldset=langid, fieldvalue={tempruorder}]
%		}
%		\map[overwrite]{
%			\step[fieldsource=langid, match=russian, final]
%			\step[fieldsource=presort, 
%			match=\regexp{(.+)}, 
%			replace=\regexp{aa$1}]
%		}
%		\map{
%			\step[fieldsource=langid, match=russian, final]
%			\step[fieldset=presort, fieldvalue={az}]
%		}
%		\map[overwrite]{
%			\step[fieldsource=langid, notmatch=russian, final]
%			\step[fieldsource=presort, 
%			match=\regexp{(.+)}, 
%			replace=\regexp{za$1}]
%		}
%		\map{
%			\step[fieldsource=langid, notmatch=russian, final]
%			\step[fieldset=presort, fieldvalue={zz}]
%		}
%		\map{
%			\step[fieldsource=langid, match={tempruorder}, final]
%			\step[fieldset=langid, null]
%		}
%	}
%}
%Сортировка списка литературы (конец)


\ifnumequal{\value{mediadisplay}}{1}{
    \DeclareSourcemap{
        \maps{%
            \map{% использование media=text по умолчанию
                \step[fieldset=media, fieldvalue=text]
            }
        }
    }
}{}
\ifnumequal{\value{mediadisplay}}{2}{
    \DeclareSourcemap{
        \maps{%
            \map[overwrite]{% удаление всех записей media
                \step[fieldset=media, null]
            }
        }
    }
}{}
\ifnumequal{\value{mediadisplay}}{3}{
    \DeclareSourcemap{
        \maps{
            \map[overwrite]{% стираем значения всех полей media=text
                \step[fieldsource=media,match={text},final]
                \step[fieldset=media, null]
            }
        }
    }
}{}
\ifnumequal{\value{mediadisplay}}{4}{
    \DeclareSourcemap{
        \maps{
            \map[overwrite]{% стираем значения всех полей media=eresource
                \step[fieldsource=media,match={eresource},final]
                \step[fieldset=media, null]
            }
        }
    }
}{}

%%% Создание команд для вывода списка литературы %%%
\newcommand*{\insertbibliofull}{
    \printbibliography[keyword=bibliofull,section=0,title=\bibtitlefull]
    \ifnumequal{\value{draft}}{0}{
      \printbibliography[heading=nobibheading,env=counter,keyword=bibliofull,section=0]
    }{}
}
\newcommand*{\insertbiblioauthor}{
    \printbibliography[heading=pubgroup, section=0, keyword=biblioauthor, title=\bibtitleauthor]
}
\newcommand*{\insertbiblioauthorimportant}{
    \printbibliography[heading=pubgroup, section=2, keyword=biblioauthor, title=\bibtitleauthorimportant]
}

% Вариант вывода печатных работ автора, с группировкой по типу источника.
% Порядок команд `\printbibliography` должен соответствовать порядку в файле common/characteristic.tex
\newcommand*{\insertbiblioauthorgrouped}{
    \section*{\bibtitleauthor}
    \ifsynopsis
    \printbibliography[heading=pubsubgroup, section=0, keyword=biblioauthorvak,    title=\bibtitleauthorvak,resetnumbers=true]
    \else
    \printbibliography[heading=pubsubgroup, section=0, keyword=biblioauthorvak,    title=\bibtitleauthorvak,resetnumbers=false]
    \fi
    \printbibliography[heading=pubsubgroup, section=0, keyword=biblioauthorwos,    title=\bibtitleauthorwos,resetnumbers=false]%
    \printbibliography[heading=pubsubgroup, section=0, keyword=biblioauthorscopus, title=\bibtitleauthorscopus,resetnumbers=false]%
    \printbibliography[heading=pubsubgroup, section=0, keyword=biblioauthorconf,   title=\bibtitleauthorconf,resetnumbers=false]%
    \printbibliography[heading=pubsubgroup, section=0, keyword=biblioauthorother,  title=\bibtitleauthorother,resetnumbers=false]%
}

\newcommand*{\insertbiblioexternal}{
    \printbibliography[heading=pubgroup,    section=0, keyword=biblioexternal,     title=\bibtitlefull]
}
   % Реализация пакетом biblatex через движок biber
}

% Вывести информацию о выбранных опциях в лог сборки
\typeout{Selected options:}
\typeout{Draft mode: \arabic{draft}}
\typeout{Font: \arabic{fontfamily}}
\typeout{AltFont: \arabic{usealtfont}}
\typeout{Bibliography backend: \arabic{bibliosel}}
\typeout{Precompile images: \arabic{imgprecompile}}
% Вывести информацию о версиях используемых библиотек в лог сборки
\listfiles

\begin{document}
\input{common/renames} 
\thispagestyle{empty}
\begin{center}
    \thesisOrganization
\end{center}

\noindent%
\begin{tabularx}{\textwidth}{@{}lXr@{}}%
    & & \large{На правах рукописи}\\
    \IfFileExists{images/logo.png}{\includegraphics[height=2.5cm]{logo}}{\rule[0pt]{0pt}{3.5cm}}  & &
    \ifnumequal{\value{showperssign}}{0}{%
        \rule[0pt]{0pt}{1.5cm}
    }{
        \includegraphics[height=1.4cm]{images/personal-signature.png}
    }\\
\end{tabularx}

\vspace{0pt plus1fill} %число перед fill = кратность относительно некоторого расстояния fill, кусками которого заполнены пустые места
\begin{center}
\textbf {\large \thesisAuthor}
\end{center}

\vspace{0pt plus3fill} %число перед fill = кратность относительно некоторого расстояния fill, кусками которого заполнены пустые места
\begin{center}
\textbf {\Large %\MakeUppercase
\thesisTitle}

\vspace{0pt plus3fill} %число перед fill = кратность относительно некоторого расстояния fill, кусками которого заполнены пустые места
{\large Специальность \thesisSpecialtyNumber\ "---\par <<\thesisSpecialtyTitle>>}

\ifdefined\thesisSpecialtyTwoNumber
{\large Специальность \thesisSpecialtyTwoNumber\ "---\par <<\thesisSpecialtyTwoTitle>>}
\fi


\vspace{0pt plus1.5fill} %число перед fill = кратность относительно некоторого расстояния fill, кусками которого заполнены пустые места
\Large{Автореферат}\par
\large{диссертации на соискание учёной степени\par \thesisDegree}
\end{center}

\vspace{0pt plus4fill} %число перед fill = кратность относительно некоторого расстояния fill, кусками которого заполнены пустые места
{\centering\thesisCity~--- \thesisYear\par}

\newpage
% оборотная сторона обложки
\thispagestyle{empty}
\noindent Работа выполнена в {\thesisInOrganization}.

\vspace{0.008\paperheight plus1fill}
\noindent%
\begin{tabularx}{\textwidth}{@{}lX@{}}
    \ifdefined\supervisorTwoFio
    Научные руководители:   & \supervisorRegalia\par
                              \ifdefined\supervisorDead
                              \framebox{\textbf{\supervisorFio}}
                              \else
                              \textbf{\supervisorFio}
                              \fi
                              \par
                              \vspace{0.013\paperheight}
                              \supervisorTwoRegalia\par
                              \ifdefined\supervisorTwoDead
                              \framebox{\textbf{\supervisorTwoFio}}
                              \else
                              \textbf{\supervisorTwoFio}
                              \fi
                              \par
                              \vspace{0.013\paperheight}
                              \supervisorThreeRegalia\par
                              \ifdefined\supervisorThreeDead
                              \framebox{\textbf{\supervisorThreeFio}}
                              \else
                              \textbf{\supervisorThreeFio}
                              \fi
                              \vspace{0.013\paperheight}\\
    \else
    Научный руководитель:   & \supervisorRegalia\par
                              \ifdefined\supervisorDead
                              \framebox{\textbf{\supervisorFio}}
                              \else
                              \textbf{\supervisorFio}
                              \fi
                              \vspace{0.013\paperheight}\\
                              
    Консультанты:   &         \consultOneRegalia\par
                              \ifdefined\consultOneDead
                              \framebox{\textbf{\consultOneFio}}
                              \else
                              \textbf{\consultOneFio}
                              \fi
                              \par
                              \vspace{0.013\paperheight}
                              \consultTwoRegalia\par
                              \ifdefined\consultTwoDead
                              \framebox{\textbf{\consultTwoFio}}
                              \else
                              \textbf{\consultTwoFio}
                              \fi
                              \vspace{0.013\paperheight}\\
    \fi
    Официальные оппоненты:  &
    \ifnumequal{\value{showopplead}}{0}{\vspace{13\onelineskip plus1fill}}{%
        \textbf{\opponentOneFio,}\par
        \opponentOneRegalia,\par
        \opponentOneJobPlace,\par
        \opponentOneJobPost\par
        \vspace{0.01\paperheight}
        \textbf{\opponentTwoFio,}\par
        \opponentTwoRegalia,\par
        \opponentTwoJobPlace,\par
        \opponentTwoJobPost
    \ifdefined\opponentThreeFio
        \par
        \vspace{0.01\paperheight}
        \textbf{\opponentThreeFio,}\par
        \opponentThreeRegalia,\par
        \opponentThreeJobPlace,\par
        \opponentThreeJobPost
    \fi
    }%
    \vspace{0.013\paperheight} \\
    % \ifdefined\leadingOrganizationTitle
    % Ведущая организация:    &
    % \ifnumequal{\value{showopplead}}{0}{\vspace{6\onelineskip plus1fill}}{%
    %     \leadingOrganizationTitle
    % }%
    % \fi
\end{tabularx}
\vspace{0.008\paperheight plus1fill}

\noindent Защита состоится \defenseDate~на~заседании диссертационного совета \defenseCouncilNumber~при \defenseCouncilTitle~по адресу: \defenseCouncilAddress.

\vspace{0.008\paperheight plus1fill}
\noindent С диссертацией можно ознакомиться в библиотеке НИЯУ МИФИ.

\vspace{0.008\paperheight plus1fill}
\noindent Отзывы на автореферат в двух экземплярах, заверенные печатью учреждения, просьба направлять по адресу: \defenseCouncilAddress, ученому секретарю диссертационного совета~\defenseCouncilNumber.

\vspace{0.008\paperheight plus1fill}
\noindent{Автореферат разослан \synopsisDate.}

\noindent Телефон для справок: \defenseCouncilPhone.

\vspace{0.008\paperheight plus1fill}
\noindent%
\begin{tabularx}{\textwidth}{@{}%
>{\raggedright\arraybackslash}b{18em}@{}
>{\centering\arraybackslash}X
r
@{}}
    Ученый секретарь\par
    диссертационного совета\par
    \defenseCouncilNumber,\par
    \defenseSecretaryRegalia
    &
    \ifnumequal{\value{showsecrsign}}{0}{}{%
        % \includegraphics[width=2cm]{secretary-signature.png}%
    }%
    &
    \defenseSecretaryFio
\end{tabularx}
        % Титульный лист
%\mainmatter                   % В том числе начинает нумерацию страниц арабскими цифрами с единицы
\mainmatter*                  % Нумерация страниц не изменится, но начнётся с новой страницы
\section*{Общая характеристика работы}

\newcommand{\actuality}{\underline{\textbf{\actualityTXT}}}
\newcommand{\progress}{\underline{\textbf{\progressTXT}}}
\newcommand{\aim}{\underline{{\textbf\aimTXT}}}
\newcommand{\tasks}{\underline{\textbf{\tasksTXT}}}
\newcommand{\novelty}{\underline{\textbf{\noveltyTXT}}}
\newcommand{\influence}{\underline{\textbf{\influenceTXT}}}
\newcommand{\methods}{\underline{\textbf{\methodsTXT}}}
\newcommand{\defpositions}{\underline{\textbf{\defpositionsTXT}}}
\newcommand{\reliability}{\underline{\textbf{\reliabilityTXT}}}
\newcommand{\probation}{\underline{\textbf{\probationTXT}}}
\newcommand{\contribution}{\underline{\textbf{\contributionTXT}}}
\newcommand{\publications}{\underline{\textbf{\publicationsTXT}}}

{\actuality}

Подавляющее большинство из $\approx$500 эксплуатируемых и сооружаемых ядерных энергоблоков представляют собой легководные реакторы на тепловых нейтронах \cite{PRISHome}, работающие на ядерном топливе из низкообогащенного урана (НОУ). Работа каждого из энергоблоков создает необходимость в обеспечении его топливом и выборе способа обращения с выгруженным из него облученным ядерным топливом (ОЯТ).

Основным материалом для производства топлива реакторов на тепловых нейтронах является природный уран, обогащаемый в каскадах газовых центрифуг. Но основную часть его мировых запасов можно добыть только при высоких операционных затратах, которые оцениваются как неконкурентоспособные для ядерной генерации энергии по сравнению с другими источниками \cite{Uranium2022,WorldDistributionUranium2018,hartardCompetitionConflictsResource2015}. 

Еще одним вызовом для ядерной промышленности является обращение с ОЯТ, общемировая масса которого превышает 400 килотонн и прирастает ежегодно на $>$15 килотонн \cite{kaygorodcevProblemyPerspektivyRazvitiya2021,UseReprocessedUranium2020WNA}. При этом, основным материалом облученного ядерного топлива является уран, составляющий $\approx$90-95\% его массы, за вычетом конструкционных материалов, концентрация изотопа $^{235}$U в котором, как правило, выше, чем в природном уране, что делает целесообразным его повторное использование \cite{24NikipelovNikipelovSudby}. Его вовлечение в производство ядерного топлива реакторов на тепловых нейтронах может позволить существенно сократить объем захоронения радиоактивных отходов и снизить потребности в природном уране.
% Однако стоит отметить, что реакторы на тепловых нейтронах являются реакторами-<<сжигателями>>, то есть в среднем воспроизводят делящихся материалов значительно меньше, чем распадается в активной зоне реактора в процессе облучения топлива. Этот факт говорит о том, что для реакторов данного типа невозможно полное замыкание ядерного топливного цикла, поскольку для их полноценного обеспечения топливом потребуются внешние источники делящихся материалов, а также обогащение исходной урановой смеси до требуемого содержания $^{235}$U, характерного для топлива легководного реактора ($\approx$5\%). 

Вовлечение регенерата сопряжено с рядом проблем, так как при облучении ядерного топлива в активной зоне реактора образуются искусственные изотопы урана, в первую очередь, $^{232}$U и $^{236}$U. Кроме того, как правило, возрастает и концентрация природного изотопа $^{234}$U. Изотоп $^{232}$U опасен тем, что является родоначальником цепочки распадов, среди дочерних продуктов которых есть,  в частности, $^{208}$Tl, представляющий собой источник жесткого гамма-излучения, обуславливающего высокий уровень радиоактивного фона. Поэтому при производстве уранового топлива существуют нормативные ограничения на допустимое содержание $^{232}$U в низкообогащенном уране. На текущий момент в РФ допустимые концентрации (в мас. долях) $^{232}$U в НОУ не должны превышать предельно допустимых значений: $5\cdot10^{-7}$\% (или, для некоторых случаев $2\cdot10^{-7}$\%). Проблема, связанная с изотопом $^{236}$U, состоит в том, что он является паразитным поглотителем нейтронов в ядерном топливе и, следовательно, отрицательно воздействует на реактивность реактора и глубину выгорания топлива. Для компенсации отрицательного влияния $^{236}$U и получения заданных ядерно-физических характеристик реактора нужно повышать среднее начальное обогащение топлива по $^{235}$U.  При этом, концентрации изотопов $^{232}$U, $^{234}$U и $^{236}$U (четных изотопов), возрастают при обогащении регенерированного урана в ординарных (трехпоточных) каскадах газовых центрифуг, используемых для обогащения природного урана. Фактически это означает необходимость развития способов обогащения регенерированного урана с учетом требований к изотопному составу производимого обогащенного продукта, отвечающих действующим техническим условиям на товарный низкообогащенный уран.

Настоящая работа посвящена разработке эффективных способов решения второй проблемы, связанной с обогащением регенерированного урана.
Перейдем к анализу проблем, возникающих в ее контексте для технологий разделения изотопов. На сегодняшний день предложен ряд технических решений, позволяющих решить задачу обогащения регенерированного урана до концентраций $^{235}$U, требуемых в современных топливных циклах энергетических реакторов на тепловых нейтронах (в частности отечественных ВВЭР), при одновременном выполнении принятых ограничений на содержание $^{232}$U в ядерном топливе и реализации необходимого дообогащения регенерата по $^{235}$U для компенсации негативного влияния $^{236}$U. Тем не менее, далеко не все из них способны решить задачу обогащения регенерата с одновременной коррекцией его изотопного состава в условиях, когда исходное содержание четных изотопов может меняться в широком диапазоне. Последнее обстоятельство особо важно в контексте рассмотрения перспективных реакторов, имеющих относительно высокую глубину выгорания топлива и, как следствие, состав ОЯТ которых может характеризоваться повышенным содержанием четных изотопов (кратно больше предельно допустимых значений). Помимо этого, необходимо учитывать, что замыкание топливного цикла реакторов на тепловых нейтронах подразумевает многократное обращение урана в топливе, что будет обуславливать дополнительное накопление четных изотопов в регенерате от цикла к циклу, учитывая, что при таком подходе исходное топливо на каждом цикле будет содержать четные изотопы еще до загрузки в реактор.

Очевидно, что вопросы коррекции изотопного состава регенерированного урана лежат в области теории и практики разделения изотопных смесей, что делает актуальной для разделительной науки задачу поиска эффективных способов обогащения регенерата урана с одновременной коррекцией его изотопного состава в условиях развития тенденции повышения глубины выгорания топлива и многократного использования урана в нем (многократный рецикл урана). В дополнение к этому важен выбор оптимальной каскадной схемы, которая должна обеспечить максимально эффективное использование ресурса регенерированного урана при минимальных затратах работы разделения.

Разработка способов решения указанных задач возможна с использованием теории каскадов для разделения многокомпонентных изотопных смесей, описывающей массоперенос компонентов в многоступенчатых разделительных установках и позволяющей находить оптимальные условия такого процесса.


{\aim} диссертационной работы является разработка способов обогащения регенерированного урана в каскадах центрифуг при его многократном использовании в регенерированном ядерном топливе для реакторов на тепловых нейтронах.

Для~достижения поставленной цели решены следующие {\tasks}:
\begin{enumerate}[leftmargin=0.5cm]
  \item Выявлены физические ограничения решения задачи обогащения регенерата произвольного изотопного состава в одиночном каскаде и в простых модификациях двойного каскада при одновременном выполнении условий на концентрации изотопов $^{232}$U, $^{234}$U и $^{236}$U в получаемом продукте --- низкообогащенном уране, и расходовании заданной массы регенерата на единицу получаемого продукта.
  \item Предложены модификации двойных каскадов, позволяющие корректировать изотопный состав регенерата по концентрациям изотопов $^{232}$U, $^{234}$U и $^{236}$U с одновременным расходованием максимального количества подлежащего обогащению регенерата при различных исходных концентрациях четных изотопов в нем. Разработаны и апробированы методики расчета и оптимизации предложенной модификации двойного каскада. Показана возможность использования предложенной схемы при различных внешних условиях, а также различных концентрациях четных изотопов в исходном регенерированном уране.
  \item Обоснованы способы вовлечения загрязненной четными изотопами фракции, возникающей в двойных каскадах при очистке от $^{232}$U, с учетом полной или частичной подачи данной фракции: а) в отдельный двойной каскад, осуществляющий наработку низкообогащенного урана для последующей топливной кампании реактора; б) перемешивании этой фракции с потоками обедненного урана и низкообогащенного урана для получения дополнительной массы товарного НОУ; в) в третий каскад с предварительным перемешиванием ее с природным, обедненным и/или низкообогащенным ураном. Выявлены достоинства и недостатки каждого из способов, что позволяет обозначить возможные области их применения. Для предложенной в случае (в) системы каскадов разработана методика оптимизации её параметров по различным критериям эффективности. На основе разработанной методики показана возможность обеспечить экономию природного урана в цикле вплоть до 30\% по отношению к открытому топливному циклу.
  \item Изучены физические закономерности изменения изотопного состава регенерата урана в зависимости от выбора параметров модифицированного двойного каскада при обогащении регенерированного урана с различным исходным содержанием четных изотопов в питающей смеси. Это позволяет обосновать выбор оптимальных с точки зрения заданных критериев эффективности концентраций $^{235}$U в отборах первого и второго каскадов. Показано, что повышение концентрации $^{235}$U в отборе второго каскада обеспечивает более эффективную работу каскадной схемы в целом, позволяя достичь одновременной экономии природного урана и работы разделения по сравнению с открытым ядерным топливным циклом.
  % \item Обобщение и систематизация подходов к выбору каскадной схемы, позволяющих эффективное обогащение регенерированного урана в условиях однократного и многократного рецикла.
  % \item Определение физических закономерностей изменения изотопного состава регенерированного урана и параметров модифицированного двойного каскада для его дообогащения при многократном рецикле урана (отдельно и совместно с плутонием) в топливе реакторов типа ВВЭР.
\end{enumerate}

{\novelty}
\begin{enumerate}[leftmargin=0.5cm]
  \item Разработаны способы обогащения регенерированного урана на основе построения тройных и двойных каскадных схем для вовлечения фракции, загрязненной четными изотопами при обогащении регенерированного урана в двойных каскадных схемах в условиях широкого диапазона изменения внешних условий (концентрации четных изотопов в обогащаемом регенерате и товарном продукте, величины ограничений на концентрации четных изотов и др.).
  \item Предложены методики расчета  и оптимизации различных модификаций двойных каскадов, позволяющие корректировать изотопный состав регенерата по концентрациям изотопов $^{232}$U, $^{234}$U и $^{236}$U с одновременным расходованием всего подлежащего обогащению регенерата при различных исходных концентрациях четных изотопов в нем и различных внешних условиях.
  \item Изучены физические закономерности изменения интегральных характеристик модифицированных двойных и тройных каскадов и  изотопного состава регенерата при его обогащении в них для различных внешних условий и различных составов поступившего в обогащение регенерата. На основе полученных результатов выявлены пределы изменения ключевых параметров каскадных схем, обеспечивающих наиболее эффективные режимы их работы с точки зрения заданных критериев. Показана принципиальная возможность одновременного снижения снижения расхода природного урана и работы разделения по отношению к  открытому топливному циклу.
  \item Предложены способы вовлечения высокоактивного нештатного отхода, образующегося в процессе обогащения регенерированного урана в модифицированном двойном каскадe, в воспроизводство ядерного топлива. Оценена величина дополнительной экономии природного урана, возникающей в топливном цикле, за счет предложенных способов, которая может достигать 7\%.
\end{enumerate}

{\influence} 
\begin{enumerate}[leftmargin=0.5cm]
  \item Разработаны модификации двойных и тройных каскадов, позволяющие обогащать регенерированный уран с одновременным выполнением ограничений на концентрации четных изотопов и вовлечением требуемой массы регенерата.
  \item Разработаны методики оптимизации параметров предложенных в диссертации двойного и тройного каскадов, позволяющие находить наиболее эффективные с точки зрения таких критериев, как расход работы разделения, расход природного урана, степень извлечения $^{235}$U, наборы их параметров, при одновременном возврате всей массы регенерированного урана в цикл и выполнении ограничений по концентрациям четных изотопов. Предложенные методики оптимизации систем каскадов могут быть адаптированы к расчету и оптимизации параметров различных вариантов каскадных схем для разделения многокомпонентных смесей неурановых элементов.
  \item Разработанные каскадные схемы и методики их расчета и оптимизации могут быть использованы в расчетных группах на предприятиях и организациях, связанных как с проектированием и построением разделительных каскадов, так и непосредственным производством изотопной продукции (АО «Уральский электрохимический комбинат», АО «Сибирский химический комбинат», АО «ТВЭЛ», АО «Восточно-Европейский головной научно-исследовательский и проектный институт энергетических технологий», АО «ПО «ЭХЗ» и др.). Имеется акт об использовании результатов диссертационной работы в НИЦ <<Курчатовский институт>> от 05.12.2023 г.
  \item Разработанные каскадные схемы и методики их расчета могут лечь в основу имитационных моделей и цифровых двойников технологий топливного цикла реакторов на тепловых нейтронах, использующих регенерированное урановое топливо.  
\end{enumerate}

{\methods}
Исследование проводит систематизацию научно-технической литературы, посвященной заявленной теме.
Применены подходы, известные в современной теоретической физике, и в частности, в теории разделения изотопов в каскадах.
В работе теоретически обоснованы принципы построения анализируемых каскадов, разработаны программные коды расчета и оптимизации их параметров для различных постановок задач, проведено их компьютерное моделирование.
При разработке программных кодов использована теория квазиидеального каскада. При подготовке программных кодов использованы современные программные средства языков программирования Julia и Python и подключаемых библиотек, таких как NLsolve.jl, Optim.jl, SciPy, предназначенных для решения систем нелинейных уравнений и оптимизационных процедур, Matplotlib и PGFPlots.jl для визуализации результатов.

{\defpositions}
\begin{enumerate}[leftmargin=0.5cm]
  \item Способы обогащения регенерата урана с одновременным выполнением условий на концентрации четных изотопов и максимальным вовлечением исходного материала в многокаскадных схемах в широком диапазоне изменения внешних условий. Методики оптимизации предложенных каскадных схем (модифицированный двойной каскад, тройной каскад).
  \item Критерии определения возможности/невозможности получения необходимого количества конечного продукта на основе регенерированного урана различного исходного состава путем его обогащения в одиночных и двойных каскадах.
  \item Способы вовлечения загрязненной четными изотопами фракции, получаемой в двойных каскадах при обогащении регенерата, в воспроизводство ядерного топлива.
\end{enumerate}

{\reliability}.
Надежность, достоверность и обоснованность научных положений и выводов, сделанных в диссертации, следует из корректности постановки задач, физической обоснованности применяемых приближений, использования методов, ранее примененных в аналогичных исследованиях, взаимной согласованности результатов. Корректность результатов вычислительных экспериментов гарантируется тестами и операторами проверки соответствия ограничениям, верифицирующими строгое выполнение заданных условий и соблюдение условий сходимости балансов (массовых и покомпонентных).

{\probation}
Результаты, изложенные в материалах диссертации, доложены и обсуждены на конференциях:
\begin{itemize}[leftmargin=0.4cm]
  \item V Международная научная конференция молодых ученых, аспирантов и студентов «Изотопы: технологии, материалы и применение», г. Томск, Россия, 2018 г.;
  \item VI Международная научная конференция молодых ученых, аспирантов и студентов «Изотопы: технологии, материалы и применение», г. Томск, Россия, 2020 г.;
  \item 15th International Workshop on Separation Phenomena in Liquids and Gases (SPLG-2019), г. Уси, Китай, 2019 г.;
  \item 16th International Workshop on Separation Phenomena in Liquids and Gases (SPLG-2021), г. Москва, Россия, 2021 г.;
  \item XVII International conference and School for young scholars “Physical chemical processes in atomic systems”, г. Москва, Россия, 2019 г..
\end{itemize}

По теме диссертации опубликовано 9 печатных работ, в том числе 5 в изданиях, индексированных в международной системе цитирования Scopus, и 4 -- в журналах из перечня ВАК. Автор принимал участие в следующих проектах, поддержанных Российским научным фондом (РНФ), в которых были использованы некоторые из результатов диссертационной работы: 
\begin{itemize}[leftmargin=0.4cm]
  \item Разработка каскадных схем для эффективного получения изотопно-модифицированных материалов для топливных циклов перспективных ядерных реакторов и других приложений (2018---2020 гг.);
  \item Оптимизация стационарного и нестационарного массопереноса в многокаскадных схемах для получения стабильных изотопов и обогащения регенерированного урана (2020---2022 гг.).
\end{itemize}


{\contribution} Автор принимал участие разработке каскадных схем, написании программных кодов, проведении вычислительных экспериментов, а также в обработке и анализе результатов вычислительных экспериментов.


\section*{Содержание работы}
Во \underline{\textbf{введении}} обоснована актуальность разработки каскадных схем для обогащения регенерированного урана, вытекающая из задач долгосрочного устойчивого развития ядерной энергетики, а также из существующих на сегодня ограничений/недостатков ранее предложенных схем. Сформулирована цель исследования, состоящая в разработке эффективных способов обогащения регенерированного урана в каскадах центрифуг при его многократном использовании в регенерированном ядерном топливе для реакторов на тепловых нейтронах. Помимо этого, во \underline{\textbf{введении}} сформулированы научная новизна и практическая значимость выполненной работы, изложены основные положения, выносимые на защиту, обоснована достоверность полученных в работе результатов и представлены сведения об их апробации.

\underline{\textbf{Первая глава}} посвящена критическому анализу ранее предложенных каскадных схем обогащения регенерированного урана, а также краткому обзору источников по промышленному опыту обогащения регенерата урана. Проанализирована проблема четных изотопов $^{232,234,236}$U в задаче обогащения регенерированного урана с точки зрения разделительных технологий. Известно, что изотопы $^{232,234}$U ухудшают радиационные характеристики ядерного топлива (ЯТ), содержание $^{232}$U в НОУ-продукте ограничено мерами радиационной безопасности персонала на разделительном и фабрикационном производстве значениями $2\cdot10^{-7} \%$ или $5\cdot10^{-7} \%$, предельно допустимое отношение $\frac{C_{234,{P}}}{C_{235,{P}}} = 0,02$. Изотоп $^{236}$U вносит <<паразитный>> захват тепловых нейтронов в ЯТ, приводя к необходимости повышения его обогащения по $^{235}$U, что увеличивает затраты работы разделения в цикле. 

Описан процесс многократного использования (рецикла) урана в топливе реакторов на тепловых нейтронах (рис. \ref{fig_autoref1}) и подлежащие рассмотрению в диссертационной работе его стадии, а именно: обогащение регенерированного урана в каскадах центрифуг. Одним из факторов, осложняющих многократный рецикл урана является рост концентраций четных изотопов в процессе рецикла, что требует модификации известных каскадных схем, поскольку концентрации четных изотопов могут превышать допустимые пределы в несколько раз, что для примера проиллюстрировано взятыми из открытых источников составами регенерированного урана, прошедшего несколько рециклов (таблица \ref{is_compositions_2_5autoref}). Также при реализации схемы рис. \ref{fig_autoref1} подразумевают, что при производстве свежего топлива для реактора используют весь выделенный из ОЯТ этого же реактора регенерат, что проиллюстрировано на схеме рис. \ref{fig_autoref2}.  Такой подход призван обеспечить: (1) минимизацию потерь  $^{235}$U в топливном цикле; (2) максимально эффективно использовать потенциал ОЯТ для воспроизводства топлива; (3) исключить нежелательное накопление регенерата в процессе его многократного рецикла.
\begin{table}[h]
  \centering
  \caption{{Изотопные составы регенерата различных циклов.{\label{is_compositions_2_5autoref}}}}
  \begin{tabular}{|c||c|c|c|c|c|c|}
  \hline {\tiny Состав} & {\tiny Массовое число} & 232 & 233 & 234 & 235 & 236 \\
  \hline 1 & C, \% & $6,62\cdot10^{-7}$ & $1,19\cdot10^{-6}$ & $3,28\cdot10^{-2}$ & 1,43 & $9,93\cdot10^{-1}$ \\
  2 & C, \% &  $1,03\cdot10^{-6}$ & $1,3\cdot10^{-6}$ & $3,91\cdot10^{-2}$ & 1,07 & 1,45 \\\hline
  \end{tabular}
\end{table}

\begin{figure}[ht]
  \centerfloat{\includegraphics[scale=0.021]{cascades/recycling_ru}}
  \caption{Схема многократного рециклирования урана}\label{fig_autoref1}
\end{figure}

\begin{figure}[ht]
  \centerfloat{\includegraphics[scale=0.25]{cascades/ordinary/recycling1kg_ru}}
  \caption{Схема замыкания урановой топливной составляющей}\label{fig_autoref2}
\end{figure}

Приводится формулировка задачи обогащения регенерата, которой посвящена диссертационная работа: получение заданной массы товарного НОУ требуемого обогащения по $^{235}$U из заданной массы сырьевого регенерата урана (в том числе многократно рециклированного) с одновременным выполнением ограничений на концентрации четных изотопов. 

Показана невозможность использования ординарного (трехпоточного) каскада (рис. \ref{ordinary}) для решения сформулированной выше задачи в условиях многократного рецикла урана. Такой каскад можно использовать только для обогащения составов регенерата, в которых исходные концентрации четных изотопов меньше (на порядок или более), чем их допустимые пределы в товарном НОУ, что заведомо невыполнимо при многократном рецикле урана в современных реакторах на тепловых нейтронах (таблица \ref{is_compositions_2_5autoref}).

\begin{figure}[ht]
  \centerfloat{\includegraphics[scale=0.025]{cascades/ordinary}}
  \caption{Схема ординарного трехпоточного каскада. Обозначения: $F$ --- поток питания; $P$ --- поток отбора; $W$ --- поток отвала}\label{ordinary}
\end{figure}

Анализ ранее предложенных способов обогащения регенерата позволяет условно разделить рассмотренные способы на 3 типа: (1) схемы с разбавлением четных изотопов; (2) схемы с отделением четных изотопов; (3) «гибридные» схемы (комбинируют первые два способа).
Показано, что лишь некоторые из известных способов потенциально могут решить поставленную задачу обогащения регенерата в условиях варьирования содержания четных изотопов в обогащаемом регенерате. Это делает необходимым дальнейший поиск каскадных схем для решения задачи, которые можно применять для различных исходных составов регенерированного урана, а также в случаях возможного изменения внешних условий задачи. 

\underline{\textbf{Во второй главе}} приведены основные понятия и определения теории разделения изотопов в каскадах. Введены понятия  разделительного элемента, разделительной ступени, разделительного каскада и возможных вариантов соединения ступеней в каскаде. Изложены основные сведения, необходимые для моделирования разделения многокомпонентных изотопных смесей в каскадах, описаны модели «квазиидеального» каскада, как частного случая симметричного противоточного каскада, и $R$-каскада. Рассмотрены основные варианты постановок задач расчета таких каскадов и алгоритмы их решения, которые будут использованы в 3-й и 4-й главах. 

\underline{\textbf{Третья глава}} посвящена анализу причин, затрудняющих или делающих невозможным использование описанных в главе 1 способов обогащения регенерата в условиях многократного рецикла. Рассмотрены различные варианты однокаскадных схем (рис. \ref{fig:diagram1ch3}), а также двойной каскад. В качестве схем на основе ординарного каскада рассмотрены:

\begin{enumerate}
  \item Схема с разбавлением природным ураном предварительно обогащенного регенерата (рис. \ref{fig:diagram1ch3}.a);
  \item Схема с разбавлением предварительно обогащенного регенерата низкообогащенным ураном (рис. \ref{fig:diagram1ch3}.b);
  \item Схема с разбавлением предварительно обогащенного природного урана регенератом (рис. \ref{fig:diagram1ch3}.c);
  \item Схема с разбавлением регенерата природным ураном перед подачей в ординарный трехпоточный каскад (рис. \ref{fig:diagram1ch3}.d).
\end{enumerate}

Для рассмотренных схем проведена серия вычислительных экспериментов, в которых варьировали параметры каждой из них при решении задачи обогащения регенерата, сформулированной выше. По результатам проведенных расчетов: (1) показана нецелесообразность использования схем на основе ординарных каскадов для обогащения регенерата в условиях многократного рецикла, так как они не позволяют решить задачу в большинстве случаев; (2) показана возможность обогащения регенерата с превышенными относительно допустимых пределов концентрациями четных изотопов в двойном каскаде. Однако двойной каскад позволяет только решить задачу, а именно обогатить регенерат по изотопу $^{235}$U и снизить концентрации четных изотопов, не решая, при этом, задачи полного использования регенерата. Другой проблема, связанной с использованием двойного каскада является высокое содержание $^{236}$U в получаемом НОУ-продукте (до 3\%), что приводит к необходимости повышения обогащения по $^{235}$U вплоть до 6-7\% и, соответственно, росту затрат работы разделения в топливном цикле на десятки процентов по сравнению с открытым ЯТЦ.

\begin{figure}[ht]
  \centerfloat{\includegraphics[scale=0.02]{cascades/ord_all}}
  \caption{Схемы обогащения регенерата на основе одиночного ординарного каскада. Обозначения: $E$ --- поток питающего схему регенерата, $F_n$ --- поток разбавителя (природного урана или низкообогащенного урана ($F_{leu}$)); $W$ --- поток отвального ОГФУ тяжелого конца каскада; $P$ --- товарный низкообогащенный уран}\label{fig:diagram1ch3}
\end{figure}

В третьей главе рассмотрен двойной каскад (рис. \ref{fig:double_ru_in3}), представляющий собой последовательное соединение двух каскадов, позволяющих сконцентрировать легкие четные изотопы отдельно от изотопа $^{235}$U. Для этого сначала в каскаде I обогащают изотоп $^{235}$U с одновременным обогащением изотопов $^{232}$U, $^{234}$U, $^{236}$U, а затем полученную смесь направляют на вход каскада II (рис. \ref{fig:double_ru_in3}), где она делится на две группы: в первой обогащены легкие изотопы ($^{232}$U, $^{234}$U и $^{235}$U), во второй обедняется $^{235}$U с более интенсивным обеднением $^{232}$U, $^{234}$U, что позволяет в потоке $W_2$ получить НОУ, отвечающий требованиям по концентрациям изотопов $^{232}$U, $^{234}$U с одновременной компенсацией $^{236}$U. Анализ проведенных расчетов обосновывает необходимость разработки каскадных схем для решения задачи возврата регенерированного урана в ЯТЦ.

\begin{figure}[ht]
  \centerfloat{\includegraphics[scale=0.055]{cascades/Double_core_pure}}
  \caption{Двойной каскад. Обозначения: $E$ --- поток питающего схему регенерата, $W_1$ --- поток отвального ОГФУ тяжелого конца каскада; $P$ ($W_2$) --- конечный НОУ продукт на основе регенерата; $P_2$ --- отход двойного каскада в виде высокообогащенного урана}\label{fig:double_ru_in3}
\end{figure}


В \underline{\textbf{четвертой главе}} описаны предлагаемые в диссертации способы решения задачи обогащения регенерата в условиях его многократного рецикла.
В качестве базового предложенного способа рассмотрен <<двойной модифицированный каскад>> (рис. \ref{p2left_autoref}). Данный способ можно рассматривать в качестве развития способа, предложенного в патенте АО <<СХК>> №2282904 \cite{EXTvodolazskihSposobIzotopnogoVosstanovleniya}. Идея работы предлагаемого способа состоит в следующем. В каскаде I обогащают исходный регенерат изотопами $^{232,233,234,235,236}$U. В каскаде II смесь делится на две фракции, так, чтобы в потоке тяжелой фракции ($W_2$) было понижено содержание $^{232,233,234}$U по отношению к питающей второй каскад смеси --- потоку $P_1$, при этом обогащение по $^{235}$U в потоке $W_2$ составляет величину несколько выше, чем требуется для товарного НОУ. Затем происходит разбавление потока $W_2$ смесью, не содержащей искусственных изотопов урана для выполнения ограничений по $^{232}$U и $^{236}$U, которая нарабатывается в каскаде III. В результате такого смешивания и получают финальный продукт --- товарный НОУ заданной массы и отвечающий всем требованиям по концентрациям четных изотопов. 

\begin{figure}[ht]
  \centerfloat{\includegraphics[scale=0.025]{cascades/DoubleModified23}}
  \caption{Схема модифицированного двойного каскада для обогащения регенерированного урана. Обозначения: $E$ --- поток регенерированного урана; $P_1$ --- поток отбора первого каскада, выступающий питанием второго каскада; $P_2$ --- поток отбора второго каскада; $W_1$ --- поток отвала первого каскада; $W_2$ --- поток тяжелой фракции (условный «отвал») второго каскада; $P_3$ --- поток НОУ-разбавителя на основе природного урана $F_3$; $P$ --- финальный продукт (товарный низкообогащенный уран (НОУ))}\label{p2left_autoref}
\end{figure}

Для каскадной схемы рис. \ref{p2left_autoref} в диссертационной работе предложена методика расчета и оптимизации ее параметров при решении задачи обогащения регенерата со всеми ограничениями. Предложенная методика основана на современных методах оптимизации функции многих переменных и может быть обобщена на случай оптимизации по различным критериям эффективности. Следует отметить, что предложенный подход к оптимизации схемы не имеет аналогов в литературе, поскольку впервые проведена оптимизация многокаскадной схемы как единой системы, в отличие от ранее использованных подходов с отдельной оптимизацией каждого из каскадов. С использованием разработанной методики оценена эффективность модифицированного двойного каскада при решении задачи обогащения урана в различных условиях и при оптимизации по таким критериям эффективности как:

\begin{enumerate}
  \item Минимум расхода природного урана ($(\frac{\Delta A}{P})_\text{min}$);
  \item Минимум затрат работы разделения ($(\frac{F_n}{P})_\text{min}$);
  \item Максимум степени извлечения $^{235}$U в схеме ($(Y_f)_\text{max}$);
  \item Максимум степени извлечения $^{235}$U из исходного регенерата ($(Y_{E})_\text{max}$).
\end{enumerate}  

В результате проведенных вычислительных экспериментов показано, что с использованием предложенной схемы даже для составов регенерированного урана с концентрациями четных изотопов выше предельных значений для товарного НОУ, возможно добиться экономии природного урана на уровне 15\% и выше при практически нулевом перерасходе или даже экономии затрат работы разделения. 

Проведено сравнение ранее предложенных каскадных схем и схемы рис. \ref{p2left_autoref} по ключевым характеристикам (расход природного урана, затраты работы разделения), во многом определяющим удельные затраты на товарный НОУ. Сравнение оптимальных параметров модифицированного двойного каскада с аналогичными характеристиками ранее известных способов обогащения регенерата показало преимущества предложенного способа по отношению к ним. Это выражается как в самом факте решения задачи, по отношению к способам, неспособным решить задачу, так и в лучших значениях расхода природного урана и затрат работы разделения по отношению к способам, решающим поставленную задачу. Результаты такого сравнения на примере обогащения регенерата состава 1 (таблица \ref{is_compositions_2_5autoref}) приведены в таблице \ref{allaut}. 

Для предложенного способа обогащения регенерата (рис. \ref{p2left_autoref}) проанализирована также его «устойчивость» к изменению внешних условий таких, как:
\begin{itemize}
  \item требуемое обогащение по изотопу $^{235}$U ($C_{235,P}$ от  4,4\% до  5,5\%);    
  \item величина предельно допустимой концентрации изотопа $^{232}$U в НОУ-продукте (варьировалась в интервале от $1\cdot10^{-7}$\% до $1\cdot10^{-6}$\%);
  \item расход регенерированного урана на единицу продукта ($E/P$ от 0,93 до 2,79).
\end{itemize}

Полученные результаты показали возможность решения задачи в широком диапазоне внешних условий, поскольку для всех рассмотренных комбинаций внешних параметров задачи были найдены решения, т.е. подобраны параметры модифицированного двойного каскада, обеспечившие решение задачи. Таким образом, схема применима как при текущих параметрах топливного цикла и требованиях к товарному НОУ, так и потенциально может быть применена при их изменении.

\begin{table}[ht]
  \centering
  \caption{Сравнение интегральных показателей (параметров П) схем для состава 1.{\label{allaut}}}
  \begin{tabular}{|c|c|c|c|c|c|c|}
      \hline \diagbox{{\tinyП}}{{\tiny Схема}} & $\text{1}$ & $\text{2}$ & $\text{3}$ & $\text{4}$ & $\text{5}$ & $\text{6}$\\ \hline
      $\text{$Y_{f}$}$ & 78,89 & 86,44 & 40,26 & 88,98 & 89,01 & 86,90\\ \hline
      $\text{$Y_{E}$}$ & 78,89 & 48,19 & 1,00  & ---   & 89,01 & 86,90\\ \hline
      $\text{$\delta(\frac{\Delta A}{P}), \%$}$ & 1,63 & 11,01 & $29,12$ & 11,01 & 4,17 & 4,77\\ \hline % (11.82-11.26)/11.82
      $\text{$\delta(\frac{F_n}{P}), \%$}$ & 21,1 & 15,19 & 12,86 & $17,0$ & 6,17 & $19,29$\\ \hline
      $\text{$\frac{P_{2}}{P}$}$ & $0$ & $0$ & $0$ & $0$ & $0$ & $5,17\cdot10^{-3}$\\ \hline
      $\text{$\frac{E}{P}$}$ & $4,38$ & \cellcolor{red!25}0,76 & \cellcolor{red!25}0,60 & \cellcolor{red!25}0,76 & 4,71 & $0,93$\\ \hline
      % $\text{$C_{232,P}, {\tiny \cdot10^{-7}} \%$}$ & \cellcolor{red!25}29,04 & 5,00 & 3,97 & 5,00 & 5,00 & 5,00\\ \hline
      $C_{232,P},${\tiny $\cdot10^{-7}$}\% & \cellcolor{red!25}29,04 & 5,00 & 3,97 & 5,00 & 5,00 & 5,00\\ \hline

      $\frac{C_{234,P}}{C_{235,P}}$ & \cellcolor{red!25}{\tiny $2,39\cdot10^{-2}$} & {\tiny $1,11\cdot10^{-2}$} & {\tiny $1,10\cdot10^{-2}$} & {\tiny $1,11\cdot10^{-2}$} & {\tiny $1,95\cdot10^{-2}$} & $1,20\cdot10^{-2}$\\ \hline
      $\text{$C_{235,P}, \%$}$ & $5,95$ & $5,10$ & $5,12$ & $5,12$ & $6,01$ & $5,10$\\ \hline
      $\text{$C_{236,P}, \%$}$ & $3,40$ & {\tiny $5,11\cdot10^{-1}$} & {\tiny $5,96\cdot10^{-1}$} & {\tiny $5,99\cdot10^{-1}$} & $3,62$ & $6,79\cdot10^{-1}$\\ \hline
    \end{tabular}   
\end{table}

Разработаны способы дальнейшего использования загрязненной легкими изотопами $^{232,234}$U фракции (таблица \ref{P2_compositions_autoref}), получаемой в потоке $P_2$ двойного модифицированного каскада (рис. \ref{p2left_autoref}), которая в рассмотренном случае составляет $\approx$0,3\% от массы НОУ-продукта. Использование данной фракции призвано предотвратить нежелательное накопление на разделительном производстве высокоактивных отходов, а также задействовать остаточное содержание $^{235}$U в этом потоке, которое может достигать 20\% и более. Предложены следующие 3 способа: 

\begin{table}[h]
  \centering
  \caption{{Изотопный состав $P_2$.{\label{P2_compositions_autoref}}}}
    \begin{tabular}{|c|c|c|c|c|c|}
    \hline Массовое число & 232 & 233 & 234 & 235 & 236 \\
    \hline C, \% & $2,48\cdot10^{-5}$ & $5,44\cdot10^{-5}$ & 0,69 & 19,76 & 7,34 \\ \hline
  \end{tabular}
\end{table}

\begin{itemize}
  \item Перемешивание $P_2$ с регенератом, поступающим на обогащение (рис. \ref{P2utilizationRingautoref});
  \item Получение дополнительной массы товарного НОУ (рис. \ref{P2utilizationautoref});
  \item Перемешивание $P_2$ с обедненным ураном и последующее обогащение (рис. \ref{p2_withDepU}).
\end{itemize}

\begin{figure}[ht]
  \centerfloat{\includegraphics[scale=0.02]{cascades/P2utilizationRing}}
  \caption{Схема передачи загрязненного изотопом $^{232}$U состава гексафторида урана в двойном каскаде от первой партии дообогащенного регенерированного урана к последующей. Обозначения: $E$ --- поток регенерированного урана; $P_1$ --- поток отбора первого каскада, выступающий питанием второго каскада; $W_1$ --- поток отвала первого каскада; $W_2$ --- поток тяжелой фракции (условный «отвал») второго каскада; $P_3$ --- поток НОУ-разбавителя; $P$ --- финальный продукт (товарный низкообогащенный уран (НОУ)); $P_2$ --- поток отбора второго каскада, который подается на питание последующего двойного каскада, перемешиваясь с регенератом очередного рецикла}\label{P2utilizationRingautoref}
\end{figure}

\begin{figure}[ht]
  \centerfloat{\includegraphics[scale=0.023]{cascades/P2utilization}}
  \caption{Схема независимого вовлечения в производство НОУ загрязненной изотопом $^{232}$U фракции, смешанной с обедненным и природным ураном}\label{P2utilizationautoref}
\end{figure}

Проведен сравнительный анализ предложенных вариантов вовлечения загрязненной четными изотопа фракции. Каждый из рассмотренных способов вовлечения $P_2$ в ЯТЦ демонстрирует повышение эффективности использования $^{235}$U находящегося в регенерированном уране, что позволяет получить дополнительное увеличение экономии природного урана.

Для способа вовлечения легкой фракции путем ее перемешивания с регенератом, поступающим на обогащение (рис. \ref{P2utilizationRingautoref}), проведены вычислительные эксперименты по топливоподготовке (обогащение регенерата с целью производства низкообогащенного урана) для серии частичных перегрузок топлива в реакторе (замена части ТВС активной зоны реактора)\footnote{формулировка задачи последовательного обогащения регенерата для нескольких перегрузок реактора со всеми условиями осуществлена совместно с сотрудниками НИЦ <<Курчатовский институт>>.}. Каждая из серий расчетов отличалась выбранным критерием эффективности, в качестве которых использованы $(Y_f)_\text{max}$, $(Y_{E})_\text{max}$, $(\delta(\frac{\Delta A}{P}))_\text{min}$, $(\delta(\frac{F_n}{P}))_\text{min}$, $(\frac{P_2}{P})_\text{min}$. По результатам анализа полученных закономерностей, с каждой последующей перегрузкой происходит снижение эффективности схемы по каждому из показателей.

Для способа независимого получения дополнительной массы товарного НОУ (рис. \ref{P2utilizationautoref}), была показана возможность получить дополнительную экономию природного урана относительно двойного модифицированного каскада.

Для оценки эффективности способа вовлечения в ЯТЦ легкой фракции путем ее перемешивания с обедненным ураном и последующим обогащением, представленного на рис. \ref{p2_withDepU} разработан алгоритм расчета такой многокаскадной схемы. Предложенный подход обобщается на случай использования различных критериев эффективности. Проведенные с ее помощью серии вычислительных экспериментов показывают возможность обеспечить экономию природного урана и затрат работы разделения по отношению к открытому ЯТЦ даже в случае обогащения регенерата с высоким содержанием $^{232}$U (выше предельных значений для товарного НОУ).

\begin{figure}[ht]
  \centerfloat{\includegraphics[scale=0.02]{cascades/triple_cascade23}}
  \caption{Тройной каскад для обогащения регенерированного урана. Обозначения: $E$ --- поток регенерированного урана; $P_1$ --- поток отбора первого каскада, выступающий питанием второго каскада; $P_2$ --- поток отбора второго каскада; $F_{D}$ --- поток ОГФУ-разбавителя, смешиваемого с $P_2$ перед подачей на вход третьего каскада; $W_1$ --- поток отвала первого каскада; $W_2$ --- поток тяжелой фракции (условный «отвал») второго каскада; $P_3$ --- поток НОУ-разбавителя на основе природного урана $F_3$; $P$ --- финальный продукт (товарный низкообогащенный уран (НОУ)), полученный смешиванием потоков $W_2$, $P_3$ и $P_4$, где $P_4$ --- отбор третьего каскада; $W_4$ --- отвал третьего каскада}\label{p2_withDepU}
\end{figure}

По результатам исследования схем, представляющих собой различные способы использования побочной фракции $P_2$, представлена сравнительная таблица \ref{3loopautoref}.
\begin{table}
  \centering
  \caption{Сравнение интегральных показателей способов вовлечения загрязненного продукта для состава 1. Обозначения: П --- параметр, ср. --- среднее, сп. --- способ.{\label{3loopautoref}}}
  \renewcommand{\arraystretch}{1.2}
  \begin{tabular}{|r|c|c|c|c|c|c|c|c|c|}
    \hline
    \multirow{2}{*}{П} & \multicolumn{3}{c|}{сп. 1} & \multicolumn{3}{c|}{сп. 2} & \multicolumn{3}{c|}{сп. 3}\\
    \cline{2-10}
    & {\tiny Загр.} 1 & {\tiny Загр.} 2 & ср. & {\tiny Загр.} 1 & {\tiny Загр.} 2 & ср. & {\tiny Загр.} 1 & {\tiny Загр.} 2 & ср. \\
    \hline
    $\frac{F_n}{P}$   & 6,40 & 6,49 & 6,45    & 6,40  & 6,40  & 6,40    & 6,23 & 6,23 & 6,23\\ \hline
    $\frac{\Delta A}{P}, \textit{{\tiny ЕРР}}$ & 11,26 & 11,61 & 11,43 & 11,26 & 11,29 & 11,27   & 11,66 & 11,66 & 11,66 \\ \hline
    $\frac{P_2}{P}, \%$  & 0,49 & 2,96 & 1,73    & 0,49 & 0,91 & 0,72        & 0 & 0 & 0 \\ \hline
    $\frac{E}{P}$        & 0,93 & 0,95 & 0,94    & 0,93 & 1,08 & 1,01     & 0,93 & 0,93 & 0,93 \\ \hline
  \end{tabular}
\end{table}

Как следует из анализа данных таблицы \ref{3loopautoref}, использование схемы независимого вовлечения $P_2$ (сп. 2, представленная на рис. \ref{P2utilizationautoref}) для формирования Загр. 2, позволяет, по сравнению со схемой с замыканием (сп. 1, Загр. 2, рис. \ref{P2utilizationRingautoref}), расходовать меньшее удельное количество природного урана и работы разделения (на $\approx$1,3\% и $\approx$2,7\%) на этапе производства НОУ-продукта для осуществления перегрузки топлива в реакторе, и снижать количество отхода в виде $P_2$ в более чем три раза. То есть, схема независимого вовлечения $P_2$ показывает лучшие результаты по ключевым (интегральным) показателям, по сравнению со схемой модифицированного двойного каскада.

Способ 3 (тройной каскад) позволяет задействовать все требуемое количество регенерата на каждой перегрузке, а также позволяет добиться наименьшего расхода природного урана, обеспечивая $\approx$4\% выигрыша, по сравнению со способом 1 на этапе производства урана для Загр. 2, при этом перерасходуя работу разделения лишь на $\approx$0,5\% и не производя побочного отхода с высоким содержанием $^{232,234}$U, такого как $P_2$.

\newpage

\pdfbookmark{Заключение}{conclusion}
В \underline{\textbf{заключении}} перечислены полученные ключевые результаты диссертационного исследования и сформулированы его основные выводы.

\noindent \begin{enumerate}[leftmargin=0.4cm]
\item Предложен модифицированный двойной каскад с НОУ-разбавителем из природного урана, применимый  для обогащения регенерированного урана в условиях многократного рецикла урана в топливе легководных реакторов и позволяющий получить продукт, отвечающий всем требованиям на концентрации четных изотопов. 
\noindent \begin{enumerate}[leftmargin=0.4cm]
    \item На основе теории квазиидеального каскада разработаны методики расчета и оптимизации предложенной каскадной схемы по различным критериям эффективности (затраты работы разделения, расход природного урана, степень извлечения $^{235}$U из регенерата, степень извлечения $^{235}$U из всех питающих потоков схемы). Показано, что эффективность предложенной каскадной схемы по тому или иному критерию зависит от выбранного диапазона изменения концентрации $^{235}$U в потоке легкой фракции каскада II. Наиболее выгодные с точки зрения выбранных критериев эффективности наборы параметров каскадной схемы лежат в области, где концентрация $^{235}$U в потоке легкой фракции каскада II превышает 20\%. Это означает, что при практической реализации модифицированного двойного каскада целесообразно рассматривать возможность получения в отдельных потоках такой схемы концентраций $^{235}$U, превышающих 20\%, и, в первую очередь, в потоке $P_2$. 
    \item Анализ эффективности предложенной каскадной схемы с точки зрения потерь $^{235}$U показал, что схема обеспечивает экономию природного урана по сравнению с открытым топливным циклом на уровне 15-20\% в зависимости от исходного изотопного состава регенерата. Это превышает аналогичные показатели для простейших разбавляющих схем практически вдвое.
    \item Предложенная схема позволяет полностью решить задачу обогащения регенерата в широком диапазоне внешних условий и ограничений, что создает базис для ее практической реализации и поиска наиболее эффективных режимов ее работы.
\end{enumerate}

\item Показано, что модификации ординарного каскада для обогащения и разбавления регенерированного урана принципиально не решают задачу обогащения регенерированного урана при одновременном выполнении условий на концентрации четных изотопов в товарном НОУ и обеспечения расходования заданной массы регенерата на получение этого НОУ для составов регенерата с исходным содержанием четных изотопов, превышающим предельные значения для товарного НОУ. 

Основная причина невозможности решения задачи состоит в том, что в рассматриваемых схемах число свободных параметров оказывается меньшим, чем число условий, которые необходимо одновременно удовлетворить. В результате такие схемы могут обеспечить решение задачи только в частных случаях, когда в обогащение поступает регенерированный уран с исходными концентрациями четных изотопов ниже предельных значений для товарного НОУ.

\item Обоснованы способы вовлечения загрязненной четными изотопами фракции, возникающей в двойных каскадах при очистке от $^{232}$U, с учетом полной или частичной подачи данной фракции: а) в отдельный двойной каскад, осуществляющий наработку низкообогащенного урана для последующей топливной кампании реактора; б) перемешивании этой фракции с потоками обедненного урана и низкообогащенного урана для получения дополнительной массы товарного НОУ; в) в третий каскад с предварительным перемешиванием ее с природным, обедненным и/или низкообогащенным ураном. Для каждого из способов проанализированы их достоинства и недостатки, и вытекающие из них области применения, а также рассчитаны получаемые преимущества относительно открытого ЯТЦ.
 
\item Результаты работы применимы для проведения дальнейшего технико-экономического анализа каждой из схем на основе их интегральных показателей, таких как расход природного урана, затраты работы разделения, потери $^{235}$U в цикле в контексте всей цепочки ядерного топливного цикла, а также с учетом возникающих в этой цепочке изменений при использовании регенерата урана по отношению к открытому топливному циклу. Полученные в диссертации результаты дополняют теорию каскадов для разделения изотопов. В частности, предложенные в работе методики оптимизации двойных и тройных каскадов могут быть адаптированы к случаю разделения многокомпонентных смесей неурановых изотопов в каскадах центрифуг.

\end{enumerate}


\insertbibliofull   
\pdfbookmark{Литература}{bibliography}












































% \ifdefmacro{\microtypesetup}{\microtypesetup{protrusion=false}}{} % не рекомендуется применять пакет микротипографики к автоматически генерируемому списку литературы
% \urlstyle{rm}                               % ссылки URL обычным шрифтом
% \ifnumequal{\value{bibliosel}}{0}{% Встроенная реализация с загрузкой файла через движок bibtex8
%     \renewcommand{\bibname}{\large \bibtitleauthor}
%     \nocite{*}
%     \insertbiblioauthor           % Подключаем Bib-базы
%     %\insertbiblioexternal   % !!! bibtex не умеет работать с несколькими библиографиями !!!
% }{% Реализация пакетом biblatex через движок biber
%     % Цитирования.
%     %  * Порядок перечисления определяет порядок в библиографии (только внутри подраздела, если `\insertbiblioauthorgrouped`).
%     %  * Если не соблюдать порядок "как для \printbibliography", нумерация в `\insertbiblioauthor` будет кривой.
%     %  * Если цитировать каждый источник отдельной командой --- найти некоторые ошибки будет проще.
%     %
%     %% authorvak
%     \nocite{smirnovObogashchenieRegenerirovannogoUrana2018}%
%     % \nocite{vakbib2}%
%     % %
%     % %% authorwos
%     % \nocite{wosbib1}%
%     % %
%     % %% authorscopus
%     % \nocite{smirnovApplyingEnrichmentCapacities2018}%
%     % %
%     % %% authorpathent
%     % \nocite{patbib1}%
%     % %
%     % %% authorprogram
%     % \nocite{progbib1}%
%     % %
%     % %% authorconf
%     \nocite{smirnovApplyingEnrichmentCapacities2018}%
%     % \nocite{confbib2}%
%     % %
%     % %% authorother
%     % \nocite{bib1}%
%     % \nocite{bib2}%

%     \ifnumgreater{\value{usefootcite}}{0}{
%         \begin{refcontext}[labelprefix={}]
%             \ifnum \value{bibgrouped}>0
%                 \insertbiblioauthorgrouped    % Вывод всех работ автора, сгруппированных по источникам
%             \else
%                 \insertbiblioauthor      % Вывод всех работ автора
%             \fi
%         \end{refcontext}
%     }{
%         \ifnum \totvalue{citeexternal}>0
%             \begin{refcontext}[labelprefix=A]
%                 \ifnum \value{bibgrouped}>0
%                     \insertbiblioauthorgrouped    % Вывод всех работ автора, сгруппированных по источникам
%                 \else
%                     \insertbiblioauthor      % Вывод всех работ автора
%                 \fi
%             \end{refcontext}
%         \else
%             \ifnum \value{bibgrouped}>0
%                 \insertbiblioauthorgrouped    % Вывод всех работ автора, сгруппированных по источникам
%             \else
%                 \insertbiblioauthor      % Вывод всех работ автора
%             \fi
%         \fi
%         %  \insertbiblioauthorimportant  % Вывод наиболее значимых работ автора (определяется в файле characteristic во второй section)
%         \begin{refcontext}[labelprefix={}]
%             \insertbibliofull            % Вывод списка литературы, на которую ссылались в тексте автореферата
%         \end{refcontext}
%         % Невидимый библиографический список для подсчета количества внешних публикаций
%         % Используется, чтобы убрать приставку "А" у работ автора, если в автореферате нет
%         % цитирований внешних источников.
%         \printbibliography[heading=nobibheading, section=0, env=countexternal, keyword=biblioexternal, resetnumbers=true]%
%     }
% }
% \ifdefmacro{\microtypesetup}{\microtypesetup{protrusion=true}}{}
% \urlstyle{tt} 
      % Содержание автореферата
\appendix
\begin{refsection}   
    \phantom{
\cite{nevinicaToplivnyyCiklLegkovodnogo2019,
    2024smirnovObogashchenieRegenerirovannogoUrana2018,
    % smirnovApplyingEnrichmentCapacities2018,
    vantrodionova2019,
    % smirnovMethodEnrichReprocessed2019,
    smirnovFizikotehnicheskieProblemyObogashcheniya2020,
    % gusevMultycascadeEnrichmentSchemes2020,
    smirnovAnalysisEffectRestrictions2021}}
\printbibliography[heading=subbibliography, title=\bibtitleauthor]
\end{refsection}


% \printbibliography[section=0, env=countexternal, keyword=biblioexternal, resetnumbers=true]
%%% Выходные сведения типографии
\newpage\thispagestyle{empty}

\vspace*{0pt plus1fill}

\small
\begin{center}
    \textit{\thesisAuthor}
    \par\medskip
    \thesisTitle
    \par\medskip

    Автореф. дис. на соискание ученой степени \thesisDegreeShort
    \par\bigskip

    Подписано в печать \blank[\widthof{999}].\blank[\widthof{999}].\blank[\widthof{99999}].
    Заказ № \blank[\widthof{99999999999}]

    Формат 60\(\times\)90/16. Усл. печ. л. 24. Тираж 50 экз.
    %Это не совсем формат А5, но наиболее близкий, подробнее: http://ru.wikipedia.org/w/index.php?oldid=78976454

    Типография \blank[0.5\linewidth]
\end{center}
\cleardoublepage

\end{document}