\chapter*{Список сокращений и условных обозначений}\label{acronims} % Заголовок
\addcontentsline{toc}{chapter}{Список сокращений и условных обозначений}  % Добавляем его в оглавление
\noindent
%\begin{longtabu} to \dimexpr \textwidth-5\tabcolsep {r X}
\begin{longtabu} to \textwidth {r X}
% Жирное начертание для математических символов может иметь
% дополнительный смысл, поэтому они приводятся как в тексте
% диссертации

\(q_0\) & коэффициент разделения\\
\(\theta\) & коэффициент деления потоков смеси (срез)\\
\(N\) & длина каскада (число ступеней)\\

\(\begin{rcases}
    f\\
    N+1-f
    \end{rcases}\)  &
    число ступеней в обеднительной и обогатительной частях каскада
\\

\(\begin{rcases}
    n\\
    k
    \end{rcases}\)  &
    индексы целевого ($^{235}$U) и опорного компонент разделяемой изотопной смеси
\\

\(\begin{rcases}
    F_i\\
    P_i\\
    W_i
    \end{rcases}\)  &
    потоки питания, отбора и отвала, где \textit{i} -- индекс каскада
\\

\(\begin{rcases}
    G_i\\
    G'_i\\
    G''_i
    \end{rcases}\)  &
    парциальные потоки \textit{i}-го компонента в потоках питания, отбора и отвала
\\

\(\phi _{i}\) & срез парциальных потоков \textit{i}-го компонента\\

\(\begin{rcases}
    C_{i,F}\\
    C_{i,P}\\
    C_{i,W}
    \end{rcases}\)  &
    концентрации \textit{i}-го компонента в потоках питания, отбора и отвала каскада
\\
% \($R_{ik}$\) & относительная концентрация \textit{i}-го к \textit{k}-му компоненту\\

\textbf{ЛВР} & легководный реактор \\
\textbf{ВВЭР} & водо-водяной энергетический реактор \\
\textbf{PWR} & водо-водяной энергетический реактор западного дизайна (Pressurized water reactor)\\
\textbf{ЯТЦ} & ядерный топливный цикл \\
\textbf{ЗЯТЦ} & замкнутый ядерный топливный цикл \\
\textbf{ТВС} & тепловыделяющая сборка \\
\textbf{ОТВС} & облученная тепловыделяющая сборка \\
\textbf{MOX-топливо} & ядерное топливо, состоящее из смеси диоксидов урана и плутония \\
\textbf{ОЯТ} & Облученное ядерное топливо, извлеченное из ядерного реактора после использования и для этой цели в имеющейся форме более непригодноe \\

\textbf{РАО} & Радиоактивные отходы. Существуют подклассы радиоактивных отходов: высокоактивные (ВАО), среднеактивные (САО), низкоактивные (НАО) \\



\textbf{НОУ} & низкообогащенный уран \\
\textbf{ВОУ} & высокообогащенный уран\\
\textbf{ОГФУ} & обедненный гексафторид урана\\
\textbf{RepU} & регенерированный уран \\
\textbf{E} & регенерированный уран \\
\textbf{n} & природный уран \\

\textbf{РР} & работа разделения\\
\textbf{ЕРР} & 1 кг работы разделения, единица работы по разделению изотопов. Мера усилий, затрачиваемых на разделение материала определённого изотопного состава на две фракции с отличными изотопными составами; не зависит от применяемого процесса разделения. \\

\textbf{$UF_6$} & гексафторид урана\\
\textbf{$C_{8}H_{3}F_{13}$} & фреон-346\\

\textbf{ASTM} & международное общество по испытаниям и материалам\\

\textbf{СНАУ} & система нелинейных алгебраических уравнений \\

\textbf{СМ.} & смесь изотопных составов\\

\textbf{ord.} & ординарный каскад\\

\end{longtabu}

Критерии эффективности каскадной схемы:
\begin{itemize}
    \item $(Y_f)_\text{max}$ -- максимум суммарной степени извлечения схемы (\ref{Rec2}) , соответствующий минимуму потерь $^{235}$U в схеме;
    \item $(Y_{E})_\text{max}$ -- максимум степени извлечения из регенерата (\ref{RecR2}), соответствующий минимуму потерь $^{235}$U регенерата;
    \item $(\delta(\frac{\Delta A}{P}))_\text{min}$ -- максимум экономии работы разделения, относительно референтной схемы трехпоточного каскада для обогащения природного урана, соответствующая минимуму удельного расхода работы разделения; 
    \item $(\delta(\frac{F_n}{P}))_\text{min}$\ -- максимальная экономия природного урана относительно референтной схемы трехпоточного каскада для обогащения природного урана, соответствующая минимуму удельного расхода природного урана.
\end{itemize}\label{criteria_list}

Обозначения параметров каскадных схем:
\begin{itemize}
    \item $Y_f$ -- суммарная степень извлечения $^{235}$U в схеме \ref{Rec2};
    \item $Y_{E}$ -- степень извлечения $^{235}$U схемой из регенерированного урана \ref{RecR2};
    \item $\delta(\frac{\Delta A}{P})$ -- экономия работы разделения относительно референтной схемы трехпоточного каскада для обогащения природного урана. Наибольшая экономия соответствует минимуму суммарного потока схемы \ref{GrindEQ__1_73_}. Если величина отрицательная, абсолютное значение соответствует потерям работы разделения, по сравнению с референтной схемы трехпоточного каскада для обогащения природного урана.
    \item  $\frac{F_n}{P}$ -- удельный расход природного урана на единицу производимого товарного НОУ, где $F_n$ -- поток природного урана, питающего каскад;
    \item  $\delta(\frac{F_n}{P})$ -- экономия природного урана относительно референтной схемы трехпоточного каскада для обогащения природного урана.  Наибольшая экономия соответствует минимуму удельного расхода природного урана схемы. Если величина отрицательная, абсолютное значение соответствует перерасходу природного урана, по сравнению с референтной схемы трехпоточного каскада для обогащения природного урана;
    \item $M_{k1}$ -- масса изотопа, выбранного в качестве опорного компонента при расчете $R$-каскада (\ref{GrindEQ__1_75_})--(\ref{GrindEQ__1_76_}), для первого каскада в схеме, в который поступает регенерат;
    \item $M_{k2}$ -- масса изотопа, выбранного в качестве опорного компонента при расчете $R$-каскада (\ref{GrindEQ__1_75_})--(\ref{GrindEQ__1_76_}), для второго каскада в схеме, на питание которого поступает поток легкой фракции первого каскада;
    \item $C_{232,\text{P}},C_{234,\text{P}},C_{235,\text{P}},C_{236,\text{P}}$ -- концентрации изотопов урана в конечном НОУ-продукте $P$;
    \item $C_{232,\text{x}},C_{234,\text{x}},C_{235,\text{x}},C_{236,\text{x}}$ -- концентрации изотопов урана в потоках $x$;
    \item $F_{x}$ -- выходные потоки, выраженные в килограммах гексафторида урана ($UF_6$), получаемые в схеме при производстве 1 тонны металического урана НОУ-продукта.
  \end{itemize}
  
\addtocounter{table}{-1}% Нужно откатить на единицу счетчик номеров таблиц, так как предыдущая таблица сделана для удобства представления информации по ГОСТ
