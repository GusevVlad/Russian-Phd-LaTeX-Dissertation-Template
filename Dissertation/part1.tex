% \chapter{Анализ литературы, патентов и обзор практики обогащения регенерированного урана в разделительных каскадах}\label{ch1}
\chapter{Анализ литературы, патентов и обзор опыта применения разделительных каскадов для обогащения регенерированного урана}\label{ch1}

\section{Проблема $^{232,234,236}$U при обогащении регенерированного урана}

Регенерированный уран, являющийся основным источником делящихся материалов ОЯТ, может быть обогащен с использованием передового промышленного метода обогащения природного урана --- газовой центрифуги --- для производства на его основе так называемого регенерированного уранового топлива (РУТ).

Интерес к проблеме вовлечения регенерированного урана в топливный цикл существует уже не одно десятилетие. Первые работы, посвященные этой проблеме, относятся к 1970–1980-м годам \cite{kazukihidaSimultaneousEvaluationEffects1986,sidenkoIssledovanieKaskadnyhShem,delagarzaUranium236LightWater1977,ksenof88,borodynyaIssledovanieProblemyVovlecheniya1989,raysIzgotovlenieOksidnogoTopliva1994,zhiroEkonomicheskiePreimushchestvaPererabotki1997,lebedevZamknutyyToplivnyyCikl1999,psheninZaklyuchitelnyyOtchetNIR2012}. Одновременно начали развиваться и подходы к обогащению регенерированного урана. В результате предложены различные варианты каскадных схем для обогащения регенерированного урана. Помимо этого, с использованием некоторых из предложенных каскадных схем проведен ряд исследований, направленных на изучение закономерностей многократного рецикла урана в составе РУТ и смешанных видов топлива легководных реакторов, в частности, типа ВВЭР \cite{smirnovEvolutionIsotopicComposition2012,kazukihidaSimultaneousEvaluationEffects1986,blandinskiySoglasovannyyPodhodModelirovaniyu2018,colemanEvaluationMultipleSelfrecycling2010}. 

С обогащением регенерата связаны и некоторые сложности, обусловленные появлением в уране в процессе его облучения в реакторе нежелательных искусственных изотопов $^{232,234,236}$U. Эти <<четные>> изотопы усложняют обогащение регенерата урана, поскольку их содержание в конечном товарном продукте --- низкообогащенном уране --- строго регламентировано, что требует очистки от них в процессе обогащения.

Наличие строгих ограничений на <<четные>> изотопы обусловлено их нейтронно-физическими и радиационными свойствами \cite{smirnovEvolutionIsotopicComposition2012, proselkovAnalizVozmozhnostiIspolzovaniya2003, dudnikovInfluence236UEfficacy2016}.

Для примера приведём изотопные составы регенерированного урана, соответствующего однократному и двукратному облучению в реактору типа ВВЭР (таблица \ref{is_compositions_2_5}) \cite{palkinDesignanalyticalResearchRefinement2010,nevinicaToplivnyyCiklLegkovodnogo2019}.

\begin{table}[h]
  \centering
  \caption{{Изотопные составы регенерата различных циклов. Обозначения: С --- концентрация.{\label{is_compositions_2_5}}}}
  \normalsize\begin{tabulary}{1.0\textwidth}{|c|c|c|c|c|c|c|}
  \hline Состав & Массовое число & 232 & 233 & 234 & 235 & 236 \\
  \hline 1 & C, \% & $6,62\cdot10^{-7}$ & $1,19\cdot10^{-6}$ & $3,28\cdot10^{-2}$ & 1,43 & 0,9932 \\
  2 & C, \% &  $1,03\cdot10^{-6}$ & $1,3\cdot10^{-6}$ & $3,91\cdot10^{-2}$ & 1,07 & 1,45 \\\hline
  \end{tabulary}
\end{table}

Следует подчеркнуть, что нежелательные изотопы $^{232,234,236}$U не могут быть отделены от целевого $^{235}$U химическим путём. Поэтому единственная возможность решения проблемы состоит в коррекции изотопного состава регенерата в процессе его обогащения до нужного содержания $^{235}$U. 
% Они могут быть удалены только с использованием технологий разделения изотопов, что и затрудняет обогащение регенерированного урана изотопом $^{235}$U для его возврата в ЯТЦ.

Кратко проанализируем нежелательные свойства изотопов $^{232,234,236}$U. Изотоп $^{232}$U является родоначальником длинной цепочки распадов, в которую входят нуклиды-излучатели жёстких гамма-квантов.
Основным дочерним источником интенсивного гамма-излучения (2,6 МэВ) является короткоживущий $^{208}$Tl ($t_{\frac{1}{2}}=3,65$ мин.) \cite{matveevUran232EgoVliyanie1985,abbasProliferationResistanceFeatures2013}. Гамма активность облученного урана достигает своего пикового значения через $\approx$10 лет после извлечения облученной тепловыделяющей сборки (ОТВС) из активной зоны реактора \cite{gresleyEnrichingRecyclingUranium1988}.
% Опасность на производстве также представляет еще один дочерний изотоп урана-232 --- $^{220}$Rn (торон) вследствие его эманирования в воздух рабочей зоны.

Изотоп $^{234}$U является активным $\alpha$-источником, который присутствует и в уране природного происхождения. Однако в регенерированном уране его содержание оказывается выше, чем в природной смеси \cite{matveevUran232EgoVliyanie1985,kryuchkovObogashchennyyUranDobavleniem2007}. При этом, $^{234}$U, лишь частично выгорает в ходе облучения на протяжении реакторной кампании \cite{gresleyEnrichingRecyclingUranium1988}. Поэтому действующие технические условия ограничивают содержанием данного изотопы во избежании осложнения радиационной обстановки при обращении с низкообогащенным ураном, в первую очередь, на заводах по изготовлению ядерного топлива.

$^{236}$U, являясь паразитным поглотителем тепловых нейтронов, препятствует развитию цепной ядерной реакции, тем самым ухудшая нейтронно-физические свойства ядерного топлива. Кроме того, после захвата нейтрона изотопом  $^{236}$U конечным продуктом цепочки его распада является изотоп  $^{232}$U \cite{ksenof88}. Этот фактор способствует росту концентрации $^{232}$U при многократном рецикле урана. 
Эффект отравления реактора, заключающийся в снижении его реактивности из-за захвата нейтронов изотопом  $^{236}$U, должен быть скомпенсирован дополнительным количеством делящегося $^{235}$U в топливе. Для обеспечения требуемого эквивалента уровня обогащения по $^{235}$U, к заданной концентрации $^{235}$U в продукте для случая обогащения природного урана необходимо обеспечить добавку делящегося $^{235}$U.
Ее величина определяется концентрацией $^{236}$U:
$(C_{235,P})_\textit{экв.}=(C_{235,P})_\textit{прир.}+\Delta C_{235}$, где $\Delta C_{235}$ соответствует некоторой функции. В простейшем случае компенсирующую добавку рассчитывают как линейную функции от концентрации $^{236}$U: $f(C_{236,P})=K_{236} \times C_{236,P}$, где $K_{236}$ --- это коэффициент компенсации реактивности. Его значение в зависимости от нейтронных характеристик топливной кампании может находиться в пределах 0,2--0,6 \cite{delagarzaMulticomponentIsotopeSeparation1961, borodynyaIssledovanieProblemyVovlecheniya1989,delculAnalysisReuseUranium2009}.


Отметим также, что $^{234}$U имеет тенденцию захватывать нейтрон и превращаться в делящийся $^{235}$U, что должно уменьшить необходимую компенсацию $^{236}$U \cite{dyachenkoIspolzovanieRegenerirovannogoUrana2012}. Однако во многих расчетных исследованиях этот фактор не учитывается ввиду его слабого влияния.

% Также может в некоторых случаях может быть необходимым принимать во внимание то обстоятельство, что изотопы $^{232}$U вместе с $^{234}$U привносят альфа-частицы в смесь гексафторида урана ($UF_6$ --- соединение, используемое в процессе обогащения урана \cite{orlovWayObtainUranium2015, orlovDesublimationPurificationTransporting2017}), что может приводить к его диссоциации,
% а значит к нежелательному появлению и дальнейшему осаждению в ступенях каскада легких компонентов, таких как, например, свободный фтор ($F_2$) \cite{kryuchkovObogashchennyyUranDobavleniem2007, bernhardtRadiationEffectsAlpha1958, shmelevRazrabotkaRaschetnoyModeli2012}. 

% Содержание этих изотопов в низкообогащенном продукте может регулироваться различными стандартами, такими как, например, ASTM C996 - 15 \cite{c26committeeSpecificationUraniumHexafluoride}.

\section{Задача обогащения регенерированного урана с точки зрения разделительных технологий}

Специфика задачи обогащения регенерированного урана заключается в том, что она представляет собой более сложную разделительную проблему, чем, обогащение  природного урана.
Это обусловлено тем, что, регенерированный уран нельзя рассматривать как квазибинарную изотопную смесь, что усложняет процесс разделения. Кроме того, помимо обогащения целевого изотопа --- $^{235}$U, при решении этой задачи необходимо одновременно выполнить ограничения на еще три изотопа --- $^{232}$U, $^{234}$U и $^{236}$U.

В этой связи, начиная с 1980-х годов появляются публикации и патенты, направленные на поиск эффективного решения задачи обогащения регенерированного урана \cite{ksenof88,borodynyaIssledovanieProblemyVovlecheniya1989,smirnovKaskadnyeShemyZadachah2012,sulaberidzeNekotoryhRazdelitelnyhProblemah2004,kazukihidaSimultaneousEvaluationEffects1986,sidenkoIssledovanieKaskadnyhShem,2024smirnovObogashchenieRegenerirovannogoUrana2018,prusakovKorrekciyaIzotopnogoSostava2008}. Однако для многих из них теоретическое обоснование проводили, опираясь на относительно <<чистый>> состав регенерата, соответствующий ОЯТ реакторов ВВЭР-440 или РБМК. С развитием новых поколений реакторов и изменением характерных для них глубин выгорания и уровней обогащения используемого топлива многие из предложенных на текущий момент способов не могут эффективно решить рассматриваемую задачу.
Другим немаловажным фактором является то, что в большинстве предложенных способов подразумевается, что обогащению подлежит регенерат, который получен из облученного топлива, изготовленного из природного урана. В настоящий же момент активно изучают вопросы многократного рециклирования урана, когда регенерированное урановое топливо восстанавливается и используется несколько раз \cite{rodionovaAnalizTehnikoekonomicheskihHarakteristik2019,smirnovFizikotehnicheskieProblemyObogashcheniya2020}. Суть этого процесса можно пояснить на схеме рисунка \ref{recycle}.

\begin{figure}[ht]
  \centerfloat{\includegraphics[scale=0.0225]{cascades/recycling_ru}}
  \caption{Схема многократного рециклирования урана}\label{recycle}
\end{figure}

\begin{figure}[ht]
  \centerfloat{\includegraphics[scale=0.04]{cascades/ordinary}}
  \caption{Схема ординарного трехпоточного каскада. Обозначения: $F$ --- поток питания; $P$ --- поток отбора; $W$ --- поток отвала}\label{ordinary}
\end{figure}

В соответствии со схемой рис. \ref{recycle} предполагаем, что первичная загрузка реактора осуществлена топливом, изготовленном из обогащенного природного урана. Далее, при обогащении регенерированного урана природный уран используется в качестве материала подпитки, обеспечивающего необходимое количество дополнительного $^{235}$U для изготовления требуемой массы свежего топлива. Далее, процесс рецикла урана с добавлением природного сырья повторяют $N$ раз. Заметим, что материал подпитки необходим в рассматриваемой принципиальной схеме рецикла, так как в реакторе на тепловых нейтронах не достижимо расширенное воспроизводство ядерного топлива, из-за того, что коэффициент воспроизводства делящегося нуклида $^{235}$U в таком типе реакторов меньше единицы \cite{ignatevVliyanieVidaTopliva2020}. При этом очевидно, что природный уран является наиболее удобным материалом для добавлению к регенерату при производстве топлива новой загрузке. Тем не менее, вместо природного урана можно использовать и другие доступные урановые смеси, в которых либо очень низкая, либо нулевая концентрация четных изотопов. Например, это может быть НОУ с обогащением 1-1,5\%, обедненный уран и др..


Как следует из анализа результатов исследований, посвященных вопросам многократного рецикла урана в топливе легководных реакторов, в ходе рециклирования происходит рост (до нескольких раз) концентраций четных изотопов в регенерате после облучения в реакторе \cite{smirnovEvolutionIsotopicComposition2012}. При этом ввиду относительной малости концентрации $^{232}$U на первом (или, в некоторых случаях, на первых двух) рециклах не происходит достижения концентрацией этого изотопа предельных значений в финальном продукте \cite{smirnovApplyingEnrichmentCapacities2018}.
После роста концентраций четных изотопов на первых рециклах, наблюдается их постепенный выход на «плато», начиная с $\approx$3-го рецикла, что обусловлено фиксацией концентрации изотопа $^{232}$U в продукте на уровне $5\cdot10^{-7}$\%, что доказывает возможность многократного рециклирования облученной урановой топливной составляющей.

Таким образом, при анализе вопросов замыкания топливного цикла реакторов на тепловых нейтронах с использованием регенерированного урана необходимо учитывать такие факторы, как общая тенденция к повышению глубины выгорания в современных реакторах, так и рост концентраций четных изотопов в процессе рециклирования урана \cite{andrianovaPovyshenieVygoraniyaTopliva2008}. Это факторы делают актуальными разработку каскадных обогатительных схем, позволяющих эффективно использовать регенерированный уран при производстве товарного НОУ с учетом всех описанных выше требований и ограничений, в том числе, в условиях многократного рецикла урана.

Отметим ещё один важный фактор. Очевидно, что для извлечения максимальной выгоды из осуществляемой переработки ОЯТ, целесообразно максимально вовлекать в повторное использование весь выделенный из него регенерат. Это означает, что если рассматривать отдельный реактор, то логично при получении НОУ из регенерированного урана использовать при его производстве весь выделенный из ОЯТ этого же реактора регенерат. Это будет означать, во-первых, минимизацию потерь $^{235}$U в топливном цикле, во-вторых, максимально эффективное использование потенциала ОЯТ для воспроизводства топлива, а, в-третьих, отсутствие нежелательного накопления регенерата на складах \cite{tendallNuclearEnergyEurope2011}. При этом следует сделать акцент на том, что подобное условие не является физическим требованием, а скорее призвано повысить эффективность замыкания топливного цикла реакторов на тепловых нейтронах по урановой составляющей. Схематично это условие иллюстрирует рисунок \ref{reconeto}.  

\begin{figure}[ht]
  \centerfloat{\includegraphics[scale=0.35]{cascades/ordinary/recycling1kg_ru}}
  \caption{Схема замыкания урановой топливной составляющей}\label{reconeto}
\end{figure}

Учитывая сказанное выше, задача обогащения регенерата в общем случае может быть сформулирована как: получение заданной массы товарного НОУ требуемого обогащения по $^{235}$U из сырьевого регенерата урана (в том числе многократно рециклированного) с одновременным выполнением ограничений на концентрации четных изотопов при условии расходования всей массы регенерата, выделенного из ОЯТ данного реактора.

Таким образом, с обогащением регенерата урана в каскадах газовых центрифуг связаны определенные сложности, требующие модификации подходов, принятых на разделительных производствах при обогащении природного урана. Все перечисленные факторы обуславливают актуальность разработки в области поиска оптимальных каскадных схем для обогащения регенерированного урана с учетом требований, предъявляемых к получаемому продукту --- НОУ.

Оптимальность той или иной каскадной схемы зависит от выбранных критериев эффективности. В качестве таких критериев, как правило, используют минимум затрат работы разделения и расхода природного урана для получения единицы товарного НОУ. Эти характеристики в значительной мере определяют величину удельных затрат на получение товарного НОУ.


\section{Промышленный опыт}\label{sec:ch1/sec1}

Возврат урана в топливный цикл по представленной выше схеме опирается на три ключевые технологии:
\begin{enumerate}
  \item Радиохимическую переработку ОЯТ;
  \item Изотопное обогащение регенерированного урана;
  \item Изготовление топлива на основе восстановленного отработавшего топлива.
\end{enumerate}

Что касается первого пункта, в странах, лидирующих в развитии ядерных технологий, с середины прошлого века широко используется технология гидрометаллургической переработки облученного топлива, называемая PUREX \cite{selvaduraySurveyNuclearFuel1979}. В России, технологии связанные с переработкой ОЯТ развиваются особенно успешно благодаря ориентированности отрасли на замыкание ЯТЦ \cite{balihinSostoyaniiPerspektivahRazvitiya2018, efimenkoProblemyPerspektivyRazvitiya2017}. В виду такого стратегического курса отечественной атомной отрасли, запланирован ввод новых мощностей, которые рассчитаны на переработку принимаемого ОЯТ из-за рубежа \cite{050519L3942005}. С 2016 г. на <<ФГУП ПО <<МАЯК>> осуществляется переработка партий ОТВС ВВЭР-1000 \cite{PyatyyNacionalnyyDoklad}.

Что касается технологии изотопного обогащения урановых смесей, российская атомная промышленность имеет опыт обогащения регенерированного урана из реакторов ВВЭР-440, который затем использовался в качестве топлива РБМК \cite{VVER10001200Za}. Для этого используют метод прямого обогащения в трехпоточной каскадной схеме. Такая схема реализована для производства исходного сырья для изготовления топлива РБМК на заводе РТ-1 \cite{volkVozvratUranaIz2010}. Этот вариант также апробирован для изготовления опытных тепловыделяющих сборок (ТВС) для реакторов ВВЭР, требующих более высокого уровня обогащения \cite{proselkovAnalizVozmozhnostiIspolzovaniya2003}.
% Здесь важно также отметить, что на сегодняшний день у топливного дивизиона Росатома имеется уникальный технологический задел, связанный с газоцентрифужной технологией, который уже сегодня отражен в доминирующей роли этой технологии на мировом рынке разделительных услуг за счет низкой себестоимости единицы работы разделения, которую обеспечивают энергоэффективные и долговечные разделительные аппараты.

Что касается заключительного пункта, Росатом на одном из заводов фабрикации ядерного топлива осуществлял изготовление опытных образцов тепловыделяющих сборок в том числе на основе зарубежного облученного топлива (из Франции) с повышенным содержанием $^{232}$U \cite{kislovRadiacionnyeAspektyIspolzovaniya}.

Имеющийся в России опыт рециклирования ядерного топлива базируется на смешении регенератов урана, извлекаемых из ОЯТ ВВЭР и ОЯТ транспортных реакторов с высоким содержанием $^{235}$U \cite{international2003iaea}.

При этом зарубежный опыт базируется на однократном использовании MOX-топлива \cite{international2003iaea}.

% При этом, опираясь на передовой уровень разделительной технологии, можно заключить, что задача обогащения регенерата до необходимого для повторного использования в энергетических ядерных реакторах уровня концентрации изотопа $^{235}$U может быть решена.

Таким образом, сложившаяся к текущему моменту в России научно-производственная база с наращиваемыми объемами промышленных разделительных мощностей, основанных на центробежном методе разделения, является основным аргументом в пользу готовности к вовлечению регенерата в топливный цикл легководных реакторов.

Однако, для практической реализации долгосрочных планов отрасли по замыканию ЯТЦ и расширению предложения международных топливных поставок, что предусматривает многократное рециклирование делящихся материалов, необходимо решить задачу возврата регенерата в ЯТЦ, подразумевая наличие вышеизложенных ограничений \cite{RosatomGoskorporaciyaRosatoma,panteleyOsobennostiMezhdunarodnogoSotrudnichestva2017}.

Для анализа возможности решения задачи рецикла урана в рамках поставленных ограничений с помощью ранее предложенных схем, перейдем к их подробному рассмотрению.

\section{Обзор способов обогащения регенерата урана в каскадах центрифуг}

Ниже приведены результаты критического анализа основных из предложенных к настоящему моменту каскадных схем, что позволяет охарактеризовать их достоинства и недостатки,  и сделать вывод о возможности их использования для решения задачи обогащения регенерата урана в условиях его многократного рецикла в топливе современных реакторов на тепловых нейтронах.

Очевидно, что непосредственное применение штатной схемы обогащения --- ординарного или трехпоточного каскада имеет в рассматриваемом случае существенные ограничения и в общем случае поставленную задачу не решает. Главная причина состоит в том, что подобный каскад имеет всего один выходящий поток отбора, в котором, одновременно будут концентрироваться, как целевой  $^{235}$U, так и четные изотопы. В результате ординарный каскад позволяет лишь обогащать относительно «чистые» составы регенерата, в которых исходные содержания четных изотопов меньше (на порядок или более), чем их допустимые пределы в товарном НОУ. Однако эти условия, очевидно, невыполнимы при многократном рецикле урана.
Проведенный сравнительный анализ предложенных способов обогащения регенерата позволяет условно разделить их на 3 типа: схемы с разбавлением четных изотопов,  схемы с отделением четных изотопов, «гибридные» схемы. 
Ниже проанализированы каскадные схемы каждого из указанных типов. В части \ref{integral_comp} будут представлены интегральные результаты расчетов обогащения регенерата различного исходного состава для большинства рассмотренных ниже схем с целью оценки их эффективности для решения поставленной задачи.

\subsection{Каскадные схемы с разбавлением четных изотопов}

Ряд из предложенных каскадных схем обогащения регенерата в качестве основного фактора, корректирующего изотопный состав регенерата в процессе его обогащения, используют разбавление четных изотопов урановой смесью, которая их не содержит. В качестве таких разбавителей чаще всего рассматривают природный уран, однако это могут быть также обедненный или низкообогащенный уран.

Простейшие схемы с разбавлением основаны на использовании штатного ординарного каскада. Рассмотрим такие схемы, которые могут быть реализованы следующими способами (рис. \ref{fig:diagram1}) \cite{ksenof88,borodynyaIssledovanieProblemyVovlecheniya1989,sulaberidzeNekotoryhRazdelitelnyhProblemah2004,smirnovKaskadnyeShemyZadachah2012}:

\begin{enumerate}
  \item Смешивание регенерированного урана и природного (или обедненного) урана перед подачей в каскад рис. (рис. \ref{fig:diagram1}.1).
  \item Получение обогащенной фракции из регенерата и последующее ее разбавление природной урановой смесью (рис. \ref{fig:diagram1}.2).
  \item Получение НОУ из природного урана путем его прямого обогащения, с последующим разбавлением регенератом (рис. \ref{fig:diagram1}.3).
\end{enumerate}

\begin{figure}[ht]
  \centerfloat{\includegraphics[scale=0.5]{cascades/diagram1}}
  \caption{Схемы на основе ординарного каскада}\label{fig:diagram1}
\end{figure}

Для всех вариантов схем рис. \ref{fig:diagram1} сотношение между расходом регенерата и разбавителем природного происхождения определяется пределом допустимой концентрации $^{232}$U в конечном продукте --- низкообогащенном уране. Также компенсируется отрицательная реактивность $^{236}$U с помощью добавочной концентрации $^{235}$U к той, что требуется для НОУ-топлива с заданными свойствами.

Основным преимуществом таких схем является простота реализации, поскольку нет необходимости в модификации самого каскада, так как операции разбавления осуществляются за его пределами.

В качестве недостатков таких схем можно выделить:
\begin{enumerate}
  \item отсутствие возможности очищать регенерированный уран от четных изотопов, так как такие схемы основаны исключительно на разбавлении четных изотопов до допустимых концентраций;
  \item потери работы разделения, возникающие из-за смешения потоков с различными изотопными концентрациями $^{235}$U;
  \item невозможность выполнения условия «полного использования регенерированного урана» при многократном рецикле \cite{smirnovApplyingEnrichmentCapacities2018} (см. главу \ref{ch:ch2});
  \item выполнение ограничений по концентрации $^{232}$U в продукте напрямую зависит от концентрации указанного изотопа в поступившем на обогащение регенерате;
  \item для схем рис. \ref{fig:diagram1}.1--\ref{fig:diagram1}.2 имеет место загрязнение 100\% задействованных в обогащении регенерированного урана разделительных мощностей, что делает проблематичным их дальнейшее «перепрофилирование» на обогащение природного урана, по крайней мере в случае длительной (в течение нескольких лет) работы с регенерированным ураном.
\end{enumerate}

Подытоживая рассмотрение простейших разбавляющих схем обогащения регенерата можно отметить, что их использование не позволяет очищать регенерат от чётных изотопов, вся масса которых в значительной мере переносится в отбор каскада, что затрудняет использование таких схем в условиях многократного рецикла, когда концентрации чётных изотопов возрастают. Поэтому такие каскадные схемы потенциально применимы только для обогащения относительно «чистого» состава регенерата, в котором содержание $^{232}$U меньше допустимой нормы на порядок и более, что нехарактерно для изотопных составов выгружаемого из активной зоны современных ВВЭР облученного топлива при многократном рецикле урана \cite{bormanTehnikoekonomicheskiyAnalizVozmozhnyh2012}. 

Другие варианты каскадных схем с разбавлением чётных изотопов основаны на использовании так называемых многопоточных каскадов \cite{sulaberidzeQuasiidealCascadesAdditional2006}. В отличие от предыдущих вариантов в рассматриваемом случае разбавление регенерата осуществляют непосредственно в каскаде путем подачи одного или нескольких разбавителей в качестве дополнительного питания каскада параллельно с самим регенерированным ураном. Подобное разбавление одним или несколькими разбавителями можно осуществить в каскадах с двумя или тремя внешними питаниями (рисунки \ref{fig:2_inputs}, \ref{fig:3_inputs}).

\begin{figure}[ht]
  \centerfloat{\includegraphics[scale=0.035]{cascades/2in_}}
  \caption{Каскад с дополнительным потоком питания}\label{fig:2_inputs}
\end{figure}

\begin{figure}[ht]
  \centerfloat{\includegraphics[scale=0.035]{cascades/3in_}}
  \caption{Каскад с тремя потоками питания}\label{fig:3_inputs}
\end{figure}

Ключевые отличие от рассмотренных выше схем на основе ординарного каскада в этих случаях состоит в том, что разбавление регенерата происходит непосредственно внутри каскада. При этом подача разбавителя(лей) в каскад в виде дополнительных потоков питания преследует цель минимизации потерь работы разделения при смешивание потоков с различным содержанием изотопа $^{235}$U. Это достигается за счет подачи потоков регенерированного урана и разбавителя в ступени каскада с близкими концентрациями изотопа $^{235}$U.
На рисунке \ref{fig:2_inputs} изображен вариант каскада с двумя питаниями: обогащаемый регенерированный уран, разбавитель --- природный уран. Как следует из результатов исследования \cite{smirnovEvolutionIsotopicComposition2012} в условиях многократного рецикла в топливе ВВЭР данная схема не способна обеспечить выполнение условия «полного использования регенерата», начиная со второго или третьего рецикла в зависимости от заданной величины допустимой концентрации $^{232}$U в товарном НОУ ($2\cdot10^{-7}$\% или $5\cdot10^{-7}$\%) \cite{rodionovaAnalizTehnikoekonomicheskihHarakteristik2019a,smirnovFizikotehnicheskieProblemyObogashcheniya2020}.

Схема, представленная на рисунке \ref{fig:3_inputs}, также является разбавляющей. Основное отличие от предыдущего варианта состоит в том, что разбавление регенерированного урана осуществляют с использованием комбинации разбавителей \cite{smirnovObogashchenieRegenerirovannogoUrana2014}. В некоторых случаях, природный уран ($F_n$) может быть заменен НОУ с обогащением 1,0--2,0\% \cite{smirnovObogashchenieRegenerirovannogoUrana2017}.

Использование комбинации разбавителей позволяет при заданных требованиях к продукту варьировать ключевые интегральные характеристики схемы: удельный расход природного урана и затраты работы разделения при получении товарного НОУ. Однако как и в случае с другими разбавляющими каскадными схемами при увеличении исходных концентраций четных изотопов в регенерате эффективность рассматриваемых схем с дополнительными питаниями снижается, а начиная с определенных концентраций $^{232}$U подобные схемы не могут обеспечить условие полного использования регенерата, тем самым не решая сформулированную выше в общем случае задачу обогащения регенерата. Ключевая причина снижения эффективности состоит в том, что данная схема, как и все предыдущие варианты имеет лишь один выводной поток, обогащенный по легким компонентам. В этом потоке неминуемо одновременно с целевым изотопом $^{235}$U концентрируются и все четные изотопы, включая $^{236}$U. Таким образом, комбинирование разбавителей лишь дает возможность варьировать расходные характеристики схемы, но не корректировать изотопный состав получаемого продукта.
Невозможность решения задачи обогащения регенерированного урана произвольного состава в разбавляющих каскадных схемах легко проиллюстрировать аналитической оценкой, основанной на условии баланса материальных потоков в каскаде. 
Как известно, в стационарном режиме работы, в отсутствии потерь или источников рабочего вещества, внешние параметры ординарного каскада подчиняются следующим условиям, выражающим закон сохранения вещества \cite{sulaberidzeTeoriyaKaskadovDlya2011}:
				  
\begin{equation} \label{EQ__1} 
  \begin{array}{l} {\quad \quad \quad \quad F=P+W,} \\ {F \cdot C_{i,F} =P \cdot C_{i,P} +W \cdot C_{i,W} ,\; i=1,...,m. \; ,} \end{array}
\end{equation} 
где $F$ --- поток питания каскада, $P$ --- поток отбора каскада, $W$ --- поток отвала каскада, $C_{i,F}$, $C_{i,P}$, $C_{i,W}$, --- концентрации i-го компонента в потоках $F$, $P$ и $W$, соответственно, m --- число компонентов разделяемой смеси.

В случае, если регенерат подают в каскад совместно с разбавителем (например, природным ураном), независимо от способа его подачи в каскад выполняется следующее соотношение, выражающее закон сохранения i-го компонента смеси:

\begin{equation} \label{EQ__2} 
  \begin{array}{l} {E \cdot C_{i,F}+F_n \cdot C_{i,F_n} =P \cdot C_{i,P} +W \cdot C_{i,W} \; ,\; i=1,...,m.} \end{array} ,
\end{equation} 
где $E$ --- поток питающего каскад регенерата, $F_n$ --- поток природного урана с концентрациями компонентов, которые равны нулю для изотопов $^{232}$U, $^{233}$U, $^{236}$U. Следует отметить, что $^{233}$U является еще одним искусственным изотопом урана, однако его присутствие не оказывает негативного влияния на характеристики НОУ. Напротив, его свойства близки к $^{235}$U, но в силу малости его содержания (на уровне $10\cdot10^{-8}$ --- $10\cdot10^{-7}$\%), данным фактором можно пренебречь.

Очевидно, что изотоп $^{232}$U, как наиболее легкий в разделяемой смеси, будет наиболее интенсивно обогащаться в отборе каскада и обедняться в его отвале. При этом для стандартных значений концентраций $^{235}$U в отвале каскада (0,1--0,2\%) для концентрации $^{232}$U в этом потоке будет справедливо условие  (индекс «1» соответствует изотопу $^{232}$U). Учитывая условие $W\textless F$, из соотношения \ref{EQ__2} легко получить следующее:

\begin{equation}\label{EQ_4} 
  C_{232,P} \approx \frac{E}{P} \cdot C_{232,E} \; .
\end{equation}

Из соотношения \ref{EQ_4} следует, что если необходимо выполнить условие «полного использования регенерата» и обеспечить величину $\frac{E}{P}\approx$ 0,95, что характерно для случая возврата регенерата из ОЯТ реакторов ВВЭР, то величина концентрации  может быть только больше соответствующего значения в исходном регенерате. По этой причине с использованием «разбавляющих» схем каскадов возможно обогащение только некоторых составов регенерата, соответствующих малым (до 40 МВт сут/кг) глубинам выгорания топлива и характеризующихся приемлемым исходным содержанием изотопа $^{232}$U.

Однако для современных величин глубины выгорания топлива, концентрация $^{232}$U даже в исходном регенерате может изначально превосходить величину ограничения. Например, это характерно для многократно рециклированного урана. Изменить эту ситуацию невозможно ввиду отсутствия параметров, позволяющих уменьшать относительную концентрацию изотопов $^{232}$U и $^{235}$U одновременно с обогащением последнего (и здесь и далее под относительной концентрацией компонентов понимаем отношение их абсолютных массово-долевых концентраций).

Анализ показывает, что выполнение условия «полного использования регенерата» возможно только, если концентрация $^{232}$U в исходном регенерате не превышает требуемого ограничения на выходе из каскада. Это может оказаться невыполнимым условием для случая многократного рецикла урана в топливе ВВЭР, при котором концентрация $^{232}$U в исходном регенерате, как правило, превышает допустимые ограничения уже начиная со второго рецикла \cite{rodionovaAnalizTehnikoekonomicheskihHarakteristik2019,smirnovFizikotehnicheskieProblemyObogashcheniya2020}. Данное обстоятельство фактически делает невозможным использование любого из описанных выше вариантов разбавляющих схем для решения сформулированной выше задачи обогащения регенерата урана в условиях многократного рецикла.


Помимо описанных выше вариантов разбавляющих каскадных схем, предложены и более сложные подходы, позволяющие с оговоркой получить очищенный регенерированный уран. Эффект очистки состоит в том, что в таком способе возможно получить регенерированный уран с концентрацией $^{235}$U, как в исходной смеси, но существенно сниженным содержанием четных изотопов. Пример такой каскадной схемы представлен на рисунке \ref{fig:3_out}. Она представляет собой каскад с дополнительными питанием и дополнительным отбором \cite{palkinSeparationUraniumIsotopes2010}. Основным питанием каскада выступает разбавитель, в предложенном варианте --- природный уран. Дополнительным питанием выступает обогащаемый регенерат. В отборе на конце такого каскада получают поток НОУ товарного качества. Поток дополнительного отбора представляет собой «очищенный» от чётных изотопов регенерат. Из полученного в дополнительном отборе полупродукта в дальнейшем может быть наработан товарный НОУ, для чего схему надо будет модифицировать, добавив еще один каскад.

\begin{figure}[ht]
  \centerfloat{\includegraphics[scale=0.035]{cascades/3out_}}
  \caption{Каскад с дополнительным потоком отбора для очистки регенерированного урана от минорных изотопов}\label{fig:3_out}
\end{figure}

По существу своей работы представленная на рисунке \ref{fig:3_out} каскадная схема является модификацией рассмотренной ранее схемы с двумя потоками питания. Отличие заключается в наличии потока дополнительного отбора, в котором получают очищенный регенерат. Однако наличие этого потока накладывает определенные ограничения на соотношения между потоками природного урана и регенерата, поступающих в каскад. Это обусловлено тем, что заметного снижения содержания минорных изотопов в дополнительном отборе можно добиться лишь при значительном разбавлении регенерата природным сырьем, в соотношениях, лежащих в диапазоне (1-25)/100 \cite{palkinSeparationUraniumIsotopes2010,smirnovKaskadnyeShemyZadachah2012}. Фактически это означает, что основной эффект «очистки» здесь также обусловлен разбавлением и включением дополнительного отбора на ступени с концентрацией $^{235}$U, близкой к таковой в исходном регенерате. При этом в схеме не происходит фактического отделения $^{235}$U от четных изотопов. Важно также отметить, что в представленных в \cite{palkinSeparationUraniumIsotopes2010} расчётных примерах эффективность такой схемы проверяли на примере состава регенерата с относительно невысоким содержанием изотопа $^{232}$U.
% Проведенные в рамках настоящей работы тестовые расчёты на примере обогащения повторно рециклируемого регенерата показали её неспособность решить в общем случае задачу обогащения регенерата, что затрудняет использование такой схемы для его многократного рецикла (см. Приложение 1).

Таким образом, рассматриваемая каскадная схема не может обеспечить решение сформулированной выше задачи обогащения регенерата в условиях многократного рецикла по тем же причинам, по которым подобную задачу не решают и другие «разбавляющие» схемы. Отдельного анализа требует также вопрос использования получаемого в дополнительном отборе очищенного регенерированного урана. В зависимости от входящего состава обогащаемого регенерированного урана данный материал может быть не пригоден для последующего прямого обогащения в ординарном каскаде, что ставит под сомнение целесообразность получения такого материала в принципе.
Подытоживая проведенный краткий анализ способов обогащения регенерата, основанных на его разбавлении, отметим их общие достоинства и недостатки.

К достоинствам подобных схем можно отнести следующее:

\begin{enumerate}
  \item	позволяют снижать концентрацию четных изотопов при обогащении регенерата различного исходного состава;
  \item	относительная простота реализации на основе центробежного метода разделения;
  \item	в большинстве вариантов реализации «разбавляющие» схемы позволяют осуществить процесс обогащения без превышения допустимых концентраций четных изотопов на отдельных ступенях каскада.
\end{enumerate} 

К недостаткам «разбавляющих» схем можно отнести следующее:

\begin{enumerate}
  \item	эффект снижения концентрации четных изотопов в таких схемах связан преимущественно с их разбавлением продуктами, не содержащими четных изотопов (природный уран, обедненный уран, НОУ из природного урана), что делает невозможным решение сформулированной выше задачи обогащения регенерированного урана в условиях многократного рецикла урана в топливе современных реакторов на тепловых нейтронах;
  \item	 для большинства вариантов разбавляющих схем происходит загрязнение 100\% разделительного оборудования, что может затруднить последующее его использование для обогащения смесей урана, не содержащих изотопов $^{232}$U и $^{236}$U.
\end{enumerate}

Описанные выше недостатки «разбавляющих» схем стимулировали развитие иных подходов к обогащению регенерированного урана, которые описаны ниже.


\subsection{Схемы с очисткой от $^{232}$U. Двойные каскады}\label{ch1/dvoynoy}

Простейшим вариантом каскада, реализующим отделение $^{232}$U от $^{235}$U в процессе обогащения регенерата является двойной каскад --- последовательное соединение двух каскадов (рис. \ref{fig:double_ru}). 

\begin{figure}[ht]
  \centerfloat{\includegraphics[scale=0.09]{cascades/Double_core_pure}}
  \caption{Двойной каскад}\label{fig:double_ru}
\end{figure}

Подобные каскадные схемы можно условно назвать «очищающими» от чётных изотопов. Идея работы подобных каскадов заключается в том, чтобы сконцентрировать нежелательные четные изотопы отдельно от целевого изотопа --- $^{235}$U. В отличие от рассмотренных выше схем с разбавлением четных изотопов в данном случае действительно может быть реализована очистка от них (хотя бы частично).

В простейшем варианте  реализации отделение четных изотопов от $^{235}$U может быть осуществлено следующим образом. Сначала, в первом каскаде обогащают изотоп $^{235}$U с одновременным обогащением изотопов $^{232}$U, $^{234}$U и $^{236}$U, затем полученную смесь направляют на вход второго каскада, где она делится на две группы: в первой обогащены легкие изотопы ($^{232}$U, $^{234}$U и $^{235}$U), во второй обедняется $^{235}$U с более интенсивным обеднением $^{232}$U, $^{234}$U. Таким образом, в условном «отвале» второго каскада возможно получить низкообогащенный уран, отвечающий требованиям по концентрациям изотопов $^{232}$U, $^{234}$U с одновременной компенсацией $^{236}$U.

Возможны варианты реализации двойного каскада, в которых изотоп $^{235}$U обогащают в потоке тяжелой фракции второго каскада. Например, в одной из модификаций двойного каскада перед отделением «легких» изотопов от $^{235}$U во втором каскаде, на выходе из первого каскада максимально обедняют изотоп $^{236}$U по отношению к изотопам $^{232}$U–$^{235}$U. В этом случае во втором каскаде $^{235}$U можно обогащать на «тяжелом» конце каскада с последующим разбавлением материалом, не содержащим четных изотопов, например, обедненным ураном. В результате в получаемом товарном НОУ снижены не только концентрации изотопов $^{232}$U и $^{234}$U, но и $^{236}$U, что крайне важно в условиях многократного рецикла урана, в котором $^{236}$U, будучи предшественником $^{232}$U в цепочке превращений, определяет динамику накопления изотопа $^{232}$U в ОЯТ \cite{smirnovEvolutionIsotopicComposition2012}.

Однако заметного эффекта очистки удается достичь только при высоких обогащениях по $^{235}$U на выходе из первого каскада (вплоть до 90\%). Это оказывается крайне нежелательным с учетом того, что согласно нормативным документам МАГАТЭ урановая смесь с концентрацией $^{232}$U более 20\% считается материалом прямого использования \cite{ManagementHighEnriched2005}. Кроме того, в загрязненной фракции второго каскада концентрации $^{232}$U и $^{234}$U возрастают на несколько порядков по отношению к исходной смеси, тем самым делая затруднительным обращение с подобной фракцией из-за существенного уровня удельной активности.

К основным достоинствам схем на основе двойных каскадов следует отнести:

\begin{enumerate}
  \item	возможность очистки (хотя бы частично) продукта от изотопов $^{232}$U и $^{234}$U, а не разбавления как в случае с ранее рассмотренными схемами; 
  \item	возможность обеспечить выполнение условия компенсации $^{236}$U в получаемом товарном продукте.
\end{enumerate}

Среди основных недостатков схем на основе двойных каскадов можно выделить следующие: 
\begin{enumerate}
  \item	получаемый в отборе второго каскада изотопный материал представляет собой «концентрат» изотопов $^{232}$U и $^{234}$U, что усложняет радиационную обстановку на разделительном производстве; 
  \item	из-за высоких обогащений в схеме возникают потери работы разделения, поскольку высокообогащенный в первом каскаде поток урана приходится обеднять во втором каскаде, а в некоторых случаях ещё и разбавлять обедненным ураном;
  \item	в представленной модификации двойные каскады не решают проблему очистки от изотопа $^{236}$U;
  \item в представленной модификации данная схема не обеспечивает заданного отношения между исходным регенератом и продуктом, что делает невозможным выполнение условия полного использования регенерата.
\end{enumerate}

Тем самым схема фактически не решает задачу обогащения урана в наиболее общей постановке.
Отметим, что ввиду отсутствия в простейших вариантах двойных каскадов других источников $^{235}$U, кроме самого регенерата, для наработки требуемой массы товарного НОУ для фабрикации комплекта ТВС на загрузку реактора, необходимо привлечение НОУ, полученного из других источников. В частности, недостающее количество НОУ может быть получено путем прямого обогащения природного урана до эквивалентной концентрации $^{235}$U.

Помимо описанного выше варианта двойных каскадов предложены и более сложные. Рассмотрим кратко наиболее характерные варианты.
В работе \cite{prusakovKorrekciyaIzotopnogoSostava2008} предложена модификация двойного каскада, состоящая в том, что для эффективного удаления $^{232}$U из обогащаемой смеси предложено использовать так называемый «газ-носитель» или «буферный газ» --- инертное соединение, имеющее массовое число, близкое к молекуле $^{232}$UF$_{6}$ (рис. \ref{p2_gas}) \cite{orlovWayObtainUranium2015, orlovDesublimationPurificationTransporting2017}. Процесс удаления $^{232}$U в такой схемы осуществляют следующим образом: первый каскад выделяет $^{235}$U в отборную фракцию, при этом в этом же потоке обогащен и $^{232}$U, а во втором каскаде $^{232}$U вместе с потоком буферного газа концентрируют на «легком» конце каскада, а товарный продукт (обогащенный изотопом $^{235}$U) отбирают на его отвальном («тяжелом») конце (рисунок \ref{p2_gas}). «Газ-носитель», примешиваемый к отбору первого каскада перед подачей его во второй каскад  увеличивает долю легкой фракции в каскаде и способствует более интенсивному концентрированию $^{232}$U в потоке отбора второго каскада. В результате, это уменьшает концентрацию данного изотопа в потоке тяжелой фракции второго каскада, из которого получают требуемый НОУ. В качестве «газа-носителя» предложено использовать фреон С$_{8}$H$_{3}$F$_{13}$, среднее массовое число для которого практически совпадает с массовым числом компонента $^{232}$UF$_{6}$.

\begin{figure}[ht]
  \centerfloat{\includegraphics[scale=0.035]{cascades/p2_gas_}}
  \caption{Каскад с газом-носителем. Обозначения: $V$ --- газ-носитель (фреон)}\label{p2_gas}
\end{figure}

Данная каскадная схема имеет схожие достоинства и недостатки с описанным выше вариантом двойного каскада. Характерным отличием схемы с газом-носителем является более высокая степень извлечения  $^{235}$U и более высокая степень очистки от $^{232}$U. С другой стороны, рассматриваемая схема схема имеет и характерные недостатки, которые заключаются в следующем:
\begin{enumerate}
  \item	отделение $^{232}$U от $^{235}$U за счет использования «газа-носителя» провоцирует рост концентрации $^{236}$U в получаемом товарном НОУ. Данное обстоятельство может иметь негативные последствие в условиях многократного рецикла урана в топливе реакторов на тепловых нейтронах. Это связано с тем, что рост концентрации $^{236}$U на каждом рецикле будет способствовать росту концентрации $^{232}$U \cite{smirnovEvolutionIsotopicComposition2012};
  \item	использование «газа-носителя» на разделительном производстве требует создания отдельной инфраструктуры по обращению с ним, а также отделению от него товарного гексафторида урана, что может сказаться на величине удельных затрат на получение товарного НОУ;
\end{enumerate}

Рассмотрим некоторые другие модификации двойных каскадов. В работе \cite{SposobIzotopnogoVosstanovleniyac} предложен вариант двойного каскада, состоящего из последовательно соединенных ординарного каскада и каскада с двумя внешними питаниями и дополнительным (промежуточным) потоком отбора $P'$ (рис. \ref{fig:double_crazy}).

\begin{figure}[ht]
  \centerfloat{\includegraphics[scale=0.035]{cascades/double_crazy}}
  \caption{Двойной каскад на основе пятипоточного каскада, производящий восстановленный регенерат в промежуточном потоке отбора $P'$}\label{fig:double_crazy}
\end{figure}

Как следует из рисунка \ref{fig:double_crazy} потоки отбора и отвала первого каскада ($P_1$, $W_1$) поступают в качестве двух питаний второго каскада. Такая организация потоков между каскадами позволяет добиться разделения исходной смеси на группы, компоненты которых концентрируются в различных частях второго каскада. В итоге на внутренних ступенях оказывается сконцентрирован $^{235}$U, откуда его можно отобрать, используя включенный здесь дополнительный поток отбора $P'$. На концевых ступенях происходит отбор обедненного урана в потоке отвала второго каскада $W_2$, а на другом конце каскада, в потоке лёгкой фракции получают поток $P_2$, загрязненный изотопами $^{232}$U, $^{234}$U, который фактически является отходом рассматриваемой каскадной схемы. Тем не менее, в отличие от ранее рассмотренных вариантов двойного каскада в рассматриваемой схеме возможно избежать появления фракции, высокообогащенной по $^{235}$U, что важно с точки зрения вопросов обеспечения ядерного нераспространения. Однако, в соответствии с результатами \cite{SposobIzotopnogoVosstanovleniyac} поток легкой фракции второго каскада $P_2$ имеет обогащение по $^{235}$U на уровне 20,0\%, что означает потери изотопа $^{235}$U в этом потоке. При этом обогащение $^{235}$U в потоке отбора первого каскада составляет величины в диапазоне 5,0--10,0\%.
 
Таким образом, в данной схеме удается снизить потери работы разделения по сравнению с простейшими модификациями двойных каскадов. Однако остальные недостатки двойных каскадов присущи также и данной схеме. 

Завершая обзор наиболее характерных вариантов двойных каскадных схем, отметим следующее. 
Ключевым достоинством двойных каскадов является то, что они позволяют менять соотношения между концентрациями четных изотопов и $^{235}$U. Иными словами в них происходит очистка, а не разбавление регенерата. Это обстоятельство особенно важно при рассмотрение вопросов обогащения регенерата в условиях его многократного рецикла, когда концентрации чётных изотопов урана возрастают от рецикла к рециклу.

Однако двойные каскады имеют и ряд общих недостатков, среди которых можно выделить:

\begin{itemize}
  \item наличие отхода в виде фракции, загрязненной четными изотопами. Выработка стратегии по обращению с этой фракцией требует отдельных исследований. При этом наличие таких загрязненных фракций приводит к необходимости введения дополнительных мер радиационной безопасности на производстве. В результате практическая реализации подобных мер может изменить технологические подходы, принятые на разделительных производствах и, соответственно, повлиять на удельные затраты при производстве товарного НОУ; 
  \item	двойные каскады сами по себе принципиально не могут решить задачу «полного использования регенерата», поскольку принципиально производят продукта, в несколько раз меньше, чем требуется. Это обусловлено тем, что они практически не используют других источников изотопа $^{235}$U, кроме регенерированного урана, в котором остаточной массы $^{235}$U недостаточно для формирования требуемой массы новой загрузки реактора. Учитывая сказанное выше, с использованием двойных каскадов возможно обеспечить получение только части ТВС для новой загрузки реактора. Недостающая масса НОУ может быть получена, например, из природного урана путём его обогащения в ординарном каскаде.
\end{itemize}


\subsubsection{Гибридные схемы каскадов для обогащения регенерата урана}

Учитывая, что двойные каскады в общем случае не могут полностью решить сформулированную выше задачу обогащения регенерата, к настоящему моменту предложены способы обогащения регенерата урана, в которых сочетаются характерные особенности каскадных схем с разбавлением чётных изотопов и двойных каскадов. Ниже кратко проанализированы подобные способы.
Одним из вариантом таких гибридных схем является последовательное соединение одиночного каскада с двумя питаниями и ординарного каскада (рис.\ref{f_2double}) \cite{palkinOchistkaRegenerirovannogoGeksaftorida2013}. Подобную схему можно реализовать двумя способами, отличающимися тем, какой из выходящих потоков первого каскада подают на вход второго (рис. \ref{fig:2_inputs}).

\begin{figure}[ht]
  \centerfloat{\includegraphics[scale=0.035]{cascades/2double}}
  \caption{Варианты соединения двухкаскадной схемы, состоящей из каскада с двумя потоками питания и ординарного каскада: а) случай подачи отбора первого каскада на питание второго; б) случай подачи отвала второго каскада на питание второго каскада. Обозначения: $F_{n}$ --- поток природного урана; $E$ --- поток регенерата, направленного на обогащение; $P_1$ --- поток «легкой» фракции каскада 1; $W_1$ – поток «тяжелой» фракции каскада 1; $P_2$ --- поток «легкой» фракции каскада 2; $W_2$ --- поток «тяжелой» фракции каскада 2}\label{f_2double}
\end{figure}

В рассматриваемой схеме в первом каскаде получают НОУ промежуточного обогащения, меньшего, чем требуется для получения товарного НОУ. Затем, во втором каскаде данный промежуточный материал обогащают/обедняют в одном из выходящих потоков до уровня концентрации в исходном регенерированном уране, в зависимости от того из какого потока первого каскада был получен промежуточный материал. В результате с помощью схемы на выходных потоках второго каскада производится как обогащенный товарный продукт, так и урановая смесь с концентрацией $^{235}$U на уровне исходного регенерата, но с пониженным содержанием четных изотопов. Ключевым преимуществом данной схемы, в отличие от ранее рассмотренных двухкаскадных схем, является отсутствие на каких-либо ступенях каскада концентрации $^{235}$U, превышающей уровень низкообогащенного урана.
Тем не менее, по своей сути схема является, во многом, «разбавляющей», поскольку основной эффект очистки связан с наличием в первом каскаде дополнительного питания, в котором туда поступает природный уран, выступающий в качестве разбавителя.
% Как показал анализ результатов вычислительных экспериментов, проведенных для данной схемы в рамках настоящей работы (Приложение 1),
Данная каскадная схема позволяет решить задачу обогащения регенерата в сформулированной выше общей постановке только для случая  обогащения состава регенерата с относительно невысоким исходным содержанием чётных изотопов. Это означает, что данную каскадную схему затруднительно использовать для обогащения регенерата в условиях его многократного рецикла.

Ещё одним вариантом гибридной схемы обогащения регенерированного урана можно считать каскадную схемы, основанную на очистке регенерата от четных изотопов в одиночном каскаде, имеющем так называемое <<расширение>> потока \cite{palkinRestorationIsotopicComposition2020}. В этом подходе использован принцип выделения изотопов промежуточных массовых чисел из многокомпонентных смесей стабильных изотопов в каскадах с дополнительными потоками отбора \cite{smirnovQKASKADYDLYaPOLUChENIYa2013,smirnovVliyanieProfilyaPotoka2010,palkinMnogopotochnyeKaskadyDlya2015}. 
Основная идея работы подобной схемы состоит в том, что, подобрав соответствующим образом вид функции распределения потока питания по ступеням каскада, возможно добиться концентрирования целевого промежуточного компонента на внутренних ступенях. Организовав на ступени в области максимума концентрации целевого промежуточного компонента внутри каскада поток дополнительного отбора, возможно получить фракцию с максимальным содержанием этого изотопа при более низких по отношению к нему концентрациях легких изотопов, чем в отборе на конце каскада.
Описываемый эффект продемонстрирован как на примере модельного Q-каскада, так и на примере каскада постоянной ширины \cite{smirnovDesignCascadeLocally2015}. Каскады, имеющие подобную особенность в распределении потока питания по ступеням были названы каскадами с «расширением» потока \cite{smirnovVliyanieProfilyaPotoka2010}.

Учитывая, что изотоп $^{235}$U является промежуточным по массовому числу в смеси регенерированного урана этот способ можно применить и для концентрирования данного изотопа при обогащении регенерата урана. После чего, перемешав, полученный в промежуточном отборе такого каскада обогащенный регенерат, например, с обедненным ураном можно получить НОУ товарного качества.

Подобная каскадная схема схематично изображена на рисунке \ref{fig:enl}. Принцип работы данной схемы можно описать следующим образом. На вход каскада подают поток регенерированного урана $E$. Каскад имеет три выходящих потока: поток отвала $W_1$, поток дополнительного отбора $G$ и поток основного отбора $P_1$. В потоке дополнительного отбора ($G$) достигается максимальное обогащение по $^{235}$U, которое составляет величину около 90\% или выше \cite{palkinRestorationIsotopicComposition2020}. В потоке отбора $P_1$ каскада нарабатывают смесь, высокообогащенную по $^{234}$U (до уровня 80\% и выше) и изотопу $^{232}$U (до уровня 3--10\%). Концентрация $^{235}$U в потоке $P_1$ лежит в диапазоне 10–-20\%. Материал, полученный в потоке $G$, далее необходимо перемешать с составом, имеющим низкое содержание $^{235}$U, для получения товарного продукта с одновременным снижением концентраций четных изотопов. В качестве разбавителя удобно использовать обедненный уран (поток $F_D$). После смешивания потоков $E$ и $F_D$ получают состав урана, обладающий необходимой для товарного продукта концентрацией $^{235}$U и удовлетворяющий ограничениям на концентрации четных изотопов.
Процесс очистки в данной схеме состоит в отделении легкой группы изотопов ($^{232}$U и $^{234}$U) от целевого $^{235}$U при одновременном снижении относительной концентрации $^{235}$U и $^{236}$U. Фактически данная схема очищает регенерат в процессе его обогащения одновременно от всех четных изотопов. После чего происходит разбавление обедненным ураном фракции с высоким содержанием изотопа $^{235}$U, получаемой в потоке дополнительного отбора. В процессе такого разбавления происходит также и окончательная коррекция содержания чётных изотопов в получаемой смеси.
% Как показали представленные в Приложении расчёты, проведенные для подобной каскадной схемы, её эффективность и возможность решения поставленной задачи обогащения регенерированного урана существенно зависят от состава исходной разделяемой смеси. Данный фактор затрудняет её использование в условиях многократного рецикла урана. 

\begin{figure}[ht]
  \centerfloat{\includegraphics[scale=0.035]{cascades/extend}}
  \caption{Схема каскада концентрирования $^{235}$U в дополнительном отборе и последующим разбавлением обедненного урана для получения товарного НОУ. Обозначения: $E_1$ –- поток регенерата, направленного на обогащение; $P_1$ –- поток «легкой» фракции; $W_1$ –- поток отвала; $G$ –- поток дополнительного отбора; $F_D$ –- поток обедненного урана; $P$ –- поток товарного НОУ
  }\label{fig:enl}
\end{figure}

К достоинствам схемы можно отнести следующее:

\begin{enumerate}
  \item полное отсутствие природного урана в схеме;
  \item эффект коррекции изотопного состава достигается не только за счет разбавления, но и за счет снижения относительных концентрации четных изотопов к $^{235}$U (в первую очередь, $^{236}$U) в самом каскаде.
\end{enumerate}

К недостаткам схемы можно отнести следующее:
\begin{enumerate}
  \item высокие уровни активности на разделительном производстве ввиду наличия потоков с концентрациями четных изотопов на порядки, превышающими допустимые пределы для низкообогащенного урана и уранового сырья. Возможность работы разделительного производства при уровне концентрации $^{232}$U свыше $10^{-3} \%$ и с фракцией, содержащей практически «чистый» $^{234}$U требует отдельной проработки с точки зрения вопросов радиационной безопасности и проблемы радиолиза рабочего вещества \cite{belovRADIOLIZOBOGAShchENNOGO2322019}; 
  \item данная схема в общем случае не позволяет обеспечить условие «полного использования регенерата».
\end{enumerate}


\section{Выводы из анализа предложенных способов обогащения регенерированного урана}

Итак, известные на сегодняшний день технические решения задачи обогащения регенерированного урана с одновременной коррекцией его изотопного состава основаны на:
\begin{enumerate}
  \item разбавлении регенерированного урана материалами, не содержащими четных изотопов (например, природным ураном, обедненным ураном и т.д.), на входе в разделительный каскад, на выходе из разделительного каскада или внутри каскада при подаче разбавителей в качестве дополнительных внешних питаний каскада;
  \item использовании каскадных схем, позволяющих понижать относительные концентрации чётных изотопов по отношению к $^{235}$U (преимущественно двойные каскады и их модификации);
  \item комбинировании процессов разбавления регенерата материалами, не содержащими чётных изотопов, и его очистки за счёт выведения нежелательных фракций с относительно высоким содержанием $^{232}$U и $^{234}$U в виде отдельных потоков.
\end{enumerate}

Проведенный теоретический анализ наиболее характерных каскадных схем каждого из типов позволяет сделать следующие выводы:
\begin{enumerate}
  \item наиболее перспективные варианты решения задачи обогащения регенерата в условиях многократного рецикла могут быть основаны на использовании модификаций «гибридных» каскадных схем, поскольку они одновременно позволяют корректировать как изотопный состав регенерата за счёт его частичной очистки от чётных изотопов, так и массовые расходы за счёт варьрования величин потоков разбавителей;
  \item ни одна из описанных выше схем не может быть применена для решения задача обогащения регенерата в условиях многократного рецикла в наиболее общей постановке;
  \item актуальным становится поиск каскадной схемы, позволяющей решить в общем случае задачу обогащения регенерата при различном исходном содержании регенерата в нём.
\end{enumerate}
% проведенная серия вычислительных экспериментов с целью оценки возможности обогащения регенерированного урана, прошедшего несколько рециклов, показала, что ... ни одна из
