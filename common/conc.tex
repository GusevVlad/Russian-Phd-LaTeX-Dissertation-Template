В результате проведения диссертационной работы разработаны варианты каскадных схем, позволяющие решить задачу обогащения регенерированного урана в условиях его многократного рецикла при условиях, характерных топливному циклу современных легководных реакторов российского дизайна и международным спецификациям:

\begin{enumerate}
  \item Схема двойного каскада с НОУ-разбавителем;
  \item Схема двойного каскада с НОУ-разбавителем с возвратом потока $P_2$ в цикл;
  \item Схема тройного каскада с НОУ-разбавителем и дополнительным разбавителем потока $P_2$, возвращаемого в цикл
  \item Схема независимой утилизации побочного продукта легкой фракции второго каскада схемы двойного каскада с НОУ-разбавителем.
\end{enumerate}

Для каждой из предложенных схем разработаны оригинальные методики расчета и оптимизации ее параметров по критерию минимума суммарного потока каскадной схемы, основанная на использовании современных методов условной оптимизации функций многих переменных. С использованием разработанных методик расчета и оптимизации предложенных каскадных схем продемонстрирована возможность их использования для обогащения регенерированного урана в условиях многократного рецикла на примере взятого из литературы изотопного состава регенерата урана с повышенным содержанием четных изотопов и отвечающего пятому рециклу в топливе ВВЭР.

Полученные оценки интегральных характеристик каскадных схем для топливного цикла с использованием только обедненного урана свидетельствуют о целесообразности оценки возможности реализации ЯТЦ при таких условиях с учетом масштабов доступных производственных мощностей по обогащению урана.

Для выбора конкретного варианта каскадной схемы с целью дальнейшей практической реализации необходим детальный технико-экономический анализ каждой из схем на основе их интегральных показателей (расходные характеристики, затраты работы разделения и пр.) в контексте всей цепочки стадий ЯТЦ и с учетом возникающих в этой цепочке изменений при использовании регенерата урана по отношению к открытому топливному циклу. 

Помимо этого, необходима проработка технологических проблем каждой из схем, в частности, с точки зрения возможности эксплуатации и обслуживания оборудования в условиях работы с материалами, имеющими более высокую, чем природный уран удельную активность. Например, подобные условия возникают в «очистительных» каскадах, выделяющих в легкую фракции $\alpha$-активные изотопы $^{232}$U и $^{234}$U. 



Использование уранового регенерата для производства топлива легководных энергетических реакторов позволит: 
\begin{enumerate}
  \item сократить объем захоронения радиоактивных отходов; 
  \item обеспечить экономию природного урана;
  \item сэкономить затраты работы разделения (по сравнению со случаем обогащения природного урана) при дообогащении данного материала в разделительном каскаде. 
\end{enumerate}


Переработка отработавшего ядерного топлива энергетических реакторов на тепловых нейтронах (в первую очередь, легководных) и повторное использование выделенных из него делящихся материалов является одной из приоритетных задач для ядерной отрасли РФ. Решение указанных задач позволит повысить конкурентоспособность отечественной ядерной индустрии и упрочить ее лидирующие позиции на мировом рынке услуг в области ядерного топливного цикла (ЯТЦ). Вовлечение превалирующей урановой составляющей отработанного топлива в ядерный топливный цикл реакторов, составляющих основную долю парка энергоблоков, позволит увеличить рентабельность электрогенерации на АЭС.