%% Согласно ГОСТ Р 7.0.11-2011:
%% 5.3.3 В заключении диссертации излагают итоги выполненного исследования, рекомендации, перспективы дальнейшей разработки темы.
%% 9.2.3 В заключении автореферата диссертации излагают итоги данного исследования, рекомендации и перспективы дальнейшей разработки темы.

Использование уранового регенерата для производства топлива легководных энергетических реакторов позволит: 
\begin{enumerate}
  \item сократить объем захоронения радиоактивных отходов; 
  \item обеспечить экономию природного урана;
  \item сэкономить затраты работы разделения (по сравнению со случаем обогащения природного урана) при дообогащении данного материала в разделительном каскаде. 
\end{enumerate}
Таким образом, вовлечение урановой составляющей отработанного топлива в ядерный топливный цикл реакторов, составляющих основную долю парка энергоблоков, позволит увеличить рентабельность электрогенерации на АЭС. В частности, Росатом, внедряя описанные в работе технологии, уже сегодня формирует более конкурентоспособные коммерческие предложения в части топливных поставок, а также организует эффективный переходный период на двухкомпонентную структуру ядерной энергетики. Такой подход позволяет ресурсоэффективнее воплощать глобальную стратегию замыкания ЯТЦ, осуществляя рецикл топлива с помощью имеющихся реакторных мощностей парка ВВЭР.


% \begin{enumerate}
%   \item На основе анализа \ldots
%   \item Численные исследования показали, что \ldots
%   \item Математическое моделирование показало \ldots
%   \item Для выполнения поставленных задач был создан \ldots
% \end{enumerate}
