
{\actuality}
Построение ядерной энергетики нового типа, устойчивой к ресурсным ограничениям и предусматривающей решение проблемы обращения с радиоактивными отходами, связано с реакторами на быстрых нейтронах, обладающими размножающими свойствами. То есть, такая система, называемая двухкомпонентной ядерной энергетикой, нацелена на воспроизводство делящегося материала -- энергетического  плутония -- в реакторе на быстрых нейтронах. Однако, по оценкам \cite{andrianovaPERSPEKTIVNYETOPLIVNYEZAGRUZKI2015}, в ближайшие десятилетия, по мере становления двухкомпонентной ядерно-энергетической системы, неизбежен переходный период, когда делящиеся материалы будут повторно использоваться в топливном цикле реакторов на тепловых нейтронах, так как они составляют основную часть парка энергоблоков. Основным материалом топлива является уран, составляющий $\approx$95\% за вычетом конструкционных материалов. К тому же, оценки показывают, что регенерированный уран с содержанием $^{235}$U на уровне $\approx$0,85-0,95\% экономически целесообразно дообогащать на изотопно-разделительном производстве \cite{NikipelovNikipelovSudby}. 

Итак, использование выделенного из отработавшего ядерного топлива (ОЯТ) регенерированного урана является основным достижимым в ближайшей перспективе направлением вовлечения регенерируемых материалов в топливный цикл энергетических реакторов. Выделенный из ОЯТ регенерированный уран может быть использован в составе топлива ВВЭР различными способами: центрифужное дообогащение для производства уранового топлива, дообогащение и включение в состав смешанного уран-плутониевого топлива типа REMIX.

Рецикл урана является сложной задачей ввиду присутствия в изотопном составе регенерата ряда четных изотопов. В первую очередь, это неприродные $^{232}$U и $^{236}$U. Присутствие первого затрудняет обращение с регенератом, как на стадии обогащения, так и на стадии производства твэлов. Влияние же второго сказывается на ухудшении размножающих свойств ядерного топлива, поскольку данный изотоп является паразитным поглотителем тепловых нейтронов. Вдобавок, в регенерате, по сравнению с природным ураном, на порядок выше содержание $^{234}$U При этом, ориентируясь на сегодняшние тенденции к увеличению длительности топливных циклов ВВЭР, которые связаны с повышением глубины выгорания топлива, следует принять во внимание вытекающий из этого рост содержания вредных четных изотопов в регенерате.

Итак, ввиду необходимости решения задачи эффективного вовлечения регенерированного урана в ядерный топливный цикл (ЯТЦ), существует потребность поиска и дальнейшей разработки каскадных схем, которые позволят решить задачу производства из регенерата свежего топлива, удовлетворяющего стандартным спецификациям.
На сегодняшний день, хоть и предложен ряд каскадов, которые могут быть полезны для этой задачи, их границы применимости могут быть недостаточны в условиях многократного рецикла.

Таким образом, учитывая принятое в ГК Росатом стратегическое решения перехода к замкнутому ЯТЦ, решение перечисленных задач представляется актуальным для современной разделительной науки. 

{\aim} диссертационной работы является изучение физических закономерностей
молекулярно-селективного массопереноса в ординарных и многопоточных каскадах
для разделения многокомпонентных смесей с целью дальнейшего поиска
оптимальных условий обогащения регенерированного урана в подобных каскадах при
его многократном использовании в различных видах регенерированного ядерного
топлива для реакторов на тепловых нейтронах.

Для~достижения поставленной цели необходимо было решить следующие {\tasks}:
\begin{enumerate}
  \item Анализ физических закономерностей массопереноса компонентов смеси
  регенерированного урана в ординарном каскаде.
  Выявление физических причин
  невозможности решения задачи обогащения регенерата произвольного изотопного
  состава в одиночном каскаде при одновременном выполнении условий на
  концентрации изотопов $^{232}$U, $^{234}$U и $^{236}$U в получаемом продукте – низкообогащенном уране.
  \item Физическое обоснование принципов построения двойных каскадов,
  позволяющих корректировать изотопный состав регенерата по концентрациям
  изотопов $^{232}$U, $^{234}$U и $^{236}$U с одновременным расходованием полного количества
  подлежащего обогащению регенерата при различных исходных концентрациях
  четных изотопов в нем.
  \item Обоснование физических принципов «утилизации» загрязненной четными
  изотопами фракции, возникающей в двойных каскадах, путем полной или
  частичной подачи данной фракции в третий каскад с предварительным
  перемешиванием ее с природным и/или низкообогащенным ураном.
  \item Изучение физических закономерностей изменения изотопного состава регенерата и
  интегральных характеристик модифицированных двойных каскадов и тройных
  каскадов при обогащении регенерированного урана с различным исходным
  содержанием четных изотопов.
  \item Обобщение и систематизация подходов к выбору каскадной схемы, позволяющих
  эффективное обогащение регенерированного урана в условиях однократного и
  многократного рецикла.
  \item Определение физических закономерностей изменения изотопного состава
  регенерированного урана и параметров модифицированного двойного каскада для
  его дообогащения при многократном рецикле урана (отдельно и совместно с
  плутонием) в топливе реакторов типа ВВЭР.
\end{enumerate}


{\novelty}
\begin{enumerate}
  \item Впервые предложены модификации двойных каскадов, позволяющих корректировать
  изотопный состав регенерата по концентрациям изотопов $^{232}$U, $^{234}$U и $^{236}$U с одновременным расходованием полного количества подлежащего обогащению регенерата при различных исходных концентрациях четных изотопов в нем.
  \item Обоснованы физические принципы построения тройных каскадных схем для максимально полного использования использования исходного регенерированного урана для воспроизводства топлива реакторов на тепловых нейтронах.
  \item Выполнены оригинальные исследования по изучению физических закономерностей изменения изотопного состава регенерата и интегральных характеристик модифицированных двойных и тройных каскадах при обогащении регенерированного урана с различным исходным содержанием четных изотопов.
  \item Разработан обобщенный подход к выбору каскадной схемы для эффективного обогащения регенерированного урана в условиях однократного и многократного рецикла. Для этого были предложены критерии эффективности, такие как потери $^{235}$U в схеме и из регенерата, доля центрифуг, для которых превышается пороговое значение концентрации $^{232}$U, а также методы вычисления этих критериев в составных схемах.
  \item Разработан алгоритм оптимизации модифицированных двойных и тройных каскадов для обогащения регенерата урана по различным критериям
  эффективности.
  \item Определены физические закономерности изменения изотопного состава регенерированного урана и параметров модифицированного двойного и тройного каскадов для его дообогащения при многократном рецикле урана (отдельно и совместно с плутонием) в топливе реакторов типа ВВЭР.
\end{enumerate}

{\influence} 
\begin{enumerate}
  \item Проведенный анализ физических закономерностей массопереноса компонентов смеси регенерированного урана в ординарном каскаде позволил однозначно определить условия при которых возможно/невозможно обогащение регенерированного урана различного исходного состава в одиночном каскаде.
  \item Разработанные модификации двойных и тройных каскадов позволяют эффективно решать задачу обогащения регенерированного урана с одновременным выполнением ограничений на концентрации четных изотопов и максимальным использованием исходного регенерата.
  \item Анализ результатов расчетного моделирования молекулярно-селективного массопереноса в модифицированных двойных и тройных каскадах для обогащения регенерата урана позволяет рекомендовать область практической применимости подобных схем для получения обогащенного регенерированного урана.
  \item Разработаны рекомендации по использованию результатов работы для обогащения регенерированного урана в условиях однократного и многократного рецикла в различных видах топлива.
  \item  Представленные в работе результаты могут быть использованы в расчетных группах на предприятиях и организациях, связанных как с проектированием и построением разделительных каскадов, так и непосредственным производством изотопной продукции (АО «Уральский электрохимический комбинат», АО «Сибирский химический комбинат», АО «ТВЭЛ», АО «Восточно-Европейский головной научно-исследовательский и проектный институт энергетических технологий», АО «ПО «ЭХЗ» и др.).
  \item Предложенные методики расчета могут лечь в основу технико-экономического анализа обращения с ОЯТ в части получения из восстановленного урана низкообогащенного урана, отвечающего требуемым качествам.  
  \item  Выводы работы применимы в рамках принятой ГК Росатом программы <<Сбалансированный ЯТЦ>>, нацеленной на обеспечение дополнительных конкурентных преимуществ направления зарубежных поставок ядерного топлива. Проводимое в данной работе исследование является перспективным для развития бизнеса ГК Росатом как в направлении топливных поставок, так и в обращении с облученным топливом \cite{efimenkoProblemyPerspektivyRazvitiya2017}.
  \item Разработан тренировочный программный комплекс для расчета каскада, нацеленного на возврат регенерированного урана. Код оформлен в виде лабораторной работы, которая затем внедрена в учебный процесс.
\end{enumerate}

{\methods}
Исследование проводит систематизацию научно-технической литературы, посвященной заявленной теме.
Применяются подходы, известные в современной теоретической физике, и в частности, в теории разделения изотопов в каскадах.
В ходе работы осуществляется обоснование теоретических принципов работы анализируемых каскадов, и математическое моделирование ранее не известных каскадных схем.
Для проведения расчетов использовались модельные каскады, а именно квазиидеальный каскад и его разновидность R-каскад, для которого принимается условие несмешивания пары выбранных компонентов. Рассматривался противоточный симметричный каскад ($\alpha=\beta=\sqrt{q}$).
Моделирование процессов разделения смесей изотопов урана проводилось с помощью специально разработанных специализированных компьютерных программ. Применялись современные программные средства языка программирования Julia и подключаемые библиотеки, такие как NLopt, Optim, предназначенные для решения систем нелинейных уравнений и нелинейной оптимизации, Plots.jl для визуализации результатов, и др..

{\defpositions}
\begin{enumerate}
  \item Результаты анализа физических закономерностей массопереноса компонентов смеси регенерированного урана в ординарном каскаде, позволяющие однозначно определить условия при которых возможно/невозможно обогащение регенерированного урана различного исходного состава в одиночном каскаде.
  \item Физико-математические модели, методики расчета и оптимизации модифицированных двойных и тройных каскадных схем для обогащения
  регенерата урана с одновременным выполнением условий на концентрации четных изотопов и максимальным использованием исходного материала.
  \item Методика выбора каскадной схемы обогащения регенерированного урана в условиях многократного рецикла, в зависимости от его исходного состава и принятых ограничений на концентрации четных изотопов.
\end{enumerate}
% В папке Documents можно ознакомиться в решением совета из Томского ГУ
% в~файле \verb+Def_positions.pdf+, где обоснованно даются рекомендации
% по~формулировкам защищаемых положений.

{\reliability} Надежность, достоверность и обоснованность научных положений и выводов, сделанных в диссертации, следует из корректности постановки задач, физической обоснованности применяемых приближений, использования в исследованиях методов, ранее примененных в аналогичных исследованиях, взаимной согласованности результатов исследования, а также из совпадения результатов численных экспериментов, полученных с помощью независимо разработанных методик как самим соискателем, так и другими исследователями. Корректность результатов вычислительных экспериментов гарантируется тестами и операторами проверки соответствия ограничениям, верифицирующими строгое выполнение заданных условий и соблюдение условий сходимости балансов (массовых и покомпонентных).

{\probation}
См. приложение А2.

{\contribution} Автор принимал активное участие в написании расчетных кодов, проведении вычислительных экспериментов, а также оформлении методики выбора каскадной схемы. Автором был разработан программный комплекс для сопровождения процесса принятия решений по выбору для заданной задачи каскада конфигурации, оптимальной по целевым критериям.

{\publications} 
См. приложение А1.

