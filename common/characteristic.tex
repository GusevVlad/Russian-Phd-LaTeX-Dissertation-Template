{\actuality}
Построение ядерной энергетики нового типа, устойчивой к ресурсным ограничениям и предусматривающей решение проблемы обращения с радиоактивными отходами, связано с реакторами на быстрых нейтронах, нацеленными на воспроизводство делящегося материала -- энергетического  плутония. Однако, по оценкам \cite{andrianovaPerspektivnyeToplivnyeZagruzki2015}, в ближайшие десятилетия, по мере становления двухкомпонентной ядерно-энергетической системы, неизбежен переходный период, когда делящиеся материалы будут повторно использоваться в топливном цикле реакторов на тепловых нейтронах, так как они составляют основную часть парка энергоблоков.
На сегодняшний день мире в состоянии эксплуатации и сооружения насчитывают около 500 ядерных энергоблоков, подавляющее большинство из которых представляют собой легководные реакторы \cite{PRISHome}, работающие на ядерном топливе из низкообогащенного урана. Работа каждого из энергоблоков создает необходимость в обеспечении его топливными ресурсами и выборе способа обращения с выгруженным из него отработавшим ядерным топливом (ОЯТ). Проанализируем каждую из составляющих. 
Как известно, основным материалом для производства топлива реакторов на тепловых нейтронах является природный уран, который предварительно обогащают на разделительных производствах, как правило, с использованием газовых центрифуг. В соответствии с доступными данными, мировые запасы урана  оценивают в 59 мегатонн. Но большая часть этих запасов, составляют ресурсы урана, для которых на сегодняшний день отсутствуют отработанные технологии добычи и, соответственно, стоимость такого урана не определена. Лишь только для ~8 мегатонн известно, что стоимость получения такого природного урана может быть, конечно, различной но не превысит порога в 260 долл США/кг. В связи с этим существуют прогнозы, предсказывающие неизбежный разрыв между добычей и потреблением природного урана в будущем, что может привести к значительным проблемам  с обеспечением топливом реакторов на тепловых нейтронах в перспективе 15-20 лет.  
Дополнительным вызовом для ядерной промышленности является обращение с ОЯТ в долгосрочной перспективе. На текущий момент общая масса хранимого ОЯТ составляет более 400 килотонн. С каждым годом эта масса прирастает примерно на 11 килотонн. Учитывая сложности поиска новых мест для строительства хранилищ ОЯТ, а также негативное отношнение общества к этой проблеме во многих странах, очевидно, что без выработки приемлемого со всех точек зрения решения, это затруднит развитие ядерной энергетики в будущем. 
Следует отметить, что основным материалом отработавшего ядерного топлива является уран, составляющий $\approx$90-95\%, за вычетом конструкционных материалов. Оставлшаяся часть делится между плутонием и продуктами (осколками) деления. В большинстве случаев регенерированный уран содержит $^{235}$U на уровне $\geq$0,85\%, то есть долю делящегося изотопа выше, чем в природном уране, что делает целесообразным его повторное использование и обогащение на изотопно-разделительном производстве  \cite{NikipelovNikipelovSudby}. Вовлечение регенерированного урана отдельного или совместно с плутонием в производство ядерного топлива реакторов на тепловых нейтронах может позволить как сократить объем захоронения радиоактивных отходов, так и снизить потребности в природном уране. Однако стоит отметить, что реакторы на тепловых нейтронах являются "реакторами-сжигателями", то есть в среднем воспроизводят деляющихся материалов значительно меньше, чем распадается в активной зоне реактора в процессе облучения топлива. Этот факт говорит о том, что для реакторов данного типа возможно лишь частичное замыкание топливного цикла, поскольку для их полноценного обеспечения топливом потребуются внешние источники делящихся материалов.  
Необходимо также отметить и то, что существуют и проблемы, связанные с переработкой ОЯТ и использованием регенерата урана в топливном цикле легководных реакторов на тепловых нейтронах.
Во-первых, переработка ОЯТ сама по себе представляет технологически сложную, капиталоемкую, радиационно-опасную и затратную процедуру. Лишь немногоие страны на текущий момент обладают промышленными технологиями переработки и соответствующими мощностями, позволяющими рассматривать возможность замыкания топливного цикла.
Во-вторых, при облучении ядерного топлива в активной зоне реактора образуются искусственные изотопы урана, в первую очередь, 232U и 236U. Кроме того, как правило, возрастает и концентрация природного изотопа 234U. Изотоп 232U опасен тем, что является родоначальником цепочки распадов, среди дочерних продуктов которых есть,  в частности, 208Tl, представляющий собой источник жесткого гамма-излучения, обуславливающего высокий уровень радиоактивного фона. Поэтому при производстве уранового топлива существуют нормативные ограничения на допустимое содержание 232U в низкообогащенном уране. На текущий момент в РФ допустимые концентрации (в мас. долях) 232U в НОУ не должны превышать значений 2-510-7%. Проблема, связанная с изотопов 236U состоит в том, что он вносит паразитное поглощение нейтронов в ядерное топливо и, следовательно, отрицательно воздействует на реактивность реактора и глубину выгорания топлива. При наличии в загружаемом в реактор топливе 236U для компенсации его отрицательного влияния на реактивность и для получения заданных ядерно-физических характеристик реактора нужно повышать среднее начальное обогащение топлива по 235U.  Отдельно стоит подчеркнуть, что концентрации изотопов 232U, 234U и 236U (четных изотопов), возрастают при обогащении регенерированного урана в ординарных каскадах газовых центрифуг, используемых для обогащения природного урана. Под ординарным каскадом понимают каскад, имеющий три внешних потока – питание, отбор и отвал. Фактически это означает, что обогащение регенерированного урана требует развития собственных способов реализации с учётом необходимости коррекции его изотопного состава таким образом, чтобы удовлетворить требованиям действующих технических условий на товарный низкообогащенный уран. 
На сегодняшний день предложен ряд технических решений, позволяющих решить задачу обогащения регенерированного урана до концентраций 235U, требуемых в современных топливных циклах энергетических реакторов на тепловых нейтронах (в частности отечественных ВВЭР), при одновременном выполнении принятых ограничений на содержание 232U в ядерном топливе и реализации необходимого дообогащения регенерата по 235U для компенсации негативного влияния 236U. Тем не менее далеко не все из них являются универсальными и способны решить задачу обогащения регенерата с одновременной коррекцией его изотопноого состава в условиях, когда исходное содержание четных изотопов может существенно меняться, например, в сторону увеличения. Последнее обстоятельство особо важно в контексте рассмотрения перспективных реакторов, имеющих относительно высокую глубину выгорания топлива и, как следствие, состав ОЯТ которых может характеризоваться более высоким содержанием четных изотопов. Помимо этого, необходимо учитывать, что замыкание топливного цикла реакторов на тепловых нейтронах подразумевает многократное использование урана, что будет обуславливать дополнительное накопление четных изотопов в регенерате от цикла к циклу.   
Таким образом, актуальным для разделительной науки остается вопрос поиска эффективных способов обогащения регенерата урана с одновременной коррекцией его изотопногого состава в условиях развития тенденции повышения глубины выгорания топлива и многократного использования урана в нём (многократный рецикл урана). В дополнение к этому отдельно стоит вопрос выбора оптимальной каскадной схемы для дообогащения регенерированного урана, которая должна обеспечить максимально эффективное использование ресурса регенерированного урана при минимальных затратах работы разделения.
Решение указанных задач возможно осуществить на основе активно развивающейся в последние десятилетия теории каскадов для разделения многокомпонентных изотопных смесей. Существующие в этой теории модели массопереноса компонентов в многоступенчатых разделительных установках позволяют выявить ключевые закономерности изменения интегральных характеристик таких установок в процессе обогащения регенерированного урана с целью поиска оптимальных условиях такого процесса.
 

{\aim} диссертационной работы является изучение физических закономерностей
молекулярно-селективного массопереноса в ординарных и многопоточных каскадах
для разделения многокомпонентных смесей с целью дальнейшего поиска
оптимальных условий обогащения регенерированного урана в подобных каскадах при
его многократном использовании в регенерированном ядерном топливе для реакторов на тепловых нейтронах. 

Для~достижения поставленной цели решены следующие {\tasks}:
\begin{enumerate}
  \item Анализ физических закономерностей массопереноса компонентов смеси
  регенерированного урана в ординарном каскаде.
  Выявление физических ограничений нахождения решения задачи обогащения регенерата произвольного изотопного
  состава в одиночном каскаде при одновременном выполнении условий на
  концентрации изотопов $^{232}$U, $^{234}$U и $^{236}$U в получаемом продукте – низкообогащенном уране, а также априорная оценка возможности решения этой задачи.
  \item Физическое обоснование принципов построения двойных каскадов,
  позволяющих корректировать изотопный состав регенерата по концентрациям
  изотопов $^{232}$U, $^{234}$U и $^{236}$U с одновременным расходованием максимального количества
  подлежащего обогащению регенерата при различных исходных концентрациях
  четных изотопов в нем.
  \item Обоснован способ эффективной «утилизации» загрязненной четными
  изотопами фракции, возникающей в двойных каскадах, с учетом полной или
  частичной подачи данной фракции: в третий каскад с предварительным
  перемешиванием ее с природным, обедненным и/или низкообогащенным ураном; в отдельный двойной каскад, осуществляющий наработку низкообогащенного урана для последующей топливной кампании реактора.
  \item Разработаны методы сравнения исследуемых каскадных схем, а также расчетные методики для оптимизации параметров каскадных схем.
  \item Изучены физические закономерности изменения изотопного состава регенерата, а также
  интегральных характеристик модифицированных двойных каскадов и тройных
  каскадов при обогащении регенерированного урана с различным исходным
  содержанием четных изотопов в питающей смеси.
  % \item Обобщение и систематизация подходов к выбору каскадной схемы, позволяющих
  % эффективное обогащение регенерированного урана в условиях однократного и
  % многократного рецикла.
  % \item Определение физических закономерностей изменения изотопного состава
  % регенерированного урана и параметров модифицированного двойного каскада для
  % его дообогащения при многократном рецикле урана (отдельно и совместно с
  % плутонием) в топливе реакторов типа ВВЭР.
\end{enumerate}


{\novelty}
\begin{enumerate}
  \item Впервые предложены модификации двойных каскадов, позволяющих корректировать
  изотопный состав регенерата по концентрациям изотопов $^{232}$U, $^{234}$U и $^{236}$U с одновременным расходованием полного количества подлежащего обогащению регенерата при различных исходных концентрациях четных изотопов в нем и других внешних условиях.
  \item Обоснованы физические принципы построения тройных каскадных схем для максимального вовлечения исходного регенерированного урана для воспроизводства топлива реакторов на тепловых нейтронах.
  \item Выполнены оригинальные исследования по изучению физических закономерностей изменения изотопного состава регенерата и интегральных характеристик модифицированных двойных и тройных каскадах при обогащении регенерированного урана с различным исходным содержанием четных изотопов.
  \item Разработаны методы расчетов каскадных схем, позволяющих решить задачу возврата регенерированного урана в топливный цикл в условиях многократного рециклирования.
  \item Разработан обобщенный подход к выбору каскадной схемы для эффективного обогащения регенерированного урана в условиях однократного и многократного рецикла.
  \item Разработка методик оптимизации систем каскадов (двойного и тройного каскадов) для обогащения регенерата урана по различным критериям эффективности, таким как:
  % \begin{enumerate}
  %   \item расход природного урана в цикле;
  %   \item затраты работы разделения в цикле;
  %   \item доля потерь $^{235}$U в каскадной схеме;
  %   \item доля потерь $^{235}$U из исходного регенерата;
  %   \item доля газовых центрифуг в схеме, в которых превышена предельно допустимая концентрация по $^{232}$U.
  % \end{enumerate}
  \item Разработка подхода к утилизации высокоактивного «нештатного» отхода, образующегося в процессе обогащения регенерированного урана в двойном каскаде.
  \item Определение физических закономерностей изменения изотопного состава регенерированного урана и параметров каскадных схем (в модифицированном двойном и тройном каскаде) для его дообогащения при многократном рецикле урана (отдельно и совместно с плутонием) в топливе реакторов типа ВВЭР.
\end{enumerate}

{\influence} 
\begin{enumerate}
  \item Проведенный анализ физических закономерностей массопереноса компонентов смеси регенерированного урана в ординарном каскаде позволяет однозначно определить условия при которых возможно/невозможно получение необходимого количества конечного продукта на основе регенерированного урана различного исходного состава путем обогащения в одиночном каскаде.
  \item Разработанные модификации двойных и тройных каскадов позволяют эффективно решать задачу обогащения регенерированного урана с одновременным выполнением ограничений на концентрации четных изотопов и максимальным вовлечением исходного регенерата.
  % \item Проведенный анализ результатов расчетного моделирования молекулярно-селективного массопереноса в модифицированных двойных и тройных каскадах для обогащения регенерата урана выявляет область практической применимости подобных схем для получения НОУ-продукта на основе регенерированного урана.
  \item Предложенные способы оптимизации построения каскадных схем двойного и тройного каскадов позволяют находить наиболее эффективные с точки зрения таких критериев, как расход работы разделения, расход природного урана, степень извлечения $^{235}$U, конфигурации каскадов для возврата регенерированного урана в цикл.
  \item Разработаны рекомендации по использованию результатов работы для обогащения регенерированного урана в условиях однократного и многократного рецикла в различных видах топлива. Представленные в работе результаты могут быть использованы в расчетных группах на предприятиях и организациях, связанных как с проектированием и построением разделительных каскадов, так и непосредственным производством изотопной продукции (АО «Уральский электрохимический комбинат», АО «Сибирский химический комбинат», АО «ТВЭЛ», АО «Восточно-Европейский головной научно-исследовательский и проектный институт энергетических технологий», АО «ПО «ЭХЗ» и др.). Предложенные методики расчета могут лечь в основу технико-экономического анализа использования восстановленного урана для получения низкообогащенного урана, отвечающего требуемым качествам.  
  % \item Разработан тренировочный программный комплекс для расчета каскада, нацеленного на возврат регенерированного урана. Код оформлен в виде лабораторной работы, которая внедрена в учебный процесс.
\end{enumerate}


{\methods}.
Исследование проводит систематизацию научно-технической литературы, посвященной заявленной теме.
Применены подходы, известные в современной теоретической физике, и в частности, в теории разделения изотопов в каскадах.
В ходе работы обоснованы принципы построения анализируемых каскадов, и проведено математическое моделирование каскадных схем.
Для проведения расчетов использованы схемы модельных каскадов (квазиидеальный каскад и его разновидность R-каскад, для которого выполняется условие несмешивания относительных концентраций пары выбранных компонентов). Моделирование процессов разделения смесей изотопов урана проводили с использованием разработанных в ходе выполнения работы специализированных компьютерных программ. Применены современные программные средства языков программирования Julia и Python и подключаемых библиотек, таких как NLopt, Optim, ScyPy, предназначенных для решения систем нелинейных уравнений и оптимизационных процедур, Plots.jl для визуализации результатов.

{\defpositions}
\begin{enumerate}
  \item Результаты анализа физических закономерностей массопереноса компонентов смеси регенерированного урана в ординарном каскаде, позволяющие однозначно определить условия при которых возможно/невозможно получение необходимого количества конечного продукта на основе регенерированного урана различного исходного состава путем обогащения в одиночном каскаде.
  \item Физико-математические модели, методики расчета и оптимизации модифицированных двойных и тройных каскадных схем для обогащения регенерата урана с одновременным выполнением условий на концентрации четных изотопов и максимальным вовлечением исходного материала.
  \item Методика выбора каскадной схемы обогащения регенерированного урана в условиях многократного рецикла, в зависимости от его исходного состава и принятых ограничений на концентрации четных изотопов.
\end{enumerate}

{\reliability}.
Надежность, достоверность и обоснованность научных положений и выводов, сделанных в диссертации, следует из корректности постановки задач, физической обоснованности применяемых приближений, использования методов, ранее примененных в аналогичных исследованиях, взаимной согласованности результатов, а также из совпадения результатов численных экспериментов. Корректность результатов вычислительных экспериментов гарантируется тестами и операторами проверки соответствия ограничениям, верифицирующими строгое выполнение заданных условий и соблюдение условий сходимости балансов (массовых и покомпонентных).

% {\probation}
% См. приложение А2.

{\contribution} Автор принимал активное участие разработке каскадных схем, написании расчетных кодов, проведении вычислительных экспериментов, а также в обработке и в анализе результатов численных экспериментов. Автор разработал расчетные коды, реализующие новые подходы к оптимизации рассматриваемых схем.

% {\publications} 
% См. приложение А1.

