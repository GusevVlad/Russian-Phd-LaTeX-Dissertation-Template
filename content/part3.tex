\chapter{Модельные каскады для разделения многокомпонентных смесей.}

\section{Частные случаи симметрично-противоточного каскада для разделения многокомпонентных изотопных смесей.}



\section{«Квазиидеальный» каскад и Q-каскад, их физическая взаимосвязь.}

Наиболее общая из таких моделей называется «квазиидеальным» каскадом, где предполагается постоянство по всей его длине относительных коэффициентов разделения, а также срезов парциальных потоков компонентов по каскадным ступеням \cite{yamamotoMulticomponentIsotopeSeparating1978}.
В настоящее время он используется в двух приближениях: со слабым обогащением, когда $q$-1 $\approx$ 0 (Q-каскад \cite{borisevichNewApproachOptimize2011, kolokoltsovDesignCascadesSeparating1970, zengQCascadeExplanation2012}) и произвольным обогащением, когда $q$-1 не много меньше единицы (квазиидеальный каскад \cite{sulaberidzeSpecialFeaturesEnrichment2006}).
Применение модельных каскадов значительно упрощает расчет закономерностей массообмена в каскаде для многокомпонентного разделения.

Ниже кратко рассмотрены модельные каскады, для интересующего нас случая произвольного немалого коэффициента разделения на ступенях, когда $q$-1 не много меньше единицы  -- «квазиидеальный» каскад и его частный случай R-каскад \cite{sazykinKvaziidealnyeKaskadyDlya2000}.


Рассмотрим случай симметричного противоточного каскада с постоянными по его длине относительными коэффициентами разделения $q_{ik} ,\; \alpha _{ik} ,\; \beta _{ik} $ $(i=1,\; 2,...,m;$ \textit{k}--номер «опорного» компонента). Условие постоянства относительных коэффициентов разделения обеспечивает выполнение условия постоянства величин \textit{g${}_{i}$} и $\phi _{i} $. Следовательно, соотношения (\ref{GrindEQ__1_24_})--(\ref{GrindEQ__1_27_}) приводятся к виду \cite{sulaberidzeTeoriyaKaskadovDlya2011}:

\begin{equation} \label{GrindEQ__1_52_} 
  G'_{i} (s-1)+\frac{1}{g_{i} } G'_{i} (s+1)-\frac{g_{i} +1}{g_{i} } G'_{i} (s)+\delta _{sf} Fc_{iF} =0,\; \; i\ne k, 
  \end{equation} 
  \begin{equation} \label{GrindEQ__1_53_} 
  G'_{k} (s-1)+\frac{1}{g_{k} } G'_{k} (s+1)-\frac{g_{k} +1}{g_{k} } G'_{k} (s)+\delta _{sf} Fc_{kF} =0, 
  \end{equation}

где $s$ – текущий номер ступени, отсчитываемый от «тяжелого» конца каскада к его «легкому» концу $\delta _{sf} =\left\{\begin{array}{l} {0,\; \; s\ne f} \\ {1,\; \; s=f} \end{array}\right. $

Уравнения (\ref{GrindEQ__1_52_})--(\ref{GrindEQ__1_53_}) представляют собой линейные разностные уравнения второго порядка относительно неизвестных функций $G'_{i} (s)$. Граничные условия для них имеют вид:

\begin{equation} \label{GrindEQ__1_54_} 
  \left\{\begin{array}{l} {G'_{i} (0)=G'_{i} (N+1)=0,\; \; i=1,\; 2,...,m} \\ {G'_{i} (N)=PC_{i}^{P} ,\; \; i=1,\; 2,...,m} \\ {G'_{i} (1)=g_{i} WC_{i}^{W} ,\; \; i\ne k} \\ {G''_{k} (1)=g_{k} WC_{i}^{W} .} \end{array}\right.  
\end{equation} 

Ступени с номерами $s=1$ и $s=N$ являются крайними ступенями каскада, что делает возможным формально записать $G'_{i} (0)=G'_{i} (N+1)=0$.

Решив (\ref{GrindEQ__1_52_}) и (\ref{GrindEQ__1_53_}), а также используя уравнения баланса (\ref{GrindEQ__1_21_}) и граничные условия (\ref{GrindEQ__1_54_}), можно получить уравнения связи внешних параметров такого каскада с длинами его секций и параметрами ступени. В итоге:

\begin{equation} \label{GrindEQ__1_55_} 
  \frac{P}{F} =\sum _{j=1}^{m}C_{j}^{F} \frac{1-g_{j}^{-f} }{1-g_{j}^{-N-1}} ,\; \; s=f,...,N ,                                                  
  \end{equation} 
  \begin{equation} \label{GrindEQ__1_56_} 
  \frac{W}{F} =\sum _{j=1}^{m}C_{j}^{F} \frac{g_{j}^{N+1-f} -1}{g_{j}^{N+1} -1} ,\; \; s=1,...,f-1 ,                                            
\end{equation}

\begin{equation} \label{GrindEQ__1_57_} 
  C_{i}^{P}=C_{i}^{F} \frac{1-g_{i}^{-f}}{1-g_{i}^{-N-1}} / \sum_{j=1}^{m} C_{j}^{F} \frac{1-g_{j}^{-f}}{1-g_{j}^{-N-1}}, i=1,2, \ldots, m                             
\end{equation}

\begin{equation} \label{GrindEQ__1_58_} 
  C_{i}^{W}=C_{i}^{F} \frac{g_{i}^{N+1-f}-1}{g_{i}^{N+1}-1} / \sum_{j=1}^{m} C_{j}^{F} \frac{g_{j}^{N+1-f}-1}{g_{j}^{N+1}-1}, i=1,2, \ldots, m                         
\end{equation} 

Далее, распределение потока $L(s)$, концентраций компонентов и коэффициента деления потоков по ступеням каскада можно определить по формулам \cite{sulaberidzeTeoriyaKaskadovDlya2011}:

\begin{equation} \label{GrindEQ__1_59_} 
L(s)=\sum_{j=1}^{m} G_{j}^{\prime}(s) \frac{1+g_{j}}{g_{j}}=\left\{\begin{array}{c}
  P \sum_{j=1}^{m} \frac{g_{j}+1}{g_{j}-1} C_{j}^{P}\left(1-g_{j}^{s-N-1}\right), s=f, \ldots, N \\
  W \sum_{j=1}^{m} \frac{g_{j}+1}{g_{j}-1} C_{j}^{\pi}\left(g_{j}^{s}-1\right), s=1, \ldots, f-1
  \end{array}\right.
\end{equation} 

\begin{equation} \label{GrindEQ__1_60_} 
C_{i} (s)=\frac{1+g_{j} }{g_{j} } \cdot \frac{G''_{i} (s)}{G_{i} (s)} =\left\{\begin{array}{l} {\frac{C_{i}^{P} \frac{g_{j} }{g_{j} -1} \left(1-g_{j}^{s-N-1} \right)}{\sum _{j=1}^{m}\frac{g_{j} +1}{g_{j} -1}  C_{j}^{P} \left(1-g_{j}^{s-N-1} \right)} ,\; \; s=f,...,N,} \\ {\; \frac{C_{i}^{W} \frac{g_{j} }{g_{j} -1} \left(g_{j}^{s} -1\right)}{\sum _{j=1}^{m}\frac{g_{j} +1}{g_{j} -1}  C_{j}^{W} \left(g_{j}^{s} -1\right)} ,\; \; s=1,...,f-1,} \end{array}\right.  
\end{equation} 

\begin{equation} \label{GrindEQ__1_61_} 
\begin{array}{l} {\theta (s)=\frac{\sum _{j=1}^{m}G'_{j} (s) }{\sum _{j=1}^{m}G_{j} (s) } =\left\{\begin{array}{l} {\frac{\sum _{j=1}^{m}\frac{g_{j} }{g_{j} -1} C_{j}^{P} \left(1-g_{j}^{s-N-1} \right) }{\sum _{j=1}^{m}\frac{g_{j} +1}{g_{j} -1}  C_{j}^{P} \left(1-g_{j}^{s-N-1} \right)} ,\; \; s=f,...,N,} \\ {\; \frac{\sum _{j=1}^{m}\frac{g_{j} }{g_{j} -1} C_{j}^{W} \left(g_{j}^{s} -1\right) }{\sum _{j=1}^{m}\frac{g_{j} +1}{g_{j} -1}  C_{j}^{W} \left(g_{j}^{s} -1\right)} ,\; \; s=1,...,f-1.} \end{array}\right. } \\ {\; } \end{array} 
\end{equation}

Формулу для расчета относительного суммарного потока в каскаде легко получить, суммируя (\ref{GrindEQ__1_59_}) по всем ступеням каскада

\begin{equation} \label{GrindEQ__1_62_} 
  \sum _{s=1}^{N}\frac{L(s)}{P} =\sum _{i=1}^{m}\left\{\frac{g_{i} +1}{g_{i} -1} \left[\frac{W}{P} C_{i}^{W} (f)+C_{i}^{P} \left(N+1-f\right)\right]\right\}  .   
\end{equation} 
  
Рассмотренный выше каскад отличается тем, что относительные коэффициенты разделения $q_{ik} ,\; \alpha _{ik} ,\; \beta _{ik} $ (и, соответственно, срезы парциальных компонентов $\phi _{i} ,\; \; \phi _{k} $ и параметры $g_{i} $, $g_{k} $) остаются постоянными по длине каскада. Для таких каскадов в работе \cite{sazykinKvaziidealnyeKaskadyDlya2000} был введен термин «квазиидеальный» каскад.

\section{R-каскад и его использование при оптимизации параметров каскадов для разделения многокомпонентных изотопных смесей.}

В исследованиях, как правило, когда обогащенный переработанный уран обогащается многопоточными схемами, часто используется модель R-каскада (Matched Abundance Ratio Cascade-MARC \cite{kazukihidaSimultaneousEvaluationEffects1986, delagarzaMulticomponentIsotopeSeparation1961, woodEffectsSeparationProcesses2008}).
Это частный случай `квазиидеального' каскада. Здесь условие отсутствия смешивания относительных концентраций при подаче в каждую ступень выполняется для выбранной пары компонентов (например, это могут быть изотопы $^{235}$U и $^{238}$U).

Рассмотрим R-каскад, в котором выполняется несмешивание относительных концентраций $n$-го и $k$-го компонентов смеси. Данная каскадная модель является аналогом используемого в теории разделения бинарных смесей «идеального» каскада, в «узлы» которого входят потоки с одинаковой концентрацией компонентов. R-каскады могут быть построены как в случае «слабого обогащения», так и для немалых обогащений на ступени. Рассмотрим R-каскад в случае немалых обогащений на ступени. Условие несмешения по относительным концентрациям $n$-го и $k$-го компонентов можно записать в виде:

\begin{equation} \label{GrindEQ__1_68_} 
  R'_{nk} (s-1)=R_{nk} (s)=R''_{nk} (s+1).                                                 
\end{equation} 

Вследствие (\ref{GrindEQ__1_68_}) коэффициенты $\alpha _{nk} $ и $\beta _{nk} $ совпадают для двух соседних ступеней. При постоянных полных коэффициентах разделения равенство:

\begin{equation} \label{GrindEQ__1_69_} 
  \alpha _{nk} =\beta _{nk} =\sqrt{q_{nk} }  
\end{equation} 

приводит к каскаду со ступенями симметричными относительно пары компонентов с номерами $n$ и $k$. При этом на всех ступенях каскада $\alpha _{ik} \ne \beta _{ik} \; (i\ne n)$. Учитывая, сказанное выше, (\ref{GrindEQ__1_55_})--(\ref{GrindEQ__1_58_}) могут быть переписаны виде:
  

\begin{equation} \label{GrindEQ__1_70_} 
  \frac{P}{F} =\sum _{j=1}^{m}C_{j}^{F} \frac{(R_{nk}^{W} )^{-d_{j} } -(R_{nk}^{F} )^{-d_{j} } }{(R_{nk}^{W} )^{-d_{j} } -(R_{nk}^{P} )^{-d_{j} } }  ,                                            
  \end{equation} 
  \begin{equation} \label{GrindEQ__1_71_} 
  \frac{W}{F} =\sum _{j=1}^{m}C_{j}^{F} \frac{(R_{nk}^{F} )^{-d_{j} } -(R_{nk}^{P} )^{-d_{j} } }{(R_{nk}^{W} )^{-d_{j} } -(R_{nk}^{P} )^{-d_{j} } }  ,                                        
\end{equation} 

\begin{equation} \label{GrindEQ__1_72_} 
  C_{i}^{P}=C_{i}^{F} \frac{\left(R_{n k}^{W}\right)^{-d_{i}}-\left(R_{n k}^{F}\right)^{-d_{i}}}{\left(R_{n k}^{W}\right)^{-d_{i}}-\left(R_{n k}^{P}\right)^{-d_{i}}} / \sum_{j=1}^{m} C_{j}^{F} \frac{\left(R_{n k}^{W}\right)^{-d_{j}}-\left(R_{n k}^{F}\right)^{-d_{j}}}{\left(R_{n k}^{W}\right)^{-d_{j}}-\left(R_{n k}^{P}\right)^{-d_{j}}}
\end{equation} 

\begin{equation} \label{GrindEQ__1_73_} 
  C_{i}^{W}=C_{i}^{F} \frac{\left(R_{n k}^{F}\right)^{-d_{i}}-\left(R_{n k}^{P}\right)^{-d_{i}}}{\left(R_{n k}^{W}\right)^{-d_{i}}-\left(R_{n k}^{P}\right)^{-d_{i}}} / \sum_{j=1}^{m} C_{j}^{F} \frac{\left(R_{n k}^{F}\right)^{-d_{j}}-\left(R_{n k}^{P}\right)^{-d_{j}}}{\left(R_{n k}^{W}\right)^{-d_{j}}-\left(R_{n k}^{P}\right)^{-d_{j}}}
\end{equation} 

\begin{equation} \label{GrindEQ__1_74_} 
  d_{i} =\frac{\ln q_{ik} }{\ln g_{n} } -1,              
\end{equation}

, где $R_{n k}^{F}$, $R_{n k}^{W}$ и $R_{n k}^{P}$ -- относительные концентрации целевого компонента в потоках $F$, $W$, и $P$, соответственно.

Для молекулярно-кинетических методов разделения соотношения (\ref{GrindEQ__1_15_})--(\ref{GrindEQ__1_16_}) можно записать в следующем виде:

\begin{equation} \label{GrindEQ__1_75_} 
  g_{k} =q_{0}^{-\frac{M_{k} -M_{n} }{2} } ,        
  \end{equation} 
  \begin{equation} \label{GrindEQ__1_76_} 
  g_{i} =q_{0}^{M^{*} -M_{i} } ,        
\end{equation} 

, где $M^{*} =\frac{M_{n} +M_{k} }{2} $.

Из (\ref{GrindEQ__1_75_})--(\ref{GrindEQ__1_76_}) непосредственно следует, что для всех компонентов с $M_{i} $$\mathrm{<}$$M^{*} $ величины $g_{i} $$\mathrm{>}$1, если же $M_{i} $$\mathrm{>}$$M^{*} $, то $g_{i} $$\mathrm{<}$1. Из соотношений (\ref{GrindEQ__1_72_}) и (\ref{GrindEQ__1_73_}) при выполнении условий $N-f+1>>1,\; \; f-1>>1$ («длинный каскад») следует, что в таком R-каскаде компоненты с $g_{i} $$\mathrm{>}$1 ($M_{i} $$\mathrm{<}$$M^{*} $ обогащаются к «легкому» выходящему потоку каскада, а компоненты с $g_{i} $$\mathrm{<}$1 ($M_{i} $$\mathrm{>}$$M^{*}$ обогащаются к «тяжелому» выходящему потоку каскада. Следовательно, величина параметра $M^{*}$ полностью определяет направление обогащения компонентов смеси в R-каскаде. 

Суммарный поток R-каскада равен \cite{sulaberidzeTeoriyaKaskadovDlya2011}:

\begin{equation} \label{GrindEQ__1_77_} 
  \sum _{s=1}^{N}L(s) =\sum _{j=1}^{m}\frac{PC_{j}^{P} \ln R_{nk}^{P} +WC_{j}^{W} \ln R_{nk}^{W} -FC_{j}^{F} \ln R_{nk}^{F} }{\frac{g_{j} -1}{g_{j} +1} \ln g_{n} }  .               
\end{equation} 

Среди свойств, присущих модели R-каскада особо следует выделить следующие:

\begin{enumerate}
  \item В случае $m=2$ условие несмешения (\ref{GrindEQ__1_68_}) сводится к известному условию несмешения абсолютных концентраций, которое справедливо для «идеального» каскада;
  \item	Как показано в работе \cite{songComparativeStudyModel2010}, суммарный поток R-каскада при заданных величинах концентраций целевого компонента в потоках отбора $C_{n}^{P}$ и отвала $C_{n}^{W}$ минимален, при условии соответствующего выбора номера опорного компонента. Остановимся подробнее на этом свойстве. Фактически выбор опорного компонента определяет величину $M^{*}$. При этом, строго говоря, величина $M^{*}$ для любой $m$-компонентной смеси является дискретной функцией номера опорного компонента и, соответственно, имеет ограниченный набор допустимых значений, определяемых возможным количеством «опорных» компонентов смеси. В \cite{sulaberidzeSravnenieOptimalnyhModelnyh2008} предложено формально ввести в рассмотрение «виртуальные» компоненты с исчезающее малой концентрацией (на несколько порядков меньше наименьшей концентрации «реальных» компонентов смеси) и с массовыми числами, лежащими в пределах от \textit{M${}_{1}$} до \textit{M${}_{m}$}. В этом случае значение M* может принимать любые значения в интервале от \textit{M${}_{1}$} до \textit{M${}_{m}$}. Это позволяет построить кривую зависимости суммарного потока в каскаде от величины $M^{*}$ и найти ее минимум. 
\end{enumerate}
  
Тем самым, данный подход позволяет из бесконечного множества набора параметров R-каскадов, обеспечивающих получение заданных концентраций целевого компонента в выходящих потоках, выбрать параметры такого R-каскада, который отвечает минимуму величины суммарного потока \cite{sulaberidzeSravnenieOptimalnyhModelnyh2008}. При этом полученные параметры такого R-каскада будет незначительно (менее, чем на 1\%) отличаться от параметров оптимального по величине суммарного потока каскада (при заданных концентрациях целевого компонента в потоках отбора и отвала) \cite{songComparativeStudyModel2010}. Такой R-каскад можно рассматривать как наилучший или «эталонный».

Приведенные выше свойства каскада с несмешиванием по относительным концентрациям выбранной пары компонентов (R-каскада) делают его очень удобным для численного моделирования процессов молекулярно-селективного массопереноса в каскаде казовых центрифуг для разделения многокомпонентных смесей, таких как регенерированный уран. К тому же основной целью проведения вычислительных экспериментов в диссертационной работе являлся расчет изотопных составов получаемого в схеме конечного продукта (товарного низкообогащенного урана) и оценка ключевых интегральных параметров каскадных схем (массовые расходы регенерата и обедненного урана, потоки между каскадами, затраты работы разделения и другие), выбор был сделан именно в пользу модели R-каскада. 

Следует отметить, что как таковые понятия «работа разделения» и «единица работы разделения» (ЕРР) первоначально введены только для двухкомпонентных смесей. Для многокомпонентной смеси и, в том числе, смеси регенерированного урана, понятие работы разделения является условным. Поэтому в приведенных ниже результатах под работой разделения подразумевали условную величину прямо пропорциональную числу газовых центрифуг в каскаде (или суммарному потоку каскада при условии работы центрифуг в идентичных режимах).

\section{Алгоритмы расчета параметров ординарных «квазиидеального» каскада и Q-каскада при различных заданных параметрах. Примеры расчетов.}


\section{Задания для самостоятельного выполнения.}



\clearpage