\chapter{Общие уравнения симметрично-противоточного каскада}


\section{Понятие разделительного элемента, ступени, каскада.}


\subsection{Понятие разделительного элемента}

В общем случае разделительный элемент может иметь несколько входов и выходов. В практике разделения изотопов однофазными методами, он имеет, как правило, один вход и два выхода и называется \textit{простым разделительным элементом} (рис. \ref{1_1}). В этом случае в разделительный элемент подают поток разделяемой смеси $L_{э} $ с концентрацией (мольной долей\footnote{ $\ \textrm{Е}\textrm{с}\textrm{л}\textrm{и}\ \textrm{р}\textrm{е}\textrm{ч}\textrm{ь}\ \textrm{и}\textrm{д}\textrm{е}\textrm{т}\ \textrm{о}\ \textrm{р}\textrm{а}\textrm{з}\textrm{д}\textrm{е}\textrm{л}\textrm{е}\textrm{н}\textrm{и}\textrm{и}\ \textrm{и}\textrm{з}\textrm{о}\textrm{т}\textrm{о}\textrm{п}\textrm{о}\textrm{в}\ \textrm{т}\textrm{я}\textrm{ж}\textrm{е}\textrm{л}\textrm{ы}\textrm{х}\ \textrm{э}\textrm{л}\textrm{е}\textrm{м}\textrm{е}\textrm{н}\textrm{т}\textrm{о}\textrm{в},\ \textrm{т}\textrm{о}\ \textrm{в}\ \textrm{э}\textrm{т}\textrm{о}\textrm{м}\ \textrm{с}\textrm{л}\textrm{у}\textrm{ч}\textrm{а}\textrm{е}\ \textrm{в}\textrm{е}\textrm{л}\textrm{и}\textrm{ч}\textrm{и}\textrm{н}\textrm{ы}\ \textrm{м}\textrm{о}\textrm{л}\textrm{ь}\textrm{н}\textrm{о}\textrm{й}\ \textrm{и}\ \textrm{м}\textrm{а}\textrm{с}\textrm{с}\textrm{о}\textrm{в}\textrm{о}\textrm{й}\ \textrm{к}\textrm{о}\textrm{н}\textrm{ц}\textrm{е}\textrm{н}\textrm{т}\textrm{р}\textrm{а}\textrm{ц}\textrm{и}\textrm{и}\ \textrm{п}\textrm{р}\textrm{а}\textrm{к}\textrm{т}\textrm{и}\textrm{ч}\textrm{е}\textrm{с}\textrm{к}\textrm{и}\ \textrm{с}\textrm{о}\textrm{в}\textrm{п}\textrm{а}\textrm{д}\textrm{а}\textrm{ю}\textrm{т}.$ }) ценного (целевого изотопа) $c$, а из него отбирают два потока: «обогащенный», с более высокой, чем во входящем потоке, концентрацией $c'$, и «обедненный», с более низкой концентрацией $c''$. В отсутствие потерь рабочего вещества, которые могут быть вызваны, например, его частичным разложением в процессе разделения, обогащенный поток будет равен $L'_{э} =\theta L_{э} $, а обедненный, соответственно,~~$L''_{э} =(1-\theta )L_{э} $. Отношение $\theta =\frac{L'_{э} }{L_{э} } $ называют \textit{коэффициентом деления потока}, а расход $L_{э} $ -- \textit{производительностью элемента}. 


\begin{figure}[ht]
  \centerfloat{\includegraphics[scale=0.7]{images/theory/lu15087t0po}}
  \caption{Схема разделительной ступени }\label{1_1}
\end{figure}


\subsection{Понятие разделительной ступени}


В разделительную ступень параллельно соединяют элементы с одинаковыми величинами $\theta $, $c,$ $ $$c'$ и $c''$ (рис.1.2). Она имеет суммарную производительность $L$ и выходящие потоки $L'_{} =\theta L_{} $и $L''_{} =(1-\theta )L_{} $. Такое представление о работе элементов в ступени предполагает, что все они работают в одинаковых условиях и имеют одинаковые характеристики.

Между приращением $\delta '=c'-c$ и уменьшением $\delta ''=c-c''$ концентрации целевого компонента в выходящих потоках разделительного элемента существует определенная связь. Для стационарного состояния эта связь может быть найдена из следующих уравнений материального баланса:

\begin{equation} \label{GrindEQ__1_1_} 
L=L'+L'', 
\end{equation} 
\begin{equation} \label{GrindEQ__1_2_} 
Lc=L'(c+\delta ')+L''(c-\delta ''). 
\end{equation} 

Из \ref{GrindEQ__1_1_} и \ref{GrindEQ__1_2_} следует, что 
\begin{equation} \label{GrindEQ__1_3_} 
\delta ''=\frac{L'}{L''} \delta '=\frac{\theta }{1-\theta } \delta '. 
\end{equation} 

Одной из основных разделительных характеристик ступени является \textit{полный коэффициент разделения}, который для большинства однофазных методов разделения не зависит от состава изотопной смеси
\begin{equation} \label{GrindEQ__1_4_} 
q={\frac{c'}{1-c'}  \mathord{\left/{\vphantom{\frac{c'}{1-c'}  \frac{c''}{1-c''} =\frac{R'}{R''} }}\right.\kern-\nulldelimiterspace} \frac{c''}{1-c''} =\frac{R'}{R''} } .                              
\end{equation} 

Здесь $R'$ и $R''$ - значения относительной концентрации ценного (целевого) изотопа ($R=\frac{c}{1-c} $) в потоках обогащенной и обедненной фракции. Коэффициент разделения $q$ характеризует эффект разделения, достигаемый в одном элементе или ступени, и может зависеть от производительности ступени $L$ и коэффициента деления потока $\theta $:
\begin{equation} \label{GrindEQ__1_5_} 
q=q(L,\theta ), 
\end{equation} 

где $q(L,\theta )$ - некоторая функция, которую определяют по результатам теоретических или экспериментальных исследований.

В соответствии со сказанным к числу основных параметров ступени относятся восемь величин: $L,\; L',\; L'',\; c,\; c',\; c'',\; q,\; \theta $, связанных независимыми соотношениями \ref{GrindEQ__1_1_}~--\ref{GrindEQ__1_5_}. Причём из перечисленных параметров свободными (независимыми) являются только три. Как правило, в качестве свободных параметров рассматривают величины $L,\; \; c\; $ и $\theta $. Однако, в зависимости от рассматриваемой задачи могут быть выбраны их различные комбинации.

Для удобства анализа эффекта разделения в элементе (ступени) могут быть введены дополнительные параметры и характеристики. Например, коэффициенты разделения по обогащённой $\alpha $ и обеднённой $\beta $ фракциям, рассчитываемые как
\begin{equation} \label{GrindEQ__1_6_} 
\alpha ={\frac{c'}{1-c'}  \mathord{\left/{\vphantom{\frac{c'}{1-c'}  \frac{c}{1-c} }}\right.\kern-\nulldelimiterspace} \frac{c}{1-c} }  , \beta ={\frac{c}{1-c}  \mathord{\left/{\vphantom{\frac{c}{1-c}  \frac{c''}{1-c''} }}\right.\kern-\nulldelimiterspace} \frac{c''}{1-c''} } .        
\end{equation} 

Эти параметры характеризуют величину эффекта разделения в обогащённой и обеднённой фракции по отношению к концентрации в потоке питания. Кроме того, для описания процесса разделения удобно пользоваться коэффициентами обогащения $\varepsilon $, $\varepsilon '$, $\varepsilon ''$, равными
\begin{equation} \label{GrindEQ__1_7_} 
\varepsilon =q-1, \varepsilon '=\alpha -1, \varepsilon ''=1-{1 \mathord{\left/{\vphantom{1 \beta }}\right.\kern-\nulldelimiterspace} \beta } .             
\end{equation} 

Набор параметров $\varepsilon $, $\varepsilon '$, $\varepsilon ''$ позволяет определить \textit{полное обогащение ступени}
\begin{equation} \label{GrindEQ__1_8_} 
\delta =c'-c''=\delta '+\delta '',                          
\end{equation} 

где $\delta '=c'-c$, $\delta ''=c-c''$. Если выразить $c'$ и $c''$ из \ref{GrindEQ__1_6_} и использовать \ref{GrindEQ__1_7_}, то в результате получим
\begin{equation} \label{GrindEQ__1_9_} 
\delta '=\frac{\varepsilon 'c(1-c)}{1+\varepsilon 'c}  ,  \delta ''=\frac{\varepsilon ''c(1-c)}{1-\varepsilon ''c} .              
\end{equation} 

Отсюда видно, что при заданных коэффициентах $\varepsilon '$, $\varepsilon ''$ зависимости $\delta '$ и $\delta ''$ от $c$ имеют максимумы, не совпадающие друг с другом. В случае «слабого обогащения» ($\varepsilon =q-1<<1$) наибольшие значения $\delta '$ и $\delta ''$ достигаются в одной точке $c=0,5$, а формулы \ref{GrindEQ__1_9_} существенно упрощаются:
\begin{equation} \label{GrindEQ__1_10_} 
\delta '=\varepsilon 'c(1-c),  \delta ''=\varepsilon ''c(1-c).              
\end{equation} 

При подстановке \ref{GrindEQ__1_10_} в \ref{GrindEQ__1_8_} имеем
\begin{equation} \label{GrindEQ__1_11_} 
\delta =\varepsilon c(1-c),  \varepsilon '=\varepsilon (1-\theta ), \varepsilon ''=\theta \varepsilon .    
\end{equation} 

Согласно \ref{GrindEQ__1_3_} и \ref{GrindEQ__1_8_} обогащения $\delta $, $\delta '$, $\delta ''$ связаны друг с другом балансовыми соотношениями
\begin{equation} \label{GrindEQ__1_12_} 
\delta '=(1-\theta )\delta ,  \delta ''=\theta \delta .                             
\end{equation} 

Следовательно, если коэффициенты разделения не зависят от параметров $L$ и $\theta $, можно путем уменьшения $\theta $ повысить концентрацию ценного компонента в обогащённом потоке $c'$, произведя таким образом «перераспределение» полного обогащения $\delta $ в выходных потоках. При этом концентрация в обеднённом потоке $c''$ будет приближаться к концентрации во входном потоке $A$. Очевидно, что возможность такого изменения обогащений $\delta '$ и $\delta ''$ связана с условиями сохранения материального баланса в разделительном элементе.

\subsection{Понятие каскада.}


Для получения требуемых концентраций ценного (целевого) изотопа ступени соединяют в последовательную цепочку -- каскад, умножающий эффект разделения в одиночной разделительной ступени. Простейшей схемой последовательного соединения ступеней является так называемый \textit{простой каскад} (рис.1.4). Его отличительным признаком является подача обогащенной фракции на питание следующей ступени и выведение потоков обедненной фракции ступеней из процесса дальнейшей переработки. В такой схеме поток питания каскада \textit{F} (от английского слова Feed) подают в первую ступень, поток отбора \textit{P} (Product) является потоком обогащенной фракции последней n-ой ступени. Отвальный поток каскада \textit{W} (Waste) образуют обедненные потоки ступеней (могут не смешиваться друг с другом). Так как эти потоки не участвуют в процессе обогащения, то простой каскад, по существу, является прямоточным. Для разделения изотопов, когда разделяемое вещество, как правило, является дорогим, простой каскад является неэффективным. Это обусловлено существенным сокращением потоков питания ступеней и выведением в отвал потоков с концентрацией ценного (целевого) изотопа, близкой к концентрации в обогащенных фракциях. Поэтому при разделении изотопов применяют более эффективную с точки зрения экономии сырья, а также  имеющую ряд других преимуществ, противоточную (рециркуляционную) схему, в которой обедненная ценным компонентом фракция возвращается в каскад для дальнейшей переработки. Простейшая схема такого каскада приведена на рис.1.5. В этой схеме обогащенный поток $L'_{s} =\theta _{s} L_{s} $ из произвольной $s$-ой ступени подается на вход последующей $s+1$-ой ступени, а обедненный $L''_{s} =(1-\theta _{s} )L_{s} $-- на вход $s-1$-ой ступени. При таком соединении на входе в $s$-ую ступень смешиваются потоки из предыдущей $s-1$-ой ступени и из последующей $s+1$-ой. Такой каскад является \textit{противоточным}, а способ соединения ступеней с помощью внешних коммуникаций, т.е. таких коммуникаций, в которых передаются уже разделенные потоки, называют\textit{ внешним каскадированием}.

Рис.1.4. Схема простого каскада


Изображенная на рис.1.5. схема характерна тем, что между любыми соседними ступенями можно провести поперечное сечение (на рисунке изображено двойной пунктирной линией), пересекающее только две коммуникации. Такой каскад называют \textit{симметричным.}

Если потоки направляют не в соседние предыдущую и последующую ступени, а через одну или через несколько ступеней, то такой противоточный каскад называется \textit{несимметричным} (рис.1.6).


% \includegraphics*[width=4.27in, height=1.21in]{image2}

Рис.1.5. Схема соединения ступеней в симметричном противоточном каскаде



Отметим, что при внешнем каскадировании разделительная ступень считается заданной ячейкой схемы, для которой коэффициент разделения и его зависимость от коэффициента деления потоков должны быть известны, после чего сам процесс разделения оказывается для построения каскадов несущественным. Тем самым теория построения "внешних" потоков оказывается независимой от конкретного метода разделения.



\noindent Рис. 1.6. Пример соединения ступеней в несимметричный каскад с подачей потока питания через одну ступень в прямом направлении


\section{Получение уравнений симметрично-противоточного каскада для случая произвольного числа компонентов и произвольных коэффициентов разделения.}

Рассмотрим симметричный каскад, состоящий из \textit{N} ступеней. 

Пусть на вход ступени с номером $s=f$ подают поток питания $F$ с концентрацией $A_{F} $. Поток, обогащенный ценным (целевым) изотопом, отбирается с правого конца каскада $(s=N)$ (сокращенно: отбор), а обедненный поток - с левого конца каскада $(s=1)$ (сокращенно: отвал). Соответственно обозначим концентрации в потоках отбора $A_{P} $ и отвала $A_{W} $. Ступени каскада нумеруются последовательно от 1 на отвале до \textit{N} на отборе. Часть каскада от точки подачи питания $(s=f,\; \; f+1,\; { ...}\; {,}\; \; {N)}$ называется обогатительной, а часть слева $(s=1,\; \; 2,\; \; {...}\; {,}\; \; f{ -1)}$ - обеднительной.

В симметричной противоточной схеме можно использовать частичный или полный возврат обогащённых или обеднённых потоков отбора или отвала на вход соответствующей ступени $(s=N$ или $s=1)$. Такие коммутации потоков называют "закрутками" и обычно их применяют на концевых ступенях каскада [5] (рис. 1.8).



% \noindent \includegraphics*[width=4.40in, height=1.51in]{image3}

Рис 1.7. Схема симметричного каскада для разделения бинарных смесей



\noindent 

Рис.1.8. Схемы закруток потоков : а) -- на отвале; б) -- на отборе



Внешними параметрами каскада являются шесть переменных, определяющих внешние рабочие условия: $F,\; P,\; W$ - потоки питания, отбора и отвала каскада; $A_{F} ,\; A_{P} ,\; A_{W}$ - концентрации в соответствующих потоках. К внутренним относятся: \textit{N} -- общее количество ступеней в каскаде, \textit{f} -- номер ступени, на вход которой подают поток питания, параметры ступеней: $L_{S} ,\; L'_{S} ,\; L''_{S} $ - входной и два выходных потока на s-ой ступени каскада, $A_{S} ,\; A'_{S} ,\; A''_{S} \; $- концентрации в соответствующих потоках; $q_{S} ,\; \alpha _{S} ,\; \beta _{S} $ - коэффициенты разделения и $\theta _{S} $ - коэффициенты деления потоков $(s=\overline{1,N)}$.

В стационарном состоянии каскада внутренние параметры каскада можно выразить через внешние параметры каскада и уравнения разделения в ступени. Проведем поперечное сечение между некоторой \textit{s} -- ой ступенью и соседней с ней \textit{s}+1 -- ой ступенью обогатительной части каскада (обозначено на рис. 1.7 пунктиром) и рассмотрим часть каскада, находящуюся справа от этого мысленного сечения. Потоки разделяемого вещества и потоки ценного (целевого) изотопа входящие в эту часть каскада $L'_{S} =\theta _{S} L_{S} $ и $L'_{S} A'_{S} =\theta _{S} L_{S} A_{s}^{/} $ и выходящие из нее $L''_{S+1} =(1-\theta _{S+1} )L_{S+1} $ и $L''_{S+1} A''_{S+1} =(1-\theta _{S+1} )L_{S+1} A''_{S+1} $ связаны уравнениями материального баланса:

\begin{equation} \label{GrindEQ__1_71_} 
\theta _{S} L_{S} -(1-\theta _{S+1} )L_{S+1} =P,                
\end{equation} 
\begin{equation} \label{GrindEQ__1_72_} 
\theta _{S} L_{S} c'_{S} -(1-\theta _{S+1} )L_{S+1} c''_{S+1} =Pc_{P}.     
\end{equation} 

В этих уравнениях через, $c'_{s}$ и $c''_{s}$ обозначены концентрации ценного (целевого) изотопа соответственно на выходах из ступени.

Аналогичные соотношения можно записать для обеднительной части каскада

\begin{equation} \label{GrindEQ__1_73_} 
  \theta _{s} L_{s} -(1-\theta _{s+1} )L_{s+1} =-W, 
\end{equation}

\begin{equation} \label{GrindEQ__1_74_} 
  \theta _{s} L_{s} c'_{s} -(1-\theta _{s+1})L_{s+1} c''_{s+1} =-WC_{w}. 
\end{equation} 

Для ступени с номером \textit{s=f}, на вход которой подают поток питания \textit{F}, уравнения материального баланса имеют вид
\begin{equation} \label{GrindEQ__1_75_} 
L_{f} =\theta _{f-1} L_{f-1} +(1-\theta _{f+1} )L_{f+1} +F,     
\end{equation}

\begin{equation} \label{GrindEQ__1_76_}
  L_{f} c_{f} = \theta _{f-1} L_{f-1} c'_{f-1} + (1-\theta _{f+1})L_{f+1} c''_{f+1} +Fc_{F}.  
\end{equation}

При использовании формул \ref{GrindEQ__1_71_} -- \ref{GrindEQ__1_74_} следует иметь в виду, что $L_{0} =0$ и $L_{N+1} =0$. Концентрации $c_{S} ,\; c'_{S} $ и $c''_{S} $ на каждой ступени связаны соотношениями \ref{GrindEQ__1_4_}, \ref{GrindEQ__1_6_}, \ref{GrindEQ__1_7_}, а внешние параметры при отсутствии потерь вещества в ступенях каскада должны удовлетворять уравнениям материального баланса:
\begin{equation} \label{GrindEQ__1_77_} 
F=P+W,                                       
\end{equation} 
\begin{equation} \label{GrindEQ__1_78_} 
Fc_{F} {}^{} =Pc_{P} {}^{} +Wc_{W} {}^{} .                      
\end{equation} 

Вводя для разности концентраций на входах двух произвольных соседних  ступеней обозначение
\begin{equation} \label{GrindEQ__1_79_} 
\Delta _{S} =c_{S+1} -c_{S} ,\; (s=1,N-1) 
\end{equation} 

и, вычитая соотношение \ref{GrindEQ__1_71_}, умноженное на $c_{S} $, из \ref{GrindEQ__1_72_}, получим
\begin{equation} \label{GrindEQ__1_80_} 
\Delta _{S} =\frac{\theta _{S} L_{S} }{(1-\theta _{S+1} )L_{S+1} } \delta '_{S} +\delta ''_{S+1} -\frac{P(c_{P} -c_{S} )}{(1-\theta _{S+1} )L_{S+1} } ,            
\end{equation} 

где величины $\delta '_{S} =c'_{S} -c_{S} $ и$\delta ''_{S} =c_{S} -c''_{S} $ определяются соотношениями \ref{GrindEQ__1_9_}, \ref{GrindEQ__1_7_}. Для обеднительной части каскада справедливы точно такие же уравнения, только в правой части вместо \textit{P} и $Pc_{P} $ следует подставлять --\textit{W} и $-Wc_{W} $, т.е.
\begin{equation} \label{GrindEQ__1_81_} 
\Delta _{S} =\frac{\theta _{S} L_{S} }{(1-\theta _{S+1} )L_{S+1} } \delta '_{S} +\delta ''_{S+1} -\frac{W(c_{S} -c_{W} )}{(1-\theta _{S+1} )L_{S+1} }  
\end{equation} 

С помощью уравнений \ref{GrindEQ__1_71_} -- \ref{GrindEQ__1_78_} можно рассчитать распределения концентраций и коэффициентов деления потоков по ступеням каскада, если известны коэффициенты разделения $q_{S} ,\; \alpha _{S} ,\; \beta _{S} $ и полное число ступеней в каскаде \textit{N}, номер ступени\textit{ f}, в которую вводится поток питания, и зависимость потока $L_{S} $ от номера ступени. Подобного рода задачи обычно решают численными методами с применением ЭВМ.

В случае «слабого обогащения», когда величина обогащения мала по сравнению с концентрацией во входящем в ступень потоке, т.е. $\delta _{S} /c_{S} <<1$ система уравнений, определяющих каскад \ref{GrindEQ__1_73_} -- \ref{GrindEQ__1_76_} или \ref{GrindEQ__1_80_} -- \ref{GrindEQ__1_81_} может быть подвергнута значительным упрощениям.

Если обогащение на ступени мало, то для получения на каскаде требуемых изменений концентраций, как правило, нужно большое число ступеней $(N>>1)$, т.е. каскад должен быть "длинным". В этом случае можно считать, что все параметры каскада от ступени к ступени изменяются незначительно, а величина потока изотопной смеси, проходящего через произвольную ступень, намного превосходит величину потока отбора, т.е. $L_{S} \approx L_{S+1} $, $\delta '_{S} \approx \delta '_{S+1} $, $\delta ''_{S} \approx \delta ''_{S+1} $, $\vartheta _{S} \approx \vartheta _{S+1} $ и $P/L_{S} <<1.$ Так как число ступеней в каскаде велико, а изменение параметров при переходе от ступени к ступени мало, то можно представить \textit{s} как непрерывно меняющуюся переменную, а параметры каскада $L,\; \theta ,$ и \textit{N} непрерывными функциями от этой переменной.

С учетом сказанного из уравнения баланса \ref{GrindEQ__1_71_} следует $\theta _{S} \approx 1-\theta _{S} $, т.е.
\begin{equation} \label{GrindEQ__1_82_} 
\theta _{S} \cong \frac{1}{2}  
\end{equation} 

Условие \ref{GrindEQ__1_82_} выражает основное свойство симметричного каскада с малым обогащением на отдельной ступени. Потоки в ступенях этого каскада делятся почти пополам. Полагая в \ref{GrindEQ__1_80_} $\theta _{S} \approx \frac{1}{2} $, $L_{S} \approx L_{S+1} $, $\delta '_{S} \approx \delta '_{S+1} $, $\delta ''_{S} \approx \delta ''_{S+1} $ и учитывая, что согласно \ref{GrindEQ__1_8_}, \ref{GrindEQ__1_10_} -- \ref{GrindEQ__1_12_}

\begin{equation} \label{GrindEQ__1_83_} 
\delta '_{S} =\delta ''_{S} =\frac{1}{2} \mathrm{\; }\varepsilon c_{S} (1-c_{S} ),                    
\end{equation} 

получим
\begin{equation} \label{GrindEQ__1_84_} 
\Delta _{S} =\varepsilon c_{S} (1-c_{S} )-\frac{P(c_{P} -c_{S} )}{\frac{1}{2} L_{S+1} }  
\end{equation} 

Считая параметры каскада непрерывными функциями от переменной \textit{s} и заменяя $\Delta _{S} $ на $\frac{dc}{ds} $, перепишем предыдущее соотношение в виде
\begin{equation} \label{GrindEQ__1_85_} 
\frac{dc}{ds} =\varepsilon c(1-c)-\frac{2P(c_{P} -c)}{L} ,                    
\end{equation} 

в которых $c=c(s)$ и $L=L(s)$ - соответственно распределение концентраций и потоков вдоль каскада. 

Соответствующее уравнение для обеднительной части каскада будет иметь вид
\begin{equation} \label{GrindEQ__1_86_} 
\frac{dc}{ds} =\varepsilon c(1-c)-\frac{2W(c-c_{W} )}{L} .                     
\end{equation} 

При этом потоки отбора, отвала и питания и концентрации в этих потоках по-прежнему связаны двумя уравнениями баланса \ref{GrindEQ__1_77_} и \ref{GrindEQ__1_78_}. Минимальный поток питания для каждой ступени, соответствующий данному отбору \textit{P} и концентрации $c_{P} $, можно найти из условия равенства нулю градиента концентрации \ref{GrindEQ__1_85_}, т.е.

\begin{equation} \label{GrindEQ__1_87_} 
\left(\frac{\varepsilon L}{2P} \right)_{\min } =\frac{c_{P} -c}{c(1-c)} .                                   
\end{equation} 

Уравнение \ref{GrindEQ__1_85_} можно рассматривать как частный случай общего уравнения \ref{GrindEQ__1_80_} для приращения концентраций в применении к случаю слабого обогащения. Решение задачи в этом случае гораздо проще, потому что вместо уравнения \ref{GrindEQ__1_80_} в конечных разностях мы имеем обыкновенное дифференциальное уравнение первого порядка и еще потому, что для нахождения распределения концентраций в каскаде с заданным распределением потоков $L_{S} $ достаточно проинтегрировать только одно уравнение.

Наибольшие изменения концентраций при переходе от одной ступени к другой имеют место в безотборном режиме, когда $P=W=F=0$. Такой режим можно организовать в заполненном разделяемой смесью каскаде при наличии "закруток" на концевых ступенях. Поскольку в этом случае $c''_{S} =c'_{S-1} $, то
\begin{equation} \label{GrindEQ__1_88_} 
R'_{1} =q_{1} R''_{1} ,    
\end{equation} 
\begin{equation} \label{GrindEQ__1_89_} 
R'_{S} =q_{S} R''_{S} ,    
\end{equation} 

и степень разделения в каскаде $Q=R'_{N}/R''_{1}$ достигает максимальной величины

\begin{equation} \label{GrindEQ__1_90_} 
Q=\prod _{S=1}^{N}q_{S},                      
\end{equation} 

где $R'_{N} $ и $R''_{{1}} $ - относительные концентрации ценного (целевого) изотопа на концах каскада. Если все ступени в каскаде имеют одинаковые полные коэффициенты разделения, т.е. $q_{S} \equiv q$, то соотношение \ref{GrindEQ__1_90_} может быть преобразовано к виду
\begin{equation} \label{GrindEQ__1_91_} 
N=\ln Q/\ln q,     
\end{equation} 

известному как формула Фенске [12]. Она определяет минимально возможное число ступеней, необходимое для достижения заданного значения степени разделения каскада. Характерно, что число ступеней в каскаде \textit{N} не зависит от формы каскада, т.е. конкретного распределения $L_{S} $.

В случае слабых одинаковых обогащений на ступенях из \ref{GrindEQ__1_85_} и \ref{GrindEQ__1_86_} для безотборного режима каскада имеем
\begin{equation} \label{GrindEQ__1_92_} 
\frac{dc}{ds} =\varepsilon c(1-c),    
\end{equation} 

откуда после интегрирования получаем экспоненциальный закон изменения концентраций по ступеням
\begin{equation} \label{GrindEQ__1_93_} 
R_{2} =R_{1} \exp (\varepsilon s_{12} ),   
\end{equation} 

здесь $R_{1} ,\; R_{2} $ - относительные концентрации ступеней, работающих на участке каскада, определяемом концентрациями \textit{c}${}_{1}$ и \textit{c}${}_{2}$; \textit{s}${}_{12}$ -- соответствующее количество ступеней. В режимах работы каскада с непрерывным отбором и отвалом $(P\ne 0,\; W\ne 0)$ изменения концентраций на ступенях, а, следовательно, и степень разделения каскада будут меньше.

\textbf{Критерии эффективности работы каскада}

В задачах проектирования каскадов целесообразно определять их параметры, исходя из принятого критерия эффективности. Возможны два принципиальных подхода к выбору критериев.

Первый подход предполагает, что параметры ступеней могут быть выбраны из физических соображений, не связанных непосредственно с поставленной целью разделения. Физический критерий эффективности выражается требованием, чтобы энтропия при соединении потоков на входе каждой ступени не возрастала, т.е. чтобы термодинамическая работа, связанная с изменением концентрации при разделении смеси, не терялась. Для этого необходимо, чтобы концентрации различных потоков на входе в каждую ступень были одинаковыми. Для каскада с тремя внешними потоками, представленного на рис.1.5, это соответствует выполнению условий
\begin{equation} \label{GrindEQ__1_94_} 
\begin{array}{l} {c'_{S-1} =c_{S} =c''_{S+1} ,} \\ {\; \; \; \; \; \; \; \; c_{f} =c_{F} ,} \end{array} 
\end{equation} 

или $\begin{array}{l} {R'_{S-1} =R_{S} =R''_{S+1} ,} \\ {\; \; \; \; \; \; \; \; R_{f} =R_{F} .} \end{array}$                                       \ref{GrindEQ__1_95_}

Соотношения \ref{GrindEQ__1_94_}, \ref{GrindEQ__1_95_} называются условиями несмешения, а каскад, удовлетворяющий этим требованиям -- \textit{идеальным}. 

В другом подходе определяют практические потребности изотопного производства. Создание крупного разделительного предприятия (завода) связано с минимизацией удельных материальных затрат на производство обогащенного продукта. Эта задача весьма сложная в силу того, что необходимо определить большое число параметров, влияющих на затраты производства. Задачу оптимизации каскада можно упростить, учитывая специфику метода разделения. В общем случае задача оптимизации может быть записана в виде
\begin{equation} \label{GrindEQ__1_96_} 
\psi =\psi (u_{1} ,u_{2} ,...,u_{k} )\to \min (\max ), 
\end{equation} 

где $\Psi $ - целевая функция (показатель эффективности); $u_{1} ,\, u_{2} ,\, ...,\, u_{k} $ - независимые параметры каскада; $\min (\max )$- значение целевой функции (минимум или максимум) при оптимальных значениях независимых параметров. При использовании молекулярно-кинетических методов разделения удельные затраты на производство обогащённого продукта, как правило, пропорциональны суммарному количеству элементов в каскаде. В соответствии с этим при заданных внешних параметрах в качестве критерия оптимизации можно принять минимум суммарного количества разделительных элементов:
\begin{equation} \label{GrindEQ__1_97_} 
\Psi =\sum _{s=1}^{N}Z_{S}  \to \min , 
\end{equation} 

где \textit{Z${}_{S}$} -- число разделительных элементов в \textit{s} -- ой ступени каскада. Величина $\sum _{s=1}^{N}Z_{S}  $ при работе элементов с заданными одинаковыми потоками \textit{L${}_{\textrm{Э}}$} может быть представлена в виде
\begin{equation} \label{GrindEQ__1_98_} 
\sum _{s=1}^{n}Z_{S} =\frac{\sum _{s=1}^{N}L_{S}  }{L_{-} }  .    
\end{equation} 

и, следовательно, в этом случае целью оптимизации является минимизация суммарного потока питания ступеней
\begin{equation} \label{GrindEQ__1_99_} 
\psi =\sum _{s=1}^{N}L_{S}  \to \min .    
\end{equation} 

Данный критерий предусматривает, что все внешние параметры каскада варьируются в допустимой области их изменения до получения минимального суммарного потока. Каскад, отвечающий требованию \ref{GrindEQ__1_99_} будем называть \textit{оптимальным.}







