% \chapter*{Введение}                         % Заголовок
% \addcontentsline{toc}{chapter}{Введение}    % Добавляем его в оглавление


\section{Востребованность изотопных смесей (бинарных и многокомпонентных) в различных областях науки и техники (примеры).}

Востребованность изотопных смесей (бинарных и многокомпонентных) в различных областях науки и техники (примеры).

\section{Принципы каскадирование одиночных разделительных элементов. Роль теории каскадов для разделения изотопных смесей при изучении физических закономерностей массопереноса компонентов в подобных установках.}

Принципы каскадирование одиночных разделительных элементов. Роль теории каскадов для разделения изотопных смесей при изучении физических закономерностей массопереноса компонентов в подобных установках.

\section{Понятие и роль модельных каскадов в общей теории.}

....


Под поверочным расчетом каскада подразумевают следующую задачу:
Задано: состав исходной разделяемой смеси,  число ступеней в каскаде и величины питающих их потоков, величины внешнего потока питания и одного из выходящих потоков каскада (отбора или отвала), параметры ступени (например, относительные коэффициенты разделения ступеней и др.).
Подлежат определению: концентрации всех компонентов в потоках отбора  и отвала и распределение концентраций компонентов по ступеням каскада. 
Поверочный расчет каскада необходим при исследовании оптимального управления процессом разделения, при изменении режимов работы и отдельных параметров разделительного каскада \cite{sulaberidzeTeoriyaKaskadovDlya2011}. Основные трудности поверочного расчета связаны с тем, что неизвестные концентрации компонентов в потоках отбора и отвала сами явно входят в основные уравнения переноса (или их граничные условия). Невозможность аналитического решения этих уравнений вызывает необходимость разработки численных методов, малочувствительных к заданию начальных приближений для концентраций компонентов в выходящих потоках. На сегодняшний день предложены различные методы поверочного расчета, которые позволяют численно решить данную задачу \cite{sulaberidzeTeoriyaKaskadovDlya2011, sazykinUsovershenstvovannyyMetodRascheta1997, wuCalculationMethodsDetermining1988, holpanovEffektivnyyMetodRascheta1998, potapovCalculationSquaredoffCascades1996, zengRobustEfficientCalculation2000}.

Под проектировочным расчетом каскада обычно подразумевают следующую задачу \cite{sulaberidzeTeoriyaKaskadovDlya2011}.
Задано: состав исходной разделяемой смеси, один из выходящих потоков каскада (отбор или отвал), концентрации одного из компонентов (целевого или ключевого) в потоках отбора и отвала.
Подлежат определению: все внутренние параметры каскада (распределение потока и концентраций компонентов по ступеням каскада и др.), концентрации остальных компонентов (всех кроме ключевого) в потоках отбора и отвала. 
При этом, очевидно, что найденные параметры каскада должны соответствовать оптимальным условиям разделения в каскаде.

