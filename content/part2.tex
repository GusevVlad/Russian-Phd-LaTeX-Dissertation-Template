\chapter{Модельные каскады для разделения бинарных смесей.}


Трудности решения (\ref{GrindEQ__1_24_})--(\ref{GrindEQ__1_27_}), (\ref{GrindEQ__1_35_})--(\ref{GrindEQ__1_38_}) и (\ref{GrindEQ__1_39_})--(\ref{GrindEQ__1_40_}) в общем случае, стимулировали развитие упрощенных подходов, которые позволяют получить аналитическое решение для данных систем при введении определенных предположений. Полученные в результате таких упрощений физико-математические модели симметрично-противоточного каскада сохраняют закономерности молекулярно-селективного массопереноса, но позволяют заметно упростить соответствующие расчетные процедуры для определения оптимальных параметров каскада. Такие каскады получили название модельных \cite{minenkoTeoriiKaskadovDlya1965, delagarzaMulticomponentIsotopeSeparation1961, zhigalovskiyLekcionnyeMaterialyPo1999, kolokoltsovDesignCascadesSeparating1970, kolokolcovVoprosuPostroeniiKaskadov1970, minenkoPredelnoeObogashcheniePromezhutochnyh1972, yamamotoMulticomponentIsotopeSeparating1978, wuStudyMulticomponentIsotope, borisevichRascheteKaskadovDopolnitelnym1993, woodCriterionEffiencyMultiisotope1999, sulaberidzeOsobennostiObogashcheniyaKomponentov2006, sazykinKvaziidealnyeKaskadyDlya2000, sulaberidzeSravnenieOptimalnyhModelnyh2008}.

Модельные каскады действуют как физически эквивалентные представления и, как показано в \cite{sulaberidzeClassificationModelCascades2020}, могут быть выведены из <<обобщенного модельного каскада>>, которым является симметрично-противоточный каскад с постоянными по его длине относительными коэффициентами разделения).

Целесообразной областью применения теории модельных каскадов является ее использование при предварительном рассмотрении актуальных проблем современной теории разделения многокомпонентных изотопных смесей в каскадах и смежных с разделительной наукой областей, таких, например, как ядерная энергетика.
% В рамках данной работы анализ строится на основе теории модельных каскадов, однако выявленные физические закономерности массопереноса в рассмотренных каскадных схемах справедливы и в близких к реализуемых в производственных условиях прямоугольных и прямоугольно секционированных каскадов.

Далее рассмотрим подробнее математические модели, нашедшие свое применение в расчетных исследованиях.


\section{«Идеальный» каскад}

\subsection{«Идеальный» каскад для разделения бинарных смесей как частный случай симметрично-противоточного каскада с постоянными коэффициентами разделения.}

\subsection{«Идеальный» каскад в случае «слабого обогащения» и немалых обогащений на ступенях.}

\subsection{Сопоставление «идеального» и оптимального каскадов для разделения бинарных смесей.}



\section{Алгоритм расчета параметров «идеального» каскада для различных исходных условий. Примеры расчетов.}



\section{Задания для самостоятельного выполнения.}

