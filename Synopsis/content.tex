\section*{Общая характеристика работы}

\newcommand{\actuality}{\underline{\textbf{\actualityTXT}}}
\newcommand{\progress}{\underline{\textbf{\progressTXT}}}
\newcommand{\aim}{\underline{{\textbf\aimTXT}}}
\newcommand{\tasks}{\underline{\textbf{\tasksTXT}}}
\newcommand{\novelty}{\underline{\textbf{\noveltyTXT}}}
\newcommand{\influence}{\underline{\textbf{\influenceTXT}}}
\newcommand{\methods}{\underline{\textbf{\methodsTXT}}}
\newcommand{\defpositions}{\underline{\textbf{\defpositionsTXT}}}
\newcommand{\reliability}{\underline{\textbf{\reliabilityTXT}}}
\newcommand{\probation}{\underline{\textbf{\probationTXT}}}
\newcommand{\contribution}{\underline{\textbf{\contributionTXT}}}
\newcommand{\publications}{\underline{\textbf{\publicationsTXT}}}

{\actuality}
Построение ядерной энергетики нового типа, устойчивой к ресурсным ограничениям и предусматривающей решение проблемы обращения с радиоактивными отходами, связано с реакторами на быстрых нейтронах, нацелеными на воспроизводство делящегося материала -- энергетического  плутония -- в реакторе на быстрых нейтронах. Однако, по оценкам \cite{andrianovaPerspektivnyeToplivnyeZagruzki2015}, в ближайшие десятилетия, по мере становления двухкомпонентной ядерно-энергетической системы, неизбежен переходный период, когда делящиеся материалы будут повторно использоваться в топливном цикле реакторов на тепловых нейтронах, так как они составляют основную часть парка энергоблоков.

% При этом, регенерированный уран в виде соединений $UO_3$ или $UNH$ имеет нулевую цену на выходе предприятия по переработку отработанного уранового топлива (Завода РТ) \cite{gresleyEnrichingRecyclingUranium1988}.

На сегодняшний день мире в состоянии эксплуатации и сооружения насчитывается порядка 500 ядерных энергоблоков, подавляющее большинство из которых относятся к типу легководных реакторов \cite{PRISHome}. Такие реакторы потенциально могут использовать топливо, изготовленное на основе регенерированных материалов (урана и плутония). В связи с чем большой практический интерес представляет реализация многократного использования делящихся материалов из ОЯТ в топливном цикле имеющихся и сооружаемых энергоблоков. В этом случае на протяжении эксплуатационного времени жизни энергетического реактора осуществляется замыкание топливного цикла с использованием накопленного в ОЯТ урана и плутония.

Основным материалом отработавшего ядерного топлива является уран, составляющий $\approx$90-95\% за вычетом конструкционных материалов. Так как регенерированный уран содержит $^{235}$U на уровне $\geq$0,85\%, то есть долю делящегося изотопа выше, чем в природном уране, дообогащать его на изотопно-разделительном производстве представляется экономически целесообразно \cite{NikipelovNikipelovSudby}.

Использование выделенного из отработавшего ядерного топлива (ОЯТ) регенерированного урана является основным достижимым в ближайшей перспективе направлением вовлечения регенерируемых материалов в топливный цикл энергетических реакторов. Выделенный из ОЯТ регенерированный уран может быть использован в составе топлива ВВЭР различными способами:
\begin{itemize}
  \item центрифужное дообогащение для производства уранового топлива;
  \item включение (в некоторых случаях с предварительным дообогащением) в состав смешанного уран-плутониевого топлива типа MOX или REMIX.
\end{itemize}
 
Рецикл урана является сложной задачей ввиду присутствия в изотопном составе регенерата ряда четных изотопов. В первую очередь, это неприродные $^{232}$U и $^{236}$U. Присутствие первого, ввиду того, что он является опосредованным источником жесткого гамма-излучения, затрудняет обращение с регенератом, как на стадии обогащения, так и на стадии производства твэлов \cite{matveevUran232EgoVliyanie1985}. Влияние же второго сказывается в ухудшении размножающих свойств ядерного топлива, поскольку данный изотоп является паразитным поглотителем тепловых нейтронов. Вдобавок, в регенерате, по сравнению с природным ураном, на порядок выше содержание $^{234}$U, что вносит в смесь регенерата дополнительную нежелательную радиоактивность. При этом, ориентируясь на сегодняшние тенденции к увеличению длительности топливных циклов ВВЭР, которые связаны с повышением глубины выгорания топлива, следует принять во внимание вытекающий из этого рост содержания вредных четных изотопов в регенерате \cite{smirnovEvolutionIsotopicComposition2012}.

Необходимость решения проблемы эффективного многократного вовлечения облученного урана в ядерный топливный цикл (ЯТЦ), тесно связана с поиском и дальнейшей разработкой каскадных схем, которые позволяют эффективно производить из регенерата НОУ, удовлетворяющий стандартным спецификациям.
На сегодняшний день предложен ряд каскадных схем, которые могут быть применены для решения этой задачи, однако их возможности могут быть недостаточны при актуальных параметрах топливного цикла легководных реакторов. Поэтому возникает потребность поиска новых схем, которые могут быть применены для возврата урана в ядерный топливный цикл более эффективно. 

% Таким образом, учитывая принятое в ГК <<Росатом>> стратегическое решение о переходе к замкнутому ЯТЦ, решение перечисленных задач представляется актуальным для современной разделительной науки. 

{\aim} диссертационной работы является изучение физических закономерностей
молекулярно-селективного массопереноса в ординарных и многопоточных каскадах
для разделения многокомпонентных смесей с целью дальнейшего поиска
оптимальных условий обогащения регенерированного урана в подобных каскадах при
его многократном использовании в различных видах регенерированного ядерного
топлива для реакторов на тепловых нейтронах. 

Для~достижения поставленной цели решены следующие {\tasks}:
\begin{enumerate}
  \item Анализ физических закономерностей массопереноса компонентов смеси
  регенерированного урана в ординарном каскаде.
  Выявление физических ограничений нахождения решения задачи обогащения регенерата произвольного изотопного
  состава в одиночном каскаде при одновременном выполнении условий на
  концентрации изотопов $^{232}$U, $^{234}$U и $^{236}$U в получаемом продукте – низкообогащенном уране, а также априорная оценка возможности решения этой задачи.
  \item Физическое обоснование принципов построения двойных каскадов,
  позволяющих корректировать изотопный состав регенерата по концентрациям
  изотопов $^{232}$U, $^{234}$U и $^{236}$U с одновременным расходованием максимального количества
  подлежащего обогащению регенерата при различных исходных концентрациях
  четных изотопов в нем.
  \item Обоснованы физические принципы эффективной «утилизации» загрязненной четными
  изотопами фракции, возникающей в двойных каскадах, с учетом полной или
  частичной подачи данной фракции в третий каскад с предварительным
  перемешиванием ее с природным, обедненным и/или низкообогащенным ураном.
  \item Обоснованы физические принципы эффективной «утилизации» загрязненной четными
  изотопами фракции, возникающей в двойных каскадах, путем замыкания, заключающемся в ее отправке в отдельный двойной каскад, осуществляющий наработку низкообогащенного урана для последующей топливной кампании реактора.
  \item Разработаны методы оценки исследуемых каскадных схем, а также расчетные методики для оптимизации параметров каскадных схем.
  % \item Изучение физических закономерностей изменения изотопного состава регенерата и
  % интегральных характеристик модифицированных двойных каскадов и тройных
  % каскадов при обогащении регенерированного урана с различным исходным
  % содержанием четных изотопов.
  % \item Обобщение и систематизация подходов к выбору каскадной схемы, позволяющих
  % эффективное обогащение регенерированного урана в условиях однократного и
  % многократного рецикла.
  % \item Определение физических закономерностей изменения изотопного состава
  % регенерированного урана и параметров модифицированного двойного каскада для
  % его дообогащения при многократном рецикле урана (отдельно и совместно с
  % плутонием) в топливе реакторов типа ВВЭР.
\end{enumerate}


{\novelty}
\begin{enumerate}
  \item Впервые предложены модификации двойных каскадов, позволяющих корректировать
  изотопный состав регенерата по концентрациям изотопов $^{232}$U, $^{234}$U и $^{236}$U с одновременным расходованием полного количества подлежащего обогащению регенерата при различных исходных концентрациях четных изотопов в нем и других внешних условиях.
  \item Обоснованы физические принципы построения тройных каскадных схем для максимального вовлечения исходного регенерированного урана для воспроизводства топлива реакторов на тепловых нейтронах.
  \item Выполнены оригинальные исследования по изучению физических закономерностей изменения изотопного состава регенерата и интегральных характеристик модифицированных двойных и тройных каскадах при обогащении регенерированного урана с различным исходным содержанием четных изотопов.
  \item Разработаны методы расчетов каскадных схем, позволяющих решить задачу возврата регенерированного урана в топливный цикл в условиях многократного рециклирования.
  \item Разработан обобщенный подход к выбору каскадной схемы для эффективного обогащения регенерированного урана в условиях однократного и многократного рецикла.
  \item Разработка методик оптимизации систем каскадов (двойного и тройного каскадов) для обогащения регенерата урана по различным критериям эффективности, таким как:
  \begin{enumerate}
    \item расход природного урана в цикле;
    \item затраты работы разделения в цикле;
    \item доля потерь $^{235}$U в каскадной схеме;
    \item доля потерь $^{235}$U из исходного регенерата;
    \item доля газовых центрифуг в схеме, в которых превышена предельно допустимая концентрация по $^{232}$U.
  \end{enumerate}
  \item Разработка подхода к утилизации высокоактивного «нештатного» отхода, образующегося в процессе обогащения регенерированного урана в двойном каскаде.
  \item Определение физических закономерностей изменения изотопного состава регенерированного урана и параметров каскадных схем (в модифицированном двойном и тройном каскаде) для его дообогащения при многократном рецикле урана (отдельно и совместно с плутонием) в топливе реакторов типа ВВЭР.
\end{enumerate}

{\influence} 
\begin{enumerate}
  \item Проведенный анализ физических закономерностей массопереноса компонентов смеси регенерированного урана в ординарном каскаде позволяет однозначно определить условия при которых возможно/невозможно получение необходимого количества конечного продукта на основе регенерированного урана различного исходного состава путем обогащения в одиночном каскаде.
  \item Разработанные модификации двойных и тройных каскадов позволяют эффективно решать задачу обогащения регенерированного урана с одновременным выполнением ограничений на концентрации четных изотопов и максимальным вовлечением исходного регенерата.
  \item Проведенный анализ результатов расчетного моделирования молекулярно-селективного массопереноса в модифицированных двойных и тройных каскадах для обогащения регенерата урана выявляет область практической применимости подобных схем для получения НОУ-продукта на основе регенерированного урана.
  \item Предложенные способы оптимизации построения каскадных схем двойного и тройного каскадов позволяют находить наиболее эффективные конфигурации каскадов для возврата регенерированного урана в цикл.
  \item Разработаны рекомендации по использованию результатов работы для обогащения регенерированного урана в условиях однократного и многократного рецикла в различных видах топлива. Представленные в работе результаты могут быть использованы в расчетных группах на предприятиях и организациях, связанных как с проектированием и построением разделительных каскадов, так и непосредственным производством изотопной продукции (АО «Уральский электрохимический комбинат», АО «Сибирский химический комбинат», АО «ТВЭЛ», АО «Восточно-Европейский головной научно-исследовательский и проектный институт энергетических технологий», АО «ПО «ЭХЗ» и др.). Предложенные методики расчета могут лечь в основу технико-экономического анализа обращения с ОЯТ в части получения из восстановленного урана низкообогащенного урана, отвечающего требуемым качествам.  
  % \item Разработан тренировочный программный комплекс для расчета каскада, нацеленного на возврат регенерированного урана. Код оформлен в виде лабораторной работы, которая внедрена в учебный процесс.
\end{enumerate}


{\methods}.
Исследование проводит систематизацию научно-технической литературы, посвященной заявленной теме.
Применены подходы, известные в современной теоретической физике, и в частности, в теории разделения изотопов в каскадах.
В ходе работы обоснованы теоретические принципы построения анализируемых каскадов, и проведено математическое моделирование каскадных схем.
Для проведения расчетов использованы схемы модельных каскадов (квазиидеальный каскад и его разновидность R-каскад, для которого выполняется условие несмешивания относительных концентраций пары выбранных компонентов). Моделирование процессов разделения смесей изотопов урана проводили с использованием разработанных в ходе выполнения работы специализированных компьютерных программ. Применены современные программные средства языков программирования Julia и Python и подключаемых библиотек, таких как NLopt, Optim, ScyPy, предназначенных для решения систем нелинейных уравнений и нелинейной оптимизации, Plots.jl для визуализации результатов.

{\defpositions}
\begin{enumerate}
  \item Результаты анализа физических закономерностей массопереноса компонентов смеси регенерированного урана в ординарном каскаде, позволяющие однозначно определить условия при которых возможно/невозможно получение необходимого количества конечного продукта на основе регенерированного урана различного исходного состава путем обогащения в одиночном каскаде.
  \item Физико-математические модели, методики расчета и оптимизации модифицированных двойных и тройных каскадных схем для обогащения регенерата урана с одновременным выполнением условий на концентрации четных изотопов и максимальным вовлечением исходного материала.
  \item Методика выбора каскадной схемы обогащения регенерированного урана в условиях многократного рецикла, в зависимости от его исходного состава и принятых ограничений на концентрации четных изотопов.
\end{enumerate}

{\reliability}.
Надежность, достоверность и обоснованность научных положений и выводов, сделанных в диссертации, следует из корректности постановки задач, физической обоснованности применяемых приближений, использования методов, ранее примененных в аналогичных исследованиях, взаимной согласованности результатов, а также из совпадения результатов численных экспериментов, полученных с помощью независимо разработанных методик других исследователей. Корректность результатов вычислительных экспериментов гарантируется тестами и операторами проверки соответствия ограничениям, верифицирующими строгое выполнение заданных условий и соблюдение условий сходимости балансов (массовых и покомпонентных).

% {\probation}
% См. приложение А2.

{\contribution} Автор принимал активное участие разработке каскадных схем, написании расчетных кодов, проведении вычислительных экспериментов, а также в обработке результатов численных экспериментов. Автор разработал расчетные коды, реализующие новые подходы к оптимизации рассматриваемых схем.

% {\publications} 
% См. приложение А1.

 % Характеристика работы по структуре во введении и в автореферате не отличается (ГОСТ Р 7.0.11, пункты 5.3.1 и 9.2.1), потому её загружаем из одного и того же внешнего файла, предварительно задав форму выделения некоторым параметрам

%Диссертационная работа была выполнена при поддержке грантов \dots

%\underline{\textbf{Объем и структура работы.}} Диссертация состоит из~введения,
%четырех глав, заключения и~приложения. Полный объем диссертации
%\textbf{ХХХ}~страниц текста с~\textbf{ХХ}~рисунками и~5~таблицами. Список
%литературы содержит \textbf{ХХX}~наименование.

\section*{Содержание работы}
Во \underline{\textbf{введении}}на основе обзора источников обоснована актуальность разработки схем для обогащения регенерированного урана. Помимо этого, во введении сформулирована цель, поставлены задачи, доказана научная новизна и практическая значимость выполненной работы. Вынесены на защиту основные положения, обоснована достоверность полученных в работе результатов и представлены сведения об их апробации.


\underline{\textbf{Первая глава}} посвящена теоретическому введению в проблему поиска схем каскадов для обогащения восстановленного урана. Изложены основные теоретические сведения, необходимые для моделирования разделения многокомпонентных изотопных смесей в каскадах.

\underline{\textbf{Вторая глава}} посвящена постановке расчетной задачи и исследованию ограничений известных схем на основе ординарного каскада, а также двойного каскада, при решении задачи возврата восстановленного регенерата в топливный цикл легководных реакторов в режиме многократного рециклирования.
Во второй главе:
\begin{enumerate}
  \item описывается расчетная методика и ее основные предположения для используемых математический моделей.
  \item выявляются физические причины невозможности решения задачи обогащения регенерата произвольного изотопного состава в одиночном каскаде при одновременном выполнении условий на  концентрации изотопов. Формулируется рекомендация переходить к составным схемам, обеспечивающим пространственное разделение для отделения изотопов легкой фракции $^{232,234,236}$U в получаемом продукте -- низкообогащенном уране.
  \item рассматриваются ограничения схемы двойного каскада.
  \item описывается задача подбора оптимальной конфигурации каскада для заданных условий. Обсуждается вопрос критериев эффективности, ориентируясь на которые автоматизируется подбор конкретной схемы.
\end{enumerate}


\underline{\textbf{Третья глава}} посвящена исследованию возможности решения задачи эквивалентного возврата регенерированного урана и анализу предлагаемых для этого модификаций каскадных схем.
В третьей главе:
\begin{enumerate}
  \item предлагается классифицировать каскады, ис­пользуемые для обогащения регенерата, как разбавляющие или очищающие. Так, например, схема двойного каскада является очищающей, в отличие от схем, осно­ванных на ординарном каскаде, работающих на принципе разбавления.
  \item рассмотрен путь вовлечения загрязненной четными изотопами фракции, возникающей в двойных каскадах, чтобы избежать потерь $^{235}$U, сконцентрированного в этом потоке.
  \item рассматривается область применения схемы двойного каскада с замыканием, ввиду ограничения, связанного с накоплением легких четных изотопов $^{232,234}$U в загрязненной фракции с ростом числа рециклов
  \item анализируется возможность вывода из системы части загрязненного потока легкой фракции второго каскада в контексте потерь $^{235}$U, стоимости обращения с нештатным отходом, а также технической реализуемостью концентрации изотопа $^{234}$U на уровне выше $^{235}$U.
  \item рассматривается вопрос целесообразности выхода за верхние пороговые значения концентрации $^{235}$U для НОУ.
  \item анализируется схема, обеспечивающая умеренный перерасход работы разделения за счет введения природного урана. Такой подход одновременно позволяет избежать потери РР при смешении изотопных смесей с большой разницей концентраций $^{235}$U.
  \item осуществляется сопоставление рассмотренных схем по критериям:
  \begin{enumerate}
    \item Расход Природного урана
    \item Затраты работы разделения
    \item доли потерь $^{235}$U в схеме
    \item доли потерь $^{235}$U из исходного регенерата
    \item Доли ГЦ в схеме, в которых превышено ПДК по $^{232}$U или превышен порог ASTM равный $1\cdot10^{-7}$\%.
  \end{enumerate}  
\end{enumerate}

\pdfbookmark{Заключение}{conclusion}                                  % Закладка pdf
В \underline{\textbf{заключении}} приведены основные.

По итогу исследования выдвигаются рекомендации по использованию результатов работы для обогащения регенерированного урана в условиях однократного и многократного рецикла в различных видах топлива.

%% Согласно ГОСТ Р 7.0.11-2011:
%% 5.3.3 В заключении диссертации излагают итоги выполненного исследования, рекомендации, перспективы дальнейшей разработки темы.
%% 9.2.3 В заключении автореферата диссертации излагают итоги данного исследования, рекомендации и перспективы дальнейшей разработки темы.

В результате проведения диссертационной работы разработаны варианты каскадных схем, позволяющие решить задачу обогащения регенерированного урана в условиях его многократного рецикла при условиях, характерных топливному циклу современных легководных реакторов российского дизайна и международным спецификациям:

\begin{enumerate}
  \item Схема двойного каскада с НОУ-разбавителем;
  \item Схема двойного каскада с НОУ-разбавителем с возвратом потока $P_2$ в цикл;
  \item Схема тройного каскада с НОУ-разбавителем и дополнительным разбавителем потока $P_2$, возвращаемого в цикл
  \item Схема независимой утилизации побочного продукта легкой фракции второго каскада схемы двойного каскада с НОУ-разбавителем.
\end{enumerate}

Для каждой из предложенных схем разработаны оригинальные методики расчета и оптимизации ее параметров по критерию минимума суммарного потока каскадной схемы, основанная на использовании современных методов условной оптимизации функций многих переменных. С использованием разработанных методик расчета и оптимизации предложенных каскадных схем продемонстрирована возможность их использования для обогащения регенерированного урана в условиях многократного рецикла на примере взятого из литературы изотопного состава регенерата урана с повышенным содержанием четных изотопов и отвечающего пятому рециклу в топливе ВВЭР.

-	характерным недостатком схемы 1 является наличие отхода с высоким содержанием четных изотопов (на 1-2 порядка выше, чем пределы для товарного НОУ) и 235U (до 20% или до 90%, в зависимости от выбранного режима работы каскадной схемы). Одним из вариантом обращения с подобным отходом может стать его перемешивание с отвалом первого каскада при обогащении регенерата. Оценки показали, что в этом случае возможно получить обедненный уран с приемлемым содержанием 232U (не выше 5·10-7%).
-	характерными недостатками схемы 2 являются: 1) монотонный рост концентраций четных изотопов от рецикла к рециклу как в исходном регенерате, так и в получаемом товарном НОУ, вследствие возврата значительной части четных изотопов на вход каскадной схемы; 2) рост потерь 235U от первоначальной массы, поступившей на вход каскадной схемы, в процессе рецикла.
-	характерным недостатком схемы 3 являются дополнительные затраты работы разделения по отношению к схемам 1 и 2, возникающие при обогащении разбавленного обедненным ураном отхода второго каскада схемы, загрязненного четными изотопами.
-	 Анализ эффективности предложенных каскадных схем с точки зрения потерь 235U показал, что перспективными вариантами для дальнейшей технико-экономической проработки являются каскадные схемы 1 и 3.
-	 Схема 1 обеспечивает экономию чистого 235U в диапазоне ~700-850 кг в зависимости от номера рецикла, по сравнению с открытым топливным циклом на обедненном уране с концентрацией 235U – 0,13%. В результате при реализации семи последовательных рециклов в такой схеме можно сэкономить массу 235U, эквивалентную массе данного изотопа, содержащейся приблизительно в 2,94 комплектов ТВС, изготовленных из природного урана для загрузки в эквивалентный реактор. При этом схема 1 на каждом рецикле позволяет извлечь более 80% от массы 235U из исходного регенерированного урана, поступившего на обогащение. Отметим, что основная доля (более 90% или ~4 т) потерь изотопа 235U в схеме 1 на каждом рецикле обусловлена потерями данного изотопа в отвале каскада 3, который нарабатывает НОУ-разбавитель из обедненного урана. Однако данные потери можно уменьшить при изменение отвальной концентрации 235U в каскаде 3 на оптимальное с точки зрения экономических затрат значение. Кроме того, эти потери не связаны с решением основной задачи – минимизацией потерь изотопа 235U из регенерированного уранового сырья. 
-	Несмотря на видимое отсутствие потерь 235U в загрязненной фракции схемы 2, в данной схеме, ввиду искусственного повышения содержания четных в получаемом продукте с каждой последующей перегрузкой возрастает масса отхода (P2) и, соответственно, масса направляемого в него изотопа 235U. Тем самым эффект от возврата изотопа 235U в цикл нивелируется его потерями вследствие увеличения потока загрязненной фракции, которое происходит из-за роста концентраций четных изотопов в исходной смеси. В результате схема 2 не обеспечивает преимуществ в экономии 235U в процессе рециклирования, по отношению к схемам 1 и 3.
-	Схема 3 обеспечивает экономию чистого 235U в диапазоне ~550-700 кг в зависимости от номера рецикла, по сравнению с открытым топливным циклом на обедненном уране с концентрацией 235U – 0,13%. Таким образом, по этому показателю схема 3 оказывается немного менее эффективной, чем схема 1, но более эффективной, чем схема 2.
Оценки показали, что при реализации семи последовательных рециклов в такой схеме можно сэкономить массу 235U, эквивалентную массе данного изотопа, содержащейся приблизительно в 2,7 комплектах ТВС, изготовленных из природного урана для загрузки в эквивалентный реактор. Отметим, что основная доля потерь изотопа 235U в схеме 3 на каждом рецикле обусловлена потерями данного изотопа в отвалах каскадов 3 и 4. Частично данные потери можно уменьшить при изменении отвальной концентрации 235U в каскаде 3 на оптимальное с точки зрения экономических затрат значение. 
Таким образом, с точки зрения эффективности вовлечения 235U в топливный цикл схема 3 несколько уступает схеме 1. Достоинством схемы 3 можно считать отсутствие нештатных отходов в виде загрязненной четными изотопами фракции.
3.	Для определения эффективности предложенных каскадных схем 1 и 3 проведено сравнение их интегральных характеристик с ординарным каскадом в условиях открытого топливного цикла на обедненном уране и природном уране, а также с доступными данными для метода обогащения регенерированного урана на базе АО «ПО «ЭХЗ» в условиях топливного цикла с использованием природного урана. Результаты проведенного сравнения показали, что предлагаемые решения (схемы 1 и 3) обеспечивают преимущества в интегральных характеристиках (затраты работы разделения и расход сырья), как по сравнению ординарным каскадом, так и по сравнению со схемой АО «ПО «ЭХЗ». Например, схема 1 позволяет при затратах работы разделения, практически эквивалентных случаю открытого ЯТЦ на природном уране, обеспечить экономию природного урана на уровне 16%, что соответствует схеме АО «ПО «ЭХЗ», у которой, однако, затраты работы разделения по отношению к открытому ЯТЦ больше на 10%. 
Схема 3 при приблизительно равных со схемой АО «ПО «ЭХЗ» показателях по работе разделения (на 10% выше, чем для открытого ЯТЦ) обеспечивает более высокую экономию природного урана, составляющую (19% против 16% у схемы АО «ПО «ЭХЗ»).
В случае реализации ЯТЦ только с использованием обедненного урана с обогащением не выше 0,13% во всех вариантах происходит существенное (в несколько раз) увеличение затрат работы разделения. Однако схемы 1 и 3 позволяют решить задачу, имея преимущества в величине работы разделения 12 и 10%, соответственно. Важно подчеркнуть, что в случае использования только обедненного урана указанные величины преимущества по работе разделения соответствуют примерно 1000 млн. ЕРР при изготовлении 252 т товарного НОУ (по металлу).
Таким образом, предлагаемые схемы оказываются более эффективными с точки зрения комбинации таких интегральных характеристик, как затраты работы разделения и расход природного урана (в случае реализации топливного цикла с природным ураном).
Полученные оценки интегральных характеристик каскадных схем для топливного цикла с использованием только обедненного урана свидетельствуют о целесообразности оценки возможности реализации ЯТЦ при таких условиях с учетом масштабов доступных производственных мощностей по обогащению урана.
4.	Для выбора конкретного варианта каскадной схемы с целью дальнейшей практической реализации необходим детальный технико-экономический анализ каждой из схем на основе их интегральных показателей (расходные характеристики, затраты работы разделения и пр.) в контексте всей цепочки стадий ЯТЦ и с учетом возникающих в этой цепочке изменений при использовании регенерата урана по отношению к открытому топливному циклу. 
Помимо этого, необходима проработка технологических проблем каждой из схем, в частности, с точки зрения возможности эксплуатации и обслуживания оборудования в условиях работы с материалами, имеющими более высокую, чем природный уран удельную активность. Например, подобные условия возникают в «очистительных» каскадах, выделяющих в легкую фракции α-активные изотопы 232U и 234U. 



Использование уранового регенерата для производства топлива легководных энергетических реакторов позволит: 
\begin{enumerate}
  \item сократить объем захоронения радиоактивных отходов; 
  \item обеспечить экономию природного урана;
  \item сэкономить затраты работы разделения (по сравнению со случаем обогащения природного урана) при дообогащении данного материала в разделительном каскаде. 
\end{enumerate}

Таким образом, вовлечение урановой составляющей отработанного топлива в ядерный топливный цикл реакторов, составляющих основную долю парка энергоблоков, позволит увеличить рентабельность электрогенерации на АЭС.

% В частности, Росатом, внедряя описанные в работе технологии, уже сегодня формирует более конкурентоспособные коммерческие предложения в части топливных поставок, а также организует эффективный переходный период на двухкомпонентную структуру ядерной энергетики. Такой подход позволяет ресурсоэффективнее воплощать глобальную стратегию замыкания ЯТЦ, осуществляя рецикл топлива в том числе с помощью имеющихся реакторных мощностей парка ВВЭР.


% \begin{enumerate}
%   \item На основе анализа \ldots
%   \item Численные исследования показали, что \ldots
%   \item Математическое моделирование показало \ldots
%   \item Для выполнения поставленных задач был создан \ldots
% \end{enumerate}


\insertbibliofull   



\pdfbookmark{Литература}{bibliography}