\section*{Общая характеристика работы}

\newcommand{\actuality}{\underline{\textbf{\actualityTXT}}}
\newcommand{\progress}{\underline{\textbf{\progressTXT}}}
\newcommand{\aim}{\underline{{\textbf\aimTXT}}}
\newcommand{\tasks}{\underline{\textbf{\tasksTXT}}}
\newcommand{\novelty}{\underline{\textbf{\noveltyTXT}}}
\newcommand{\influence}{\underline{\textbf{\influenceTXT}}}
\newcommand{\methods}{\underline{\textbf{\methodsTXT}}}
\newcommand{\defpositions}{\underline{\textbf{\defpositionsTXT}}}
\newcommand{\reliability}{\underline{\textbf{\reliabilityTXT}}}
\newcommand{\probation}{\underline{\textbf{\probationTXT}}}
\newcommand{\contribution}{\underline{\textbf{\contributionTXT}}}
\newcommand{\publications}{\underline{\textbf{\publicationsTXT}}}

{\actuality}
Построение ядерной энергетики нового типа, устойчивой к ресурсным ограничениям и предусматривающей решение проблемы обращения с радиоактивными отходами, связано с реакторами на быстрых нейтронах, нацелеными на воспроизводство делящегося материала -- энергетического  плутония -- в реакторе на быстрых нейтронах. Однако, по оценкам \cite{andrianovaPerspektivnyeToplivnyeZagruzki2015}, в ближайшие десятилетия, по мере становления двухкомпонентной ядерно-энергетической системы, неизбежен переходный период, когда делящиеся материалы будут повторно использоваться в топливном цикле реакторов на тепловых нейтронах, так как они составляют основную часть парка энергоблоков.

% При этом, регенерированный уран в виде соединений $UO_3$ или $UNH$ имеет нулевую цену на выходе предприятия по переработку отработанного уранового топлива (Завода РТ) \cite{gresleyEnrichingRecyclingUranium1988}.

На сегодняшний день мире в состоянии эксплуатации и сооружения насчитывается порядка 500 ядерных энергоблоков, подавляющее большинство из которых относятся к типу легководных реакторов \cite{PRISHome}. Такие реакторы потенциально могут использовать топливо, изготовленное на основе регенерированных материалов (урана и плутония). В связи с чем большой практический интерес представляет реализация многократного использования делящихся материалов из ОЯТ в топливном цикле имеющихся и сооружаемых энергоблоков. В этом случае на протяжении эксплуатационного времени жизни энергетического реактора осуществляется замыкание топливного цикла с использованием накопленного в ОЯТ урана и плутония.

Основным материалом отработавшего ядерного топлива является уран, составляющий $\approx$90-95\% за вычетом конструкционных материалов. Так как регенерированный уран содержит $^{235}$U на уровне $\geq$0,85\%, то есть долю делящегося изотопа выше, чем в природном уране, дообогащать его на изотопно-разделительном производстве представляется экономически целесообразно \cite{NikipelovNikipelovSudby}.

Использование выделенного из отработавшего ядерного топлива (ОЯТ) регенерированного урана является основным достижимым в ближайшей перспективе направлением вовлечения регенерируемых материалов в топливный цикл энергетических реакторов. Выделенный из ОЯТ регенерированный уран может быть использован в составе топлива ВВЭР различными способами:
\begin{itemize}
  \item центрифужное дообогащение для производства уранового топлива;
  \item включение (в некоторых случаях с предварительным дообогащением) в состав смешанного уран-плутониевого топлива типа MOX или REMIX.
\end{itemize}
 
Рецикл урана является сложной задачей ввиду присутствия в изотопном составе регенерата ряда четных изотопов. В первую очередь, это неприродные $^{232}$U и $^{236}$U. Присутствие первого, ввиду того, что он является опосредованным источником жесткого гамма-излучения, затрудняет обращение с регенератом, как на стадии обогащения, так и на стадии производства твэлов \cite{matveevUran232EgoVliyanie1985}. Влияние же второго сказывается в ухудшении размножающих свойств ядерного топлива, поскольку данный изотоп является паразитным поглотителем тепловых нейтронов. Вдобавок, в регенерате, по сравнению с природным ураном, на порядок выше содержание $^{234}$U, что вносит в смесь регенерата дополнительную нежелательную радиоактивность. При этом, ориентируясь на сегодняшние тенденции к увеличению длительности топливных циклов ВВЭР, которые связаны с повышением глубины выгорания топлива, следует принять во внимание вытекающий из этого рост содержания вредных четных изотопов в регенерате \cite{smirnovEvolutionIsotopicComposition2012}.

Необходимость решения проблемы эффективного многократного вовлечения облученного урана в ядерный топливный цикл (ЯТЦ), тесно связана с поиском и дальнейшей разработкой каскадных схем, которые позволяют эффективно производить из регенерата НОУ, удовлетворяющий стандартным спецификациям.
На сегодняшний день предложен ряд каскадных схем, которые могут быть применены для решения этой задачи, однако их возможности могут быть недостаточны при актуальных параметрах топливного цикла легководных реакторов. Поэтому возникает потребность поиска новых схем, которые могут быть применены для возврата урана в ядерный топливный цикл более эффективно. 

% Таким образом, учитывая принятое в ГК <<Росатом>> стратегическое решение о переходе к замкнутому ЯТЦ, решение перечисленных задач представляется актуальным для современной разделительной науки. 

{\aim} диссертационной работы является изучение физических закономерностей
молекулярно-селективного массопереноса в ординарных и многопоточных каскадах
для разделения многокомпонентных смесей с целью дальнейшего поиска
оптимальных условий обогащения регенерированного урана в подобных каскадах при
его многократном использовании в различных видах регенерированного ядерного
топлива для реакторов на тепловых нейтронах. 

Для~достижения поставленной цели решены следующие {\tasks}:
\begin{enumerate}
  \item Анализ физических закономерностей массопереноса компонентов смеси
  регенерированного урана в ординарном каскаде.
  Выявление физических ограничений нахождения решения задачи обогащения регенерата произвольного изотопного
  состава в одиночном каскаде при одновременном выполнении условий на
  концентрации изотопов $^{232}$U, $^{234}$U и $^{236}$U в получаемом продукте – низкообогащенном уране, а также априорная оценка возможности решения этой задачи.
  \item Физическое обоснование принципов построения двойных каскадов,
  позволяющих корректировать изотопный состав регенерата по концентрациям
  изотопов $^{232}$U, $^{234}$U и $^{236}$U с одновременным расходованием максимального количества
  подлежащего обогащению регенерата при различных исходных концентрациях
  четных изотопов в нем.
  \item Обоснованы физические принципы эффективной «утилизации» загрязненной четными
  изотопами фракции, возникающей в двойных каскадах, с учетом полной или
  частичной подачи данной фракции в третий каскад с предварительным
  перемешиванием ее с природным, обедненным и/или низкообогащенным ураном.
  \item Обоснованы физические принципы эффективной «утилизации» загрязненной четными
  изотопами фракции, возникающей в двойных каскадах, путем замыкания, заключающемся в ее отправке в отдельный двойной каскад, осуществляющий наработку низкообогащенного урана для последующей топливной кампании реактора.
  \item Разработаны методы оценки исследуемых каскадных схем, а также расчетные методики для оптимизации параметров каскадных схем.
  % \item Изучение физических закономерностей изменения изотопного состава регенерата и
  % интегральных характеристик модифицированных двойных каскадов и тройных
  % каскадов при обогащении регенерированного урана с различным исходным
  % содержанием четных изотопов.
  % \item Обобщение и систематизация подходов к выбору каскадной схемы, позволяющих
  % эффективное обогащение регенерированного урана в условиях однократного и
  % многократного рецикла.
  % \item Определение физических закономерностей изменения изотопного состава
  % регенерированного урана и параметров модифицированного двойного каскада для
  % его дообогащения при многократном рецикле урана (отдельно и совместно с
  % плутонием) в топливе реакторов типа ВВЭР.
\end{enumerate}


{\novelty}
\begin{enumerate}
  \item Впервые предложены модификации двойных каскадов, позволяющих корректировать
  изотопный состав регенерата по концентрациям изотопов $^{232}$U, $^{234}$U и $^{236}$U с одновременным расходованием полного количества подлежащего обогащению регенерата при различных исходных концентрациях четных изотопов в нем и других внешних условиях.
  \item Обоснованы физические принципы построения тройных каскадных схем для максимального вовлечения исходного регенерированного урана для воспроизводства топлива реакторов на тепловых нейтронах.
  \item Выполнены оригинальные исследования по изучению физических закономерностей изменения изотопного состава регенерата и интегральных характеристик модифицированных двойных и тройных каскадах при обогащении регенерированного урана с различным исходным содержанием четных изотопов.
  \item Разработаны методы расчетов каскадных схем, позволяющих решить задачу возврата регенерированного урана в топливный цикл в условиях многократного рециклирования.
  \item Разработан обобщенный подход к выбору каскадной схемы для эффективного обогащения регенерированного урана в условиях однократного и многократного рецикла.
  \item Разработка методик оптимизации систем каскадов (двойного и тройного каскадов) для обогащения регенерата урана по различным критериям эффективности, таким как:
  \begin{enumerate}
    \item расход природного урана в цикле;
    \item затраты работы разделения в цикле;
    \item доля потерь $^{235}$U в каскадной схеме;
    \item доля потерь $^{235}$U из исходного регенерата;
    \item доля газовых центрифуг в схеме, в которых превышена предельно допустимая концентрация по $^{232}$U.
  \end{enumerate}
  \item Разработка подхода к утилизации высокоактивного «нештатного» отхода, образующегося в процессе обогащения регенерированного урана в двойном каскаде.
  \item Определение физических закономерностей изменения изотопного состава регенерированного урана и параметров каскадных схем (в модифицированном двойном и тройном каскаде) для его дообогащения при многократном рецикле урана (отдельно и совместно с плутонием) в топливе реакторов типа ВВЭР.
\end{enumerate}

{\influence} 
\begin{enumerate}
  \item Проведенный анализ физических закономерностей массопереноса компонентов смеси регенерированного урана в ординарном каскаде позволяет однозначно определить условия при которых возможно/невозможно получение необходимого количества конечного продукта на основе регенерированного урана различного исходного состава путем обогащения в одиночном каскаде.
  \item Разработанные модификации двойных и тройных каскадов позволяют эффективно решать задачу обогащения регенерированного урана с одновременным выполнением ограничений на концентрации четных изотопов и максимальным вовлечением исходного регенерата.
  \item Проведенный анализ результатов расчетного моделирования молекулярно-селективного массопереноса в модифицированных двойных и тройных каскадах для обогащения регенерата урана выявляет область практической применимости подобных схем для получения НОУ-продукта на основе регенерированного урана.
  \item Предложенные способы оптимизации построения каскадных схем двойного и тройного каскадов позволяют находить наиболее эффективные конфигурации каскадов для возврата регенерированного урана в цикл.
  \item Разработаны рекомендации по использованию результатов работы для обогащения регенерированного урана в условиях однократного и многократного рецикла в различных видах топлива. Представленные в работе результаты могут быть использованы в расчетных группах на предприятиях и организациях, связанных как с проектированием и построением разделительных каскадов, так и непосредственным производством изотопной продукции (АО «Уральский электрохимический комбинат», АО «Сибирский химический комбинат», АО «ТВЭЛ», АО «Восточно-Европейский головной научно-исследовательский и проектный институт энергетических технологий», АО «ПО «ЭХЗ» и др.). Предложенные методики расчета могут лечь в основу технико-экономического анализа обращения с ОЯТ в части получения из восстановленного урана низкообогащенного урана, отвечающего требуемым качествам.  
  % \item Разработан тренировочный программный комплекс для расчета каскада, нацеленного на возврат регенерированного урана. Код оформлен в виде лабораторной работы, которая внедрена в учебный процесс.
\end{enumerate}


{\methods}.
Исследование проводит систематизацию научно-технической литературы, посвященной заявленной теме.
Применены подходы, известные в современной теоретической физике, и в частности, в теории разделения изотопов в каскадах.
В ходе работы обоснованы теоретические принципы построения анализируемых каскадов, и проведено математическое моделирование каскадных схем.
Для проведения расчетов использованы схемы модельных каскадов (квазиидеальный каскад и его разновидность R-каскад, для которого выполняется условие несмешивания относительных концентраций пары выбранных компонентов). Моделирование процессов разделения смесей изотопов урана проводили с использованием разработанных в ходе выполнения работы специализированных компьютерных программ. Применены современные программные средства языков программирования Julia и Python и подключаемых библиотек, таких как NLopt, Optim, ScyPy, предназначенных для решения систем нелинейных уравнений и нелинейной оптимизации, Plots.jl для визуализации результатов.

{\defpositions}
\begin{enumerate}
  \item Результаты анализа физических закономерностей массопереноса компонентов смеси регенерированного урана в ординарном каскаде, позволяющие однозначно определить условия при которых возможно/невозможно получение необходимого количества конечного продукта на основе регенерированного урана различного исходного состава путем обогащения в одиночном каскаде.
  \item Физико-математические модели, методики расчета и оптимизации модифицированных двойных и тройных каскадных схем для обогащения регенерата урана с одновременным выполнением условий на концентрации четных изотопов и максимальным вовлечением исходного материала.
  \item Методика выбора каскадной схемы обогащения регенерированного урана в условиях многократного рецикла, в зависимости от его исходного состава и принятых ограничений на концентрации четных изотопов.
\end{enumerate}

{\reliability}.
Надежность, достоверность и обоснованность научных положений и выводов, сделанных в диссертации, следует из корректности постановки задач, физической обоснованности применяемых приближений, использования методов, ранее примененных в аналогичных исследованиях, взаимной согласованности результатов, а также из совпадения результатов численных экспериментов, полученных с помощью независимо разработанных методик других исследователей. Корректность результатов вычислительных экспериментов гарантируется тестами и операторами проверки соответствия ограничениям, верифицирующими строгое выполнение заданных условий и соблюдение условий сходимости балансов (массовых и покомпонентных).

% {\probation}
% См. приложение А2.

{\contribution} Автор принимал активное участие разработке каскадных схем, написании расчетных кодов, проведении вычислительных экспериментов, а также в обработке результатов численных экспериментов. Автор разработал расчетные коды, реализующие новые подходы к оптимизации рассматриваемых схем.

% {\publications} 
% См. приложение А1.

 % Характеристика работы по структуре во введении и в автореферате не отличается (ГОСТ Р 7.0.11, пункты 5.3.1 и 9.2.1), потому её загружаем из одного и того же внешнего файла, предварительно задав форму выделения некоторым параметрам

%Диссертационная работа была выполнена при поддержке грантов \dots

%\underline{\textbf{Объем и структура работы.}} Диссертация состоит из~введения,
%четырех глав, заключения и~приложения. Полный объем диссертации
%\textbf{ХХХ}~страниц текста с~\textbf{ХХ}~рисунками и~5~таблицами. Список
%литературы содержит \textbf{ХХX}~наименование.

\section*{Содержание работы}
Во \underline{\textbf{введении}} обоснована актуальность разработки схем для обогащения регенерированного урана. Cформулирована цель исследования, доказана научная новизна, а также практическая значимость выполненной работы. Вынесены на защиту основные положения, обоснована достоверность полученных в работе результатов и представлены сведения об их апробации. Изложена постановка задачи замыкания ядерного топливного цикла по урановой компоненте топлива легководных реакторов в условиях многократного рецикла.

\underline{\textbf{Первая глава}} посвящена теоретическому введению в проблему поиска схем каскадов для обогащения восстановленного урана. Изложены основные теоретические сведения, необходимые для моделирования разделения многокомпонентных изотопных смесей в каскадах. Представлен обзор каскадных схем, посвященных возврату регенерированного урана в ядерный топливный цикл, известных к началу написания диссертационной работы.

На основе обзора, приведенного в первой главе, во \underline{\textbf{второй главе}}  выявляются ограничения известных схем. Исследование, проводимое во второй главе показывает  нецелесообразность использования схем на основе ординарных каскадов. Также исследуются границы применимости двойного немодифицированного каскада для решения поставленной задачи. Выявлены физические принципы, на основании которых можно априорно судить о неприменимости некоторых каскадных схем для задачи возврата восстановленного регенерата в топливный цикл легководных реакторов в режиме многократного рециклирования. В этой главе формулируется рекомендация перехода к составным схемам на основе двойного каскада, обеспечивающим <<пространственное>> разделение для отделения изотопов легкой фракции $^{232,234,236}$U в получаемом продукте -- низкообогащенном уране.
Во второй главе:
\begin{enumerate}
  \item описывается расчетная методика и ее основные предположения для используемых математический моделей.
  \item выявляются физические причины затруднений при решении задачи обогащения регенерата произвольного изотопного состава в одиночном каскаде при одновременном выполнении условий на концентрации нецелевых изотопов.
  \item рассматриваются ограничения двухкаскадной схемы и устанавливается необходимость ее модификации.
\end{enumerate}

\underline{\textbf{Третья глава}} вытекает из анализа ограничений известных схем, проведенного во второй главе и из намеченного в ней пути решения задачи возврата в ЯТЦ регенерата в условиях многократного рецикла. В главе демонстрируется способ решения поставленной во введении задачи с помощью двойного модифицированного каскада, который испытывается для различных исходных смесей питающего регенерата, характерных для легководного реактора. Для этой схемы исследуются закономерности массопереноса изотопов урана при разных условиях и разрабатываются подходы к ее оптимизации на различные критерии. Далее прорабатываются способы устранения основной проблемы расмотренной схемы двойного модифицированного каскада -- загрязненной легкими изотопами $^{232,234}$U побочно производимой фракции. Рассматриваются известные способы утилизации изотопной композиции схожего состава, а также, исходя из присутствия в побочной фракции большого количества $^{235}$U, предлагаются способы его вовлечения с помощью схем с замыканием или дополнительных одиночных каскадов (построение тройного каскада). Для предлагаемых каскадных схем описывается методика расчета и оптимизации. В данной главе исследуются как возможности решения задачи полного возврата массы регенерированного урана, так и варианты замыкания ядерного топливного цикла по урановой компоненте вне рамок этого условия. Тогда как первый вариант нацелен на рециклирование топлива в системе отдельно взятого энергоблока, второй предлагает рассмотрение воспроизводимого топливообеспечения для парка реакторов как единой системы.
Итак, в третьей главе:
\begin{enumerate}
  \item рассмотрены способы обращения с загрязненной четными изотопами фракции, возникающей в двойных каскадах. В первую очередь исследуется путь вовлечения этой фракции в производство свежего низкообогащенного урана, что позволяет избежать потерь $^{235}$U, сконцентрированного в этом побочном потоке. Как альтернатива анализируется возможность вывода из системы части загрязненного потока легкой фракции второго каскада в контексте потерь $^{235}$U, стоимости обращения с нештатным отходом, а также технической реализуемостью достижения в этом потоке концентрации изотопа $^{234}$U на уровне по порядку величины сопоставимом с $^{235}$U.
  \item рассматривается область применения схемы двойного каскада с замыканием, ввиду ограничения, связанного с накоплением легких четных изотопов $^{232,234}$U в загрязненной фракции с ростом числа рециклов.
  \item рассматривается вопрос целесообразности выхода за верхние пороговые значения концентрации $^{235}$U для НОУ.
  \item анализируются потенциальные преимущества решения задачи рециклирования урана вне условия эквивалентного возврата.
  \item осуществляется сопоставление рассмотренных схем по критериям:
  \begin{enumerate}
    \item расход природного урана
    \item затраты работы разделения
    \item доля потерь $^{235}$U в схеме
    \item доля потерь $^{235}$U из исходного регенерата
    \item доля ГЦ в схеме, в которых превышена предельно допустимая концентрация по $^{232}$U  
  \end{enumerate}  
\end{enumerate}
Таким образом, глава посвящена анализу предложенных в диссертации каскадных схем для решения задачи замыкания ядерного топливного цикла легководных реакторов по урану. 

\pdfbookmark{Заключение}{conclusion}                                  % Закладка pdf
В \underline{\textbf{заключении}} приведены основные выводы, сделанные в ходе диссертационного исследования и перечислены полученные результаты.

По итогу исследования выдвигаются рекомендации по использованию результатов работы для обогащения регенерированного урана в условиях однократного и многократного рецикла в различных видах топлива.

\noindent \begin{enumerate}[leftmargin=0.4cm]
\item Предложен модифицированный двойной каскад с НОУ-разбавителем из природного урана, применимый  для обогащения регенерированного урана в условиях многократного рецикла урана в топливе легководных реакторов и позволяющий получить продукт, отвечающий всем требованиям на концентрации четных изотопов. 
\noindent \begin{enumerate}[leftmargin=0.4cm]
    \item На основе теории квазиидеального каскада разработаны методики расчета и оптимизации предложенной каскадной схемы по различным критериям эффективности (затраты работы разделения, расход природного урана, степень извлечения $^{235}$U из регенерата, степень извлечения $^{235}$U из всех питающих потоков схемы). Показано, что эффективность предложенной каскадной схемы по тому или иному критерию зависит от выбранного диапазона изменения концентрации $^{235}$U в потоке легкой фракции каскада II. Наиболее выгодные с точки зрения выбранных критериев эффективности наборы параметров каскадной схемы лежат в области, где концентрация $^{235}$U в потоке легкой фракции каскада II превышает 20\%. Это означает, что при практической реализации модифицированного двойного каскада целесообразно рассматривать возможность получения в отдельных потоках такой схемы концентраций $^{235}$U, превышающих 20\%, и, в первую очередь, в потоке $P_2$. 
    \item Анализ эффективности предложенной каскадной схемы с точки зрения потерь $^{235}$U показал, что схема обеспечивает экономию природного урана по сравнению с открытым топливным циклом на уровне 15-20\% в зависимости от исходного изотопного состава регенерата. Это превышает аналогичные показатели для простейших разбавляющих схем практически вдвое.
    \item Предложенная схема позволяет полностью решить задачу обогащения регенерата в широком диапазоне внешних условий и ограничений, что создает базис для ее практической реализации и поиска наиболее эффективных режимов ее работы.
\end{enumerate}

\item Показано, что модификации ординарного каскада для обогащения и разбавления регенерированного урана принципиально не решают задачу обогащения регенерированного урана при одновременном выполнении условий на концентрации четных изотопов в товарном НОУ и обеспечения расходования заданной массы регенерата на получение этого НОУ для составов регенерата с исходным содержанием четных изотопов, превышающим предельные значения для товарного НОУ. 

Основная причина невозможности решения задачи состоит в том, что в рассматриваемых схемах число свободных параметров оказывается меньшим, чем число условий, которые необходимо одновременно удовлетворить. В результате такие схемы могут обеспечить решение задачи только в частных случаях, когда в обогащение поступает регенерированный уран с исходными концентрациями четных изотопов ниже предельных значений для товарного НОУ.

\item Обоснованы способы вовлечения загрязненной четными изотопами фракции, возникающей в двойных каскадах при очистке от $^{232}$U, с учетом полной или частичной подачи данной фракции: а) в отдельный двойной каскад, осуществляющий наработку низкообогащенного урана для последующей топливной кампании реактора; б) перемешивании этой фракции с потоками обедненного урана и низкообогащенного урана для получения дополнительной массы товарного НОУ; в) в третий каскад с предварительным перемешиванием ее с природным, обедненным и/или низкообогащенным ураном. Для каждого из способов проанализированы их достоинства и недостатки, и вытекающие из них области применения, а также рассчитаны получаемые преимущества относительно открытого ЯТЦ.
 
\item Результаты работы применимы для проведения дальнейшего технико-экономического анализа каждой из схем на основе их интегральных показателей, таких как расход природного урана, затраты работы разделения, потери $^{235}$U в цикле в контексте всей цепочки ядерного топливного цикла, а также с учетом возникающих в этой цепочке изменений при использовании регенерата урана по отношению к открытому топливному циклу. Полученные в диссертации результаты дополняют теорию каскадов для разделения изотопов. В частности, предложенные в работе методики оптимизации двойных и тройных каскадов могут быть адаптированы к случаю разделения многокомпонентных смесей неурановых изотопов в каскадах центрифуг.

\end{enumerate}


\insertbibliofull   



\pdfbookmark{Литература}{bibliography}