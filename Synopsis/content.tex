\section*{Общая характеристика работы}

\newcommand{\actuality}{\underline{\textbf{\actualityTXT}}}
\newcommand{\progress}{\underline{\textbf{\progressTXT}}}
\newcommand{\aim}{\underline{{\textbf\aimTXT}}}
\newcommand{\tasks}{\underline{\textbf{\tasksTXT}}}
\newcommand{\novelty}{\underline{\textbf{\noveltyTXT}}}
\newcommand{\influence}{\underline{\textbf{\influenceTXT}}}
\newcommand{\methods}{\underline{\textbf{\methodsTXT}}}
\newcommand{\defpositions}{\underline{\textbf{\defpositionsTXT}}}
\newcommand{\reliability}{\underline{\textbf{\reliabilityTXT}}}
\newcommand{\probation}{\underline{\textbf{\probationTXT}}}
\newcommand{\contribution}{\underline{\textbf{\contributionTXT}}}
\newcommand{\publications}{\underline{\textbf{\publicationsTXT}}}

{\actuality}

Подавляющее большинство из $\approx$500 эксплуатируемых и сооружаемых ядерных энергоблоков представляют собой легководные реакторы на тепловых нейтронах \cite{PRISHome}, работающие на ядерном топливе из низкообогащенного урана (НОУ). Работа каждого из энергоблоков создает необходимость в обеспечении его топливом и выборе способа обращения с выгруженным из него облученным ядерным топливом (ОЯТ).

Основным материалом для производства топлива реакторов на тепловых нейтронах является природный уран, обогащаемый в каскадах газовых центрифуг. Но основную часть его мировых запасов можно добыть только при высоких операционных затратах, которые оцениваются как неконкурентоспособные для ядерной генерации энергии по сравнению с другими источниками \cite{Uranium2022,WorldDistributionUranium2018,hartardCompetitionConflictsResource2015}. 

Еще одним вызовом для ядерной промышленности является обращение с ОЯТ, общемировая масса которого превышает 400 килотонн и прирастает ежегодно на $>$15 килотонн \cite{kaygorodcevProblemyPerspektivyRazvitiya2021,UseReprocessedUranium2020WNA}. При этом, основным материалом облученного ядерного топлива является уран, составляющий $\approx$90-95\% его массы, за вычетом конструкционных материалов, концентрация изотопа $^{235}$U в котором, как правило, выше, чем в природном уране, что делает целесообразным его повторное использование \cite{24NikipelovNikipelovSudby}. Его вовлечение в производство ядерного топлива реакторов на тепловых нейтронах может позволить существенно сократить объем захоронения радиоактивных отходов и снизить потребности в природном уране.
% Однако стоит отметить, что реакторы на тепловых нейтронах являются реакторами-<<сжигателями>>, то есть в среднем воспроизводят делящихся материалов значительно меньше, чем распадается в активной зоне реактора в процессе облучения топлива. Этот факт говорит о том, что для реакторов данного типа невозможно полное замыкание ядерного топливного цикла, поскольку для их полноценного обеспечения топливом потребуются внешние источники делящихся материалов, а также обогащение исходной урановой смеси до требуемого содержания $^{235}$U, характерного для топлива легководного реактора ($\approx$5\%). 

Вовлечение регенерата сопряжено с рядом проблем, так как при облучении ядерного топлива в активной зоне реактора образуются искусственные изотопы урана, в первую очередь, $^{232}$U и $^{236}$U. Кроме того, как правило, возрастает и концентрация природного изотопа $^{234}$U. Изотоп $^{232}$U опасен тем, что является родоначальником цепочки распадов, среди дочерних продуктов которых есть,  в частности, $^{208}$Tl, представляющий собой источник жесткого гамма-излучения, обуславливающего высокий уровень радиоактивного фона. Поэтому при производстве уранового топлива существуют нормативные ограничения на допустимое содержание $^{232}$U в низкообогащенном уране. На текущий момент в РФ допустимые концентрации (в мас. долях) $^{232}$U в НОУ не должны превышать предельно допустимых значений: $5\cdot10^{-7}$\% (или, для некоторых случаев $2\cdot10^{-7}$\%). Проблема, связанная с изотопом $^{236}$U, состоит в том, что он является паразитным поглотителем нейтронов в ядерном топливе и, следовательно, отрицательно воздействует на реактивность реактора и глубину выгорания топлива. Для компенсации отрицательного влияния $^{236}$U и получения заданных ядерно-физических характеристик реактора нужно повышать среднее начальное обогащение топлива по $^{235}$U.  При этом, концентрации изотопов $^{232}$U, $^{234}$U и $^{236}$U (четных изотопов), возрастают при обогащении регенерированного урана в ординарных (трехпоточных) каскадах газовых центрифуг, используемых для обогащения природного урана. Фактически это означает необходимость развития способов обогащения регенерированного урана с учетом требований к изотопному составу производимого обогащенного продукта, отвечающих действующим техническим условиям на товарный низкообогащенный уран.

Настоящая работа посвящена разработке эффективных способов решения второй проблемы, связанной с обогащением регенерированного урана.
Перейдем к анализу проблем, возникающих в ее контексте для технологий разделения изотопов. На сегодняшний день предложен ряд технических решений, позволяющих решить задачу обогащения регенерированного урана до концентраций $^{235}$U, требуемых в современных топливных циклах энергетических реакторов на тепловых нейтронах (в частности отечественных ВВЭР), при одновременном выполнении принятых ограничений на содержание $^{232}$U в ядерном топливе и реализации необходимого дообогащения регенерата по $^{235}$U для компенсации негативного влияния $^{236}$U. Тем не менее, далеко не все из них способны решить задачу обогащения регенерата с одновременной коррекцией его изотопного состава в условиях, когда исходное содержание четных изотопов может меняться в широком диапазоне. Последнее обстоятельство особо важно в контексте рассмотрения перспективных реакторов, имеющих относительно высокую глубину выгорания топлива и, как следствие, состав ОЯТ которых может характеризоваться повышенным содержанием четных изотопов (кратно больше предельно допустимых значений). Помимо этого, необходимо учитывать, что замыкание топливного цикла реакторов на тепловых нейтронах подразумевает многократное обращение урана в топливе, что будет обуславливать дополнительное накопление четных изотопов в регенерате от цикла к циклу, учитывая, что при таком подходе исходное топливо на каждом цикле будет содержать четные изотопы еще до загрузки в реактор.

Очевидно, что вопросы коррекции изотопного состава регенерированного урана лежат в области теории и практики разделения изотопных смесей, что делает актуальной для разделительной науки задачу поиска эффективных способов обогащения регенерата урана с одновременной коррекцией его изотопного состава в условиях развития тенденции повышения глубины выгорания топлива и многократного использования урана в нем (многократный рецикл урана). В дополнение к этому важен выбор оптимальной каскадной схемы, которая должна обеспечить максимально эффективное использование ресурса регенерированного урана при минимальных затратах работы разделения.

Разработка способов решения указанных задач возможна с использованием теории каскадов для разделения многокомпонентных изотопных смесей, описывающей массоперенос компонентов в многоступенчатых разделительных установках и позволяющей находить оптимальные условия такого процесса.


{\aim} диссертационной работы является разработка способов обогащения регенерированного урана в каскадах центрифуг при его многократном использовании в регенерированном ядерном топливе для реакторов на тепловых нейтронах.

Для~достижения поставленной цели решены следующие {\tasks}:
\begin{enumerate}[leftmargin=0.5cm]
  \item Выявлены физические ограничения решения задачи обогащения регенерата произвольного изотопного состава в одиночном каскаде и в простых модификациях двойного каскада при одновременном выполнении условий на концентрации изотопов $^{232}$U, $^{234}$U и $^{236}$U в получаемом продукте --- низкообогащенном уране, и расходовании заданной массы регенерата на единицу получаемого продукта.
  \item Предложены модификации двойных каскадов, позволяющие корректировать изотопный состав регенерата по концентрациям изотопов $^{232}$U, $^{234}$U и $^{236}$U с одновременным расходованием максимального количества подлежащего обогащению регенерата при различных исходных концентрациях четных изотопов в нем. Разработаны и апробированы методики расчета и оптимизации предложенной модификации двойного каскада. Показана возможность использования предложенной схемы при различных внешних условиях, а также различных концентрациях четных изотопов в исходном регенерированном уране.
  \item Обоснованы способы вовлечения загрязненной четными изотопами фракции, возникающей в двойных каскадах при очистке от $^{232}$U, с учетом полной или частичной подачи данной фракции: а) в отдельный двойной каскад, осуществляющий наработку низкообогащенного урана для последующей топливной кампании реактора; б) перемешивании этой фракции с потоками обедненного урана и низкообогащенного урана для получения дополнительной массы товарного НОУ; в) в третий каскад с предварительным перемешиванием ее с природным, обедненным и/или низкообогащенным ураном. Выявлены достоинства и недостатки каждого из способов, что позволяет обозначить возможные области их применения. Для предложенной в случае (в) системы каскадов разработана методика оптимизации её параметров по различным критериям эффективности. На основе разработанной методики показана возможность обеспечить экономию природного урана в цикле вплоть до 30\% по отношению к открытому топливному циклу.
  \item Изучены физические закономерности изменения изотопного состава регенерата урана в зависимости от выбора параметров модифицированного двойного каскада при обогащении регенерированного урана с различным исходным содержанием четных изотопов в питающей смеси. Это позволяет обосновать выбор оптимальных с точки зрения заданных критериев эффективности концентраций $^{235}$U в отборах первого и второго каскадов. Показано, что повышение концентрации $^{235}$U в отборе второго каскада обеспечивает более эффективную работу каскадной схемы в целом, позволяя достичь одновременной экономии природного урана и работы разделения по сравнению с открытым ядерным топливным циклом.
  % \item Обобщение и систематизация подходов к выбору каскадной схемы, позволяющих эффективное обогащение регенерированного урана в условиях однократного и многократного рецикла.
  % \item Определение физических закономерностей изменения изотопного состава регенерированного урана и параметров модифицированного двойного каскада для его дообогащения при многократном рецикле урана (отдельно и совместно с плутонием) в топливе реакторов типа ВВЭР.
\end{enumerate}

{\novelty}
\begin{enumerate}[leftmargin=0.5cm]
  \item Разработаны способы обогащения регенерированного урана на основе построения тройных и двойных каскадных схем для вовлечения фракции, загрязненной четными изотопами при обогащении регенерированного урана в двойных каскадных схемах в условиях широкого диапазона изменения внешних условий (концентрации четных изотопов в обогащаемом регенерате и товарном продукте, величины ограничений на концентрации четных изотов и др.).
  \item Предложены методики расчета  и оптимизации различных модификаций двойных каскадов, позволяющие корректировать изотопный состав регенерата по концентрациям изотопов $^{232}$U, $^{234}$U и $^{236}$U с одновременным расходованием всего подлежащего обогащению регенерата при различных исходных концентрациях четных изотопов в нем и различных внешних условиях.
  \item Изучены физические закономерности изменения интегральных характеристик модифицированных двойных и тройных каскадов и  изотопного состава регенерата при его обогащении в них для различных внешних условий и различных составов поступившего в обогащение регенерата. На основе полученных результатов выявлены пределы изменения ключевых параметров каскадных схем, обеспечивающих наиболее эффективные режимы их работы с точки зрения заданных критериев. Показана принципиальная возможность одновременного снижения снижения расхода природного урана и работы разделения по отношению к  открытому топливному циклу.
  \item Предложены способы вовлечения высокоактивного нештатного отхода, образующегося в процессе обогащения регенерированного урана в модифицированном двойном каскадe, в воспроизводство ядерного топлива. Оценена величина дополнительной экономии природного урана, возникающей в топливном цикле, за счет предложенных способов, которая может достигать 7\%.
\end{enumerate}

{\influence} 
\begin{enumerate}[leftmargin=0.5cm]
  \item Разработаны модификации двойных и тройных каскадов, позволяющие обогащать регенерированный уран с одновременным выполнением ограничений на концентрации четных изотопов и вовлечением требуемой массы регенерата.
  \item Разработаны методики оптимизации параметров предложенных в диссертации двойного и тройного каскадов, позволяющие находить наиболее эффективные с точки зрения таких критериев, как расход работы разделения, расход природного урана, степень извлечения $^{235}$U, наборы их параметров, при одновременном возврате всей массы регенерированного урана в цикл и выполнении ограничений по концентрациям четных изотопов. Предложенные методики оптимизации систем каскадов могут быть адаптированы к расчету и оптимизации параметров различных вариантов каскадных схем для разделения многокомпонентных смесей неурановых элементов.
  \item Разработанные каскадные схемы и методики их расчета и оптимизации могут быть использованы в расчетных группах на предприятиях и организациях, связанных как с проектированием и построением разделительных каскадов, так и непосредственным производством изотопной продукции (АО «Уральский электрохимический комбинат», АО «Сибирский химический комбинат», АО «ТВЭЛ», АО «Восточно-Европейский головной научно-исследовательский и проектный институт энергетических технологий», АО «ПО «ЭХЗ» и др.). Имеется акт об использовании результатов диссертационной работы в НИЦ <<Курчатовский институт>> от 05.12.2023 г.
  \item Разработанные каскадные схемы и методики их расчета могут лечь в основу имитационных моделей и цифровых двойников технологий топливного цикла реакторов на тепловых нейтронах, использующих регенерированное урановое топливо.  
\end{enumerate}

{\methods}
Исследование проводит систематизацию научно-технической литературы, посвященной заявленной теме.
Применены подходы, известные в современной теоретической физике, и в частности, в теории разделения изотопов в каскадах.
В работе теоретически обоснованы принципы построения анализируемых каскадов, разработаны программные коды расчета и оптимизации их параметров для различных постановок задач, проведено их компьютерное моделирование.
При разработке программных кодов использована теория квазиидеального каскада. При подготовке программных кодов использованы современные программные средства языков программирования Julia и Python и подключаемых библиотек, таких как NLsolve.jl, Optim.jl, SciPy, предназначенных для решения систем нелинейных уравнений и оптимизационных процедур, Matplotlib и PGFPlots.jl для визуализации результатов.

{\defpositions}
\begin{enumerate}[leftmargin=0.5cm]
  \item Способы обогащения регенерата урана с одновременным выполнением условий на концентрации четных изотопов и максимальным вовлечением исходного материала в многокаскадных схемах в широком диапазоне изменения внешних условий. Методики оптимизации предложенных каскадных схем (модифицированный двойной каскад, тройной каскад).
  \item Критерии определения возможности/невозможности получения необходимого количества конечного продукта на основе регенерированного урана различного исходного состава путем его обогащения в одиночных и двойных каскадах.
  \item Способы вовлечения загрязненной четными изотопами фракции, получаемой в двойных каскадах при обогащении регенерата, в воспроизводство ядерного топлива.
\end{enumerate}

{\reliability}.
Надежность, достоверность и обоснованность научных положений и выводов, сделанных в диссертации, следует из корректности постановки задач, физической обоснованности применяемых приближений, использования методов, ранее примененных в аналогичных исследованиях, взаимной согласованности результатов. Корректность результатов вычислительных экспериментов гарантируется тестами и операторами проверки соответствия ограничениям, верифицирующими строгое выполнение заданных условий и соблюдение условий сходимости балансов (массовых и покомпонентных).

{\probation}
Результаты, изложенные в материалах диссертации, доложены и обсуждены на конференциях:
\begin{itemize}[leftmargin=0.4cm]
  \item V Международная научная конференция молодых ученых, аспирантов и студентов «Изотопы: технологии, материалы и применение», г. Томск, Россия, 2018 г.;
  \item VI Международная научная конференция молодых ученых, аспирантов и студентов «Изотопы: технологии, материалы и применение», г. Томск, Россия, 2020 г.;
  \item 15th International Workshop on Separation Phenomena in Liquids and Gases (SPLG-2019), г. Уси, Китай, 2019 г.;
  \item 16th International Workshop on Separation Phenomena in Liquids and Gases (SPLG-2021), г. Москва, Россия, 2021 г.;
  \item XVII International conference and School for young scholars “Physical chemical processes in atomic systems”, г. Москва, Россия, 2019 г..
\end{itemize}

По теме диссертации опубликовано 9 печатных работ, в том числе 5 в изданиях, индексированных в международной системе цитирования Scopus, и 4 -- в журналах из перечня ВАК. Автор принимал участие в следующих проектах, поддержанных Российским научным фондом (РНФ), в которых были использованы некоторые из результатов диссертационной работы: 
\begin{itemize}[leftmargin=0.4cm]
  \item Разработка каскадных схем для эффективного получения изотопно-модифицированных материалов для топливных циклов перспективных ядерных реакторов и других приложений (2018---2020 гг.);
  \item Оптимизация стационарного и нестационарного массопереноса в многокаскадных схемах для получения стабильных изотопов и обогащения регенерированного урана (2020---2022 гг.).
\end{itemize}


{\contribution} Автор принимал участие разработке каскадных схем, написании программных кодов, проведении вычислительных экспериментов, а также в обработке и анализе результатов вычислительных экспериментов.


\section*{Содержание работы}
Во \underline{\textbf{введении}} обоснована актуальность разработки каскадных схем для обогащения регенерированного урана, вытекающая из задач долгосрочного устойчивого развития ядерной энергетики, а также из существующих на сегодня ограничений/недостатков ранее предложенных схем. Сформулирована цель исследования, состоящая в разработке эффективных способов обогащения регенерированного урана в каскадах центрифуг при его многократном использовании в регенерированном ядерном топливе для реакторов на тепловых нейтронах. Помимо этого, во \underline{\textbf{введении}} сформулированы научная новизна и практическая значимость выполненной работы, изложены основные положения, выносимые на защиту, обоснована достоверность полученных в работе результатов и представлены сведения об их апробации.

\underline{\textbf{Первая глава}} посвящена критическому анализу ранее предложенных каскадных схем обогащения регенерированного урана, а также краткому обзору источников по промышленному опыту обогащения регенерата урана. Проанализирована проблема четных изотопов $^{232,234,236}$U в задаче обогащения регенерированного урана с точки зрения разделительных технологий. Известно, что изотопы $^{232,234}$U ухудшают радиационные характеристики ядерного топлива (ЯТ), содержание $^{232}$U в НОУ-продукте ограничено мерами радиационной безопасности персонала на разделительном и фабрикационном производстве значениями $2\cdot10^{-7} \%$ или $5\cdot10^{-7} \%$, предельно допустимое отношение $\frac{C_{234,{P}}}{C_{235,{P}}} = 0,02$. Изотоп $^{236}$U вносит <<паразитный>> захват тепловых нейтронов в ЯТ, приводя к необходимости повышения его обогащения по $^{235}$U, что увеличивает затраты работы разделения в цикле. 

Описан процесс многократного использования (рецикла) урана в топливе реакторов на тепловых нейтронах (рисунок \ref{fig_autoref1}) и подлежащие рассмотрению в диссертационной работе его стадии, а именно: обогащение регенерированного урана в каскадах центрифуг. Одним из факторов, осложняющих многократный рецикл урана является рост концентраций четных изотопов в процессе рецикла, что требует модификации известных каскадных схем, поскольку концентрации четных изотопов могут превышать допустимые пределы в несколько раз, что для примера проиллюстрировано взятыми из открытых источников составами регенерированного урана, прошедшего несколько рециклов (таблица \ref{is_compositions_2_5autoref}). Также при реализации схемы, приведенной на рисунке \ref{fig_autoref1}, подразумевают, что при производстве свежего топлива для реактора используют весь выделенный из ОЯТ этого же реактора регенерат, что проиллюстрировано на рисунке \ref{fig_autoref2}.  Такой подход призван обеспечить: (1) минимизацию потерь  $^{235}$U в топливном цикле; (2) максимально эффективно использовать потенциал ОЯТ для воспроизводства топлива; (3) исключить нежелательное накопление регенерата в процессе его многократного рецикла.

\begin{table}[h]
  \caption{Изотопные составы регенерата различных циклов.{\label{is_compositions_2_5autoref}}}
  % \centering
  \fontsize{7pt}{7pt}\selectfont
  \begin{tabularx}{\textwidth}{|Y||Y|Y|Y|Y|Y|Y|}
  \hline
  {\tiny Состав} & {\tiny Массовое число} & 232 & 233 & 234 & 235 & 236 \\
  \hline
  1 & C, \% & $6,62\cdot10^{-7}$ & $1,19\cdot10^{-6}$ & $3,28\cdot10^{-2}$ & 1,43 & $9,93\cdot10^{-1}$ \\\hline
  2 & C, \% &  $1,03\cdot10^{-6}$ & $1,30\cdot10^{-6}$ & $3,91\cdot10^{-2}$ & 1,07 & 1,45 \\
  \hline
  \end{tabularx}
\end{table}


\begin{figure}[ht]
  \centerfloat{\includegraphics[scale=0.021]{cascades/recycling_ru}}
  \caption{Схема многократного рециклирования урана}\label{fig_autoref1}
\end{figure}

\begin{figure}[ht]
  \centerfloat{\includegraphics[scale=0.25]{cascades/ordinary/recycling1kg_ru}}
  \caption{Схема замыкания урановой топливной составляющей}\label{fig_autoref2}
\end{figure}

Приводится формулировка задачи обогащения регенерата, которой посвящена диссертационная работа: получение заданной массы товарного НОУ требуемого обогащения по $^{235}$U из заданной массы сырьевого регенерата урана (в том числе многократно рециклированного) с одновременным выполнением ограничений на концентрации четных изотопов. 

Показана невозможность использования ординарного (трехпоточного) каскада (рисунок \ref{ordinary}) для решения сформулированной выше задачи в условиях многократного рецикла урана. Такой каскад можно использовать только для обогащения составов регенерата, в которых исходные концентрации четных изотопов меньше (на порядок или более), чем их допустимые пределы в товарном НОУ, что заведомо невыполнимо при многократном рецикле урана в современных реакторах на тепловых нейтронах (таблица \ref{is_compositions_2_5autoref}).

\begin{figure}[ht]
  \centerfloat{\includegraphics[scale=0.025]{cascades/ordinary}}
  \caption{Схема ординарного трехпоточного каскада. Обозначения: $F$ --- поток питания; $P$ --- поток отбора; $W$ --- поток отвала}\label{ordinary}
\end{figure}

Анализ ранее предложенных способов обогащения регенерата позволяет условно разделить рассмотренные способы на 3 типа: (1) схемы с разбавлением четных изотопов; (2) схемы с отделением четных изотопов; (3) «гибридные» схемы (комбинируют первые два способа).
Показано, что лишь некоторые из известных способов потенциально могут решить поставленную задачу обогащения регенерата в условиях варьирования содержания четных изотопов в обогащаемом регенерате. Это делает необходимым дальнейший поиск каскадных схем для решения задачи, которые можно применять для различных исходных составов регенерированного урана, а также в случаях возможного изменения внешних условий задачи. 

\underline{\textbf{Во второй главе}} приведены основные понятия и определения теории разделения изотопов в каскадах. Введены понятия  разделительного элемента, разделительной ступени, разделительного каскада и возможных вариантов соединения ступеней в каскаде. Изложены основные сведения, необходимые для моделирования разделения многокомпонентных изотопных смесей в каскадах, описаны модели «квазиидеального» каскада, как частного случая симметричного противоточного каскада, и $R$-каскада с несмешиванием относительных концентраций двух заданных компонентов смеси. Рассмотрены основные варианты постановок задач расчета таких каскадов и алгоритмы их решения, которые будут использованы в 3-й и 4-й главах. 

\underline{\textbf{Третья глава}} посвящена анализу причин, затрудняющих или делающих невозможным использование описанных в главе 1 способов обогащения регенерата в условиях многократного рецикла. Рассмотрены различные варианты однокаскадных схем (рисунок \ref{fig:diagram1ch3}), а также двойной каскад. В качестве схем на основе ординарного каскада рассмотрены:

\begin{enumerate}[leftmargin=0.4cm]
  \item Схема с разбавлением природным ураном предварительно обогащенного регенерата (рисунок \ref{fig:diagram1ch3}.a);
  \item Схема с разбавлением предварительно обогащенного регенерата низкообогащенным ураном (рисунок \ref{fig:diagram1ch3}.b);
  \item Схема с разбавлением предварительно обогащенного природного урана регенератом (рисунок \ref{fig:diagram1ch3}.c);
  \item Схема с разбавлением регенерата природным ураном перед подачей в ординарный трехпоточный каскад (рисунок \ref{fig:diagram1ch3}.d).
\end{enumerate}

Для рассмотренных схем проведена серия вычислительных экспериментов, в которых варьировали параметры каждой из них при решении задачи обогащения регенерата, сформулированной выше. По результатам проведенных расчетов: (1) показана нецелесообразность использования схем на основе ординарных каскадов для обогащения регенерата в условиях многократного рецикла, так как они не позволяют решить задачу в большинстве случаев; (2) показана возможность обогащения регенерата с превышенными относительно допустимых пределов концентрациями четных изотопов в двойном каскаде. Однако двойной каскад позволяет только решить задачу, а именно обогатить регенерат по изотопу $^{235}$U и снизить концентрации четных изотопов, не решая, при этом, задачи полного использования регенерата. Другой проблема, связанной с использованием двойного каскада является высокое содержание $^{236}$U в получаемом НОУ-продукте (до 3\%), что приводит к необходимости повышения обогащения по $^{235}$U вплоть до 6-7\% и, соответственно, росту затрат работы разделения в топливном цикле на десятки процентов по сравнению с открытым ЯТЦ.

\begin{figure}[ht]
  \centerfloat{\includegraphics[scale=0.02]{cascades/ord_all}}
  \caption{Схемы обогащения регенерата на основе одиночного ординарного каскада. Обозначения: $E$ --- поток питающего схему регенерата, $F_n$ --- поток разбавителя (природного урана или низкообогащенного урана ($F_{leu}$)); $W$ --- поток отвального ОГФУ тяжелого конца каскада; $P$ --- товарный низкообогащенный уран}\label{fig:diagram1ch3}
\end{figure}

В третьей главе рассмотрен двойной каскад (рисунок \ref{fig:double_ru_in3}), представляющий собой последовательное соединение двух каскадов, позволяющих сконцентрировать легкие четные изотопы отдельно от изотопа $^{235}$U. Для этого сначала в каскаде I обогащают изотоп $^{235}$U с одновременным обогащением изотопов $^{232}$U, $^{234}$U, $^{236}$U, а затем полученную смесь направляют на вход каскада II (рисунок \ref{fig:double_ru_in3}), где она делится на две группы: в первой обогащены легкие изотопы ($^{232}$U, $^{234}$U и $^{235}$U), во второй обедняется $^{235}$U с более интенсивным обеднением $^{232}$U, $^{234}$U, что позволяет в потоке $W_2$ получить НОУ, отвечающий требованиям по концентрациям изотопов $^{232}$U, $^{234}$U с одновременной компенсацией $^{236}$U. Анализ проведенных расчетов обосновывает необходимость разработки каскадных схем для решения задачи возврата регенерированного урана в ЯТЦ.

\begin{figure}[ht]
  \centerfloat{\includegraphics[scale=0.05]{cascades/Double_core_pure}}
  \caption{Двойной каскад. Обозначения: $E$ --- поток питающего схему регенерата, $W_1$ --- поток отвального ОГФУ тяжелого конца каскада; $P$ ($W_2$) --- конечный НОУ продукт на основе регенерата; $P_2$ --- отход двойного каскада в виде высокообогащенного урана}\label{fig:double_ru_in3}
\end{figure}

В \underline{\textbf{четвертой главе}} описаны предлагаемые в диссертации способы решения задачи обогащения регенерата в условиях его многократного рецикла.
В качестве базового предложенного способа рассмотрен <<модифицированный двойной каскад>> (рисунок \ref{p2left_autoref}). Данный способ можно рассматривать в качестве развития способа, предложенного в патенте АО <<СХК>> №2282904 \cite{EXTvodolazskihSposobIzotopnogoVosstanovleniya}. Идея работы предлагаемого способа состоит в следующем. В каскаде I обогащают исходный регенерат изотопами $^{232,233,234,235,236}$U. В каскаде II смесь делится на две фракции, так, чтобы в потоке тяжелой фракции ($W_2$) было понижено содержание $^{232,233,234}$U по отношению к питающей второй каскад смеси --- потоку $P_1$, при этом обогащение по $^{235}$U в потоке $W_2$ составляет величину несколько выше, чем требуется для товарного НОУ. Затем происходит разбавление потока $W_2$ смесью, не содержащей искусственных изотопов урана для выполнения ограничений по $^{232}$U и $^{236}$U, которая нарабатывается в каскаде III. В результате такого смешивания и получают финальный продукт --- товарный НОУ заданной массы и отвечающий всем требованиям по концентрациям четных изотопов. 

\begin{figure}[ht]
  \centerfloat{\includegraphics[scale=0.023]{cascades/DoubleModified23}}
  \caption{Схема модифицированного двойного каскада для обогащения регенерированного урана. Обозначения: $E$ --- поток регенерированного урана; $P_1$ --- поток отбора первого каскада, выступающий питанием второго каскада; $P_2$ --- поток отбора второго каскада; $W_1$ --- поток отвала первого каскада; $W_2$ --- поток тяжелой фракции (условный «отвал») второго каскада; $P_3$ --- поток НОУ-разбавителя на основе природного урана $F_3$; $P$ --- финальный продукт (товарный низкообогащенный уран (НОУ))}\label{p2left_autoref}
\end{figure}

Для каскадной схемы, приведенной на рисунке \ref{p2left_autoref}, в диссертационной работе предложена методика расчета и оптимизации ее параметров при решении задачи обогащения регенерата со всеми ограничениями. Предложенная методика основана на современных методах оптимизации функции многих переменных и может быть обобщена на случай оптимизации по различным критериям эффективности. Следует отметить, что предложенный подход к оптимизации схемы не имеет аналогов в литературе, поскольку впервые проведена оптимизация многокаскадной схемы как единой системы, в отличие от ранее использованных подходов с отдельной оптимизацией каждого из каскадов. С использованием разработанной методики оценена эффективность модифицированного двойного каскада при решении задачи обогащения урана в различных условиях и при оптимизации по таким критериям эффективности как:

\begin{enumerate}[leftmargin=0.4cm]
  \item Минимум расхода природного урана ($(\frac{\Delta A}{P})_\text{min}$);
  \item Минимум затрат работы разделения ($(\frac{F_n}{P})_\text{min}$);
  \item Максимум степени извлечения $^{235}$U в схеме ($(Y_f)_\text{max}$);
  \item Максимум степени извлечения $^{235}$U из исходного регенерата ($(Y_{E})_\text{max}$).
\end{enumerate}  

В результате проведенных вычислительных экспериментов показано, что с использованием предложенной схемы даже для составов регенерированного урана с концентрациями четных изотопов выше предельных значений для товарного НОУ, возможно добиться экономии природного урана на уровне 15\% и выше при практически нулевом перерасходе или даже экономии затрат работы разделения. 

Проведено сравнение ранее предложенных каскадных схем и схемы, приведенной на рисунке \ref{p2left_autoref}, по ключевым характеристикам (расход природного урана, затраты работы разделения), во многом определяющим удельные затраты на товарный НОУ. Сравнение оптимальных параметров модифицированного двойного каскада с аналогичными характеристиками ранее известных способов обогащения регенерата показало преимущества предложенного способа по отношению к ним. Это выражается как в самом факте решения задачи, по отношению к способам, неспособным решить задачу, так и в лучших значениях расхода природного урана и затрат работы разделения по отношению к способам, решающим поставленную задачу. Результаты такого сравнения на примере обогащения регенерата состава 1 (таблица \ref{is_compositions_2_5autoref}) приведены в таблице \ref{allaut}. 

Для предложенного способа обогащения регенерата (рисунок \ref{p2left_autoref}) проанализирована также его «устойчивость» к изменению внешних условий таких, как:
\begin{itemize}[leftmargin=0.4cm]
  \item требуемое обогащение по изотопу $^{235}$U ($C_{235,P}$ от  4,4\% до  5,5\%);    
  \item величина предельно допустимой концентрации изотопа $^{232}$U в НОУ-продукте (варьировалась в интервале от $1\cdot10^{-7}$\% до $1\cdot10^{-6}$\%);
  \item расход регенерированного урана на единицу продукта ($E/P$ от 0,93 до 2,79).
\end{itemize}

Полученные результаты показали возможность решения задачи в широком диапазоне внешних условий, поскольку для всех рассмотренных комбинаций внешних параметров задачи были найдены решения, т.е. подобраны параметры модифицированного двойного каскада, обеспечившие решение задачи. Таким образом, схема применима как при текущих параметрах топливного цикла и требованиях к товарному НОУ, так и потенциально может быть применена при их изменении.

\begin{table}[ht]
  \centering
  \caption{Сравнение интегральных показателей (параметров П) схем (C) для состава 1.{\label{allaut}}}
  \fontsize{7pt}{7pt}\selectfont
  \begin{tabularx}{\textwidth}{|Y|Y|Y|Y|Y|Y|Y|}
      \hline \diagbox{{\tinyП}}{{\tiny С}} & $\text{1}$ & $\text{2}$ & $\text{3}$ & $\text{4}$ & $\text{5}$ & $\text{6}$\\ \hline
      $\text{$Y_{f}$}$, \% & 78,89 & 86,44 & 40,26 & 88,98 & 89,01 & 87,59\\ \hline
      $\text{$Y_{E}$}$, \% & 78,89 & 48,19 & 1,00  & ---   & 89,01 & 87,56\\ \hline
      $\text{$\delta(\frac{\Delta A}{P}), \%$}$ & -1,63 & -11,01 & $-29,12$ & -11,01 & -4,17 & -4,77\\ \hline % (11.82-11.26)/11.82
      $\text{$\delta(\frac{F_n}{P}), \%$}$ & 21,14 & 15,19 & 12,86 & $17,02$ & 6,17 & $19,29$\\ \hline
      $\text{$\frac{P_{2}}{P}$}$ & $0$ & $0$ & $0$ & $0$ & $0$ & $2,87\cdot10^{-3}$\\ \hline
      $\text{$\frac{E}{P}$}$ & $4,38$ & \cellcolor{red!25}0,76 & \cellcolor{red!25}0,60 & \cellcolor{red!25}0,76 & 4,71 & $0,93$\\ \hline
      $\text{$C_{232,P}, \%$}$ & \cellcolor{red!25} $2,90\cdot10^{-6}$ &  $5,00\cdot10^{-7}$ &  $3,97\cdot10^{-7}$ & $5,00\cdot10^{-7}$ &  $5,00\cdot10^{-7}$ & $5,00\cdot10^{-7}$\\ \hline
      $\frac{C_{234,P}}{C_{235,P}}$ & \cellcolor{red!25} $2,39\cdot10^{-2}$ & $1,11\cdot10^{-2}$ & $1,10\cdot10^{-2}$ &  $1,11\cdot10^{-2}$ & $1,95\cdot10^{-2}$ & $1,20\cdot10^{-2}$\\ \hline
      $\text{$C_{235,P}, \%$}$ & $5,95$ & $5,10$ & $5,12$ & $5,12$ & $6,01$ & $5,11$\\ \hline
      $\text{$C_{236,P}, \%$}$ & $3,40$ & {\tiny $5,11\cdot10^{-1}$} & {\tiny $5,96\cdot10^{-1}$} & {\tiny $5,99\cdot10^{-1}$} & $3,62$ & $6,79\cdot10^{-1}$\\ \hline
    \end{tabularx}
\end{table}

Разработаны способы дальнейшего использования загрязненной легкими изотопами $^{232,234}$U фракции (таблица \ref{P2_compositions_autoref}), получаемой в потоке $P_2$ двойного модифицированного каскада (рисунок \ref{p2left_autoref}), которая в рассмотренном случае составляет $\approx$0,3\% от массы НОУ-продукта. Использование данной фракции призвано предотвратить нежелательное накопление на разделительном производстве высокоактивных отходов, а также задействовать остаточное содержание $^{235}$U в этом потоке, которое может достигать 20\% и более. Предложены следующие 3 способа: 

\begin{table}[h]
  \centering
  \caption{{Изотопный состав $P_2$.{\label{P2_compositions_autoref}}}}
  \fontsize{7pt}{7pt}\selectfont
    \begin{tabularx}{\textwidth}{|Y|Y|Y|Y|Y|Y|}
    \hline Массовое число & 232 & 233 & 234 & 235 & 236 \\
    \hline C, \% & $2,48\cdot10^{-5}$ & $5,44\cdot10^{-5}$ & 0,69 & 20,00 & 7,34 \\ \hline
  \end{tabularx}
\end{table}

\begin{itemize}[leftmargin=0.4cm]
  \item Перемешивание $P_2$ с регенератом, поступающим на обогащение (рисунок \ref{P2utilizationRingautoref});
  \item Получение дополнительной массы товарного НОУ (рисунок \ref{P2utilizationautoref});
  \item Перемешивание $P_2$ с обедненным ураном и последующее обогащение (рисунок \ref{p2_withDepU}).
\end{itemize}

\begin{figure}[ht]
  \centerfloat{\includegraphics[scale=0.02]{cascades/P2utilizationRing}}
  \caption{Схема передачи загрязненного изотопом $^{232}$U состава гексафторида урана в двойном каскаде от первой партии дообогащенного регенерированного урана к последующей. Обозначения: $E$ --- поток регенерированного урана; $P_1$ --- поток отбора первого каскада, выступающий питанием второго каскада; $W_1$ --- поток отвала первого каскада; $W_2$ --- поток тяжелой фракции (условный «отвал») второго каскада; $P_3$ --- поток НОУ-разбавителя; $P$ --- финальный продукт (товарный низкообогащенный уран (НОУ)); $P_2$ --- поток отбора второго каскада, который подается на питание последующего двойного каскада, перемешиваясь с регенератом очередного рецикла}\label{P2utilizationRingautoref}
\end{figure}

\begin{figure}[ht]
  \centerfloat{\includegraphics[scale=0.0215]{cascades/P2utilization}}
  \caption{Схема независимого вовлечения в производство НОУ загрязненной изотопом $^{232}$U фракции, смешанной с обедненным и природным ураном}\label{P2utilizationautoref}
\end{figure}

Проведен сравнительный анализ предложенных вариантов вовлечения загрязненной четными изотопа фракции. Каждый из рассмотренных способов вовлечения $P_2$ в ЯТЦ демонстрирует повышение эффективности использования $^{235}$U находящегося в регенерированном уране, что позволяет получить дополнительное увеличение экономии природного урана.

Для способа вовлечения легкой фракции путем ее перемешивания с регенератом, поступающим на обогащение (рисунок \ref{P2utilizationRingautoref}), проведены вычислительные эксперименты по топливоподготовке (обогащение регенерата с целью производства низкообогащенного урана) для серии частичных перегрузок топлива в реакторе (замена части ТВС активной зоны реактора)\footnote{формулировка задачи последовательного обогащения регенерата для нескольких перегрузок реактора со всеми условиями осуществлена совместно с сотрудниками НИЦ <<Курчатовский институт>>.}. Каждая из серий расчетов отличалась выбранным критерием эффективности, в качестве которых использованы $(Y_f)_\text{max}$, $(Y_{E})_\text{max}$, $(\delta(\frac{\Delta A}{P}))_\text{min}$, $(\delta(\frac{F_n}{P}))_\text{min}$, $(\frac{P_2}{P})_\text{min}$. По результатам анализа полученных закономерностей, с каждой последующей перегрузкой происходит снижение эффективности схемы по каждому из показателей.

Для способа независимого получения дополнительной массы товарного НОУ (рисунок \ref{P2utilizationautoref}), была показана возможность получить дополнительную экономию природного урана относительно двойного модифицированного каскада.

Для оценки эффективности способа вовлечения в ЯТЦ легкой фракции путем ее перемешивания с обедненным ураном и последующим обогащением, представленного на рисунке \ref{p2_withDepU} разработан алгоритм расчета такой многокаскадной схемы. Предложенный подход обобщается на случай использования различных критериев эффективности. Проведенные с ее помощью серии вычислительных экспериментов показывают возможность обеспечить экономию природного урана и затрат работы разделения по отношению к открытому ЯТЦ даже в случае обогащения регенерата с высоким содержанием $^{232}$U (выше предельных значений для товарного НОУ).

\begin{figure}[ht]
  \centerfloat{\includegraphics[scale=0.02]{cascades/triple_cascade23}}
  \caption{Тройной каскад для обогащения регенерированного урана. Обозначения: $E$ --- поток регенерированного урана; $P_1$ --- поток отбора первого каскада, выступающий питанием второго каскада; $P_2$ --- поток отбора второго каскада; $F_{D}$ --- поток ОГФУ-разбавителя, смешиваемого с $P_2$ перед подачей на вход третьего каскада; $W_1$ --- поток отвала первого каскада; $W_2$ --- поток тяжелой фракции (условный «отвал») второго каскада; $P_3$ --- поток НОУ-разбавителя на основе природного урана $F_3$; $P$ --- финальный продукт (товарный низкообогащенный уран (НОУ)), полученный смешиванием потоков $W_2$, $P_3$ и $P_4$, где $P_4$ --- отбор третьего каскада; $W_4$ --- отвал третьего каскада}\label{p2_withDepU}
\end{figure}

По результатам исследования схем, представляющих собой различные способы использования побочной фракции $P_2$, представлена сравнительная таблица \ref{3loopautoref}.
\begin{table}
  \centering
  \caption{Сравнение интегральных показателей способов вовлечения загрязненного продукта для состава 1. Обозначения: П --- параметр, ср. --- среднее, сп. --- способ, о.е. --- относительные единицы.{\label{3loopautoref}}}
  \fontsize{7pt}{7pt}\selectfont
  \begin{tabularx}{1.01\textwidth}{|Y|Y|Y|Y|Y|Y|Y|Y|Y|Y|}
    \hline
    \multirow{2}{*}{П} & \multicolumn{3}{c|}{сп. 1} & \multicolumn{3}{c|}{сп. 2} & \multicolumn{3}{c|}{сп. 3}\\
    \cline{2-10}
    & {\tiny Загр.} 1 & {\tiny Загр.} 2 & ср. & {\tiny Загр.} 1 & {\tiny Загр.} 2 & ср. & {\tiny Загр.} 1 & {\tiny Загр.} 2 & ср. \\
    \hline
    $\frac{F_n}{P}$   & 6,40 & 6,49 & 6,45    & 6,40  & 6,40  & 6,40    & 6,23 & 6,23 & 6,23\\ \hline
    $\frac{\Delta A}{P}, \textit{{\tiny о.e.}}$ & 11,26 & 11,61 & 11,43 & 11,26 & 11,29 & 11,27   & 11,66 & 11,66 & 11,66 \\ \hline
    $\frac{P_2}{P}, \%$  & 0,49 & 2,96 & 1,73    & 0,49 & 0,91 & 0,72        & 0 & 0 & 0 \\ \hline
    $\frac{E}{P}$        & 0,93 & 0,95 & 0,94    & 0,93 & 1,08 & 1,01     & 0,93 & 0,93 & 0,93 \\ \hline
  \end{tabularx}
\end{table}

Как следует из анализа данных таблицы \ref{3loopautoref}, использование схемы независимого вовлечения $P_2$ (сп. 2, представленная на рисунок \ref{P2utilizationautoref}) для формирования Загр. 2, позволяет, по сравнению со схемой с замыканием (сп. 1, Загр. 2, рисунок \ref{P2utilizationRingautoref}), расходовать меньшее удельное количество природного урана и работы разделения (на $\approx$1,3\% и $\approx$2,7\%) на этапе производства НОУ-продукта для осуществления перегрузки топлива в реакторе, и снижать количество отхода в виде $P_2$ в более чем три раза. То есть, схема независимого вовлечения $P_2$ показывает лучшие результаты по ключевым (интегральным) показателям, по сравнению со схемой модифицированного двойного каскада.

Способ 3 (тройной каскад) позволяет задействовать все требуемое количество регенерата на каждой перегрузке, а также позволяет добиться наименьшего расхода природного урана, обеспечивая $\approx$4\% выигрыша, по сравнению со способом 1 на этапе производства урана для Загр. 2, при этом перерасходуя работу разделения лишь на $\approx$0,5\% и не производя побочного отхода с высоким содержанием $^{232,234}$U, такого как $P_2$.

% \newpage

\pdfbookmark{Заключение}{conclusion}
В \underline{\textbf{заключении}} перечислены полученные ключевые результаты диссертационного исследования и сформулированы его основные выводы:
\noindent \begin{enumerate}[leftmargin=0.4cm]
\item Предложен модифицированный двойной каскад с НОУ-разбавителем из природного урана, применимый  для обогащения регенерированного урана в условиях многократного рецикла урана в топливе легководных реакторов и позволяющий получить продукт, отвечающий всем требованиям на концентрации четных изотопов. 
\noindent \begin{enumerate}[leftmargin=0.4cm]
    \item На основе теории квазиидеального каскада разработаны методики расчета и оптимизации предложенной каскадной схемы по различным критериям эффективности (затраты работы разделения, расход природного урана, степень извлечения $^{235}$U из регенерата, степень извлечения $^{235}$U из всех питающих потоков схемы). Показано, что эффективность предложенной каскадной схемы по тому или иному критерию зависит от выбранного диапазона изменения концентрации $^{235}$U в потоке легкой фракции каскада II. Наиболее выгодные с точки зрения выбранных критериев эффективности наборы параметров каскадной схемы лежат в области, где концентрация $^{235}$U в потоке легкой фракции каскада II превышает 20\%. Это означает, что при практической реализации модифицированного двойного каскада целесообразно рассматривать возможность получения в отдельных потоках такой схемы концентраций $^{235}$U, превышающих 20\%, и, в первую очередь, в потоке $P_2$. 
    \item Анализ эффективности предложенной каскадной схемы с точки зрения потерь $^{235}$U показал, что схема обеспечивает экономию природного урана по сравнению с открытым топливным циклом на уровне 15-20\% в зависимости от исходного изотопного состава регенерата. Это превышает аналогичные показатели для простейших разбавляющих схем практически вдвое.
    \item Предложенная схема позволяет полностью решить задачу обогащения регенерата в широком диапазоне внешних условий и ограничений, что создает базис для ее практической реализации и поиска наиболее эффективных режимов ее работы.
\end{enumerate}

\item Показано, что модификации ординарного каскада для обогащения и разбавления регенерированного урана принципиально не решают задачу обогащения регенерированного урана при одновременном выполнении условий на концентрации четных изотопов в товарном НОУ и обеспечения расходования заданной массы регенерата на получение этого НОУ для составов регенерата с исходным содержанием четных изотопов, превышающим предельные значения для товарного НОУ. 

Основная причина невозможности решения задачи состоит в том, что в рассматриваемых схемах число свободных параметров оказывается меньшим, чем число условий, которые необходимо одновременно удовлетворить. В результате такие схемы могут обеспечить решение задачи только в частных случаях, когда в обогащение поступает регенерированный уран с исходными концентрациями четных изотопов ниже предельных значений для товарного НОУ.

\item Обоснованы способы вовлечения загрязненной четными изотопами фракции, возникающей в двойных каскадах при очистке от $^{232}$U, с учетом полной или частичной подачи данной фракции: а) в отдельный двойной каскад, осуществляющий наработку низкообогащенного урана для последующей топливной кампании реактора; б) перемешивании этой фракции с потоками обедненного урана и низкообогащенного урана для получения дополнительной массы товарного НОУ; в) в третий каскад с предварительным перемешиванием ее с природным, обедненным и/или низкообогащенным ураном. Для каждого из способов проанализированы их достоинства и недостатки, и вытекающие из них области применения, а также рассчитаны получаемые преимущества относительно открытого ЯТЦ.
 
\item Результаты работы применимы для проведения дальнейшего технико-экономического анализа каждой из схем на основе их интегральных показателей, таких как расход природного урана, затраты работы разделения, потери $^{235}$U в цикле в контексте всей цепочки ядерного топливного цикла, а также с учетом возникающих в этой цепочке изменений при использовании регенерата урана по отношению к открытому топливному циклу. Полученные в диссертации результаты дополняют теорию каскадов для разделения изотопов. В частности, предложенные в работе методики оптимизации двойных и тройных каскадов могут быть адаптированы к случаю разделения многокомпонентных смесей неурановых изотопов в каскадах центрифуг.

\end{enumerate}


\insertbibliofull   
\pdfbookmark{Литература}{bibliography}