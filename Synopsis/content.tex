\section*{Общая характеристика работы}

\newcommand{\actuality}{\underline{\textbf{\actualityTXT}}}
\newcommand{\progress}{\underline{\textbf{\progressTXT}}}
\newcommand{\aim}{\underline{{\textbf\aimTXT}}}
\newcommand{\tasks}{\underline{\textbf{\tasksTXT}}}
\newcommand{\novelty}{\underline{\textbf{\noveltyTXT}}}
\newcommand{\influence}{\underline{\textbf{\influenceTXT}}}
\newcommand{\methods}{\underline{\textbf{\methodsTXT}}}
\newcommand{\defpositions}{\underline{\textbf{\defpositionsTXT}}}
\newcommand{\reliability}{\underline{\textbf{\reliabilityTXT}}}
\newcommand{\probation}{\underline{\textbf{\probationTXT}}}
\newcommand{\contribution}{\underline{\textbf{\contributionTXT}}}
\newcommand{\publications}{\underline{\textbf{\publicationsTXT}}}

{\actuality}
Построение ядерной энергетики нового типа, устойчивой к ресурсным ограничениям и предусматривающей решение проблемы обращения с радиоактивными отходами, связано с реакторами на быстрых нейтронах, нацелеными на воспроизводство делящегося материала -- энергетического  плутония -- в реакторе на быстрых нейтронах. Однако, по оценкам \cite{andrianovaPerspektivnyeToplivnyeZagruzki2015}, в ближайшие десятилетия, по мере становления двухкомпонентной ядерно-энергетической системы, неизбежен переходный период, когда делящиеся материалы будут повторно использоваться в топливном цикле реакторов на тепловых нейтронах, так как они составляют основную часть парка энергоблоков.

% При этом, регенерированный уран в виде соединений $UO_3$ или $UNH$ имеет нулевую цену на выходе предприятия по переработку отработанного уранового топлива (Завода РТ) \cite{gresleyEnrichingRecyclingUranium1988}.

На сегодняшний день мире в состоянии эксплуатации и сооружения насчитывается порядка 500 ядерных энергоблоков, подавляющее большинство из которых относятся к типу легководных реакторов \cite{PRISHome}. Такие реакторы потенциально могут использовать топливо, изготовленное на основе регенерированных материалов (урана и плутония). В связи с чем большой практический интерес представляет реализация многократного использования делящихся материалов из ОЯТ в топливном цикле имеющихся и сооружаемых энергоблоков. В этом случае на протяжении эксплуатационного времени жизни энергетического реактора осуществляется замыкание топливного цикла с использованием накопленного в ОЯТ урана и плутония.

Основным материалом отработавшего ядерного топлива является уран, составляющий $\approx$90-95\% за вычетом конструкционных материалов. Так как регенерированный уран содержит $^{235}$U на уровне $\geq$0,85\%, то есть долю делящегося изотопа выше, чем в природном уране, дообогащать его на изотопно-разделительном производстве представляется экономически целесообразно \cite{NikipelovNikipelovSudby}.

Использование выделенного из отработавшего ядерного топлива (ОЯТ) регенерированного урана является основным достижимым в ближайшей перспективе направлением вовлечения регенерируемых материалов в топливный цикл энергетических реакторов. Выделенный из ОЯТ регенерированный уран может быть использован в составе топлива ВВЭР различными способами:
\begin{itemize}
  \item центрифужное дообогащение для производства уранового топлива;
  \item включение (в некоторых случаях с предварительным дообогащением) в состав смешанного уран-плутониевого топлива типа MOX или REMIX.
\end{itemize}
 
Рецикл урана является сложной задачей ввиду присутствия в изотопном составе регенерата ряда четных изотопов. В первую очередь, это неприродные $^{232}$U и $^{236}$U. Присутствие первого, ввиду того, что он является опосредованным источником жесткого гамма-излучения, затрудняет обращение с регенератом, как на стадии обогащения, так и на стадии производства твэлов \cite{matveevUran232EgoVliyanie1985}. Влияние же второго сказывается в ухудшении размножающих свойств ядерного топлива, поскольку данный изотоп является паразитным поглотителем тепловых нейтронов. Вдобавок, в регенерате, по сравнению с природным ураном, на порядок выше содержание $^{234}$U, что вносит в смесь регенерата дополнительную нежелательную радиоактивность. При этом, ориентируясь на сегодняшние тенденции к увеличению длительности топливных циклов ВВЭР, которые связаны с повышением глубины выгорания топлива, следует принять во внимание вытекающий из этого рост содержания вредных четных изотопов в регенерате \cite{smirnovEvolutionIsotopicComposition2012}.

Необходимость решения проблемы эффективного многократного вовлечения облученного урана в ядерный топливный цикл (ЯТЦ), тесно связана с поиском и дальнейшей разработкой каскадных схем, которые позволяют эффективно производить из регенерата НОУ, удовлетворяющий стандартным спецификациям.
На сегодняшний день предложен ряд каскадных схем, которые могут быть применены для решения этой задачи, однако их возможности могут быть недостаточны при актуальных параметрах топливного цикла легководных реакторов. Поэтому возникает потребность поиска новых схем, которые могут быть применены для возврата урана в ядерный топливный цикл более эффективно. 

% Таким образом, учитывая принятое в ГК <<Росатом>> стратегическое решение о переходе к замкнутому ЯТЦ, решение перечисленных задач представляется актуальным для современной разделительной науки. 

{\aim} диссертационной работы является изучение физических закономерностей
молекулярно-селективного массопереноса в ординарных и многопоточных каскадах
для разделения многокомпонентных смесей с целью дальнейшего поиска
оптимальных условий обогащения регенерированного урана в подобных каскадах при
его многократном использовании в различных видах регенерированного ядерного
топлива для реакторов на тепловых нейтронах. 

Для~достижения поставленной цели решены следующие {\tasks}:
\begin{enumerate}
  \item Анализ физических закономерностей массопереноса компонентов смеси
  регенерированного урана в ординарном каскаде.
  Выявление физических ограничений нахождения решения задачи обогащения регенерата произвольного изотопного
  состава в одиночном каскаде при одновременном выполнении условий на
  концентрации изотопов $^{232}$U, $^{234}$U и $^{236}$U в получаемом продукте – низкообогащенном уране, а также априорная оценка возможности решения этой задачи.
  \item Физическое обоснование принципов построения двойных каскадов,
  позволяющих корректировать изотопный состав регенерата по концентрациям
  изотопов $^{232}$U, $^{234}$U и $^{236}$U с одновременным расходованием максимального количества
  подлежащего обогащению регенерата при различных исходных концентрациях
  четных изотопов в нем.
  \item Обоснованы физические принципы эффективной «утилизации» загрязненной четными
  изотопами фракции, возникающей в двойных каскадах, с учетом полной или
  частичной подачи данной фракции в третий каскад с предварительным
  перемешиванием ее с природным, обедненным и/или низкообогащенным ураном.
  \item Обоснованы физические принципы эффективной «утилизации» загрязненной четными
  изотопами фракции, возникающей в двойных каскадах, путем замыкания, заключающемся в ее отправке в отдельный двойной каскад, осуществляющий наработку низкообогащенного урана для последующей топливной кампании реактора.
  \item Разработаны методы оценки исследуемых каскадных схем, а также расчетные методики для оптимизации параметров каскадных схем.
  % \item Изучение физических закономерностей изменения изотопного состава регенерата и
  % интегральных характеристик модифицированных двойных каскадов и тройных
  % каскадов при обогащении регенерированного урана с различным исходным
  % содержанием четных изотопов.
  % \item Обобщение и систематизация подходов к выбору каскадной схемы, позволяющих
  % эффективное обогащение регенерированного урана в условиях однократного и
  % многократного рецикла.
  % \item Определение физических закономерностей изменения изотопного состава
  % регенерированного урана и параметров модифицированного двойного каскада для
  % его дообогащения при многократном рецикле урана (отдельно и совместно с
  % плутонием) в топливе реакторов типа ВВЭР.
\end{enumerate}


{\novelty}
\begin{enumerate}
  \item Впервые предложены модификации двойных каскадов, позволяющих корректировать
  изотопный состав регенерата по концентрациям изотопов $^{232}$U, $^{234}$U и $^{236}$U с одновременным расходованием полного количества подлежащего обогащению регенерата при различных исходных концентрациях четных изотопов в нем и других внешних условиях.
  \item Обоснованы физические принципы построения тройных каскадных схем для максимального вовлечения исходного регенерированного урана для воспроизводства топлива реакторов на тепловых нейтронах.
  \item Выполнены оригинальные исследования по изучению физических закономерностей изменения изотопного состава регенерата и интегральных характеристик модифицированных двойных и тройных каскадах при обогащении регенерированного урана с различным исходным содержанием четных изотопов.
  \item Разработаны методы расчетов каскадных схем, позволяющих решить задачу возврата регенерированного урана в топливный цикл в условиях многократного рециклирования.
  \item Разработан обобщенный подход к выбору каскадной схемы для эффективного обогащения регенерированного урана в условиях однократного и многократного рецикла.
  \item Разработка методик оптимизации систем каскадов (двойного и тройного каскадов) для обогащения регенерата урана по различным критериям эффективности, таким как:
  \begin{enumerate}
    \item расход природного урана в цикле;
    \item затраты работы разделения в цикле;
    \item доля потерь $^{235}$U в каскадной схеме;
    \item доля потерь $^{235}$U из исходного регенерата;
    \item доля газовых центрифуг в схеме, в которых превышена предельно допустимая концентрация по $^{232}$U.
  \end{enumerate}
  \item Разработка подхода к утилизации высокоактивного «нештатного» отхода, образующегося в процессе обогащения регенерированного урана в двойном каскаде.
  \item Определение физических закономерностей изменения изотопного состава регенерированного урана и параметров каскадных схем (в модифицированном двойном и тройном каскаде) для его дообогащения при многократном рецикле урана (отдельно и совместно с плутонием) в топливе реакторов типа ВВЭР.
\end{enumerate}

{\influence} 
\begin{enumerate}
  \item Проведенный анализ физических закономерностей массопереноса компонентов смеси регенерированного урана в ординарном каскаде позволяет однозначно определить условия при которых возможно/невозможно получение необходимого количества конечного продукта на основе регенерированного урана различного исходного состава путем обогащения в одиночном каскаде.
  \item Разработанные модификации двойных и тройных каскадов позволяют эффективно решать задачу обогащения регенерированного урана с одновременным выполнением ограничений на концентрации четных изотопов и максимальным вовлечением исходного регенерата.
  \item Проведенный анализ результатов расчетного моделирования молекулярно-селективного массопереноса в модифицированных двойных и тройных каскадах для обогащения регенерата урана выявляет область практической применимости подобных схем для получения НОУ-продукта на основе регенерированного урана.
  \item Предложенные способы оптимизации построения каскадных схем двойного и тройного каскадов позволяют находить наиболее эффективные конфигурации каскадов для возврата регенерированного урана в цикл.
  \item Разработаны рекомендации по использованию результатов работы для обогащения регенерированного урана в условиях однократного и многократного рецикла в различных видах топлива. Представленные в работе результаты могут быть использованы в расчетных группах на предприятиях и организациях, связанных как с проектированием и построением разделительных каскадов, так и непосредственным производством изотопной продукции (АО «Уральский электрохимический комбинат», АО «Сибирский химический комбинат», АО «ТВЭЛ», АО «Восточно-Европейский головной научно-исследовательский и проектный институт энергетических технологий», АО «ПО «ЭХЗ» и др.). Предложенные методики расчета могут лечь в основу технико-экономического анализа обращения с ОЯТ в части получения из восстановленного урана низкообогащенного урана, отвечающего требуемым качествам.  
  % \item Разработан тренировочный программный комплекс для расчета каскада, нацеленного на возврат регенерированного урана. Код оформлен в виде лабораторной работы, которая внедрена в учебный процесс.
\end{enumerate}


{\methods}.
Исследование проводит систематизацию научно-технической литературы, посвященной заявленной теме.
Применены подходы, известные в современной теоретической физике, и в частности, в теории разделения изотопов в каскадах.
В ходе работы обоснованы теоретические принципы построения анализируемых каскадов, и проведено математическое моделирование каскадных схем.
Для проведения расчетов использованы схемы модельных каскадов (квазиидеальный каскад и его разновидность R-каскад, для которого выполняется условие несмешивания относительных концентраций пары выбранных компонентов). Моделирование процессов разделения смесей изотопов урана проводили с использованием разработанных в ходе выполнения работы специализированных компьютерных программ. Применены современные программные средства языков программирования Julia и Python и подключаемых библиотек, таких как NLopt, Optim, ScyPy, предназначенных для решения систем нелинейных уравнений и нелинейной оптимизации, Plots.jl для визуализации результатов.

{\defpositions}
\begin{enumerate}
  \item Результаты анализа физических закономерностей массопереноса компонентов смеси регенерированного урана в ординарном каскаде, позволяющие однозначно определить условия при которых возможно/невозможно получение необходимого количества конечного продукта на основе регенерированного урана различного исходного состава путем обогащения в одиночном каскаде.
  \item Физико-математические модели, методики расчета и оптимизации модифицированных двойных и тройных каскадных схем для обогащения регенерата урана с одновременным выполнением условий на концентрации четных изотопов и максимальным вовлечением исходного материала.
  \item Методика выбора каскадной схемы обогащения регенерированного урана в условиях многократного рецикла, в зависимости от его исходного состава и принятых ограничений на концентрации четных изотопов.
\end{enumerate}

{\reliability}.
Надежность, достоверность и обоснованность научных положений и выводов, сделанных в диссертации, следует из корректности постановки задач, физической обоснованности применяемых приближений, использования методов, ранее примененных в аналогичных исследованиях, взаимной согласованности результатов, а также из совпадения результатов численных экспериментов, полученных с помощью независимо разработанных методик других исследователей. Корректность результатов вычислительных экспериментов гарантируется тестами и операторами проверки соответствия ограничениям, верифицирующими строгое выполнение заданных условий и соблюдение условий сходимости балансов (массовых и покомпонентных).

% {\probation}
% См. приложение А2.

{\contribution} Автор принимал активное участие разработке каскадных схем, написании расчетных кодов, проведении вычислительных экспериментов, а также в обработке результатов численных экспериментов. Автор разработал расчетные коды, реализующие новые подходы к оптимизации рассматриваемых схем.

% {\publications} 
% См. приложение А1.



\section*{Содержание работы}
Во \underline{\textbf{введении}} обоснована актуальность разработки схем для обогащения регенерированного урана, вытекающая из задач долгосрочного развития ядерной энергетики, а также из существующих на сегодня ограничений ранее предложенных схем. Cформулирована цель исследования, состоящая в теоретическом обосновании эффективных способов обогащения регенерированного урана в каскадах цен­трифуг при его многократном использовании в регенерированном ядерном топливе для реакторов на тепловых нейтронах.

\underline{\textbf{Первая глава}} посвящена теоретическому введению в проблему поиска схем каскадов для обогащения восстановленного урана. Проанализирована проблема четных изотопов $^{232,234,236}$U в задаче обогащения регенерированного урана с точки зрения разделительных технологий. $^{232,234}$U ухудшают радиационные характеристики ЯТ, содержание $^{232}$U в НОУ-продукте ограничено мерами радиационной безопасности персонала на разделительном и фабрикационном производстве значениями 2·10-7\% или 5·10-7\%, предельно допустимое отношение $\frac{C_{234,{P}}}{C_{235,{P}}} = 0,02$. $^{236}$U захватывает тепловые нейтроны, приводя к необходимости повышения обогащения по $^{235}$U. Описывается процесс многократного использования в качестве ядерного топлива урановой смеси (рис. \ref{fig_autoref1}), и приводится схема (рис. \ref{fig_autoref2}), соответствующая принципу получения НОУ из регенерированного урана, используя при его производстве весь выделенный из ОЯТ этого же реактора регенерат. Этот принцип отвечает (1) минимизации потерь  $^{235}$U в топливном цикле (2) максимально эффективному использованию потенциала ОЯТ для воспроизвод­ства топлива (3) отсутствию нежелательного накопления регенерата.

\begin{figure}[ht]
  \centerfloat{\includegraphics[scale=0.35]{theory/recycling_ru}}
  \caption{Схема многократного рециклирования урана}\label{fig_autoref1}
\end{figure}

\begin{figure}[ht]
  \centerfloat{\includegraphics[scale=0.25]{cascades/ordinary/recycling1kg_ru}}
  \caption{Схема замыкания урановой топливной составляющей}\label{fig_autoref2}
\end{figure}

Приводится формулировка задачи обогащения регенерата, которой посвящена диссертационная работа: получение заданной массы товарного НОУ требуемого обогащения по $^{235}$U из сырьевого регенерата урана (в том числе многократно рециклированного) с одновременным выполнением ограничений на концентрации четных изотопов при условии расходования всей массы реге­нерата, выделенного из ОЯТ данного реактора.

Рассматривается промышленный опыт возврата урана в ядерный топливный цикл легководных реакторов. Кратко обозреваются три ключевые технологии, необходимые для повторного использования регенерированного урана:
(1) Радиохимическую переработку ОЯТ; (2) Изотопное обогащение регенерированного урана; (3) Изготовление топлива на основе восстановленного отработавшего топлива.

Ввиду того, что ординарный трехпоточный каскад \ref{ordinary} позво­ляет обогащать только относительно «чистые» составы регенерата, то есть такие, в которых исходные содержания четных изотопов меньше (на порядок или более), чем их допустимые пределы в товарном НОУ, такой каскад неприменим при многократном рецикле урана.

\begin{figure}[ht]
  \centerfloat{\includegraphics[scale=0.6]{cascades/ordinary/ordinary}}
  \caption{Схема ординарного трехпоточного каскада. $F$ -- поток питания; $P$ -- поток отбора; $W$ -- поток отвала.}\label{ordinary}
\end{figure}

Представлен обзор каскадных схем, посвященных возврату регенерированного урана в ядерный топливный цикл, известных к началу написания диссертационной работы. Проведен анализ применимости существующих схем для решения задачи в условиях многократного рецикла урановой составляющей, на основе которого сделан вывод о необходимости дальнейшего поиска каскадных схем.
Проведенный сравнительный анализ предложенных способов обогащения регенерата позволяет условно разделить их на 3 типа: схемы с разбавлением четных изотопов, схемы с отделением четных изотопов и «гибридные» схемы.

По результатам анализа известных схем на основе ординарного каскада сделан выводы об ограничении их применимости из чего вытекает необходимость дальнейшего поиска каскадных схем для обогащения регенерата с целью его возврата в топливный цикл.

\underline{\textbf{Во второй главе}} приведены основы теории разделения изотопов в каскадах. Вводятся ключевые понятия теории каскадов, такие как понятия разделительного элемента, разделительной ступени, а также раскрываются основные принципы коммутации разделительных элементов в каскаде. Смесь изотопов регенерированного урана, в отличие от природного урана является многокомпонентной (>2 изотопов) смесью, поэтому для нее теория разделения бинарной смеси не является удовлетворительной для моделирования процессов массопереноса. Исходя из этого, в главе изложены основные теоретические сведения, необходимые для моделирования разделения многокомпонентных изотопных смесей в каскадах, в частности симметричный противоточный каскад и описывающая для него массоперенос в общем виде система уравнений. Приводится понятие модельного каскада, позволяющего упростить вычислительную процедуру расчета параметров каскада, при этом сохранив адекватность процессу разделения для рассматриваемого типа изотопной смеси, соответствующего смеси регенерированного урана. В качестве такого модельного каскада приводится «Квазиидеальный» каскад как частный случай симметричного противоточного каскада. Для поставленных в диссертации задач обосновывается выбор модели $R$-каскада с несмешиванием относительных концентраций двух заданных компонентов смеси($n$-го и $k$-го, для которых условие несмешения записывается как \ref{GrindEQ_R_autoref}) как частного случая квазиидеального каскада.
\begin{equation} \label{GrindEQ_R_autoref} 
  R'_{nk} (s-1)=R_{nk} (s)=R''_{nk} (s+1).                                                 
\end{equation} 

В завершении, в главе осуществляется анализ типичных постановок задач расчёта параметров ординарного каскада на примере модельных каскадов. В качестве таких задач рассматриваются поверочный и проектировочный типы расчетов, и для проектировочного расчёта, подразумевающего определение параметров $R$-каскада при заданных концентрациях целевого компонента в выходящих потоках каскада, описывается алгоритм решения.

На основе обзора, приведенного в первой главе, в \underline{\textbf{третьей главе}} выявляются ограничения известных схем. 
В главе:
\begin{enumerate}
  \item формулируется постановка задачи, отвечающая условиям возврата регенерированного урана в ядерный топливный цикл в режиме многократного рециклирования;
  \item описывается расчетная методика и ее основные предположения для используемых математический моделей;
  \item выявляются физические причины затруднений при решении задачи обогащения регенерата произвольного изотопного состава в одиночном каскаде при одновременном выполнении условий на концентрации нецелевых изотопов;
  \item рассматриваются ограничения двухкаскадной схемы и устанавливается необходимость ее модификации.
\end{enumerate}

Исследование, проводимое в этой главе показывает: (1) нецелесообразность использования схем на основе ординарных каскадов; (2) границы применимости двойного немодифицированного каскада, для решения поставленной задачи.

В качестве схем на основе ординарного каскада рассмотрены:

\begin{enumerate}
  \item схема с разбавлением природным ураном предварительно обогащенного регенерата (рис. \ref{fig:diagram1ch3}.b);
  \item схема с разбавлением предварительно обогащенного регенерата низкообогащенным ураном (рис. \ref{fig:diagram1ch3}.b);
  \item схема с разбавлением предварительно обогащенного природного урана регенератом (рис. \ref{fig:diagram1ch3}.c);
  \item схема с разбавлением регенерата природным ураном перед подачей в ординарный трехпоточный каскад (рис. \ref{fig:diagram1ch3}.a).
\end{enumerate}

\begin{figure}[ht]
  \centerfloat{\includegraphics[scale=0.47]{cascades/ord_all}}
  \caption{Схемы на основе ординарного каскада. Обозначения: $E$ -- поток питающего схему регенерата, $F_n$ -- поток разбавителя (природного урана или низкообогащенного урана); $W$ -- поток отвального ОГФУ тяжелого конца каскада; $P$ -- товарный низкообогащенный уран}\label{fig:diagram1ch3}
\end{figure}

Выявлены физические принципы, на основании которых можно априорно судить о неприменимости некоторых каскадных схем для задачи возврата восстановленного регенерата в топливный цикл легководных реакторов в режиме многократного рециклирования. Описывается подход позволяющий аналитически оценить возможность применения схем на основе простейших модификаций ординарного каскада, ис­ходя из изотопного состава регенерата.

Так как полученные для схем на основе ординарных каскадов результаты показали невозможность применением <<разбавляющих>> схем удовлетво­рить одновременно и условие полного возврата массы регенерата в цикл и ограничения на содержание четных изотопов, для составов с относительно высоким исходным содержанием $^{232}$U, что связано с нарастанием в потоке, формирующем НОУ-продукт, <<легких>> изотопов (в первую очередь $^{232}$U). Отсюда, в данной главе предлагается оценить применимость двойных каскадов, которые смогут обеспечить частичную очистку обогащаемого регенерата от четных изотопов $^{232,234}$U.

Приводится постановка задачи численного эксперимента для двойного каскада и описывается его методика (алгоритм расчета и оптимизации). Полученные результаты свидетельствуют о принципиальной применимости двойного каскада для решения поставленной задачи, однако возникает ряд проблем, связанных с (1) необходимостью последующего привлечения дополнительных источников $^{235}$U для формирования требуемой для загрузки реактора массы НОУ; (2) высоким содержанием $^{236}$U в НОУ-продукте, приводящим к необходимости повышения обогащения по $^{235}$U вплоть до 6-7\%.
В данной главе исследуются как возможности решения задачи полного возврата массы регенерированного урана, так и варианты замыкания ядерного топливного цикла по урановой компоненте вне рамок этого условия. Тогда как первый вариант нацелен на рециклирование топлива для отдельно взятого энергоблока, второй предлагает рассмотрение воспроизводимого топливообеспечения для парка реакторов.

В главе 3 формулируется рекомендация перехода к составным схемам на основе двойного каскада, обеспечивающим <<пространственное>> разделение для отделения изотопов легкой фракции (в первую очередь $^{232}$U) в получаемом продукте -- низкообогащенном уране.

\underline{\textbf{Четвертая глава}} вытекает из анализа ограничений известных схем, проведенного во второй главе и из намеченного в ней пути решения задачи возврата в ЯТЦ регенерата в условиях многократного рецикла.
Глава посвящена анализу предложенных в диссертации каскадных схем для решения задачи замыкания ядерного топливного цикла легководных реакторов по урану.

В начале главы 4 демонстрируется способ решения поставленной во введении задачи с помощью двойного модифицированного каскада \ref{p2left_autoref}, который испытывается для различных исходных смесей питающего регенерата, характерных для легководного реактора.

\begin{figure}[ht]
  \centerfloat{\includegraphics[scale=0.035]{cascades/DoubleModified23}}
  \caption{Схема модифицированного двойного каскада для обогащения регенерированного урана. Обозначения: $E$ -- поток регенерированного урана; $P_1$ -- поток отбора первого каскада, выступающий питанием второго каскада; $P_2$ -- поток отбора второго каскада; $W_1$ -- поток отвала первого каскада; $W_2$ -- поток тяжелой фракции (условный «отвал») второго каскада; $P_3$ -- поток НОУ-разбавителя; $P$ -- финальный продукт (товарный низкообогащенный уран (НОУ))}\label{p2left_autoref}
\end{figure}

Для этой схемы приводится постановка задача и методики расчета каскадной схемы, с помощью которых исследуются закономерности массопереноса изотопов урана при разных условиях и разрабатываются подходы к ее оптимизации на различные критерии эффективности. С помощью разработанных методик, оценивается эффективность модифицированного двойного каскада при различных расчетных оптимизационных критериях:
\begin{enumerate}
  \item минимум расхода природного урана ($(\delta(\frac{\Delta A}{P}))_\text{min}$);
  \item минимум затрат работы разделения ($(\delta(\frac{F_{NU}}{P}))_\text{min}$);
  \item максимум степени извлечения $^{235}$U в схеме ($(Y_f)_\text{max}$);
  \item максимум степени извлечения $^{235}$U из исходного регенерата ($(Y_{E})_\text{max}$).
\end{enumerate}  

Показано, что с использованием предложенной схемы даже для составов с относительно высоким содержанием четных изото­пов (выше предельных значений для товарного НОУ) возможно добиться экономии природного урана на уровне 15\% и выше при практически нулевом перерасходе или даже экономии затрат работы разделения, а также предпочтительность затрат работы разделения в качестве критерия эффективности в рассмотренных случаях.

Имея общую постановку задачи, соответствующую условиям, необходимым для возврата регенерата в в топливный цикл в условиях многократного рецикла, проводится сравнение интегральных параметров модифицированного двойного каскада с аналогичными параметрами для других ранее рассмотренных в работе способов обогащения регенерата урана.

Для двойного модифицированного каскада \ref{p2left_autoref} исследуются закономерности массопереноса, из которых делаются выводы о возможности выбора наиболее выгодных параметров с точки зрения критерев эффективности, в частности $C_{235,{P_2}} > 20\%$. Для данной схемы также проводится анализ «устойчивости» предложенного подхода обогащения регенерата к изменению внешних условий, результаты которого показывают возможность применения рассматриваемой каскадной схемы при изменении таких внешних условий, как:
\begin{itemize}
  \item требуемое обогащение по изотопу $^{235}$U ($C_{235,P}$);    
  \item величина предельно допустимой концентрации изотопа $^{232}$U в НОУ-продукте;
  \item расход регенерированного урана на единицу продукта ($E/P$).
\end{itemize}.

Для двойного модифицированного каскада \ref{p2left_autoref} также отмечается значимая возможность изолировать разделительные мощности для обогащения регенерированного урана (где разделительное оборудование будет подвержено загрязнению минорными изотопами -- в рассматриваемом случае каскады I и II) от разделительных мощностей, обогащающих не содержащий $^{232,236}$U природный уран или ОГФУ.

Далее в главе 4 прорабатываются способы устранения основной проблемы расмотренной схемы двойного модифицированного каскада -- загрязненной легкими изотопами $^{232,234}$U побочно производимой фракции $P_2$.

Рассматриваются следующие способы утилизации $P_2$, исходя из присутствия в нем существенного количества $^{235}$U:
\begin{itemize}
  \item путем перемешивания $P_2$ с регенератом, поступающим на обогащение;
  \item путем получения дополнительной массы товарного НОУ;
  \item путем перемешивания $P_2$ с обедненным ураном и последующим обогащением.
\end{itemize}.

Каждый из рассмотренных способов вовлечения $P_2$ в ЯТЦ демонстрирует повышение эффективности использования $^{235}$U находящего­ся в регенерированном уране, что позволяет получить дополнительное увеличение экономии природного урана.

Для способа утилизации легкой фракции путем ее перемешивания с регенератом, поступающим на обогащение, проведены вычислительные эксперименты по топливоподготовке (обогащение регенерата с целью производства низкообогащенного урана) для серии частичных перегрузок топлива в реакторе (замена части ТВС активной зоны реактора). Каждая из серий расчетов отличалась выбранным критерием эффективности, в качестве которых использованы $(Y_f)_\text{max}$, $(Y_{E})_\text{max}$, $(\delta(\frac{\Delta A}{P}))_\text{min}$, $(\delta(\frac{F_{NU}}{P}))_\text{min}$, $(\frac{P_2}{P})_\text{min}$. 

Для способа утилизации легкой фракции путем ее перемешивания с обедненным ураном и последующим обогащением, представленного на рис. \ref{p2_withDepU}, приводится расчетный алгоритм, с помощью которого осуществляется оценка эффективности тройного каскада по различным критериям.

\begin{figure}[ht]
  \centerfloat{\includegraphics[scale=0.035]{cascades/triple_cascade23}}
  \caption{Тройной каскад для обогащения регенерированного урана. Обозначения: $E$ -- поток регенерированного урана; $P_1$ -- поток отбора первого каскада, выступающий питанием второго каскада; $P_2$ -- поток отбора второго каскада; $F_{D}$ -- поток ОГФУ-разбавителя, смешиваемого с $P_2$ перед подачей на вход третьего каскада; $W_1$ -- поток отвала первого каскада; $W_2$ -- поток тяжелой фракции (условный «отвал») второго каскада; $P_3$ -- поток НОУ-разбавителя; $P$ -- финальный продукт (товарный низкообогащенный уран (НОУ)), полученный смешиванием потоков $W_2$, $P_3$ и $P_4$, где $P_4$ -- отбор третьего каскада; $W_4$ -- отвал третьего каскада.}\label{p2_withDepU}
\end{figure}


Выводы, относящихся ко всем рассмотренным в главе 4 схемам:
\begin{enumerate}
    \item схемы на основе двойного каскада, использующие НОУ-разбавитель, принципиально пригодны для решения задачи обогащения регенерированного урана в рамках многократного рецикла урановой составляющей топлива легководных реакторов. При этом каждая из схем имеет собственные достоинства и недостатки;
    \item характерным недостатком схемы, не предполагающей утилизацию нештатного отхода, образующегося в потоке $P_2$, является проблема с обращением с этим материалом, с высоким содержанием как четных изотопов (на 1-2 порядка выше, чем пределы для товарного НОУ) и $^{235}$U (до 20\% или, в некоторых случаях, до 90\%, в зависимости от выбранного режима работы каскадной схемы). Одним из вариантов обращения с ним, помимо схемы независимой утилизации побочного продукта легкой фракции второго каскада схемы двойного каскада с НОУ-разбавителем (рис. \ref{P2utilization}), может стать его перемешивание с отвалом первого каскада при обогащении регенерата. Оценки показали, что в этом случае возможно получить обедненный уран с приемлемым содержанием $^{232}$U (не выше $5\cdot10^{-7}$\%);
    \item характерными недостатком схемы двойного каскада с НОУ-разбавителем с возвратом потока $P_2$ в цикл (рис. \ref{P2utilizationRing}) является возврат значительной части четных изотопов на вход каскадной схемы;
    \item характерным недостатком схемы тройного каскада (рис. \ref{p2_withDepU}) являются дополнительные затраты работы разделения по отношению к схемам двойного каскада с НОУ-разбавителем, возникающие при обогащении разбавленного обедненным ураном отхода второго каскада схемы, загрязненного четными изотопами.
\end{enumerate}

Ключевая из предложенных схем -- схема двойного каскада с НОУ-разбавителем -- на каждом из рассмотренных рециклах позволяет извлечь более 80\% от массы $^{235}$U из исходного регенерированного урана, поступившего на обогащение.

\pdfbookmark{Заключение}{conclusion}                                  % Закладка pdf
В \underline{\textbf{заключении}} приведены основные выводы, сделанные в ходе диссертационного исследования и перечислены полученные результаты.

По итогу исследования выдвигаются рекомендации по использованию результатов работы для обогащения регенерированного урана в условиях однократного и многократного рецикла в различных видах топлива.

Перспективы дальнейшей разработки темы состоят в проведении технико-экономического анализа для оценки эффективности представленных схем в контексте всей цепочки ядерного топливного цикла, а также с учетом возникающих в этой цепочке изменений при использовании регенерата урана по отношению к открытому топливному циклу. Помимо этого, необходима проработка технологических проблем каждой из схем, в частности, с точки зрения возможности эксплуатации и обслуживания оборудования в условиях работы с материалами, имеющими более высокую, чем природный уран удельную активность. Например, подобные условия возникают в каскадах, концентрирующих в легкой фракции $\alpha$-активные изотопы $^{232,234}$U.

\noindent \begin{enumerate}[leftmargin=0.4cm]
\item Предложен модифицированный двойной каскад с НОУ-разбавителем из природного урана, применимый  для обогащения регенерированного урана в условиях многократного рецикла урана в топливе легководных реакторов и позволяющий получить продукт, отвечающий всем требованиям на концентрации четных изотопов. 
\noindent \begin{enumerate}[leftmargin=0.4cm]
    \item На основе теории квазиидеального каскада разработаны методики расчета и оптимизации предложенной каскадной схемы по различным критериям эффективности (затраты работы разделения, расход природного урана, степень извлечения $^{235}$U из регенерата, степень извлечения $^{235}$U из всех питающих потоков схемы). Показано, что эффективность предложенной каскадной схемы по тому или иному критерию зависит от выбранного диапазона изменения концентрации $^{235}$U в потоке легкой фракции каскада II. Наиболее выгодные с точки зрения выбранных критериев эффективности наборы параметров каскадной схемы лежат в области, где концентрация $^{235}$U в потоке легкой фракции каскада II превышает 20\%. Это означает, что при практической реализации модифицированного двойного каскада целесообразно рассматривать возможность получения в отдельных потоках такой схемы концентраций $^{235}$U, превышающих 20\%, и, в первую очередь, в потоке $P_2$. 
    \item Анализ эффективности предложенной каскадной схемы с точки зрения потерь $^{235}$U показал, что схема обеспечивает экономию природного урана по сравнению с открытым топливным циклом на уровне 15-20\% в зависимости от исходного изотопного состава регенерата. Это превышает аналогичные показатели для простейших разбавляющих схем практически вдвое.
    \item Предложенная схема позволяет полностью решить задачу обогащения регенерата в широком диапазоне внешних условий и ограничений, что создает базис для ее практической реализации и поиска наиболее эффективных режимов ее работы.
\end{enumerate}

\item Показано, что модификации ординарного каскада для обогащения и разбавления регенерированного урана принципиально не решают задачу обогащения регенерированного урана при одновременном выполнении условий на концентрации четных изотопов в товарном НОУ и обеспечения расходования заданной массы регенерата на получение этого НОУ для составов регенерата с исходным содержанием четных изотопов, превышающим предельные значения для товарного НОУ. 

Основная причина невозможности решения задачи состоит в том, что в рассматриваемых схемах число свободных параметров оказывается меньшим, чем число условий, которые необходимо одновременно удовлетворить. В результате такие схемы могут обеспечить решение задачи только в частных случаях, когда в обогащение поступает регенерированный уран с исходными концентрациями четных изотопов ниже предельных значений для товарного НОУ.

\item Обоснованы способы вовлечения загрязненной четными изотопами фракции, возникающей в двойных каскадах при очистке от $^{232}$U, с учетом полной или частичной подачи данной фракции: а) в отдельный двойной каскад, осуществляющий наработку низкообогащенного урана для последующей топливной кампании реактора; б) перемешивании этой фракции с потоками обедненного урана и низкообогащенного урана для получения дополнительной массы товарного НОУ; в) в третий каскад с предварительным перемешиванием ее с природным, обедненным и/или низкообогащенным ураном. Для каждого из способов проанализированы их достоинства и недостатки, и вытекающие из них области применения, а также рассчитаны получаемые преимущества относительно открытого ЯТЦ.
 
\item Результаты работы применимы для проведения дальнейшего технико-экономического анализа каждой из схем на основе их интегральных показателей, таких как расход природного урана, затраты работы разделения, потери $^{235}$U в цикле в контексте всей цепочки ядерного топливного цикла, а также с учетом возникающих в этой цепочке изменений при использовании регенерата урана по отношению к открытому топливному циклу. Полученные в диссертации результаты дополняют теорию каскадов для разделения изотопов. В частности, предложенные в работе методики оптимизации двойных и тройных каскадов могут быть адаптированы к случаю разделения многокомпонентных смесей неурановых изотопов в каскадах центрифуг.

\end{enumerate}


\insertbibliofull   
% \insertbiblioauthor
\pdfbookmark{Литература}{bibliography}